\documentclass[letterpaper,10pt]{article}
\usepackage[margin=1in]{geometry}
\usepackage{hyperref}
\usepackage{graphicx}
\usepackage[justification=centering,font=small,labelfont=bf]{caption}
\usepackage{fancyhdr}
\usepackage{fontspec}
\defaultfontfeatures{Ligatures=TeX}
\usepackage[backend=biber,
date=iso,
seconds=true,
style=numeric,
bibencoding=utf8,
]{biblatex}
\addbibresource{\jobname.bib}

\pagestyle{fancy}
\rhead{DSSCAW Technical Report \#002}

\title{$\mu$nandfs:\\
A NAND Filesystem for Embedded Platforms\thanks{
 \href{https://www.dsscaw.com/}{Dirty South Supercomputing} on behalf
 of \href{https://www.vakaros.com/}{Vakaros} of Atlanta, GA.
}\\
}
\author{Nick Black, Consulting Scientist\\
\texttt{nickblack@linux.com}
}

%%%%%%%%%%%%%%%%%%%%%%%%%%%%%%%%%%%%%%%%%%%%%%%%%%%%%%%%%%%%%%%%%%%%%%%%
\begin{document}
%%%%%%%%%%%%%%%%%%%%%%%%%%%%%%%%%%%%%%%%%%%%%%%%%%%%%%%%%%%%%%%%%%%%%%%%
\maketitle
\thispagestyle{fancy}
\date{}
\begin{abstract}
\setlength{\parindent}{0pt}
I was tasked with designing and implementing a persistent associative array
mapping names to arbitrary data\footnote{I.e. a single-directory filesystem, often called a \textit{blobstore}.}
using the Nordic Semiconductor nRF52840 and two Winbond W25N01GV gigabit SLC
NAND chips, along with necessary drivers. The requirements permitted 4KB of RAM,
allowed no use of other persistent storage, and mandated a fully asynchronous
API running on ``bare metal'' (no OS, realtime or otherwise). I detail my
resulting design, and demonstrate its generally performant and robust
fulfillment of these specs. I also describe its pathological worst case
behaviors.
\end{abstract}
%%%%%%%%%%%%%%%%%%%%%%%%%%%%%%%%%%%%%%%%%%%%%%%%%%%%%%%%%%%%%%%%%%%%%%%%
\section{Introduction}
The nRF52840\parencite{nrf52840} SoC pairs an ARM Cortex-M4F with 1MB of
NOR flash and 256KB of RAM, along with a wealth of interconnection
capabilities. This storage is shared with the ``S140
SoftDevice''\parencite{s140}, a closed-source BlueTooth stack, which consumes
a bit more than 100KB.

%%%%%%%%%%%%%%%%%%%%%%%%%%%%%%%%%%%%%%%%%%%%%%%%%%%%%%%%%%%%%%%%%%%%%%%%
\printbibliography
\end{document}
