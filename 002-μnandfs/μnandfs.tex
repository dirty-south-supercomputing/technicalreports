\documentclass[letterpaper,10pt]{article}
\usepackage[margin=1in]{geometry}
\usepackage{hyperref}
\usepackage{graphicx}
\usepackage[justification=centering,font=small,labelfont=bf]{caption}
\usepackage{fancyhdr}
\usepackage{fontspec}
\defaultfontfeatures{Ligatures=TeX}
\usepackage{xeCJK}
\usepackage[backend=biber,
date=iso,
seconds=true,
style=numeric,
bibencoding=utf8,
]{biblatex}
\addbibresource{\jobname.bib}

\pagestyle{fancy}
\rhead{DSSCAW Technical Report \#002}

\title{$\mu$nandfs:\\
A NAND Blobstore for Memory-Starved Platforms\thanks{
 \href{https://www.dsscaw.com/}{Dirty South Supercomputing} on behalf
 of \href{https://www.vakaros.com/}{Vakaros} of Atlanta, GA.
}\\
}
\author{Nick Black, Consulting Scientist\\
\texttt{nickblack@linux.com}
}

%%%%%%%%%%%%%%%%%%%%%%%%%%%%%%%%%%%%%%%%%%%%%%%%%%%%%%%%%%%%%%%%%%%%%%%%
\begin{document}
%%%%%%%%%%%%%%%%%%%%%%%%%%%%%%%%%%%%%%%%%%%%%%%%%%%%%%%%%%%%%%%%%%%%%%%%
\maketitle
\thispagestyle{fancy}
\date{}
\begin{abstract}
I was tasked with designing and implementing a persistent associative array
mapping names to arbitrary data---i.e. a single-directory filesystem, often
called a \textit{blobstore}---using the Nordic Semiconductor nRF52840 and two
Winbond W25N01GV gigabit SLC NAND chips. The contract also required necessary
QSPI drivers. The requirements permitted 4KB of RAM, permitted 4KB of RAM,
allowed no use of other persistent storage, and mandated a fully asynchronous
API running on ``bare metal'' (no OS, realtime or otherwise). I detail my
resulting deliverable, $\mu$nandfs, and demonstrate its generally performant
and robust fulfillment of these specs. I also describe its pathological worst
case behaviors.
\end{abstract}
%%%%%%%%%%%%%%%%%%%%%%%%%%%%%%%%%%%%%%%%%%%%%%%%%%%%%%%%%%%%%%%%%%%%%%%%
\section{Introduction}
The nRF52840\parencite{nrf52840} SoC pairs an ARM Cortex-M4F with 1MB of
NOR flash and 256KB of RAM, along with a wealth of interconnection
capabilities. This storage is shared with the ``S140
SoftDevice''\parencite{s140}, a closed-source BlueTooth stack, which consumes
slightly more than 100KB of RAM and significant flash. One QSPI and three SPI
masters are available, and can be clocked up to 32MHz. Two Winbond W25N01GV\parencite{winbond}
128MB NAND flashes were added to the PCB, each capable of QSPI at up to 104MHz.
Nordic's nRF5 SDK\parencite{nrf52sdk} version 15.3.0 was linked into our binary,
and the DUT was probed via 10-pin J-Link\parencite{segger} connection from
an nRF52-DK\parencite{nrf52dk}.

%%%%%%%%%%%%%%%%%%%%%%%%%%%%%%%%%%%%%%%%%%%%%%%%%%%%%%%%%%%%%%%%%%%%%%%%
\printbibliography
\end{document}
