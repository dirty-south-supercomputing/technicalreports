\section{A brief history of character graphics}
\label{sec:terminals}
The earliest terminals making use of glyphs\footnote{Konrad Zuse's Z3, generally
 considered the first programmable digital computer, communicated with its
operator through a matrix of blinkenlights and a not unsteampunkish keyboard that resembled the
Burroughs typewriters of its era\cite{zuse}.} printed them to paper, and are of
interest to us only so far as our modern term ``tty'' is rather dubiously
derived from ``TeleTYpewriter'', as these cantankerous contraptions were
known\footnote{Though we do hear of their Snoopy calendars in the songs of
legend\cite{quiche}.} (people had less experience abbreviating in those days).

These devices most typically printed 72 characters per line (CPL), a limit that
has persisted in strange places\cite{pandoc} through the modern era. Another constant
you'll see from time to time is 132 CPL, derived from line printers such as the
IBM 1403, the DEC LP11, and the Centronics 101\cite{ibm1403}. Most common,
however, is the 80 column line originating in 1928's 7¾x3¼x0.007in IBM
Computer Card (as designed by Clair D.\ Lake, deriving from the 1890 U.\ S.\
Census cards of Herman Hollerith\ldots themselves borrowing from Joseph
Jacquard's automation in 1804 of punched card loom control technology pioneered
by Basile Bouchon in 1725\cite{cards}). To this day, so long as your wacky
output device can do 80 columns, eh, that's good enough. In all these cases,
the limit arises from the number of characters that could be printed, using the
technology of the time, on their feeder paper (8.5in and 14in in the case of
printers).

On, then, to the ``Glass TTYs'' (ugh) and Visual Display Units of the 1970s.
Pictured in Figure~\ref{fig:terminals} are the Computer Terminal Corporation
Datapoint 3300, the Lear Sigler, Inc.\ ADM-3A, the Hazeltine 1500, and the
Soroc IQ-120. Lacking microcontrollers, and generally implementing no
independent control sequences, such devices are today often known as ``dumb
terminals'' (this term was originally a registered trademark of Lear Sigler,
see Figure~\ref{fig:adm3}). Already the 80x24 ``standard'' (it is not a standard) was emerging
(the DEC contemporaries listed were already pretty ``smart'', using proprietary
control codes):

\begin{table}[h]
\begin{center}
  \begin{tabular}{ |c|c|c| }
    \hline
    IBM 2260 Model 1 & 1965 & 40x6 \\
    \hline
    Datapoint 3300 & 1969 & 72x25 \\
    \hline
    DEC VT05 & 1970 & 72x24 \\
    \hline
    IBM 3277 Model 2 & 1971 & 80x24 \\
    \hline
    Textronix 4010 & 1972 & 74x35 \\
    \hline
    DECscope VT52 & 1974 & 80x24 \\
    \hline
    LSI ADM-3A & 1976 & 80x12, 80x24 \\
    \hline
    Hazeltine 1500 & 1977 & 80x24 \\
    \hline
    Sororc IQ-120 & 1977 & 80x24 \\
    \hline
  \end{tabular}
\caption{Some historical terminals and their resolutions.}
\end{center}
\end{table}

Why 80x24 (or 80x25, as you'll also see)\cite{infoworld80}? The 80 almost certainly arises from
the desire to display an entire punched card (this \textit{is} a standard---see
ANSI X3.21-1967/FIPS PUB 13, ``Rectangular Holes in Twelve-Row Punched
Cards'')\cite{sonicdelay}. The origin of 24 is less clear. 24 is highly
composite (it has more divisors than any smaller number), and it is the largest
integer divisible by all natural numbers not larger than its square root. There
are of course 24 hours in a day. 24 divides the scanline counts of both NTSC
and PAL at 480 and 576, respectively. 24 rows of 80 columns at a byte per
column utilize 93.75\% of a 2KiB memory, leaving exactly 128 bytes left over,
and everyone loves a good power of 2.

The Aaronites, Levite descendents of Moses's brother Aaron, the first
\texthebrew{כהן גדול}
(High Priest),
form the priestly \texthebrew{כֹּהֲנִים}; they were divided into 24 courses. The Buddha's Dharma Chakra (Wheel of Dhamma)
in its Ashoka form sends forth 24 spokes. But perhaps I grow esoteric, and even
speculative\ldots in truth, 80x24 almost certainly owes its questionable
existence to IBM's punched cards, IBM 2260 and 3270 wanting compatibility with
IBM printers, the upstart DEC wanting compatibility with IBM software for their
VT52 and legendary VT100, and the VT100 subsequently becoming a \textit{de facto} standard for
four decades.

\begin{figure}
  \centering \includegraphics[width=.9\linewidth]{media/dumbterminals.jpg}
\caption[Dumb terminals of the 1970s.]{Clockwise starting from upper left: a dork and his LSI ADM (``American
  Dream Machine'', supposedly). That poor woman with the Sororc IQs looks stoned out
  of her gourd. Grizzly Adams rocks a Datapoint 3300, but really his mind is on
  seeing Skynyrd shred it this weekend. Finally, we have Frank the Cocaine
  Ranger and his Electric Hazeltine 1500 Band. \textit{Gott im Himmel}, the 70s were \textit{unseemly}.}
\label{fig:terminals}
\end{figure}

\begin{figure}
  \begin{minipage}{0.5\textwidth}
  \centering
    \includegraphics[width=.5\linewidth]{media/digital-terms.jpeg}
    \caption{Digital Equipment Corporation terminals of the 1970s and 1980s.}
  \end{minipage}\hfill
  \begin{minipage}{0.4\textwidth}
  \centering
    \includegraphics[width=.45\linewidth]{media/vt220-charset.png}
    \caption[VT220 glyph dump.]{The VT220's glyphs from a ROM dump. VT100 implemented most of the
      first seven columns. Note the existence of box-drawing characters\cite{crttypography}.}
  \end{minipage}\hfill
\end{figure}

\begin{figure}
  \centering
  \includegraphics[width=.8\linewidth]{media/adm3.jpeg}
  \caption{A strange time.}
  \label{fig:adm3}
\end{figure}

\subsection{The DEC VTxxx terminals and ANSI X3.64-1979}

Introduced in August 1978, the VT100 ushered in a new era of smart terminals
using commodity (Intel) microprocessors, implementing portions of the upcoming
ANSI X3.64 standard (itself based on 1976's first edition of ECMA-48) along
with DEC extensions\footnote{The VT100 \textit{did not} implement all of X3.64,
nor was X3.64 derived from the VT100. The VT100 didn't do color, nor did it
insert or delete lines. It furthermore implemented several features outside
the scope of ECMA-48's first edition.} This series would go on to sell over
six million units, and it was a rare vendor that didn't include some degree
of DEC VT compatibility. Each major iteration of the series was designed to
encompass all functionality of prior iterations, beginning with the VT100's
faithful emulation of the earlier era's VT52\cite{vt52}. The VT102 cut down on the cost
and size of the VT100, and included the 132-CPL mode by default; they were
otherwise essentially the same device\cite{vt100}.

The original \texttt{xterm} was written as an emulator of the VAXStation 100
(VS100), and slowly acquired scattered features from the VT100, ANSI, and other
sources\cite{xtermfaq}. Thomas E.\ Dickey (the current maintainer of \Gls{ncurses}
and xterm) began working on XTerm in the mid-90s, and by 1996 had added the
\texttt{decTerminalID} resource following the addition of much VT220 compatibility.

\textbf{FIXME keep goin'\ldots}

\subsection{The Curses API}
