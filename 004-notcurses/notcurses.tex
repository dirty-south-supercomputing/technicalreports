% arara: xelatex: {shell: true}
% arara: biber
% arara: makeglossaries
% arara: xelatex: {shell: true}
% arara: xelatex: {shell: true}
% arara: xelatex: {shell: true}
\documentclass[letterpaper,10pt]{article}
\usepackage[margin=1in]{geometry}
\usepackage{newfloat}
\usepackage{hyperref}
\usepackage{graphicx}
\usepackage[font=small,labelfont=bf]{caption}
%\usepackage{draftwatermark}
\usepackage{fancyhdr}
\usepackage{makecell}
\usepackage{wrapfig}
\usepackage{parskip}
\usepackage[section]{placeins}
\usepackage{epigraph}
%\usepackage{sourcecodepro}
\usepackage{fontspec}
\usepackage[toc,nonumberlist,xindy]{glossaries}
\usepackage{xelatexemoji}
\usepackage{relsize}
%\setmonofont[Scale=0.7]{Source Code Pro}
\setmonofont[Scale=0.8]{Unifont}
\defaultfontfeatures{Ligatures=TeX}
\usepackage[table]{xcolor}
%\usepackage{pdfpages}
\usepackage[titletoc,title]{appendix}
\usepackage{minted}
\usepackage{xeCJK}
\usepackage{tipa}
\usepackage{polyglossia}
\usepackage{arabxetex}
\setmainlanguage{english}
\setotherlanguages{hebrew,french,bulgarian,russian,greek}
\newfontfamily\cyrillicfont[Script=Cyrillic]{Noto Sans}
\newfontfamily\hebrewfont[Scale=0.8,Script=Hebrew]{Cardo}
\newfontfamily\greekfont{Noto Sans}
\definecolor{dsscawpurp}{HTML}{b079b0}
\definecolor{dsscawpurpcap}{HTML}{6c286c}
\usepackage[font={color=dsscawpurpcap},labelfont={sc}]{caption}
\usepackage[backend=biber,
date=iso,
seconds=true,
style=numeric,
bibencoding=utf8,
]{biblatex}

\tracinglostchars=2

%\SetWatermarkText{DRAFT}
%\SetWatermarkScale{1}

\hypersetup{
  colorlinks=true,
  linkcolor=blue,
  pdftitle={Hacking the Planet with Notcurses},
}

\addbibresource{\jobname.bib}
\usemintedstyle{friendly}
\newenvironment{denseitemize}{
  \begin{itemize}
      \setlength{\itemsep}{0pt}
}{
  \end{itemize}
}
% An attractive 'C++'
\newcommand\CC{C\nolinebreak\hspace{-.05em}\raisebox{.4ex}{\relsize{-3}{\textbf{+}}}\nolinebreak\hspace{-.10em}\raisebox{.4ex}{\relsize{-3}{\textbf{+}}}\hspace{.2em}}

\pagestyle{fancy}
\rhead{
  \includegraphics[height=\fontcharht\font`\D,keepaspectratio=true]{../dsscaw-hdr.pdf}
  \textcolor{dsscawpurp}{DSSCAW Technical Report \#004}
}

\title{Hacking the Planet (with Notcurses)\\
A Guide to TUIs and Character Graphics
}
\author{Nick Black, Consulting Scientist\\
\texttt{nickblack@linux.com}
}

\makeglossaries
\setglossarypreamble{When possible, I have followed the definitions of
  RFC 2978\cite{rfc2978} and the Glossary of Unicode Terms\cite{unicodeglossary}.}
\loadglsentries{glossary}

%%%%%%%%%%%%%%%%%%%%%%%%%%%%%%%%%%%%%%%%%%%%%%%%%%%%%%%%%%%%%%%%%%%%%%%%
\begin{document}
%%%%%%%%%%%%%%%%%%%%%%%%%%%%%%%%%%%%%%%%%%%%%%%%%%%%%%%%%%%%%%%%%%%%%%%%
%\includepdf{media/frontcover.pdf}
\date{March 24, 2020}
\maketitle
\date{}
\vspace{1in}
\begin{center}
\includegraphics[width=.75\linewidth]{htp-with-notcurses.png}
\end{center}
\thispagestyle{empty}
%%%%%%%%%%%%%%%%%%%%%%%%%%%%%%%%%%%%%%%%%%%%%%%%%%%%%%%%%%%%%%%%%%%%%%%%

\clearpage
\pagenumbering{roman}

%%%%%%%%%%%%%%%%%%%%%%%%%%%%%%%%%%%%%%%%%%%%%%%%%%%%%%%%%%%%%%%%%%%%%%%%
\vspace*{1.25in}
\begin{figure}[!htb]
\centering
\includegraphics[width=1\linewidth]{media/ibm3279.jpg}
\caption[]{A programmer at her IBM 3279 2A \textit{(source: Jonathan Schilling under CCASA4)}.}
\end{figure}
\clearpage
\vspace*{1in}
\begin{center}
  \textit{For T.\ S.\ Eliot, il miglior fabbro.} \\
  \vspace{.25in}
  \textit{For Jeanette Martin, for exhortations to go H.A.M. \\
  \vspace{.25in}
  For Jim Greenlee, for speaking rigor to my programming.\\
  \vspace{.25in}
    For Prof.\ Hyesoon Kim, for introducing me to the glorious world
    inside the die.\\
  \vspace{.25in}
    For Prof.\ Richard Vuduc, for demonstrating serenity in brilliance, and kindness in dominance.\\}
  \vspace{1in}\ldots but mostly for Emily.
\end{center}
\clearpage
%%%%%%%%%%%%%%%%%%%%%%%%%%%%%%%%%%%%%%%%%%%%%%%%%%%%%%%%%%%%%%%%%%%%%%%%

\tableofcontents
%\vfill
%\begin{center}
%\includegraphics[width=1\linewidth]{media/widechars.png}
%\end{center}

\cleardoublepage
\phantomsection
\addcontentsline{toc}{section}{\listfigurename}
\listoffigures
\addcontentsline{toc}{section}{List of Listings}
\listoflistings
\addcontentsline{toc}{section}{\listtablename}
\listoftables
\clearpage
%%%%%%%%%%%%%%%%%%%%%%%%%%%%%%%%%%%%%%%%%%%%%%%%%%%%%%%%%%%%%%%%%%%%%%%%

\cleardoublepage
%%%%%%%%%%%%%%%%%%%%%%%%%%%%%%%%%%%%%%%%%%%%%%%%%%%%%%%%%%%%%%%%%%%%%%%%
\addcontentsline{toc}{section}{Foreward}
\section*{Foreward}
Hacking the Planet with Notcurses: A Guide to Character Graphics and TUIs.
Copyright © 2020 Nick Black.
ISBN: 9798620069491

This edition corresponds to version 1.2.4 of the Notcurses library, released
2020-03-24. Notcurses can be downloaded from
\url{https://github.com/dankamongmen/notcurses}. This document can be
downloaded from~\url{https://nick-black.com/htp-notcurses.pdf}.

Licensed under the Apache License, Version 2.0 (the ``License''); you may not
use this document except in compliance with the License. You may obtain a copy
of the License at \url{http://www.apache.org/licenses/LICENSE-2.0}.

Unless required by applicable law or agreed to in writing, software
distributed under the License is distributed on an ``AS IS'' BASIS,
WITHOUT WARRANTIES OR CONDITIONS OF ANY KIND, either express or implied.
See the License for the specific language governing permissions and
limitations under the License.

The entirety of this work is Free Documentation, written for love and released
to instruct. If you'd like to show thanks for my efforts, I encourage a donation to
the~\href{https://www.thefire.org/}{Foundation for Individual Rights in Education} (\url{https://www.thefire.org/}).
Alternatively, buy the paperback!

This work was prepared on a Debian Unstable Linux workstation and an Arch Linux laptop,
using Vim, \XeLaTeX, and the GIMP. A FreeBSD 12 machine was emulated with QEMU.

Tetris © The Tetris Company, LLC.
\textit{Hackers (1995)} © United Artists Pictures.
\textit{House of Leaves (2000)} © Penguin Random House.
``Ruins with Rain'' © Mark Ferrari/Living Worlds.
``Final Fantasy'' © Square Enix Co Ltd.
``Super Mario Bros.'' © Nintendo of America.
``Ninja Gaiden'' © Koei Tecmo America.
``Street Fighter II'' and ``Mega Man 2'' © Capcom of America.
Please don't sue me.

\addcontentsline{toc}{subsection}{¡Peligro!}
\subsection*{¡Peligro!}

The code written for this book attempts to minimize use of vertical space
(fewer pages → cheaper book) without eliding error checking (or crossing into
the realms of the grotesque). Error handling is a fundamental slog of C
programming, one that inevitably complicates reliable applications.

These listings cannot be considered examples of good general style\ldots but they \textit{do} get the job done.

Three irregular idioms show up frequently:

\begin{denseitemize}
\item{Use of |= to collect non-zero return values from each of a series of
      non-interdependent function calls.}
\item{Right-hand-side conditionals fed into |=, e.g. \texttt{r |= (printf("dank") < 0);}.}
\item{Extensive use of ||'s short-circuiting property.}
\end{denseitemize}

\addcontentsline{toc}{subsection}{Errata}
\subsection*{Errata}
A list of errata for this First Edition is kept at
\begin{center}
\url{https://nick-black.com/dankwiki/index.php/Hacking\_The\_Planet!\_with\_Notcurses}
\end{center}
Please contact me with corrections, either via mail at \href{mailto:nickblack@linux.com}{nickblack@linux.com},
or via PR on
\begin{center}
\url{https://github.com/dirty-south-supercomputing/technicalreports}
\end{center}

\clearpage
%%%%%%%%%%%%%%%%%%%%%%%%%%%%%%%%%%%%%%%%%%%%%%%%%%%%%%%%%%%%%%%%%%%%%%%%
\pagenumbering{arabic}

\epigraph{Our fine arts were developed, their types and uses were established, in times
very different from the present, by men whose power of action upon things was
insignificant in comparison with ours. But the amazing growth of our
techniques, the adaptability and precision they have attained, the ideas and
habits they are creating, make it a certainty that profound changes are
impending in the ancient craft of the Beautiful.}{Paul Valéry}
\section{Introduction}

I implemented Notcurses in the winter of 2019 after having a few patches
rejected from NCURSES. The first commit was pushed 2019-11-16. It proved to
be seductive as hell, and it was only with difficulty that I tore myself away
following three months of hard work. I started writing this manuscript
2020-02-12, following the 1.1.8 release. By that time, Notcurses subsumed large
chunks of NCURSES, adding a great deal more. The project had three major goals:

\begin{denseitemize}
\item to provide NCURSES-like functionality with 24-bit color, safety in the
    presence of multithreading, and full Unicode support,
\item to reduce the amount of boilerplate code necessary for the UIs of my
    TUI applications, including \textit{growlight} and \textit{omphalos}, and
\item to portably facilitate the most vivid character graphics possible.
\end{denseitemize}

Many people asked how such a thing was useful. My usual response was that
numerous devices don't present a bitmap interface, that X11 GUIs run remotely
over SSH are effectively unusable beyond a local network, that plenty of
machines don't have a GUI environment installed, that there are obvious
applications for large outdoor displays, and that Sixel isn't well-supported
across different terminal emulators. It seems impossible in an age of
gigatransistor graphics cards, but the text environment still presents
perceivably less latency than most GUI toolkits. That I was able to remove
thousands of lines of NCURSES code from my applications was a nice side
benefit.

In truth, the main reasons were that it was fun, and I wanted to see how far
I could push it.

As I write this, Notcurses is present in Arch's AUR, and is awaiting promotion
from the Debian Incoming queue. Written as a C core, it enjoys \CC, Python, and
Rust wrappers. I have submitted it as a backend to NEStopia and RetroArch, and
intend to integrate it into Mesa as an OpenGL backend. So long as one can live
with the limited resolution available when a screen is divided into rectangular
cells, it can handle any graphics thrown at it. I hope to see it displace
NCURSES as the go-to character graphics library for new applications (there is
little value in porting existing applications to Notcurses, since an unchanged
application wouldn't take advantage of its advanced features).

While the X/Open Curses specification is unlikely to ever go away (nor should
it, as a lowest-common-denominator interface to devices Notcurses is unlikely
to ever support), I believe Notcurses to present a superior interface and
implementation for modern TUI applications.

The console ain't dead! Hack on, hax0rs.

\vfill

\begin{flushright}
  \textit{---February--March 2020, Atlanta}
\end{flushright}

\cleardoublepage
%%%%%%%%%%%%%%%%%%%%%%%%%%%%%%%%%%%%%%%%%%%%%%%%%%%%%%%%%%%%%%%%%%%%%%%%

\vfill
\begin{figure}
\centering
\includegraphics[width=.65\linewidth]{media/chunli-box-front.png}
\caption[]{A rendered scene from the ``chunli'' demo (see Chapter~\ref{sec:ncdemo}),
  using some of the advanced capabilities of Notcurses. The Chun-Li sprite has
  been loaded from a transparent PNG (Chapter~\ref{sec:libav}) atop boxes
  (Chapter~\ref{sec:boxes}) drawn using Unicode (Chapter~\ref{section:unicode})
  and linear interpolations (Chapter~\ref{sec:lerps}). Along the top is a
  menu (Chapter~\ref{sec:menus}) and an independent plane (Chapter~\ref{sec:planes});
  both can be controlled with mice (Chapter~\ref{sec:input}). In the center,
  the desktop can be seen through the transparent background of the terminal.}
\end{figure}
\vfill
\cleardoublepage

%%%%%%%%%%%%%%%%%%%%%%%%%%%%%%%%%%%%%%%%%%%%%%%%%%%%%%%%%%%%%%%%%%%%%
\section{Right, what's all this, then?}
\label{sec:start}
\epigraph{A terminal is at the end of an electric wire, a shell is the home of a turtle, tty is a strange abbreviation and a console is a kind of cabinet.}{Gilles Leblanc\cite{gillesSO}}
Character graphics, aka text mode, aka the display side of a terminal, is
visualization that works with fonts rather than a pixel framebuffer\footnote{``Contones'', as raster graphics are known to printers.}
or a vector canvas\footnote{Nothing keeps you from implementing character graphics
with pixels or vectors, of course.}. There is furthermore an expectation that
this font is a fixed-width one---that all rendered glyphs are integer multiples
of some narrowest non-trivial glyph.

Given the same display hardware,

\begin{denseitemize}
\item{Character graphics are usually strictly less powerful than pure raster graphics, and}
\item{their lower effective resolution typically implies lower bandwidth requirements.}
\end{denseitemize}

A TUI (text user interface) is a holistic model, view, and controller implemented
using character graphics. TUIs, like WIMP\footnote{Windows, icons, menus, pointers, a paradigm so pervasive that
the industry collectively treasures a Wiemarian\cite{thirdreich} memory of Xerox PARC's noble
engineers stabbed in the back by management (not unlike the Oatmeal-fostered\cite{fuckoatmeal}
myopia regarding Edison and Tesla. I'll take Thomas Alva over Matthew Inman any
day). It too often goes unmentioned that the Alto and Star were as unusable as they were visionary\cite{lightningdealers}.
This is of course still superior to Java, which isn't even visionary.} GUIs,
freely move the cursor around their rectilinear display, as opposed to
line-oriented CLIs and their ineluctable marches through the scrolling region.

Given the same interactive task,\footnote{These relations are not
fundamental, but emerge from the grim meathook realities of GUI toolkits.}

\begin{denseitemize}
\item{A TUI implementation is almost certainly a smaller memory and disk footprint than a GUI,}
\item{a good TUI implementation might introduce less latency, and}
\item{a properly-done TUI implementation can often be significantly more portable.}
\end{denseitemize}

It can also be a big pile of character graphics garbage. A TUI offers
less resolution, less flexibility, and (due to monospaced fonts) less total
text space. Applications must be carefully designed for the limitations of
a dynamic textual environment.

For over two decades, NCURSES (a free software implementation of the X/Open Curses\cite{cursesosi}
specification, plus extensions\cite{ncursesfaq}) has been a ubiquitous go-to for implementing
TUIs. Maintainter (and author, in large part) Thomas E.\ Dickey
exemplifies conservative and fastidious stewardship. Perfectly lovely TUIs can
be built using NCURSES (as seen in Figure~\ref{fig:ncurses-tuis}), but it \textit{does}
have its origins in the 8-bit era, and shows its age.

\begin{figure}[!hbtp]
  \centering
    \includegraphics[width=.4\linewidth]{media/tui-ncmpcpp.png}
    \hfill
    \includegraphics[width=.4\linewidth]{media/tui-omphalos.png}
    \caption[NCURSES TUIs: Ncmpcpp and Omphalos.]
    {Left: \texttt{ncmpcpp}, a \CC application
      that has driven my Music Player Daemon since 2008 or so.
      Right: \texttt{omphalos}, a C network exploration tool
      written using NCURSES in its extended mode.}
  \label{fig:ncurses-tuis}
\end{figure}

\begin{figure}[!htb] \centering
    \includegraphics[width=.4\linewidth]{media/tui-mapscii.png}
    \hfill
    \includegraphics[width=.4\linewidth]{media/tui-growlight.png}
    \caption[Non-NCURSES TUIs: Mapscii and Growlight.]{Left: \texttt{mapscii}, a
    node.js application, blew my mind when I first saw it. The high resolution
    is achieved by using Braille characters, trading away some color control.
    Right: \texttt{growlight}, a disk manager, began life as an NCURSES C
    program, but was ported to Notcurses in 2019.}
  \label{fig:notncurses-tuis}
\end{figure}

Implementing a TUI will usually require, at a minimum:
\begin{denseitemize}
\item{Receiving input from user devices, including keyboards and mice,}
\item{some manner of user configuration flow (menus, etc.),}
\item{watching for some other event(s) from the system, and},
\item{juggling these various components without wastefully polling, nor
       introducing undue latency, and enforcing safe synchronized access to
       the graphics interface.}
\end{denseitemize}

Perhaps most terrifyingly, it will require user interface design. Notcurses
attempts to assist with this by providing numerous ready-made widgets.

\pagebreak

This text has two goals:
\begin{denseitemize}
\item{To provide a firm footing for design and implementation of character
    graphics and TUIs, elucidating the dimensions of design, along with difficulties
    to avoid, and}
\item{to serve as ``narrative reference'' for my Notcurses
      library, and as a starting place for newcomers.}
\end{denseitemize}

\begin{figure}[!htb]
\centering \includegraphics[width=.5\linewidth]{media/emacs-xerox.jpg}
  \caption{Put not your trust in hackers making a fetish of Xerox PARC.}
  \label{fig:xeroxemacs}
\end{figure}

Cell graphics are primarily the realm of \textit{terminals}, which for the
purposes of this book encompass any means by which input devices act to drive
some process generating glyph-based output to a display. This includes hardware
terminals (inputs integrated with displays, connected to a computer as a unit),
operating system consoles (text-mode interfaces operating with the graphics
engine directly connected to the terminal driver), terminal multiplexers (tools
like \texttt{screen}, \texttt{tmux}, and \texttt{mosh}, providing a
memory-persistent virtual terminal with which other terminals can interact),
and terminal emulators (applications which present a virtual terminal atop the
shared input and raster output methods of a graphical user environment).
There's some vagueness and variety involved with these terms.

At its heart, a terminal is a line discipline plus two buffers: an input buffer
to collect user-generated events (possibly from multiple devices), and an
output buffer to be processed and displayed. The buffers can be modeled as byte
streams, mutating the output at the time of their display (in contrast to e.g.\
a framebuffer, where the entirety of the screen is present at any given time).
The earliest terminals were electromechanical teletypes, reproducing their
input as line-based print on paper. These gave rise to ``dumb terminals''
(cathode-ray displays with a scrolling rectilinear output area). ``Smart
terminals'' followed, with the ability to move freely within their display
area, and also to extract and act upon ``control codes'' embedded in the output
stream. The text modes of the first video cards were designed around the
capabilities of these smart terminals. This brings us to the present, wherein
high-powered LED displays have their pixels summoned up and ordered into
formations suitable for the reconstruction of 1970s technology (a history of
terminals is presented in Appendix~\ref{sec:terminals}).

The machine on which I'm preparing this \XeLaTeX\ contains a
TU104 GPU consisting of over thirteen billion 12nm-process transistors,
rendering its output to a 3440x1440 (almost five megapixel) display. Deep
within its silicon heart remains a VGA 80x25 text mode engine\cite{vga}, inherited
largely unchanged from the EGA, the CGA\footnote{The Color Graphics Adapter was
unmitigated trash, but you could do some crazy things with it. People were
still finding things out this decade\cite{cga1024}, resulting in the
nigh-obscene ``8088 MPH'' demo that won Revision 2015\cite{revision2015}.},
the IBM Monochrome Display Adapter\footnote{The
history of video display standards since 1981's MDA is a story of imprecision,
dashed hopes, and idle dreams. Good luck finding authoritative references
for anything beyond \texttt{int 10h} real mode operation prior to version 1.0
of the SuperVGA VESA BIOS Extension\cite{videostandards}, released 1989-10-01\cite{vesa}.},
and before that smart terminals\footnote{As early as 1971, the block-oriented
IBM 3277 Model 2 ``green screen'' shipped with 80x24.}.


I mainly use this modern marvel to drive terminal emulators of 80 columns.

\cleardoublepage
%%%%%%%%%%%%%%%%%%%%%%%%%%%%%%%%%%%%%%%%%%%%%%%%%%%%%%%%%%%%%%%%%%%%%%%%

\section{Using direct mode with standard I\/O}
\label{sec:direct}
\epigraph{Unscrew the locks from the doors!\\Unscrew the doors themselves from their jambs!}{Walt Whitman, \textit{Song of Myself}}
Many tools don't intend to be full-screen TUI applications, but instead
implement that purest of UNIX interfaces: newline-delimited text, oblivious
to screen geometry, capable of being fed as input to other, similar programs.
For such tools, the full Notcurses capabilities are neither necessary nor
desirable. These programs are typically non-interactive: humans might peruse
their outputs and prepare their inputs, but they effectively run as a batch
task.

 Such tools might still want to colorize and otherwise style their output, at
least when being output to a terminal. This can be accomplished using the
\texttt{ncdirect} subset of Notcurses, and is known as \textit{direct mode}. Direct
mode functionality should not usually be mixed with other Notcurses calls.
Unlike full Notcurses, there is no explicit rendering step in direct mode, and
it is intended to be mixed among other use of standard I/O. Essentially, direct
mode ``styles your \texttt{printf()}s.'' Similarly to full Notcurses, direct mode
requires a valid and correct terminfo database entry, supplied via either the
\texttt{termtype} parameter to \texttt{ncdirect\_init()} or the \texttt{TERM} environment
variable. It does \textit{not}, however, require any particular encoding or
other locale properties\cite{setlocale} (full Notcurses requires a
properly-configured ASCII or UTF-8 locale).

Enter direct mode via a call to \texttt{ncdirect\_init()} with a successful
return of a non-\texttt{NULL} pointer to \texttt{struct ncdirect}. It is
typical to invoke this function as \texttt{ncdirect\_init(NULL, stdout)}. In this case, the terminal type must be present in the
\texttt{TERM} environment variable (this should have been done by the
terminal). The buffering and blocking status of \texttt{fp} will not be
changed. \texttt{NULL} is returned for any number of possible errors.
Otherwise, the \texttt{struct ncdirect} is ready to go, and should be cleaned
up with \texttt{ncdirect\_stop()}.

\begin{listing}[!htb]
\begin{minted}{C}
// Initialize a direct-mode notcurses context on the connected terminal at 'fp'. 'fp' must be a tty. You'll usually
// want stdout. Direct mode supportes a limited subset of notcurses routines which directly affect 'fp', and neither
// supports nor requires notcurses_render(). This can be used to add color and styling to text in the standard
// output paradigm. Returns NULL on error, including any failure initializing terminfo.
struct ncdirect* ncdirect_init(const char* termtype, FILE* fp);

// Release 'nc' and any associated resources. 0 on success, non-0 on failure.
int ncdirect_stop(struct ncdirect* nc);
\end{minted}
\caption{Initializing and stopping direct mode.}
\end{listing}

Between these two calls, inject stylizing control codes into the \texttt{FILE*} with
the \texttt{ncdirect} (the \texttt{stylebits} values are detailed in Chapter~\ref{sec:attribute}).
As detailed in Chapter~\ref{sec:channels}, the terminal has a ``default foreground color''
and ``default background color''. Return to these default colors with
\texttt{ncdirect\_fg\_default()} and \texttt{ncdirect\_bg\_default()}.

\begin{listing}[!htb]
\begin{minted}{C}
int ncdirect_bg_rgb8(struct ncdirect* n, unsigned r, unsigned g, unsigned b);
int ncdirect_fg_rgb8(struct ncdirect* n, unsigned r, unsigned g, unsigned b);
int ncdirect_fg(struct ncdirect* n, unsigned rgb);
int ncdirect_bg(struct ncdirect* n, unsigned rgb);
int ncdirect_styles_set(struct ncdirect* n, unsigned stylebits);
int ncdirect_styles_on(struct ncdirect* n, unsigned stylebits);
int ncdirect_styles_off(struct ncdirect* n, unsigned stylebits);
int ncdirect_clear(struct ncdirect* n);
int ncdirect_fg_default(struct ncdirect* n);
int ncdirect_bg_default(struct ncdirect* n);
\end{minted}
\caption{The \texttt{ncdirect} styling API.}
\end{listing}

Direct mode provides helpers for determining the terminal geometry.

\begin{listing}[!htb]
\begin{minted}{C}
int ncdirect_dim_x(const struct ncdirect* nc);
int ncdirect_dim_y(const struct ncdirect* nc);
\end{minted}
\caption{Geometry discovery with \texttt{ncdirect}.}
\end{listing}

Direct mode allows the cursor to be disabled, enabled, and moved in
two-dimensional space. Either \texttt{y} or \texttt{x} may be specified as -1
to maintain location on the associated axis.

\begin{listing}[!htb]
\begin{minted}{C}
int ncdirect_cursor_move_yx(struct ncdirect* n, int y, int x);
int ncdirect_cursor_enable(struct ncdirect* nc);
int ncdirect_cursor_disable(struct ncdirect* nc);
\end{minted}
\caption{Cursor management with \texttt{ncdirect}.}
\end{listing}

\subsection{Example: presenting \textit{\textcolor{blue}{House} of Leaves}}
Mark Z. Danielewski's experimental 2000 novel \textit{\textcolor{blue}{House} of Leaves}\cite{danielewski2000house} prints each
instance of the word \textcolor{blue}{house} in blue, even when it is a subword:

\begin{figure}[!htb]
\centering \includegraphics[width=.5\linewidth]{house-blue.png}
\caption[]{An excerpt from page 123 of \textit{\textcolor{blue}{House} of Leaves}.}
\label{fig:houseofleaves}
\end{figure}

We can easily write code to reproduce this effect for standard input and output.
Listing~\ref{list:holformatter} works as expected (see
Figure~\ref{fig:houseout}), but there are a few things worth noting about its
code. First, observe how much of the logic is devoted to checking and
propagating errors! Perhaps contrary to common expectation, reliable
code---especially when that code's primary effect is to write to
stdout---generally needs to check the results of e.g. \texttt{printf()} (what
happens if we're redirected to a file, and the disk is full?). A language
making use of exceptions would reduce if not eliminate this nonsense.

\begin{listing}[!htb]
\inputminted[]{C}{code/hol-formatter.c}
\caption{\texttt{hol-formatter.c}, a streaming formatter.}
\label{list:holformatter}
\end{listing}

\begin{figure}[!htb]
\centering \includegraphics[width=.75\linewidth]{hol-formatted.png}
\caption[\texttt{hol-formatter} as run on OCRd input.]{\texttt{hol-formatter} as run on our input. We use \texttt{tesseract} for OCR, with solid results.}
\label{fig:houseout}
\end{figure}

So long as we're dealing with either ASCII or UTF-8 input, our simple, old-skool
\texttt{tolower(3)} is satisfactory \textit{for this problem}. The key
observation is that UTF-8 encoded text can be compared for equality by
a structure-oblivious~\texttt{memcmp(3)}, as of course can ASCII.
Unless we need to color e.g.~\textcolor{blue}{\texttt{ℏöûⓈᴇ}} (maybe we should,
maybe we shouldn't) this is safe, simple, and sufficient. If we \textit{do}
wish to collapse distinct but by some measure similar EGCs, we should normalize
input as prescribed by Unicode Standard Annex \#15\cite{annex15}.

We don't switch from blue to some other specified color, because we don't know
the background color of the terminal. Some people, possibly aliens, don't favor
a dark terminal background. If the terminal background were white, and we had
just used e.g. \texttt{ncdirect\_fg(n, 0xffffff)}, text following
``\textcolor{blue}{house}'' would be invisible.

One might observe that a user with a blue background will have invisible
``\textcolor{blue}{house}'' text. This is a real issue, one lacking a perfect
solution\footnote{Applying \texttt{NCSTYLE\_STANDOUT} might or might not help.}.
It is not generally possible to discover the RGB values of the default colors.
I suppose all one can do is rest easy, serene in the belief that white
backgrounds are one thing, but people with chromatic backgrounds deserve
whatever happens to them.

\subsection{Example: colorizing a dumb game}
Imagine we've written the simple guessing game in Listing~\ref{list:guessgame}.

\begin{listing}[!htb]
\inputminted[]{C}{code/hilostdio.c}
\caption{\texttt{hilostdio.c}, a simple guessing game.}
\label{list:guessgame}
\end{listing}

The correct approach for a player is binary search, and for an $N$-bit
\texttt{long}, we expect to guess the number in no more than $N$ tries. Let's
color the output to indicate how bad of a guess was offered. We'll use red for
low guesses, blue for high guesses, and break the 256 shades of each (assuming
the other two components to be fixed) uniformly across the $N$ levels of
logarithmic distance\footnote{This would be a good place to employ \gls{gamma correction}.}.
If we wanted to do this (see Listing~\ref{list:hilodirect}) without direct use of RGB color,
we'd either need accept fewer shades, or be forced to reprogram the palette.

\begin{listing}[!htb]
\inputminted[]{C}{code/hilodirect.c}
\caption{\texttt{hilodirect.c}, a colorized version of the guessing game.}
\label{list:hilodirect}
\end{listing}

Stepping through the orders of magnitude\footnote{\texttt{\_\_builtin\_clzl()}
is a compiler intrinsic for \textit{count leading zeroes}. Exhaustive methods
for fast clzl can be found in \cite{hackerdelight}. Demonstrating that
absolute value of the difference of leading zeroes is a $lg_{2}$ difference
is left as an exercise for the reader.}, we get the expected gradient
(Figure~\ref{fig:colorguess}). Were we to actually play, the response would
converge to a balanced, strong green as we approached the correct answer.

\begin{figure}[!htb]
\centering \includegraphics[width=.75\linewidth]{media/hilodirect.png}
\caption{Colorized output from~\texttt{hilodirect.c}.}
\label{fig:colorguess}
\end{figure}

\subsection{Advanced coëxistence with stdio}
It is most common to initialize Notcurses with \texttt{stdout}, whether in
direct mode or fullscreen mode. This isn't the only way to operate, though.
By opening the tty directly using \texttt{/dev/tty}%\cite{tty4}%
, and providing
this \texttt{FILE*} to Notcurses, a program passing its standard output to
another process can make concurrent use of Notcurses on the display, in either
direct or fullscreen mode. This is how the \texttt{notcurses-pipe} program
works\footnote{See \url{https://github.com/dankamongmen/notcurses/issues/381}.}.

For programs that need to write to the terminal, but want to ``overlay'' some
Notcurses, fullscreen mode won't work (though the program could be run in an
\texttt{ncprocess} widget; see Chapter~\ref{sec:uiwidgets}). Direct mode, however, is
a possibility. I've not yet written the example\footnote{Send me patches! Or
I'll do it\ldots eventually \url{https://github.com/dankamongmen/notcurses/issues/382}.}, but it is possible to, for
instance, periodically acquire the current cursor position, move elsewhere on
the screen, update a HUD, and return to the departure position. Scrolling could
be addressed by retaining a copy of any obliterated output. This would suffer
a startup period of one screen, during which the area scrolled above the HUD
would be cleared. This could be avoided by aligning the HUD with the top of
the terminal.

\subsection{Use in multithreaded environments}
\label{sec:directthreads}
Direct mode calls reduce to a cached terminfo lookup and \texttt{fprint(3)}
calls on the provided \texttt{FILE*}. The former is read-only; all necessary
elements are acquired from terminfo at the time of context creation. The latter
has the same thread semantics as \texttt{fprintf(3)}: while it is \textit{safe}
for multiple threads to concurrently print to the same \texttt{FILE*}, there are
no guarantees of ordering or even atomicity. Given the existence of multibyte
UTF-8 output, let alone potentially lengthy escape sequences, it's thus practically
necessary that multiple threads working with the same \texttt{FILE*} work exclusively.

Multiple threads may freely call read-only functions such as \texttt{ncdirect\_fg()}.

\cleardoublepage

%%%%%%%%%%%%%%%%%%%%%%%%%%%%%%%%%%%%%%%%%%%%%%%%%%%%%%%%%%%%%%%%%%%%%%%%
\section{Using fullscreen mode}
\label{sec:fullscreen}
\epigraph{This is how space begins, with words only, signs traced on the blank page. To describe space: to name it, to trace it, like those portolano-makers who saturated the coastlines with the names of harbours, the names of capes, the names of inlets, until in the end the land was only separated from the sea by a continuous ribbon of text. Is the aleph, that place in Borges from which the entire world is visible simultaneously, anything other than an alphabet?}{Georges Perec, \textit{Species of Spaces}}
From this chapter forward, we will be using the fullscreen mode of Notcurses,
opening up all of its capabilities. This comes at a cost: while fullscreen mode
is being used, it is not safe to use standard I/O in conjunction with the
terminal controlled by Notcurses. Doing so is likely to (at a minimum) corrupt
the screen. If \texttt{stdout} and \texttt{stderr} are attached to the same
terminal (as they usually are in an interactive session), and \texttt{stdout}
is provided to Notcurses, output to \texttt{stderr} will corrupt the display
just as thoroughly as output to \texttt{stdout}. If your fullscreen Notcurses
program intends to log to \texttt{stderr}, you should first ensure that
it has been redirected or is otherwise going somewhere different than
\texttt{stdout}. Note that simply rerendering the output will \textit{not}
necessarily clean up corruption, even following \texttt{ncplane\_erase()}
operations, since Notcurses optimizes its rendering based on its concept of the
screen. A call to \texttt{notcurses\_refresh()} will be necessary to sync
the physical screen to Notcurses's concept thereof.

It is possible for the screen to be corrupted by external agents. For this
reason, Ctrl+L is by tradition bound to screen redrawing. You should hook this
input up to \texttt{notcurses\_refresh()} unless you have good reasons not to
do so (this is not default behavior of Notcurses only because Notcurses does
not itself drive the reading of input). It is sadly not possible for such
corruption to be efficiently and generally detected.

It is possible for the attached terminal to be resized, especially (but not
only) for terminal emulators in GUI windowing environments\footnote{This could
also happen when refitting a \texttt{screen} or \texttt{tmux} session.
Even on the Linux or FreeBSD console, this can happen due to a change in video
resolution.}. Notcurses can detect such events, and synthesizes
\texttt{NCKEY\_RESIZE} inputs in response to them. If the screen shrinks, the
excess data relative to the constant origin will no longer be displayed (i.e.
the material in the upper left will be retained). If the screen is enlarged,
any data uncovered will be displayed, and the new area will otherwise be empty.
Some widgets can intelligently resize themselves in the face of screen
geometry changes (see Chapter~\ref{sec:uiwidgets}).

Notcurses prepares a given terminal for fullscreen mode in \texttt{notcurses\_init()}

\begin{listing}[!htb]
\begin{minted}{C}
// Initialize a notcurses context on the connected terminal at 'fp'. 'fp' must
// be a tty. You'll usually want stdout. Returns NULL on error, including any
// failure initializing terminfo.
struct notcurses* notcurses_init(const notcurses_options* opts, FILE* fp);

// Destroy a notcurses instance, restoring the terminal to its original state.
int notcurses_stop(struct notcurses* nc);
\end{minted}
\caption{Initializing and stopping fullscreen mode.}
\end{listing}

Before calling \texttt{notcurses\_init()} (and usually as one of the first lines
of the program) it is necessary to set the current locale via the standard
library function \texttt{setlocale()}. A coverage of ANSI/ISO C locales is beyond
the scope of this text, but it is usually sufficient to call
\texttt{setlocale(LC\_ALL, "")}, relying on the user's configured \texttt{LANG}
environment variable. Notcurses only supports those locales using
US-ASCII or UTF-8 encodings (see Chapter~\ref{section:unicode} for more
information on character encodings), and its capabilities on US-ASCII
are \textit{severely} constrained. \texttt{notcurses\_init()} will return an
error for any other encoding (see Figure~\ref{fig:encodingfail}).

\begin{figure}[!htb]
\centering \includegraphics[width=.7\linewidth]{media/notcurses-init-fails.png}
\caption{Notcurses refusing to start due to an unsupported character encoding.}
\label{fig:encodingfail}
\end{figure}

By default (assuming the \texttt{enter\_ca\_mode} terminfo capability is expressed),
Notcurses attempts to enter the ``\gls{smcup}''. Using the alternate screen
implies:
\begin{denseitemize}
\item{The screen will be cleared upon entry,}
\item{Output will not be appended to the scrollback buffer, and}
\item{On exit, output will be cleared.}
\end{denseitemize}
Whether or not the original screen contents are restored is terminal-dependent
(if the \texttt{non\_rev\_rmcup} terminfo capability is defined, the original
contents will \textit{not} be restored). The alternate screen is generally
useful, but some users don't like it, so it's wise to expose this via a
configuration option. Disabling use of the alternate screen can be done via the
\texttt{notcurses\_options} field \texttt{inhibit\_alternate\_screen}.

Successful creation of a \texttt{struct notcurses} implies the existence of
a \texttt{struct ncplane}, the ``standard plane''\footnote{\texttt{ncplane}s,
discussed in depth in Chapter~\ref{ncplane}, are the fundamental drawing surfaces of Notcurses.}.
This standard plane cannot be destroyed without destroying the containing
Notcurses context, nor can it be moved or resized by the user. Its size always
matches Notcurses's concept of the terminal's screen size, and its origin
always corresponds precisely to the terminal's origin\footnote{Some will note
clear similarities to the X ``root window''\cite{joyofx}.}. Aside from these
restrictions, the standard plane is a drawable surface like any other
\texttt{ncplane}---it can be moved along the z-axis, written to with arbitrary
glyphs and styles, made transparent, etc.

Once you're done using a \texttt{struct notcurses}, it's important to destroy
it with \texttt{notcurses\_stop()}, even if your process exits abnormally. By
default, Notcurses registers signal handlers for most fatal signals. These
handlers will call \texttt{notcurses\_stop()} and then pass the signal to the
original actions. You can disable this with the \texttt{no\_quit\_sighandlers}
field of \texttt{notcurses\_options}, but there aren't very many good reasons
to do so.

\subsection{The \texttt{notcurses\_options} structure}
The first parameter to \texttt{notcurses\_init()} is a (possibly \texttt{NULL})
\texttt{notcurses\_options}. This structure has been defined such that the
default options are equivalent to a zero-initialized structure. Passing \texttt{NULL}
is thus equivalent to passing a zero-initialized \texttt{notcurses\_options}\footnote{Except
it's strong against changes to \texttt{notcurses\_options}'s size!}.
The fields therein include:
\begin{denseitemize}
\item{\texttt{const char* termtype}: The name of the terminfo database entry to
    use. If \texttt{NULL}, the value of the environment variable \texttt{TERM}
    is used. Failure to initialize the terminfo database will result in a
    \texttt{notcurses\_init()} failure.} A defined but invalid or suboptimal
    entry can result in garbage, missing output, poor performance, reduced
    colors, and unsightly weight gain.
\item{\texttt{bool inhibit\_alternate\_screen}: As noted above, this prevents
    Notcurses from making use of the alternate screen, even if the \texttt{enter\_ca\_mode}
    terminfo capability is defined. It's best to wire this up to a user-managed
    option. Not using the alternate screen can look weird upon return to the
    shell (see Figure~\ref{fig:altscreen}).

\begin{figure}[!htb]
\centering \includegraphics[width=.7\linewidth]{media/no-alternate-screen.png}
\caption[Inhibiting use of the alternate screen.]{\texttt{notcurses-demo} can be invoked with \texttt{-k} to avoid
  using the alternate screen. Here, we see its output left on the screen as
  we return to our shell.}
\label{fig:altscreen}
\end{figure}
  }
\item{\texttt{bool retain\_cursor}: Notcurses hides the cursor by default.
    Set this to keep the cursor visible (the cursor can be turned on and off
    at runtime with \texttt{notcurses\_cursor\_enable()} and
    \texttt{notcurses\_cursor\_disable()}).}
\item{\texttt{bool suppress\_banner}: At startup, Notcurses emits some
    diagnostics and/or warnings, including version information and details
    about the current terminal. At shutdown, it prints performance statistics.
    These outputs \textit{do not} go to the alternate screen. Set this
    field to disable these outputs, but be aware that doing so might hide
    important warnings (see Figure~\ref{fig:banner}).

    \begin{figure}[!htb]
      \centering \includegraphics[width=.7\linewidth]{media/notcurses-banner.png}
      \caption[Notcurses initialization warnings.]{Initializing Notcurses without 24-bit color support will
        generate a warning, hopefully provoking your users to set it up.}
      \label{fig:banner}
    \end{figure}
}
\item{\texttt{bool no\_quit\_sighandlers}, \texttt{bool no\_winch\_sighandler}:
    As noted above, Notcurses by default registers signal actions for the normally fatal
    \texttt{SIGABRT}, \texttt{SIGINT}, \texttt{SIGQUIT}, and \texttt{SIGSEGV}.
    These handlers will call \texttt{notcurses\_stop()} before propagating the
    signal to the original actions. This is usually desirable, as the screen
    will not otherwise be restored to its previous state. In addition, \texttt{SIGWINCH}
    is caught in order to generate \texttt{NCKEY\_RESIZE} inputs. If you
    disable these handlers, you'll almost certainly want to replace them with
    similar functionality.}
\item{\texttt{FILE* renderfp}: If not \texttt{NULL}, this designates a file
    handle open for writing. In addition to the terminal, each rendered scene
    will be written to this file. This is intended for debugging.}
\item{\texttt{int margin\_t}, \texttt{int margin\_r},\texttt{int margin\_b}, \texttt{int margin\_l}}:
    Margin requests on the top, right, bottom, and left, respectively, of
    the rendering area (see Figure~\ref{fig:margins}). These requests will be satisfied on a best-effort
    basis---requesting more margin than is actually available is not an error.
    There must always be at least one row and one column available. If the
    alternate screen is being used, the margin areas will be cleared. Otherwise,
    they will be left uncleared. The margins are recomputed on a resize.
\end{denseitemize}

\begin{figure}[!htb]
    \centering
    \includegraphics[width=.75\linewidth]{media/margins.png}
    \caption{Margins can be used around the rendering area.}
    \label{fig:margins}
\end{figure}

\subsection{Functions on \texttt{notcurses} objects}
\label{sec:notcursesfuncs}
Output is not written to this top-level \texttt{struct notcurses}---that's
done with \texttt{ncplane}s---but there are a number of functions
available for these objects. Acquiring an \texttt{ncplane} for output can be
done by grabbing a reference to the standard plane, or creating a new plane.
New planes are always inserted into the top of the z-axis. All user-created
planes can be destroyed in one call with \texttt{notcurses\_drop\_planes()} (note
that it is not necessary to call this prior to \texttt{notcurses\_stop()}; the
latter cleans up all resources associated with the context).

\begin{listing}[!htb]
\begin{minted}{C}
// Get a reference to the standard plane (one matching our current idea of the
// terminal size) for this terminal. The standard plane always exists, and its
// origin is always at the uppermost, leftmost cell of the terminal.
struct ncplane* notcurses_stdplane(struct notcurses* nc);
const struct ncplane* notcurses_stdplane_const(const struct notcurses* nc);

// notcurses_stdplane(), plus free bonus dimensions written to non-NULL y/x!
static inline struct ncplane* notcurses_stddim_yx(struct notcurses* nc, int* restrict y, int* restrict x){
  struct ncplane* s = notcurses_stdplane(nc); // can't fail
  ncplane_dim_yx(s, y, x); // accepts NULL
  return s;
}

// Return our current idea of the terminal dimensions in rows and cols.
static inline void notcurses_term_dim_yx(struct notcurses* n, int* restrict rows, int* restrict cols){
  ncplane_dim_yx(notcurses_stdplane(n), rows, cols);
}

// Create a new ncplane at the specified offset (relative to the standard plane)
// and the specified size. The number of rows and columns must both be positive.
// This plane is initially at the top of the z-buffer, as if ncplane_move_top()
// had been called on it. The void* 'opaque' can be retrieved (and reset) later.
struct ncplane* ncplane_new(struct notcurses* nc, int rows, int cols, int yoff, int xoff, void* opaque);

// Return the topmost ncplane, of which there is always at least one.
struct ncplane* notcurses_top(struct notcurses* n);

// Destroy any ncplanes other than the stdplane.
void notcurses_drop_planes(struct notcurses* nc);

// Retrieve the contents of the specified cell as last rendered. The EGC is returned, or NULL on error.
// This EGC must be free()d by the caller.
char* notcurses_at_yx(struct notcurses* nc, int yoff, int xoff, uint32_t* attr, uint64_t* channels);
\end{minted}
\caption{Essential functions on \texttt{notcurses} objects.}
\end{listing}

Reading input is a per-context operation, performed with \texttt{notcurses}
objects. It is discussed in detail in Chapter~\ref{sec:input}. When reading
input, we might get the synthesized event \texttt{NCKEY\_RESIZE}\footnote{This
event is generated upon receipt of a \texttt{SIGWINCH} signal, SIGnifying WINdow
CHange.}. This indicates that the terminal has been resized, and we might want
to call \texttt{notcurses\_resize()} and get the new dimensions. As discussed
earlier, sometimes the display is externally corrupted. It's thus a good idea
to hook some UI event (usually Ctrl+L) to \texttt{notcurses\_refresh()}, which
redraws every cell on the display according to the internal Notcurses
framebuffer.

\begin{listing}[!htb]
\begin{minted}{C}
// Refresh our idea of the terminal's dimensions, reshaping the standard plane
// if necessary. References to ncplanes (and the egcpools underlying cells)
// remain valid following a resize, but the cursor might have changed position.
int notcurses_resize(struct notcurses* n, int* restrict y, int* restrict x);

// Refresh the physical screen to match what was last rendered (i.e., without
// reflecting any changes since the last call to notcurses_render()). This is
// primarily useful if the screen is externally corrupted.
int notcurses_refresh(struct notcurses* n);
\end{minted}
\caption{Dealing with external events.}
\end{listing}

Finally, \texttt{notcurses\_render()} synthesizes a terminal's worth of current
state out of all your virtual objects, schedules an optimized list of escape
sequences and encoded characters, and blits the result to the terminal. Only
through \texttt{notcurses\_render()} (and transitively through its callers) ought
your program write to the actual terminal, and only \texttt{notcurses\_render()}
has any bearing on what the user sees. Between calls, you are free to do whatever
you want in terms of moving, reordering, creating, writing upon, and destroying
planes. There will be no flicker or tearing; what you last rendered remains on
the screen. When you've got your stack how you want it, and only then, invoke
\texttt{notcurses\_render()}. It is an exclusive function---any concurrent use
of the same \texttt{struct notcurses} is undefined.

\begin{listing}[!htb]
\begin{minted}{C}
// Make the physical screen match the virtual screen. Changes made to the
// virtual screen (i.e. most other calls) will not be visible until after a
// successful call to notcurses_render().
int notcurses_render(struct notcurses* nc);
\end{minted}
\caption{Rendering syncs the physical display to our visual planes.}
\end{listing}

\subsection{Reading, rendering, rasterizing, and writing}
\label{sec:rendering}

Understanding how Notcurses translates its data structures into a terminal
display is critical for reasoning about your program in general, and particularly
relevant for maximizing performance.

During initialization of a terminal, unless \texttt{suppress\_banner} is supplied
in \texttt{notcurses\_options}, \texttt{notcurses\_init()} will print some
diagnostics to stdout, and flush the output buffer. Notcurses maintains an
internal virtual framebuffer, containing the state of the terminal as believed
to exist\footnote{Do not confuse this with the standard plane. This framebuffer
reflects rendering and rasterizing, not the output API.}. It is initialized in
\texttt{notcurses\_init()} to an empty matrix of cells, each cell having the
default foreground and background.

What happens next depends on whether the ``alternate screen'' (as described
earlier) is employed. If so, the terminal will be immediately cleared.
Otherwise, the terminal will not be altered until the first call to
\texttt{notcurses\_render()}. That first call, however, will write to every
cell of the terminal, effectively clearing any existing output. The upshot is
that it is not possible to integrate preexisting data into your TUI, regardless
of whether the alternate screen is used (aside from marginalia). This reflects
the impossibility of portably discovering the state of the terminal.

Subsequent to the first call, Notcurses---having written them---has a concept
of the display's contents. From that point on, screen updates will write only
to changed (``damaged'') cells. When only parts of the screen have changed,
this saves a tremendous amount of work. On an 80x45 terminal, if only a 10x10
region of cells have changed, we reduce our bandwidth by about
95\%\footnote{10x10 is only 2.7\% of 80x45, but there is overhead due to moving
the cursor to the region, and then positioning the cursor at the end of each
line of the region.}. These savings are multiplicative:

\begin{denseitemize}
\item{Notcurses doesn't have to \texttt{write()} the data (memory copy).}
\item{The terminal doesn't have to \texttt{read()} the data (memory copy).}
\item{The terminal doesn't need to process the data (assorted work).}
\item{The terminal doesn't need to write to the display (memory copy).}
\end{denseitemize}

Whether a cell has been updated is decided at rasterization time. Writing to
that cell between calls to \texttt{notcurses\_render()} does not necessarily
mean the cell will be considered damaged when it comes time to write. If the
cell has been damaged, it will be emitted, and the virtual framebuffer internal
to Notcurses will be updated.

Solving for the desired state of the screen is \textit{rendering}, and this is
the first step of \texttt{notcurses\_render()}. Solving for the screen means
solving for the current state of every cell, given our ordered set of
\texttt{ncplane}s. Solving for a cell means determining the extended grapheme
cluster to be rendered, determining the attributes to be applied to that EGC,
and determining the colors in which it ought be displayed. The higher a plane
is on the z-axis, the more it can impact these solutions:

\begin{denseitemize}
\item{The EGC and attribute are determined by the first plane intersecting with
      the cell having a non-null EGC at the intersecting coordinate. If there is
    no such intersecting EGC, the EGC is null, and the attribute is
    \texttt{NCSTYLE\_NORMAL}.} Null EGCs are rendered as spaces (i.\ e.\ entirely
    background color).
\item{The foreground color is determined by the first instance of a
    \texttt{CELL\_ALPHA\_OPAQUE} foreground color, or an instance of the
    default foreground color, or an instance of a palette-indexed foreground
    color, as well as any \texttt{CELL\_ALPHA\_HIGHCONTRAST} or \texttt{CELL\_ALPHA\_BLEND}
    foreground colors encountered along the way. If there is no such
    intersecting terminator, the foreground color is the color as calculated
    thus far. If \texttt{CELL\_ALPHA\_HIGHCONTRAST} is in play, the calculated
    color is then blended to stand out against the calculated background
    color.}
\item{The background color is determined independently, in the same way as the
    foreground color, except without the complicating possibility of
    \texttt{CELL\_ALPHA\_HIGHCONTRAST}.}
\end{denseitemize}

Once a cell is solved, Notcurses needn't continue inspecting lower planes at
that coordinate. Once all cells are solved, rendering is complete, and any
planes left over can be skipped entirely. Until then, Notcurses steps down from
one plane to the next, starting at the topmost plane, and updates its solution
for any intersecting unsolved cells. It is thus generally more performant to
``hide'' planes at the bottom of the stack, ideally behind a large opaque plane,
rather than moving them beyond the boundaries of the visible window. Likewise,
planes ought be no larger than necessary, so that they intersect with the
minimum number of cells. Note that there will always be at least one plane
interacting with each visible coordinate, due to the properties of the standard
plane.

Having rendered the scene, \textit{rasterization} serializes a buffer to write
to the terminal, minimizing the amount of data by moving the cursor over undamaged
regions. This is the second step of \texttt{notcurses\_render()}. Writing this
data to the terminal as it's generated is a bad idea for several reasons: it can
provoke unnecessary context switches, it results in partially-updated displays,
and it definitely involves more system calls. Notcurses instead collects it in
one or more large allocations.

Proceeding cell-by-cell from the upper left to the lower right, Notcurses
compares the rendering solution set to its internal framebuffer. If a given row
is entirely undamaged, it can be skipped. Upon discovering the leftmost damage
on a row, an absolute cursor update is performed to the damaged cell. At each
damaged cell, the EGC will be emitted, along with any necessary styling
information. It is only necessary to emit styling escapes when they change, i.\ e.\ we
can emit multiple EGCs having the same style after only issuing the appropriate
escapes once. An RGB change takes about 14 bytes, a palette index change
takes about 6, and reverting to the default 2. For single-byte simple (ASCII)
EGCs, an RGB foreground and background represent 2800\% overhead per cell!
Eliding styling escapes is thus an important secondary optimization (it's of
course most desirable to not update the cell at all). Using the ``default
color'' as only one of the foreground or background requires emitting the
\texttt{op} escape followed by the appropriate escape for changing the fore- or
background (since \texttt{op} changes both at once).

Certain EGCs are understood to be all-foreground or all-background.
\texttt{U+2588 FULL BLOCK} is all foreground. \texttt{U+0020 SPACE} is all
background. When such characters are used, notcurses will emit whichever
character requires the fewest total bytes, taking into account both the
UTF-8 encoding length and the current color state.

The upshot is that holding styling constant across a horizontal stretch is
very desirable if that range's content is going to be changing. The most
pathological input to Notcurses is text that changes its foreground and background
on a cell-to-cell basis, especially when specified as RGB, that change from
render to render. Certain terminal emulators in particular respond to the
resulting deluge of RGB escapes very poorly (see Appendix~\ref{sec:termshade}).
As examples, see the \texttt{highcontrast} and \texttt{grid} demos of
\texttt{notcurses-demo}---a large \texttt{xterm} can be brought to its knees
by these routines.

Each subsequent range of undamaged cells on a line can be skipped over with
cursor movements, but as the skip length approaches 1, it becomes less and
less advantageous to do so. Rendering performance can be very roughly
categorized as inversely proportional to the product of:

\begin{denseitemize}
\item{color changes across the rendered screen,}
\item{planar depth before an opaque glyph and background are locked in,}
\item{number of UTF-8 bytes comprising the rendered glyphs, and}
\item{screen geometry.}
\end{denseitemize}

With these buffers in hand, \texttt{notcurses\_render()} completes its task by
writing them to the terminal. This almost certainly means copying
them into a kernel buffer from which the terminal will then (following at
least one context switch and two system calls) read. Writing does not,
then, necessarily mean that the display has actually been updated, or even
that the terminal has read the data. If the terminal doesn't empty the buffer
quickly enough, however, you'll eventually run out of room and block. It is
thus critical to understand that \textbf{\texttt{notcurses\_render()} can block
for arbitrary amounts of time}\footnote{But see
\url{https://github.com/dankamongmen/notcurses/issues/214}.}. Furthermore,
if the terminal reads two renderings' worth of output at the same time, it is
likely to immediately enter the final state---you must not assume that a successful
\texttt{notcurses\_render()} is necessarily displayed within any arbitrary time,
or indeed that it corresponds with any displayed frame.

With those unhappy truths said, modern workstations ought have no problem pushing
notcurses onto commodity hardware at maximum framerates, with the terminal
faithfully reproducing each rendered scene. Even small microcontrollers ought
be able to render notcurses without user-perceptible latency. On a powerful
desktop with non-pathological output, it's easy to render in excess of
ten thousand frames per second, more than an order of magnitude beyond the
refresh capabilities of any existing or likely monitor\cite{charni}.

\subsection{Capabilities}
\label{sec:capabilities}
Different terminals expose different capabilities, and different means of
engaging them. These differences are encoded in the terminfo database\cite{terminfo}.
Notcurses hides the differences where it can, and is built around those
capabilities which are most widely supported. Some applications, however, will
want to know details of the underlying implementation. For this purpose, the
Capabilities API is provided (Listing~\ref{list:capabilities}).
\begin{listing}[!htb]
\begin{minted}{C}
// Returns a 16-bit bitmask of supported curses-style attributes (NCSTYLE_UNDERLINE, NCSTYLE_BOLD,
// etc.). The attribute is only indicated as supported if the terminal can support it together
// with color. For more information, see the "ncv" capability in terminfo(5).
unsigned notcurses_supported_styles(const struct notcurses* nc);

// Returns the number of simultaneous colors claimed to be supported, or 1 if there is no color support.
// Note that several terminal emulators advertise more colors than they actually support, downsampling internally.
int notcurses_palette_size(const struct notcurses* nc);

// Can we fade? Fading requires either the "rgb" or "ccc" terminfo capability.
bool notcurses_canfade(const struct notcurses* nc);

// Can we set the "hardware" palette? Requires the "ccc" terminfo capability.
bool notcurses_canchangecolor(const struct notcurses* nc);

// Can we load images/videos? This requires being built against FFmpeg.
bool notcurses_canopen(const struct notcurses* nc);

// Get a human-readable string describing the running notcurses version.
const char* notcurses_version(void);
\end{minted}
\caption{The capabilities API.}
\label{list:capabilities}
\end{listing}

\subsection{Statistics}
Notcurses tracks statistics across its operation, and a snapshot can be
acquired using the \texttt{notcurses\_stats()} function (Listing~\ref{list:stats}). This function cannot
fail. Most of the stats can be reset with \texttt{notcurses\_reset\_stats()}.
This function resets all cumulative stats, but not those which describe the
current state. Timings for renderings are across the breadth of
\texttt{notcurses\_render()}: they include all per-render preprocessing, output
generation, and dumping of the output (including any sleeping while blocked on
output to the terminal).

Statistics available include:
\begin{denseitemize}
\item{\texttt{renders}, \texttt{failed\_renders}: The number of successful and unsuccessful
    invocations of \texttt{notcurses\_render()}. Calls to \texttt{notcurses\_refresh()} do
    not show up in either of these stats. A render call can fail due to
    memory pressure, invalid EGCs, or a failure to successfully write to the
    output terminal.}
\item{\texttt{render\_bytes}: The number of bytes written in successful renders.
    Unsuccessful renders do not count towards the total. Dividing \texttt{renders}
  by \texttt{render\_bytes} yields the average bytes per (successful) render.}
\item{\texttt{render\_max\_bytes}, \texttt{render\_min\_bytes}: The maximum and
  minimum number of bytes emitted during a successful render.}
\item{\texttt{render\_ns}, \texttt{render\_min\_ns}, \texttt{render\_max\_ns}:
  The total, minimum, and maximum number of nanoseconds spent in \texttt{notcurses\_render()},
  whether the calls were successful or not. These timings are acquired using
  POSIX timers\cite{clockgettime} with the
  \texttt{CLOCK\_MONOTONIC}\footnote{Wouldn't \texttt{CLOCK\_MONOTONIC\_RAW} be
  superior? It would, where it's available, which isn't everywhere. It's also
  substantially more expensive than \texttt{CLOCK\_MONOTONIC} on Linux. Be
  aware, then, that NTP adjustments and time suspended \textit{do} show up in
  timings.} implementation.}
\item{\texttt{cellemissions}, \texttt{cellelisions}: The total number of EGCs
  written to output, and the number that did not need to be written due to
  being undamaged.}
\item{\texttt{fgemissions}, \texttt{fgelisions}: Foreground RGB values written to output, and the number elided.}
\item{\texttt{bgemissions}, \texttt{bgelisions}: Background RGB values written to output, and the number elided.}
\item{\texttt{defaultemissions}, \texttt{defaultelisions}: \texttt{op} escapes issued to set default colors, and the number elided.}
\item{\texttt{fbbytes}: The number of bytes devoted to framebuffers.}
\item{\texttt{planes}: The current number of planes. Will never drop below 1.}
\end{denseitemize}

\begin{listing}[!htb]
\begin{minted}{C}
typedef struct ncstats {
  // purely increasing (cumulative) stats
  uint64_t renders;          // number of successful notcurses_render() runs
  uint64_t failed_renders;   // number of aborted renders, should be 0
  uint64_t render_bytes;     // bytes emitted to ttyfp
  int64_t render_max_bytes;  // max bytes emitted for a frame
  int64_t render_min_bytes;  // min bytes emitted for a frame
  uint64_t render_ns;        // nanoseconds spent in notcurses_render()
  int64_t render_max_ns;     // max ns spent in notcurses_render()
  int64_t render_min_ns;     // min ns spent in successful notcurses_render()
  uint64_t cellelisions;     // cells we elided entirely thanks to damage maps
  uint64_t cellemissions;    // cells we emitted due to inferred damage
  uint64_t fgelisions;       // RGB fg elision count
  uint64_t fgemissions;      // RGB fg emissions
  uint64_t bgelisions;       // RGB bg elision count
  uint64_t bgemissions;      // RGB bg emissions
  uint64_t defaultelisions;  // default color was emitted
  uint64_t defaultemissions; // default color was elided
  // current state -- these can decrease
  uint64_t fbbytes;          // total bytes devoted to all active framebuffers
  unsigned planes;           // number of planes currently in existence
} ncstats;

// Acquire an atomic snapshot of the notcurses object's stats.
void notcurses_stats(struct notcurses* nc, ncstats* stats);

// Reset all cumulative stats (immediate ones, such as fbbytes, are not reset).
void notcurses_reset_stats(struct notcurses* nc, ncstats* stats);
\end{minted}
\caption{The statistics API.}
\label{list:stats}
\end{listing}

\subsection{Use in multithreaded environments}
\label{sec:fullthreads}
To facilitate maximum performance, Notcurses does not perform any locking of
its own\footnote{This might change if Notcurses begins to actively acquire
input itself, necessary for certain desirable features. In that case, the
locking would only be present in the input layer.}. All functions are safe
for multiple threads to call with regards to system and standard library
resources. Things become more complex when multiple threads wish to ``write''
to Notcurses.

As mentioned above, it is necessary that \texttt{notcurses\_render()} calls
mutually exclude themselves, and also all other functions which mutate the
context. This includes all functions which write to a plane, functions which
change the ordering of planes (including deletion of a plane), and even
statistics functions. With very few exceptions, calling any Notcurses function
concurrently with \texttt{notcurses\_render()} is an error.

Beyond this ``big rendering lock'', functions ought not generally be called
concurrently on the same \texttt{ncplane}, unless all are reading. It is
unsafe, for instance, to call \texttt{ncplane\_putegc()} (a writing function)
concurrently with \texttt{ncplane\_cursor\_yx()} (a reading function). It
might not be obvious that functions such as \texttt{ncplane\_at\_yx()} which
write to a \texttt{cell} bound to the plane are writers, but supplying the
\texttt{cell} might require writing to the plane's egcpool, and thus they are
writers. In general, a function is only a reader if its \texttt{ncplane} argument
is \texttt{const}.

Concurrent operations on different planes are safe, unless they are changing
the ordering along the z-axis. Note that \texttt{ncplane\_destroy()} updates
the z-ordering by virtue of removing an element, and thus must not be called
along with e.g. \texttt{ncplane\_move\_top()}.

\cleardoublepage

%%%%%%%%%%%%%%%%%%%%%%%%%%%%%%%%%%%%%%%%%%%%%%%%%%%%%%%%%%%%%%%%%%%%%%%%
\section{A simple notcurses render\slash event loop}
I'm not typically a fan of example-based instruction, preferring to build things
up from formal axioms. Following chapters will effect such an approach. First,
however, let's work a semi-substantial example covering a varied set of
Notcurses routines. We're going to create seven planes, one for each kind
of tetrimino, and map an image file to the background. We'll add support for
switching between the pieces, rotating them, and sliding them around the
screen. Finally, we'll deal with collisions, and fluid handling of
screen resizings. In the course of doing so, you'll learn several important
Notcurses techniques:

\begin{denseitemize}
\item{Drawing and rotating these tetriminos will involve colors, gradients, and
      transparencies. The former are fundamental drawing tools. The latter is
      all one needs for sprites.}
\item{Mapping the background will involve image decoding, scaling, and blitting.}
\item{We'll cover most of what's worth knowing regarding input.}
\end{denseitemize}

This example will be picked up and further developed in the Tetris case study
of §~\ref{section:casestudy}\footnote{Were you aware that there is a standard
for Tetris clones? There is indeed\cite{tetris}.}.

\begin{figure}[!htbp]
\centering \includegraphics[width=.5\linewidth]{media/tetriminos.png}
\caption{Piping hot tetriminos, fresh from~\textit{Spiritus Mundi}.}
\label{fig:tetriminos}
\end{figure}

Almost every Notcurses program will take the same general form, an \textit{event+render loop}:

\begin{denseitemize}
\item{Essential screen elements are laid down.}
\item{Initial state is discovered and added to the display.}
\textbf{begin loop} 
\item{Some thread, perhaps the only thread in the process, watches
    for user input. Other threads might be collecting system events that will
    change the state. Either way, an event occurs.}
\item{Thread(s) receiving events perform any necessary mutual exclusion.}
\item{Thread(s) manipulate the Notcurses virtual state via \texttt{ncplane} manipulations.}
\item{\texttt{notcurses\_render()} is called by a single thread.}
\textbf{end loop}
\item{The process terminates.}
\end{denseitemize}

\subsection{Example: moving tetriminos with a keyboard}

We could load the pieces as images from files, but given the simplicity of the
sprites, it feels simpler to just hardcode them in our source. I don't want to
leave my text editor\footnote{Vim. See Figure~\ref{fig:xeroxemacs}.} to muck
with images when we can do everything in fewer than ten lines, even in a naïve
and wasteful encoding (Listing~\ref{list:tetrimino-data}).

\begin{listing}[!htbp]
\inputminted[]{C}{code/tetrimino-data.h}
\caption{The seven canonical tetriminos (from~\texttt{tetrimino.c}).}
\label{list:tetrimino-data}
\end{listing}

To warm up and get limber, let's create seven \texttt{ncplane}s, one for each type
of tetrimino. We'll lay them out in a 2-3-2 formation. This layout will be
sloppy, for now: our function
\texttt{tetrimio\_plane()} accepts a piece ID and a coordinate.
The plane it creates has its origin at this coordinate.
We're not yet adjusting for the size of the planes themselves when coming
up with these coordinates---this shifts everything down and to the right from
symmetric divisions of the screen, as you'll see.

\begin{figure}[!htbp]
\centering \includegraphics[width=.5\linewidth]{media/screenruler.png}
\caption{Font aspect ratios center around 0.5.}
\label{fig:aspectratio}
\end{figure}

We use two display rows per game row but four display columns per game column.
This reflects what is, at least on my display and font, pretty much a 0.5
aspect ratio (Figure~\ref{fig:aspectratio}).

\begin{listing}[!htbp]
\inputminted[]{C}{code/tetrimino-display.h}
\caption{Creating a single tetrimino (from~\texttt{tetrimino.c}).}
\label{list:tetrimino-display}
\end{listing}

There are a few essential things to
take away from Listing~\ref{list:tetrimino-display}:

\begin{denseitemize}
\item{Each time we call \texttt{ncplane\_putsimple()}, the cursor is advanced
      the expected amount. If we were calling \texttt{ncplane\_putegc()} with
      a multicolumn grapheme cluster, the cursor would be advanced multiple
      columns.}
\item{Between rows, we need move the cursor ourselves, since we're skipping over
      part of the plane (this isn't true for several of the pieces, but the
      cursor update is a trivial operation, not worth avoiding).}
\item{The cursor is always initialized to the origin of a new plane, and coordinates
      supplied to \texttt{ncplane\_} functions are relative to the plane,
      \textit{not} the terminal. Coordinates can be translated among planes
      using \texttt{ncplane\_translate()}.}
\item{We only have ASCII characters in the textures right now. Were we to
      introduce any multibyte UTF-8---and we may well do so, for e.\ g.\ the
      Box Drawing Characters---the \texttt{strlen()} we size our rows by here
      would no longer fly. We'd need use \texttt{mbstowcs()}, or redefine the
      textures as \texttt{wchar\_t} arrays and use \texttt{wcslen()},
      or change up our encoding\footnote{We go with the third option, as you'll see.}. \textfrench{\textit{Rien n'est simple, mais tout est facile\ldots}}}
\end{denseitemize}

\begin{listing}[!htbp]
\inputminted[]{C}{code/tetrimino-draw.h}
\caption{Distributing the tetriminos with ``flying-v'' technique (from~\texttt{tetrimino.c}).}
\label{list:tetrimino-draw}
\end{listing}

\begin{listing}[!htbp]
\inputminted[]{C}{code/tetrimino-main.h}
\caption{A one-shot, display-only \texttt{main()} (from~\texttt{tetrimino.c}).}
\label{list:tetrimino-main}
\end{listing}

Our \texttt{main()} sets the locale, initializes the terminal, and draws the
seven pieces (Listings~\ref{list:tetrimino-draw} and~\ref{list:tetrimino-main}).
Each piece has a one-letter name; for now, we draw with those glyphs. The
pieces are monochromatic. This renders familiar shapes in the correct colors
(Figure~\ref{fig:tetriminos-1}), but other than that, it's (like most classic
Curses programs) kinda fugly. I wouldn't want to play with these pieces.

\begin{figure}[!htbp]
  \centering
  \begin{minipage}{0.30\textwidth}
    \includegraphics[width=1\linewidth]{media/tetriminos-1.png}
    \caption{Unspeakably foul.}
    \label{fig:tetriminos-1}
  \end{minipage}\hfill
  \begin{minipage}{0.30\textwidth}
    \includegraphics[width=1\linewidth]{media/tetrimino-gradient1.png}
    \caption{Adding a gradient.}
  \end{minipage}\hfill
  \begin{minipage}{0.30\textwidth}
    \includegraphics[width=1\linewidth]{media/tetrimino-gradient2.png}
    \caption{Linear expansion.}
  \end{minipage}\hfill
\end{figure}

This is hardly a render/event loop; in fact it's not a loop at all. If we were
using the alternate screen, we wouldn't see our output before flashing back to
the normal screen. Let's go ahead and hook up input. We'll be controlling one
of the tetriminos at a time---the selected piece will use boldface, and have its
color brightened\footnote{Note how we use two distinct indicators. On a monochromatic
display, we need the bold, and on a display which can't do bold, maybe we get a
color difference. Also, we want bold even if we have color, because until we
see the base color, there's no reason to think of the initial selection as
``bright''.}. We do so (Listing~\ref{list:tetrimino-switch}) with a pair of
helpers---\texttt{reduce()} and \texttt{highlight()}--- which drive the common
\texttt{blast()}. \texttt{blast()} makes use of two new functions,
\texttt{ncplane\_format()} and \texttt{ncplane\_stain()}. They allow the
attributes and channels, respectively, of a rectangular region to be changed
without altering other components. The corresponding glyph-only output routines
are likewise described in Listing~\ref{list:stain}
(Chapter~\ref{sec:staining}).

\begin{listing}[!htbp]
\inputminted[]{C}{code/tetrimino-switch.h}
\caption{Switching between pieces (from~\texttt{tetrimino-input.c}).}
\label{list:tetrimino-switch}
\end{listing}

While we're making things prettier, let's replace
those letters with some classy box-drawing characters, and improve on this
layout. We can do some simple algebraic extensions and get some linear spacing,
but it's just as easy to use trigonometric functions and get an approximation
to a circle. This is especially value as the terminal geometry changes; our
fixed linear scalings would break down as the display aspect changes, but a
good ol' circle will work on any sufficient radius
(Listing~\ref{list:tetrimino-drawcircle}).

\begin{listing}[!htbp]
\inputminted[]{C}{code/tetrimino-drawcircle.h}
\caption{Trigonometric layout: simpler, yet more accurate (from~\texttt{tetrimino-input.c}).}
\label{list:tetrimino-drawcircle}
\end{listing}

Since there's nothing else going on, it's trivial to process \texttt{stdin} in
a blocking fashion from our main thread. Let's add the code to track a selected
piece, visually highlight it, and process input (Listing~\ref{list:tetrimino-inputcore}).
The space bar will advance among the pieces in one direction,
and Tab in the other. The arrow keys and vi keys will translate the selected
piece in four directions. The parentheses will rotate the piece $\frac{\pi}{2}$ radians
clockwise and counterclockwise.

\begin{listing}[!htbp]
\inputminted[]{C}{code/tetrimino-inputcore.h}
\caption{Core input dispatch (from~\texttt{tetrimino-input.c}).}
\label{list:tetrimino-inputcore}
\end{listing}

Of course, when there's no other activity, things are easy. Quite often, we'll
have some periodic concurrent activity. Let's say we wanted to make the non-selected
tetriminos slowly rotate. The rotation isn't difficult; Notcurses provides
that functionality for you. But you don't want the rotation synced to input
activity, and you're hardly the kind of know-nothing what-not that would
busy loop on a nonblocking input interface, right? I hope you don't consider
yourself ``green'' if that's the case, because you're burning dinosaurs out
of sheer laziness. Let's do it right. There are of course four canonical
solutions to the problem of interleaving a set of asynchronous file descriptor-based
inputs against a set of periodic requirements; let's refresh ourselves:

\begin{denseitemize}
\item{No holds barred, take no prisoners state machine. No quarter asked and
    none given. Certainly the most \textit{enjoyable} choice of the four, the
    one closest to the nature of the machine, and the option with the most
    built-in job security. Casting off the crutch of the process scheduler, we
    execute in a single thread, carefully programming our
    \texttt{epoll()}\cite{epoll7} or \texttt{kqueue()}\cite{kqueue2} timeouts
    against a hierarchal, hashed timer wheel\cite{timerwheels}. As Alan Cox
    said, ``Computers are state machines. Threads are for people who don't
    understand state machines.''\footnote{Or did he? I can't find the original
    to cite. I wonder what the hell happened to Alan Cox.} Limited to a single
  CPU\ldots\ which probably isn't a problem here, but isn't exactly a great foundation
  on which to build our cathedral, \textfrench{\textit{n'est-ce pas?}}.}
\item{Two threads enter, one thread \texttt{exit()}s (hehehehe). We block on
    async I/O in our main thread. Another thread runs a timer wheel---for this,
    a loop around a \texttt{clock\_nanosleep()} will suffice. They
    lock against one another to avoid corrupting the screen or otherwise
    embarassing ourselves. The solution reeks of Java, but will work well
    enough, and any programmer---maybe even a Java programmer---ought be able
    to walk in and pick it up.}
\item{POSIX signaled timers. Noooooooooooooooooooooooooooooooooooooo.}
\item{Timer I/O multiplexing. Probably the best solution for a program of true
    scope, and in no way unfit for the task at hand. On Linux, you'll want
    \texttt{timerfd\_create()} and friends. On FreeBSD, look for \texttt{EVFILT\_TIMER}.
    Throw that into the appopriate I/O multiplexor call, spin it out among a
    few threads if you feel so inclined\cite{libtorque}, and call it a day.
  The only real disadvantage here is a lack of portability.}
\end{denseitemize}

Some will ask, ``Sounds dank, but what about on this shiny operating system I
purchased from Applooglesoft? Look, I can speak to it! Sirana, diminish my
freedoms and spy on me!'' I could give two shits about your closed-source
operating system. I presume you can purchase cloud-based Timers as a Service
(TaaS) or something.

We can now cycle through the pieces, and move the piece we've selected around
on the screen. Grab one and move it towards another piece. Experimentation will
reveal that each piece has a~\gls{boundingbox}, that there is a total ordering
among the seven pieces, and that a piece below another piece's bounding box is
robbed of its color (Figure~\ref{fig:tetrimino-badplane}). Recall from
Chapter~\ref{sec:rendering} how we solve for each coordinate of the display
grid: the EGC is a dimension distinct from the coloring channels. Each of the
planes we create is rectilinear, with the piece drawn using Unicode
\texttt{U+2588 FULL BLOCK}, and the other cells left unwritten. The foreground
for our pieces is an RGB color, and for the other cells is the default terminal
color. Upon intersecting with a lower plane, the lower plane's EGC is chosen for
rendering, but the foreground color is default terminal color (white in my
example). So the piece underneath flipping to white while another piece is
nearby makes perfect sense. The solution is simple: we set the foreground
transparent for the base character of each plane. Foreground calculation will
now bypass the plane where we haven't drawn, and the distorted plane regains
its expected colors (Figure~\ref{fig:tetrimino-trans}).

\begin{figure}[!htbp]
  \centering
  \begin{minipage}{0.30\textwidth}
    \includegraphics[width=1\linewidth]{media/tetrimino-gradient3.png}
    \caption{Trigonometry!}
  \end{minipage}\hfill
  \begin{minipage}{0.30\textwidth}
    \includegraphics[width=1\linewidth]{media/tetrimino-gradient4.png}
    \caption{Unicode Blocks.}
  \end{minipage}\hfill
  \begin{minipage}{0.30\textwidth}
    \includegraphics[width=1\linewidth]{media/tetrimino-gradient5.png}
    \caption{Better blocks.}
  \end{minipage}\hfill
\end{figure}

\begin{figure}[!htbp]
  \begin{minipage}{0.30\textwidth}
    \includegraphics[width=1\linewidth]{media/tetrimino-gradient6.png}
    \caption{Adjusting for cell aspect ratio.}
  \end{minipage}\hfill
  \begin{minipage}{0.30\textwidth}
    \includegraphics[width=1\linewidth]{media/tetrimino-gradient7.png}
    \caption{Undesirable plane interaction.}
    \label{fig:tetrimino-badplane}
  \end{minipage}\hfill
  \begin{minipage}{0.30\textwidth}
    \includegraphics[width=1\linewidth]{media/tetrimino-gradient8.png}
    \caption{Resolve it with transparent planes.}
    \label{fig:tetrimino-trans}
  \end{minipage}\hfill
\end{figure}

\begin{listing}[!htbp]
\inputminted[]{C}{code/tetrimino-databox.h}
\begin{minted}{C}

\end{minted}
\inputminted[]{C}{code/tetrimino-displayutf8.h}
\caption{Improving appearance with Unicode Block Elements (from~\texttt{tetrimino-input.c}).}
\label{list:tetrimino-displayutf8}
\end{listing}

Adding a background image is simple. We'll render it to the standard plane, our
``background'' for now (remember, newly created planes are placed at the top of
the z axis). Despite decoding the image \textit{after} creation of the pieces,
the pieces are thus not hidden. There is no need to inform Notcurses of the
image's format, or parameters thereof; just provide the file name and target
rendering plane, and we're off. See Chapter~\ref{sec:libav} for a full treatment
of multimedia in Notcurses.

\begin{listing}[!htbp]
\inputminted[]{C}{code/tetrimino-background.h}
\caption{Throwing in a background (from~\texttt{tetrimino-input.c}).}
\label{list:tetrimino-background}
\end{listing}

To rotate the unselected pieces, we'll go ahead and spawn a POSIX thread. We'll
thus need lock against piece selection. We've been leaving all the end-of-process
cleanup in the capable hands of~\texttt{notcurses\_stop()}, but it can't go
terminating threads for us, and it in any case wouldn't be safe to do so while
said thread was calling into Notcurses! We'll use POSIX
cancellation\footnote{Which isn't anywhere near as bad as it's sometimes made out to be.
Find the places where you don't want to be cancelled. Each such place needs a
cleanup handler pushed/popped around it, or cancellation disabled for its
breadth. Things get a little counterintuitive with
cancellable-but-uninterruptible system calls, but that's nothing if you came up
on BSD signals. It's hardly the most offensive wart on the great hairy ass
that is ANSI/ISO C, and---heresy, I know---I honestly prefer (most of) the
POSIX model to (most of)
the threading introduced in \CC11.} to blast the rotator thread, and~\texttt{pthread\_join()}
it to ensure safe passage through shutdown.

\begin{listing}[!htbp]
\inputminted[]{C}{code/tetrimino-thread.h}
\caption{Spin them doggies (from~\texttt{tetrimino-input.c}).}
\label{list:tetrimino-thread}
\end{listing}

As a final touch, let's slide a dark, translucent plane underneath the
selected piece, making it more visible. We create this plane prior to the
pieces, ensuring it's below all of them (but above the background). We set it
black, and its alpha to \texttt{CELL\_ALPHA\_BLEND}. This way it won't entirely
block out the background underneath the selected piece, but it will dim it,
its black blending into the grey tones underneath. The piece is unaffected.

\begin{listing}[!htbp]
\inputminted[]{C}{code/tetrimino-box.h}
\caption{Set the selection off with a coaster (from~\texttt{tetrimino-input.c}).}
\label{list:tetrimino-box}
\end{listing}

\begin{figure}[!htbp]
  \centering
  \begin{minipage}{0.30\textwidth}
    \includegraphics[width=1\linewidth]{media/tetrimino-bg.png}
    \caption{Background image, greyscaled.}
    \label{fig:tetrimino-bg}
  \end{minipage}\hfill
  \begin{minipage}{0.30\textwidth}
    \includegraphics[width=1\linewidth]{media/tetrimino-bg.png}
    \caption{Opaque highlight box.}
  \end{minipage}\hfill
  \begin{minipage}{0.30\textwidth}
    \includegraphics[width=1\linewidth]{media/tetrimino-bg.png}
    \caption{Rotation thread enabled.}
  \end{minipage}\hfill
\end{figure}

\begin{listing}[!htbp]
\inputminted[]{C}{code/tetrimino-inputmain.h}
\caption{Putting it all together (from~\texttt{tetrimino-input.c}).}
\label{list:tetrimino-inputmain}
\end{listing}

\pagebreak
\cleardoublepage
%%%%%%%%%%%%%%%%%%%%%%%%%%%%%%%%%%%%%%%%%%%%%%%%%%%%%%%%%%%%%%%%%%%%%%%%
\section{Terminal mechanics}
\label{section:tty}
\epigraph{The tty layer is one of the very few pieces of kernel code that scares the hell out of me.}{Ingo Molnar\cite{molnarhell}}
You won't often need to deal with the gritty details of terminal access and
manipulation. It's still important to understand what's going on behind the
abstraction, especially for when things go wrong. As was made clear in
Chapters~\ref{sec:direct} and~\ref{sec:fullscreen}, Notcurses requires a
proper terminal definition and a handle to a terminal device, or initialization
will fail (NCURSES requires the same). With that said, the UNIX terminal layers
have never been, and are not now for the faint of heart\footnote{When Ingo
Molnar is scared, we all ought be scared.}. Extending back to the AT\&T dark
ages, they are configured primarily through messy \texttt{ioctl}s and the
slightly-less-messy \texttt{termios} API.

More details than you probably want are available from Chapters 18 and 19 of \cite{apiue}
(general UNIX), Chapters 62 and 64 of \cite{linuxprogramming} (Linux), and Chapter
10 of \cite{freebsddesign} (FreeBSD).

Modern workstations support a variety of physical and virtual terminal devices:
\begin{denseitemize}
\item{Honest-to-Bog serial terminals, probably using the RS-232\cite{rs232}
      protocol over a D-subminiature 25-pin (DB-25M) or 9-pin (DE-9M)
      connector (see Table~\ref{table:serial}).}
\item{Virtual consoles on text-based video, plus a keyboard.}
\item{Virtual framebuffer consoles on graphics-based video, plus a keyboard.} 
\item{Terminal emulators in a graphical environment, plus brokered input devices.}
\item{Pseudoterminals hooked up to network connections.}
\end{denseitemize}

\begin{table}[!htbp]
  \centering
  \begin{tabular}{ |c|c|c|c|c| }
    \hline
    Signal & DB-25M & TIA-574 DE-9M & Yost 8P8C\cite{yost} & Originator \\
    \hline
    \hline
    Protective ground & 1 & x & x & x \\
    \hline
    Transmitted data & 2 & 3 & 3 & DTE \\
    \hline
    Received data & 3 & 2 & 6 & DCE \\
    \hline
    Request to send & 4 & 7 & 1 & DTE \\
    \hline
    Clear to send & 5 & 8 & 8 & DCE \\
    \hline
    Data set ready & 6 & 6 & 2 & DCE \\
    \hline
    Signal ground & 7 & 5 & 4, 5 & x \\
    \hline
    Carrier detect & 8 & 1 & 2 & DCE \\
    \hline
    Data terminal ready & 20 & 4 & 7 & DTE \\
    \hline
    Ring indicator & 22 & 9 & x & DCE \\
    \hline
  \end{tabular}
  \caption[RS-232/EIA-232 pin mappings]{RS-232/EIA-232 D-subminiature pins\\
    (DCE=Data circuit equipment, DTE=Data terminal equipment)}
  \label{table:serial}
\end{table}

On Linux, three kernel definitions form the core of the terminal abstraction
(the ``tty layer''). A \texttt{tty\_driver} function interface exists for each
tty implementation, and each has a different corresponding
major+minor device number pair. These implementations are enumerated in
\texttt{/proc/tty/drivers} (sample contents are listed in Table~\ref{table:procttydrivers}).
Line disciplines can be found in \texttt{/proc/tty/ldiscs}. The only line discipline
you're likely to encounter in a terminal context is \texttt{n\_tty}, the ``new tty''
discipline responsible for implementing ``cooked mode''\cite{essentialdrivers}.

POSIX.1-2017\cite{posix2017} §10.1 ``Directory Structure and Devices''
specifies three special files in the \texttt{/dev} directory, of which two are
related to the terminal/console system: \texttt{/dev/tty} is, within a process
context, a synonym for the controlling terminal associated with that process's
group (this can also be acquired with \texttt{tty(1)} or the POSIX C function
\texttt{ctermid(3)}\footnote{\texttt{/dev/tty} also uniquely supports the
\texttt{TIOCNOTTY} \texttt{ioctl(2)} for detaching a process from its
controlling terminal.}. The tty associated with an arbitrary file descriptor can
be retrieved with \texttt{ttyname\_r(3)}). \texttt{/dev/console} is a generic name for the current system console\footnote{Defined in \textit{ibid.} §3.392
``System Console'' as the device receiving messages sent by the \texttt{syslog()}
function. On Linux, it will also reproduce messages written to \texttt{/dev/kmsg}\cite{dmesg}.};
POSIX requires that the system console implement its ``General Terminal Interface'' (see
\textit{ibid.} §11). \texttt{/dev/tty0} is a special device on Linux linked to
the current virtual console (which might not be the system console). The
available system consoles can be found in \texttt{/proc/consoles}.

Initial \texttt{/dev/ttyN} devices are created internally by the kernel
(according to the value of \texttt{MAX\_NR\_CONSOLES}, by default 64) and set
up by \texttt{udev}. These are distinct devices, each usable by one virtual
console. Serial consoles show up as \texttt{/dev/ttySN}\cite{ttys4}. Each
virtual console gets a device at \texttt{/dev/vcsN} allowing access to the
glyph values of the console, a device at \texttt{/dev/vcsaN} allowing access to
the attributes at each cell, plus screen geometry, and a device at \texttt{/dev/vcsuN}
providing access to the Unicode values of each cell. If a framebuffer console
is being used, these devices are prepared atop some framebuffer device \texttt{/dev/fbN} (see
Figure~\ref{fig:framebuffers}); these mappings can be managed with \texttt{con2fbmap(1)}.

\begin{figure}[!htbp]
  \centering
  \includegraphics[width=.75\linewidth]{media/framebuffers.png}
  \caption{Three different Linux framebuffer implementations.}
  \label{fig:framebuffers}
\end{figure}

\begin{table}[!htbp]
  \centering
  \begin{tabular}{ |c|c|c|c|c|c| }
    \hline
    Name & Default node & Major & Minor & Type & sysfs \\
    \hline
    \hline
    /dev/tty & /dev/tty & 5 & 0 & system:/dev/tty & devices/virtual/tty/tty* \\
    \hline
    /dev/console & /dev/console & 5 & 1 & system:console & devices/virtual/tty/console \\
    \hline
    /dev/ptmx & /dev/ptmx & 5 & 2 & system & devices/virtual/tty/ptmx \\
    \hline
    /dev/vc/0 & /dev/vc/0 & 4 & 0 & system:vtmaster & x \\
    \hline
    usbserial & /dev/ttyUSB & 188 & 0--511 & serial & x \\
    \hline
    serial & /dev/ttyS & 4 & 64--95 & serial & x \\
    \hline
    rfcomm & /dev/rfcomm & 216 & 0--255 & serial & x \\
    \hline
    pty\_slave & /dev/pts & 136 & 0--1048575 & pty:slave & x \\
    \hline
    pty\_master & /dev/ptm & 128 & 0--1048575 & pty:master & x \\
    \hline
    unknown & /dev/tty & 4 & 1--63 & console & devices/virtual/tty/console \\
    \hline
  \end{tabular}
  \caption[Expanded contents of \texttt{/proc/tty/drivers}]{Extended content of a sample \texttt{/proc/tty/drivers}\\
    (5.5.6 kernel. \texttt{sysfs} information has been added)}
  \label{table:procttydrivers}
\end{table}

The various termios flags allow very fine control of the kernel state associated
with a given terminal. It is possible to mix flag settings arbitrarily, but three
modes are common, and have their own nomenclature:
\begin{denseitemize}
\item{\textbf{Canonical mode, aka cooked mode:}   The terminal driver buffers input until a newline is entered, while echoing
    it to the screen. Ctrl+C is mapped to \texttt{SIGINT}, Ctrl+\textbackslash\ is
    mapped to \texttt{SIGQUIT}, and Ctrl+Z is mapped to \texttt{SIGTSTP}.
    Buffered input is flushed when these signals are sent. The default mode
  under SUS4.}
\item{\textbf{Cbreak mode:} Line buffering is disabled, as is the processing
  of erase/kill characters. Interrupt and flow control translation is unaffected.
Sometimes referred to as \textit{rare mode}. This mode allows processing of input
without waiting for a newline character, while retaining e.g. Ctrl+C.}
\item{\textbf{Raw mode:} Interrupt and flow control translation is also disabled.
  This allows all keyboard input to be processed without preprocessing by the
  line discipline, but e.g. Ctrl+C behavior is lost, and (if desired) must be
  emulated in user space.}
\end{denseitemize}

\subsection{Terminals and the UNIX process model}
\label{sec:unixprocs}

You're hopefully aware that UNIX programs traditionally start with at least
three file descriptors open: 0 for \texttt{stdin}, 1 for \texttt{stdout}, and 2
for \texttt{stderr}. 
\textbf{FIXME FIXME acquisition of a tty (getty->login, ssh->pty),
  internals of kernel tty/pty devices, session groups, signal distribution,
  \texttt{/dev/tty} and \texttt{/dev/ttyXX}s}
My diagrams are adapted in part from those of Linus Akesson\cite{ttydemystified}.
\textbf{FIXME diagrams!}

\textbf{cover systemd-logind, getty, logind.conf, sd-login}
\subsection{Escape codes ANSI and otherwise}
\label{sec:escapes}

In a relic from teletypes, ISO 646, ISO 2022, and ECMA-35 described use of
non-destructive backspace to produce composed characters from spacing ones.
This had been eliminated by ISO 4873, ECMA-43, and ISO 8859.

\textbf{FIXME FIXME other stuff}

\pagebreak
\cleardoublepage
%%%%%%%%%%%%%%%%%%%%%%%%%%%%%%%%%%%%%%%%%%%%%%%%%%%%%%%%%%%%%%%%%%%%%%%%
\section{Character encodings and glyphs}\label{section:unicode}
\label{sec:charsets}
\epigraph{Thou whorseon zed, thou unnecessary letter!}{Wiliam Shakespeare, \textit{King Lear}}

We'll be well-served by becoming familiar with our core building
blocks---character sets and their rasterized forms. Even if we were to shrink
the character cell to a single pixel, we'd still need write one or another
character into it, and emit a control code to stylize it\footnote{I've seen
this idea bandied about as if it's a serious suggestion. Besides the fact
that it won't work in a console, escapes are \textit{far} less efficient than
canvases or OpenGL display lists; if your plan involves using ANSI escape
sequences to drive a 1920x1080 display via roughly two million 1x1 character
cells, you're not going to space today, or maybe ever\cite{upgoerfive}.}.

In the beginning, there are somewhat Platonic \textit{characters}, one of the
most imprecise terms in all of computer science (even when we leave aside the
data type \texttt{char}). In and of itself, a character is nothing more than an
identifier: \texttt{LATIN CAPITAL LETTER I}\footnote{Isn't that a recursive
definition? You betcha.}. There are many things a character is not:

\begin{denseitemize}
\item{A character is not necessarily unique within a character set (see diacritic vs.\ precomposed forms).}
\item{Different characters needn't be distinct glyphs\cite{nothinggoesaway}.}
\item{A character needn't limit itself to a single column, even when using a fixed-width font.}
\item{A character does not necessarily have a visual representation.}
\item{A character does not necessarily have a single meaning among different languages.}
\item{In a given encoding, all characters are not necessarily the same size.}
\item{A character cannot necessarily capture the state of the keyboard at some time.}
\item{A character cannot necessarily describe a cell of a display at some time.}
\end{denseitemize}

Collecting characters and assigning them distinct (not necessarily contiguous or
even ordered) numbers creates a \textit{code set}, and the smallest contiguous
range of numbers covering this code set is a \textit{code space}. The Unicode 13
code space is comprised of 17 contiguous planes of 65,536 code points each, for
a size of $17*2^{16}$. It defines 143,859 characters, smaller than its
code space by an order of magnitude. A map from a code set to a set of bit
sequences is a \textit{character encoding}. There are numerous standard
encodings of the Universal Character Set, each of which represents the entire
set in different ways (the best one is UTF-8, as we will see). Character sets
are registered with IANA per RFC 2978\cite{rfc2978}, and character encodings
are registered with the ISO Register of Coded Characters per ISO/IEC 2375\cite{iso2375}.

Character graphics are formed from \textit{glyphs}, the visual renderings of a
character encoding, supplied by a font. These glyphs might fuse into what
Unicode Standard Annex \#29\cite{annex29} refers to as a \textit{user-perceived
character} or \textit{grapheme cluster}. A formal algorithm for dividing a
stream of glyphs into grapheme clusters (``horizontally segmentable units of
text\cite{meaningcodepoints}'') achieves \textit{segmentation}. The current
Unicode segmentation algorithm yields \textit{extended grapheme clusters}.

These extended grapheme clusters (EGCs) can be thought of as the atoms of
character graphics. It is not generally possible to write them partially, nor
to partially overlay one atop another. Backspacing ought usually remove entire EGCs
at a time. The cursor ought cross an EGC in a single movement. In Notcurses,
each cell of a plane can hold a single EGC, which must be written as a single
unit. The actual number of columns occupied by an EGC is a property of the font
and layout engine being used.

In the modern era, Unicode is used just about everywhere, and programs must be
prepared to handle it. On the plus side, Unicode presents a single, coherent
means of representing all the world's languages, eliminating the old necessity
of switching among encodings at runtime. UNIX has its origins---and most of its
modern UI---in a much smaller character set: ASCII.

\subsection{Everyone loves ASCII}
\begin{wrapfigure}{o}{.5\textwidth}
  \fbox{%
    \parbox{0.5\textwidth}{
In the early 1960s, an IBM employee on a dare ate a hallucinogenic tapeworm
that pissed nitric acid. Stuffing his steaming, rapidly decoiling bowels
up into a brown paper bag, he was off to the hospital. Alas! On those very
hospital steps he was struck by lightning, and trepanned with great drills, and
set upon by savage hogs, and finally exploded. The tapeworm emerged, clad
in shimmering mandorla, with a name pronounceable by no human tongue. ``Adding
hallucinogenic tapeworms to a late project only makes it later,'' cautioned Fred
Brooks\cite{mythicalman}, but Gene Amdahl knew no peace. ``This
Tapeworm Queen reminds me of halcyon Wisconsin days!'' he cried, and on a prototype
Model 30 appeared her Name. Struck by the power of that Name, all began to bow,
and to weep. She vomited forth the Tapeworm Gospel. Looking upon this filth, they
thought it Good.\\
So now you know what EBCDIC is. There ended up being no fewer than 57
variants\cite{searle} of EBCDIC, worthy of Heinz. We will speak of it no
further. If you find yourself needing more information about EBCDIC, think long
and hard about your life decisions.}%
  }
\end{wrapfigure}
The first digital codes were the English telegraph system of Charles Wheatstone
and William Cooke, and American Samuel Morse's code, with which he in 1837 famously
signaled ``What hath God wrought!'' The Morse system was simpler, running
over a single line, and quickly won out. The teleprinter of Jean-Maurice-Émile Baudot
(from whom we derive the modern unit \textit{baud}) followed in
1874\cite{evolutioncodes}, with its pentabit, fixed-length Baudot code. Baudot
code required 7.42 bittimes per character; at a typical 0.022s bittime, it
could transmit just over 6 characters per second\cite{martin}.

For our purposes, the story can reasonably start with a 7-bit encoding of
128 characters: ANSI X3.4-1986, perhaps better known as ASCII (Table~\ref{table:ascii}).
The first ASCII we would recognize as such\footnote{To paraphrase G.\ H.\
Hardy, ``ASCII-1963 was clever schoolboys; '68 a Fellow from another
College\cite{ghhardy}.''} was the unpublished ASCII-1965. The ASA (ANSI
before it was ANSI) ASCII 1963 didn't even have miniscules\footnote{Uppercase
and lowercase are loanwords from printing, where the two sets were
stored in different cases of the typesetting table. ``Majuscules'' and
``miniscules'' will mark your good breeding.}. ASCII-1968 was
released as USAS X3.4-1968 to wild acclaim. The ISO/IEC 646 standard\cite{iso646}
``internationalized'' ASCII-1968 by opening up 12 graphic characters to
regional specifications (the US ASCII defined the ``International Reference
Variant'')\cite{aivosto}.

The first 32 values are non-printable control codes, first given their current
definitions in ASCII 1968. These 32 codes (known as the ``C0'' coded control
set since ISO/IEC 646) lived on through ISO/IEC 2022 (ECMA-35), ISO/IEC 6429
(ECMA-48), ISO/IEC 8859 (ECMA-94), and are reproduced in today's ISO/IEC 10646,
aka the Universal Character Set. They're common across just about every
character set of which I'm aware (EBCDIC went almost entirely its own way,
because EBCDIC), despite being largely archaic and altogether mystifying. Most
of the associated semantics have been obsoleted, and in some cases the encoded
characters are now used for different purposes than originally planned (see
Table~\ref{table:c0maps}). These characters can be entered by pressing Ctrl
along with another key from ASCII; these combinations are defined using
``caret notation''.

\begin{table}[!htb]
  \centering
  \texttt{%
  \begin{tabular}{ |c||c|c|c|c|c|c|c|c|c|c|c|c|c|c|c|c|c| }
    \hline
    & x0 & x1 & x2 & x3 & x4 & x5 & x6 & x7 & x8 & x9 & xa & xb & xc & xd & xe & xf \\
    \hline
    \hline
    0x & NUL & SOH & STX & ETX & EOT & ENQ & ACK & BEL & BS & HT & LF & VT & FF & CR & SO & SI \\
    \hline
    1x & DLE & DC1 & DC2 & DC3 & DC4 & NAK & SYN & ETB & CAN & EM & SUB & ESC & FS & GS & RS & US \\
    \hline
    2x & SP & ! & " & \cellcolor{blue!25}\# & \cellcolor{blue!25}\$ & \% & \& & ' & ( & ) & * & + & , & - & . & / \\
    \hline
    3x & 0 & 1 & 2 & 3 & 4 & 5 & 6 & 7 & 8 & 9 & : & ; & < & = & > & ? \\
    \hline
    4x & \cellcolor{blue!25}@ & A & B & C & D & E & F & G & H & I & J & K & L & M & N & O \\
    \hline
    5x & P & Q & R & S & T & U & V & W & X & Y & Z & \cellcolor{blue!25}[ & \cellcolor{blue!25}\textbackslash{} & \cellcolor{blue!25}] & \cellcolor{blue!25}\^{} & \_ \\
    \hline
    6x & \cellcolor{blue!25}\textasciigrave{} & a & b & c & d & e & f & g & h & i & j & k & l & m & n & o \\
    \hline
    7x & p & q & r & s & t & u & v & w & x & y & z & \cellcolor{blue!25}\{ & \cellcolor{blue!25}| & \cellcolor{blue!25}\} & \cellcolor{blue!25}\textasciitilde{} & DEL \\
    \hline
  \end{tabular}
  }
  \caption[ANSI X3.4-1986 (ASCII)]{ANSI X3.4-1986 (ISO/IEC 646-IRV, IA5, T.50 IRA, RFC 20)---American Standard Code for Information Interchange. Shaded characters can be replaced in regional variants.}
  \label{table:ascii}
\end{table}

Recall the \texttt{struct termios} from Chapter~\ref{section:tty}. The \texttt{c\_cc}
array of that structure defines ``special characters'' for the terminal. Aside from
CR and LF, the various caret controls can there be recovered and redefined. Should
you really want to send \texttt{SIGINT} via 'a' rather than Ctrl+C, that's where
you can do it.

\begin{table}[!htb]
  \centering
  \begin{tabular}{ |c|c|c|c|l| }
    \hline
    Character & Caret & C & \texttt{c\_cc} & Semantics \\
    \hline
    \hline
    0x00 (NUL) & \texttt{\^{}@} & & & \\
    \hline
    0x03 (ETX) & \texttt{\^{}C} & & \texttt{VINTR} & Deliver \texttt{SIGINT} \\
    \hline
    0x04 (EOT) & \texttt{\^{}D} & & \texttt{VEOF} & Indicate end-of-file \\
    \hline
    0x07 (BEL) & \texttt{\^{}G} & \textbackslash{}a & & Alert (bell) \\
    \hline
    0x08 (BS) & \texttt{\^{}H} & \textbackslash{}b & & Backspace \\
    \hline
    0x09 (HT) & \texttt{\^{}I} & \textbackslash{}t & & Proceed to next tab stop \\
    \hline
    0x0a (LF) & \texttt{\^{}J} & \textbackslash{}n & & Move down one line \\
    \hline
    0x0b (VT) & \texttt{\^{}K} & \textbackslash{}v & & Move down to next vertical tab \\
    \hline
    0x0c (FF) & \texttt{\^{}L} & \textbackslash{}f & & Move to start of line \\
    \hline
    0x0d (CR) & \texttt{\^{}M} & & & Carriage return \\
    \hline
    0x0e (SO) & \texttt{\^{}N} & & & (ISO 2022 mode) SO/LS1 \\
    \hline
    0x0f (SI) & \texttt{\^{}O} & & \texttt{VDISCARD} & (ISO 2022 mode) SI/LS0 \\
    \hline
    0x11 (DC1) & \texttt{\^{}Q} & & \texttt{VSTART} & Software flow control (resume output) \\
    \hline
    0x12 (DC2) & \texttt{\^{}R} & & \texttt{VREPRINT} & Reprint input prompt \\
    \hline
    0x13 (DC3) & \texttt{\^{}S} & & \texttt{VSTOP} & Software flow control (pause output) \\
    \hline
    0x15 (NAK) & \texttt{\^{}U} & & \texttt{VKILL} & Erase line \\
    \hline
    0x16 (SYN) & \texttt{\^{}V} & & \texttt{VLNEXT} & Literal next---inhibit translation\\
    \hline
    0x17 (ETB) & \texttt{\^{}W} & & \texttt{VWERASE} & Erase word \\
    \hline
    0x1a (SUB) & \texttt{\^{}Z} & & \texttt{VSUSP} & Deliver \texttt{SIGTSTP} \\
    \hline
    0x1b (ESC) & \texttt{\^{}[} & & & Initiate escape sequence \\
    \hline
    0x1c (FS) & \texttt{\^{}\textbackslash{}} & & \texttt{VQUIT} & Deliver \texttt{SIGQUIT} \\
    \hline
    0x1d (GS) & \texttt{\^{}]} & & & \\
    \hline
    0x1e (RS) & \texttt{\^{}\^{}} & & & \\
    \hline
    0x1f (US) & \texttt{\^{}\_} & & & \\
    \hline
  \end{tabular}
  \caption[Usual UNIX semantics of C0]{Usual UNIX semantics of C0. The exact mappings can be inspected with \texttt{stty -a}.}
  \label{table:c0maps}
\end{table}

From Table~\ref{table:c0maps}, it should be obvious that Ctrl essentially clears
the top two bits of a 7-bit ASCII input. It might be less obvious that all
miniscules are equal to the sum of 0x20 and their corresponding majuscule,
allowing lowercase and uppercase to be switched by toggling the fifth (32) bit.
Likewise, 0xDF provides a fast case-insensitive comparison mask. Numeric values
for a digit can be acquired by subtracting 0x30 ('0'). Printable characters are all
those with a 1 in any of the higher five bits, save 0x7F\footnote{Why 0x7f? Like
so many things, this comes down to punch cards. If you've punched an error on
a group of 7 holes, and want to correct it, what are your choices? The only
general solution is to interpret all 7 holes as a strikeout\cite{cardpunch}.}. Numeric values for a
letter can be acquired by subtracting 0x61 ('a') from its miniscule form. The
digits are contiguous, as are each case of letters\footnote{This is not true for
EBCDIC, where the letter sequences are based on the ``zones'' of punch cards\cite{nickgammon}.}.

\subsection{Octa- and hexabit character sets messily diverge}
US-ASCII and the regional dialects of ISO/IEC 646 were more or less sufficient
for working with English\footnote{As far as this author knows, ASCII
can faithfully represent only (modern) English, Latin, and Swahili.}, but even
the accented Latin scripts needed more room to be expressed. Other
segmentally linear, monophonemic writing systems (e.g. Cyrillic, Hangul, or
Greek) would need replace the graphic characters wholesale. The idea of
describing syllabary-based languages in seven or even eight bits is
laughable\footnote{An octet character set for ``rudimentary form'' Japanese,
containing only the \textit{katakana}, was introduced as JIS X 0201-1976.}.

\begin{figure}[!htb]
  \centering
  \includegraphics[width=.5\linewidth]{media/chart437.png}
  \caption[The legendary Code Page 00437.]{Code Page 00437, the IBM PC font immortalized by VGA \textit{(Source: \href{https://int10h.org/oldschool-pc-fonts/readme/}{VileR} under CCA3.0)}.}
  \label{fig:cp437}
\end{figure}

By the late 1970s, the eight-bit ``octet'' byte could be considered
dominant\footnote{\href{https://en.wikipedia.org/wiki/8-bit\_computing}{Wikipedia}
would have you believe (as of 2020-03, anyway) that the System/360 introduced
the eight-bit byte. The System/360 introduced a great many things, but
according to my own research, the IBM Stretch 7030 of 1961 used an 8-bit data
path three years prior\cite{ibmstretch}. The last major sub-octet processor
was probably the Intel 4004, and while the PDP-8 was a 12-bit machine, it
often operated on 6-bit characters.}. The very first ANSI/ISO C standard
(1989) established \texttt{CHAR\_BIT} to have a minimum value of 8\footnote{It
can have a larger value than 8, of course, and indeed does on many
DSPs\cite{cookcharbit}. POSIX does mandate that \texttt{CHAR\_BIT}==8.}. With
the high bit of an octet byte going largely unused as a parity bit, the
seven-bit character sets were rapidly expanded into a wide variety of eight-bit
character sets (sometimes mistakenly referred to as ``extended'' or ``high''
ASCII---the closest thing to ``extended ASCII'' might be the withdrawn ANSEL
ISO-IR 231 ``ANSI extended Latin''\cite{ansel}).

\begin{wrapfigure}{o}{0.4\textwidth}
\centering
\includegraphics[width=0.5\textwidth]{media/charset-timeline.png}
\caption{Timeline of selected sets.}
\label{fig:charset-timeline}
\end{wrapfigure}

Among the first ASCII-derived eight-bit ``code pages'' was IBM's 00437, released
with the IBM 5150 PC. It spread far and wide by virtue of being burned into the
ROM of IBM's MDA (9x14), CGA (8x8), EGA (8x14), and VGA (9x16) controllers.
CP437 used most of the extended space for semigraphics, mathematical symbols,
and some characters useful in the world as perceived by America circa 1981\footnote{The
 only suggestion that Asia existed as of CP437 is the ¥.}---CP437 has you
covered regarding the Dutch florin and the Spanish peseta\footnote{Which said
peseta didn't even have a symbol, just \texttt{₧}.}. It also introduced
the non-breaking space (at code 0xff). Interestingly, a melange of semigraphics
were assigned to the C0 control character code points, though these retained
their previous meanings (sometimes). All CP00437 characters have similar
glyphs in ISO 10646, though not necessarily at the same locations\footnote{A few
of the CP437 characters have multiple possible UCS equivalents;
selecting the correct one depends on context.}. Derivatives of CP437
usually jettisoned the mixed box drawing characters and some of the math
symbols to further expand alphabet coverage.

There are well over a thousand known code pages corresponding to regional,
corporate, and even private character sets, and I shall not even begin to
cover them all. There are code pages by Microsoft popularly known as the ``ANSI
code pages'' (despite not being based on any ANSI standard), and there are code
pages by IBM for Windows emulation which differ from the Windows code pages.
HTML5 requires that ``text/'' media types (that would previously have been interpreted
as ISO-8859-1) be interpreted as Windows-1252; Windows refers to
ISO-8859-1 as Windows-28591. Be glad that you are not working in the age of
8-bit character sets. This discordant cacophony gave rise to \textit{mojibake}
(文字化け [{\fontspec{DejaVu Serif}mod͡ʑibake}]), the garbled result of
mismatched character encoding and decoding\footnote{Japanese, from 文字 (moji
(character)) 化ける (bakeru (take a different form)). Russians know it as
\textrussian{кракозя́бры} (krakozyabry [{\fontspec{DejaVu
Serif}krɐkɐˈzʲæbrɪ̈}]) or sometimes \textrussian{бНОПНЯ} (bnopnjá
{\fontspec{DejaVu Serif}bnɐpˈnʲa}), while Bulgarians speak of
\textbulgarian{маймуница} (majmunica (monkey alphabet)). Serbs cut to a
characteristically Balkan chase with \textrussian{ђубре} (đubre (trash)). I
{\fontspec{DejaVu Serif}�} Unicode.}. ISO 4873\cite{iso4873} formalized a
split between control character sets (prefixed with 'C') and graphic character
sets (prefixed with 'G'); ISO 2022 formalized a baroque mechanism for shifting
between ``left'' (for handling 7-bit graphic characters) and ``right'' active
graphic character sets (for handling 8-bit graphic characters) from a selection
of G0--G3. Really, be quite glad that you are not working in the age of 8-bit
character sets.

From a historical perspective, the most important eight-bit character set
is likely ISO-8859-1\cite{iso8859}, based on the Multinational Character Set
of DEC's VT220, ANSI X3.4-1986, and the C1 control code set defined in ISO 6429.
Despite being a terrible attempt at encompassing the French alphabet\footnote{French
discontent eventually resulted in the ``corrected'' ISO-8859-15, published in
1999, by which time Unicode was eight years old and no one really gave a \textit{merde}\cite{french}.},
ISO-8859-1 was directly mapped to the Basic Latin and Latin-1 Supplement blocks
of UCS, together having range 0--0xff. Thus the first 128 codepoints of both 
ISO-8859-1 (all the ISO 8859 sets, actually) and UCS are inherited from ANSI
X3.4-1986 + ISO 646's C0 + ISO 2022's universal declaration of SP (space, 0x20) and
DEL (delete, 0x7f), and the second 128 codepoints of UCS are inherited
from ISO-8859-1 + ISO 6429's C1. Only the \textit{codepoints} are equal, however---UCS
does not typically encode these codepoints the same way\footnote{A notable
and critical exception is UTF-8, which encodes the first 128 codepoints the
same way as ISO 646. Read on\textellipsis}.

There were also some true 16-bit and multibyte character sets, mainly in East
Asia. You might hear of Shift JIS, EUC-JP, Big5, and GB18030. How unfortunate.

\subsection{Consilience: the Universal Character Set}
\label{sec:ucs}
\begin{figure}[!htb]
  \centering
  \includegraphics[width=1.1\linewidth]{media/unicode-growth.png}
  \caption{2020's Unicode 13.0 ships 143,859 character definitions.}
  \label{fig:unicodegrowth}
\end{figure}

\begin{table}[!htb]
  \begin{center}
    \begin{tabular}{ |c|l| }
      \hline
      Principle & Statement \\
      \hline
      \hline
      Universality & \makecell[l]{The Unicode Standard provides a single, universal repertoire.} \\
      \hline
      Efficiency & Unicode text is simple to parse and process. \\
      \hline
      Characters, not glyphs & The Unicode Standard encodes characters, not glyphs. \\
      \hline
      Semantics & Characters have well-defined semantics. \\
      \hline
      Plain text & Unicode characters represent plain text. \\
      \hline
      Logical order & The default for memory representation is logical order. \\
      \hline
      Unification & \makecell[l]{The Unicode Standard unifies duplicate characters within scripts\\across languages.} \\
      \hline
      Dynamic composition & Accented forms can be dynamically composed. \\
      \hline
      Stability & \makecell[l]{Characters, once assigned, cannot be reassigned and key properties\\are immutable.} \\
      \hline
      Convertibility & \makecell[l]{Accurate convertibility is guaranteed between the Unicode Standard\\and other widely accepted standards.} \\
      \hline
    \end{tabular}
  \end{center}
  \caption[The ten Unicode design principles.]{The 10 Design Principles of Unicode (Unicode Core Specification §2.2\cite{unicode}).}
  \label{table:ucsdesign}
\end{table}
The Universal Character Set (ISO 10646\footnote{Note that this is 10,000 more than ISO 646. Cute?},
the character set underlying Unicode) was introduced to resolve these problems,
as a ``unique, universal, and uniform character encoding''\cite{unicodehistory}. The
Unicode Consortium traces its origin to Joe Becker's paper, ``Unicode 88''\cite{unicode88}.
Often in these early documents, UCS is described as a ``16-bit character set''.
This is perhaps partially responsible for the widespread misconception that UTF-16\cite{rfc2781}
can encode all UCS code points in a single 16-bit unit. 16 bits are sufficient
to encode any given UCS \textit{plane} of up to 65,536 code points, and indeed a great
many languages are contained within the first UCS plane, the Basic Multilingual
Plane (0x0--0xffff). There are a total of 17 planes available in
UCS\footnote{Once upon a time, these other sixteen planes were known as the
Astral Planes\cite{astralplanes}.}, however, and UTF-16 requires two code
units to encode points from these other 16 planes (0x10000--0x10ffff).

\begin{table}[!htb]
  \begin{center}
    \begin{tabular}{ |c|c|l| }
      \hline
      Plane & ID & Purpose \\
      \hline
      \hline
      Basic Multilingual & 0 & Common-use characters for modern and historical scripts. \\
      \hline
      Supplementary Multilingual & 1 & Spillover from BMP. \\
      \hline
      Supplementary Ideographic & 2 & CJK spillover from BMP. \\
      \hline
      Supplementary special-purpose & 14 & Format control spillover. \\
      \hline
      Supplementary Private Use A & 15 & Local semantics. \\
      \hline
      Supplementary Private Use B & 16 & Local semantics. \\
      \hline
    \end{tabular}
  \end{center}
  \caption[The six named UCS planes.]{The six named UCS planes (Unicode Core Specification §2.8\cite{unicode}).}
  \label{table:ucsplanes}
\end{table}

\textbf{FIXME FIXME FIXME mention construction of UCS (ASCII/C0 inclusion, 8859)}

\begin{figure}[!htb]
\centering
\includegraphics[width=0.5\textwidth]{media/yearly-charsets.png}
\caption{Use of charsets over the last decade \textit{(source: \href{https://w3techs.com/technologies/history\_overview/character\_encoding/ms/y}{W3Techs})}.}
\label{fig:charsetuse}
\end{figure}

\subsection{Fixed-width fonts ain't so fixed}
Notcurses assumes that all glyphs occupy widths which are an integral multiple
of the smallest possible glyph's cell width (aka a ``fixed-width font'').
Unicode introduces characters which generally occupy two such cells, known as
wide characters (though in the end, width of a glyph is a property of the
font). It is not possible to print half of such a glyph, nor is it generally
possible to print a wide glyph on the last column of a terminal.

Notcurses does not consider it an error to place a wide character on the last
column of a line. It will obliterate any content which was in that cell, but
will not itself be rendered. The default content will not be reproduced in such
a cell, either. When any character is placed atop a wide character's left or
right half, the wide character is obliterated in its entirety. When a wide
character is placed, any character under its left or right side is annihilated,
including wide characters. It is thus possible for two wide characters to sit
at columns 0 and 2, and for both to be obliterated by a single wide character
placed at column 1.

Likewise, when rendering, a plane which would partially obstruct a wide glyph
prevents it from being rendered entirely. A pathological case would be that of
a terminal $n$ columns in width, containing $n-1$ planes, each 2 columns wide.
The planes are placed at offsets $[0\ldots n-2]$. Each plane is above the plane to
its left, and each plane contains a single wide character. Were this to be
rendered, only the rightmost glyph would be visible!

Finally, fonts and font engines which yield glyphs wider (or narrower) than
\texttt{wcwidth()} would lead Notcurses to believe can cause problems. It is
best to avoid EGCs known to be very wide, or to avoid fonts which generate very
wide glyphs. Some examples are shown in Figure~\ref{fig:wideglyphs}.

\begin{figure}[!htb]
\centering
\includegraphics[width=1\linewidth]{media/wide-unicode.png}
\caption[Some very wide Unicode glyphs]{Some very wide Unicode glyphs (font: Hack 10)}
\label{fig:wideglyphs}
\end{figure}

\subsection{Emoji}
According to Unicode Standard Annex \#51\cite{annex51}, emoji are ``pictorial
symbols that are typically presented in a colorful cartoon form and used
inline in text. They represent things such as faces, weather, vehicles and
buildings, food and drink, animals and plants, or icons that represent
emotions, feelings, or activities.''
Don't blame me, mang; I didn't do it. Emoji (Japanese 絵文字えもじ ([{\fontspec{DejaVu Serif}emodʑi}]),
絵 (picture) and our old friend 文字 (moji (character)) emerged from Japan
in the late 90s, and by June 2019 they were in the Oxford Fucking Dictionary of
the English Language\cite{oedgay}---sorry, I get kinda touchy about the OED.
Emoji represent, by far, the most kibbitzed-upon, politicized, arbitrary
element of Unicode; for us, they are also the most dangerous. Presentation of
emoji is rivaled only by bidirectional text for variation across platforms.

Some Unicode ``emoji'' support only a textual presentation, while some
support both (with different defaults for different characters). Selecting
the presentation can sometimes be done by following the base with \texttt{U+FE0E VARIATION SELECTOR-15} (text)
or \texttt{U+FE0E VARIATION SELECTOR-16} (emoji). The full data on such things
is in the Unicode-adjacent ``Data files for emoji characters'', but how closely
your local implementation will conform this week is generally unknowable. The
Fitzpatrick dermatological scale has been adopted for skin-modifying postfixes.
Microsoft, Apple, and other Californian megacorporations think
\xelatexemoji{1f52b-200e} is the best representation for
\xelatexemoji{1f52b-200d} \texttt{U+1F52B PISTOL} (I personally like
\xelatexemoji{1f52b}). That'll be important for things like
{\fontspec{Symbola}{🐊🔫👮}}, an example used in \#51 to illustrate the
importance of maintaining pictograph direction. It's a madhouse out there.

You might have seen what appeared to be emoji flags of the world. The Unicode
Consortium, in their wisdom, wanted nothing to do with such geopolitical
questions. Instead, the 26 ``Regional Indicator Symbols``---one for each
letter of the ISO basic Latin alphabet\cite{iso646}---can be used to encode
ISO 3166-1\cite{iso3166} two-letter country codes\cite{darkcorners}. Some fonts contain flag
glyphs, and some layout engines will map pairs of these symbols to flags. The
UCS \textit{does not} contain national ``flag emoji'' (it does have some other
flags, though---see Table~\ref{table:flags}\footnote{Using a green block,
it is possible to generate the flag of the Great Socialist People's Libyan Arab Jamahiriya through 2011.}).

\begin{table}[!htb]
  \centering
  \begin{tabular}{|l|l|l|}
  \hline
  Character & Version & Comments \\
  \hline
  \hline
  🎌 U+1F38C CROSSED FLAGS & Unicode 6.0 & why Japanese crossed flags? \\
  \hline
  🏁 U+1F3C1 CHEQUERED FLAG & Unicode 6.0 & ``chequered'' whatever \\
  \hline
  🏳 U+1F3F3 WAVING WHITE FLAG & Unicode 7.0 & surrender! \\
  \hline
  🏴  U+1F3F4 WAVING BLACK FLAG & Unicode 7.0 & the Nick Black flag amirite \\
  \hline
  🚩 U+1F6A9 TRIANGULAR FLAG ON POST & Unicode 6.0 & suspiciously golf-related \\
  \hline
  \end{tabular}
  \caption{Flags in Unicode 13 \xelatexemoji{1f3f4-200d-2620-fe0f}.}
  \label{table:flags}
\end{table}

There are also official and non-official EGCs that commonly render flags, including
the Jolly Roger (U+1F3F4, U+2620, \xelatexemoji{1f3f4-200d-2620-fe0f}),
Gilbert Baker's rainbow flag (U+1F3F3, U+1F308, \xelatexemoji{1f3f3-200d-1f308+fe0f}) and Monica Helms's
transgender flag (U+1F3F3, U+26A7, \xelatexemoji{1f3f3-200d-26a7-fe0f}).
If you don't like the latter, you can always throw U20E0 COMBINING ENCLOSING
CIRCLE BACKSLASH atop it:\xelatexemoji{1f3f3-200d-1f308-20e0-fe0f}. It's a big Unicode tent, and there are plenty
of EGCs for everybody.

The downside of using such complexly combined EGCs is of course that what's
intended to be a single EGC can be broken into multiple EGCs by the font
rendering system. Suddenly, instead of \xelatexemoji{1f3f4-200d-2620-fe0f}
you've got an inscrutable \texttt{🏴☠}. To add insult to injury, this will
almost certainly screw up geometry through the end of the line, since Notcurses
will have an incorrect concept of the cursor's horizontal position. This
assumes that the font in use has even the basic component glyphs---in the very
limited Linux console font, it's unlikely that anything good can come of using
composed emoji. Note furthermore that while it is sometimes possible to set the
background color of colored emoji, the foreground color can only rarely be
controlled. When used properly under aligned stars, however, emoji can make
your TUI pop like nothing else.

\subsection{Stupid Unicode tricks}
Perhaps the single most useful Unicode character is U+2580 UPPER HALF BLOCK
(\texttt{▀}), or alternatively its inverse U+2584 LOWER HALF BLOCK
(\texttt{▄}). By using these together with both a foreground and background,
it is possible to treat a region of cells as a ``framebuffer'' having twice the
vertical resolution as the number of occupied lines, and perhaps more importantly
having a cell aspect ratio of 1:1. With this technique, it's trivial to blit
RGBA as provided by e.g. image decoders and get good results (this is exactly
what's done in Notcurses's media layer, see Chapter~\ref{sec:libav}). A smart
implementation will, when possible, replace a HALF BLOCK plus two RGB specifications
with U+2588 FULL BLOCK and a single foreground RGB. A smarter implementation
still will, when possible, replace a FULL BLOCK with a space and a single
background RGB\footnote{This optimization is not possible when, for instance,
the alphas are anything other than \texttt{CELL\_ALPHA\_OPAQUE}.}, though this
optimization saves only 2 bytes compared to the dozen or so saved by eliding
an RGB escape. Notcurses effects both optimizations.

% FIXME ought we talk about e.g. awesomefont? 8x8 in a plane?

\begin{figure}[!htb]
    \centering
    \includegraphics[width=.75\linewidth]{media/blockelements.png}
    \caption{Unicode 13.0 Block Elements \textit{(source: \href{https://commons.wikimedia.org/wiki/File:UCB_Block_Elements.png}{Antonsusi} under CCA3.0)}.}
    \label{fig:blockelements}
\end{figure}

Unfortunately, it's difficult to take this technique any further. There are
U+258C LEFT HALF BLOCK (\texttt{▌}) and U+2590 RIGHT HALF BLOCK (\texttt{▐}):
they exacerbate the character cell aspect ratio problem, but can still be
useful for rendering certain surfaces in tall, thin geometries. A single cell
can be divided up to 8 ways using the eighths blocks, but only being able to
supply two colors (not to mention the resulting aspect ratio) means this can't
be easily used for 8x resolution. There are three incompletely-filled shades of
U+2588 FULL BLOCK, which could conceivably be used to increase the color range
by a factor of 4, but the 24-bit RGB space is already pretty large---resolution
is a much more precious resource.

Those readers who remember Code Page 437 might recall 0xFD, a ``middle half
block''. Given a constant background, this can be used with the other two
half blocks to increase ``vertical resolution'' by a factor of 3 (imagine a
tank drawn with upper, middle, and lower half blocks moving on a monochromatic
background). Well, that character was actually ``Solid Square/Histogram/Square Bullet''
in the CP437 definitions\cite{cp437}, and lives on as U+25A0 BLACK SQUARE (\texttt{■}).
Despite its name, it will take on the foreground color.

It might be useful to use the quadrant characters to effect a 2x2 resolution
doubling. The quadrants cannot yield a lossless reproduction for any 2x2 block
that uses more than two colors, but it might be useful to interpolate given the
small total number of possible colors. Some graduate student ought look
into this.

It is possible to use the Braille Patterns at U+2800 to get a 16x increase in
resolution, at the cost of having only two colors available per 2x4 ``8-pixel''
cell (and looking like blotter paper, depending on the font). This can be an
excellent solution for e.g. graphing, especially vertical histograms. By
examining (Figure~\ref{fig:braille}), it ought be clear that the 8 areas
map to distinct bits: counterclockwise from the upper left, the bits are 1, 2,
4, 64, 128, 32, 16, 8. This odd ordering is due to true braille only being a
6-bit (64 character) character set; this way, real braille maps to the first
64 codepoints.

\begin{figure}[!htb]
    \centering
    \includegraphics[width=1\linewidth]{media/braille-unicode.png}
    \caption{Unicode Braille characters.}
    \label{fig:braille}
\end{figure}

Certain sets of Unicode characters form cyclic groups. These can be useful for
simple, single-cell animations, e.g. work-indicating ``spinners''. I've isolated
some of these groups in Table~\ref{table:cyclics}. Smaller groups can usually
be isolated from larger groups: as an example, the eight-way HALF-SQUARE cycle
yields two four-way subcycles and four two-way subcycles.

\begin{table}[!htb]
  \centering
  \begin{tabular}{|l|l|l|}
    \hline
    Size & Glyphs & Comments \\
    \hline
    \hline
    2 & {\fontspec{Symbola}▱▰} & \makecell[l]{Parallelogram. You can get a kinda\\ neat ``barber-pole'' effect by alternating each period.} \\
    \hline
    4 & {\fontspec{Symbola}◰◳◲◱} & Note that they're out of order \\
    \hline
    6 & {\fontspec{Symbola}🞯🞰🞱🞲🞳🞴} & Five-spoked asterisk \\
    \hline
    6 & {\fontspec{Symbola}🞵🞵🞶🞷🞸🞹🞺}  & Six-spoked asterisk \\
    \hline
    5 & {\fontspec{Symbola}🞻🞼🞽🞾🞿} & \makecell[l]{Fuck your seven-spoked asterisk, fuck expecting six \\eight-spoked asterisks, and fuck you too.} \\
    \hline
    8 & \textbackslash{}|/-\textbackslash{}|/- & Approximate, but it works everywhere. \\
    \hline
    8 & {\fontspec{Symbola}◧◩⬒⬔◨◪⬓⬕} & Some fonts have different sizes. Why? Ugh. \\
    %8 & \texttt{◧◩⬒⬔◨◪⬓⬕} & Some fonts have different sizes. Why? Ugh. \\
    \hline
    10 & {\fontspec{Symbola}🞌🞍🞎'🞏'🞐'🞑'🞒'🞓'🞔'🞕🞖} & The first two are maybe cheating. \\
    \hline
  \end{tabular}
  \caption{Just a few of Unicode's many cyclic groups.}
  \label{table:cyclics}
\end{table}

Certain transformations can be safely applied to the full Latin alphabet
(see Figure~\ref{fig:unicodeweird} for more, some of which couldn't be
realized in \LaTeX).

\begin{denseitemize}
\item{{\fontspec{Unifont}ⓦⓔ ⓒⓐⓝ ⓟⓤⓣ ⓞⓤⓡ ⓣⓔⓧⓣ ⓘⓝ ⓒⓘⓡⓒⓛⓔⓢ}}
\item{stretttch text with fullwidth}
\item{{\fontspec{DejaVu Serif}𝕞𝕒𝕥𝕙 𝕕𝕠𝕦𝕓𝕝𝕖 𝕤𝕥𝕣𝕦𝕔𝕜, 𝕗𝕠𝕣 𝕥𝕙𝕠𝕤𝕖 𝕨𝕙𝕠 𝕨𝕚𝕤𝕙 𝕥𝕠 𝕝𝕠𝕠𝕜 𝕝𝕚𝕜𝕖 ℝ\^𝕟}}
\end{denseitemize}

Others can be performed on a good chunk of Latin script, but are incomplete:

\begin{denseitemize}
\item{{\fontspec{DejaVu Serif}ᴘᴀᴄᴋ ᴍʏ ʙᴏx ᴡɪᴛʜ ꜰɪᴠᴇ ᴅᴏᴢᴇɴ ʟɪqᴜᴏʀ ᴊᴜɢꜱ (smallcaps)}}
\item{{\fontspec{DejaVu Serif}ₚₐcₖ ₘy bₒₓ wᵢₜₕ fᵢᵥₑ dₒzₑₙ ₗᵢqᵤₒᵣ ⱼᵤgₛ (subscripts)}}
\item{{\fontspec{DejaVu Serif}ᴾᵃᶜᵏ ᵐʸ ᵇᵒˣ ʷⁱᵗʰ ᶠⁱᵛᵉ ᵈᵒᶻᵉⁿ ˡⁱqᵘᵒʳ ʲᵘᵍˢ (superscripts)}}
\item{{\fontspec{DejaVu Serif}dɐɔʞ ɯʎ qox ʍıʇɥ ɟıʌǝ pozǝu ןıbnoɹ ɾnƃs (inverted)}}
\item{{\fontspec{DejaVu Serif}ꟼAↄk mY dox wiTH ꟻivɘ bozɘᴎ lipUoᴙ jUgꙅ (reversed)}}
\end{denseitemize}

\begin{figure}[!htb]
    \centering
    \includegraphics[width=.75\linewidth]{media/unicode-weird.png}
    \caption{A pangram using a variety of Unicode texts.}
    \label{fig:unicodeweird}
\end{figure}

``Pseudoalphabets'' by the thousand exist. These simply substitute glyphs for
others of which they are more or less suggestive. There's no real rhyme or
reason behind these pseudoalphabets, but they can be useful if you need a
different ``font''. Some examples are shown in Figure~\ref{fig:pseudoalphabets}.

\begin{figure}[!htb]
    \centering
    \includegraphics[width=.75\linewidth]{media/pseudoalphabets.png}
    \caption{Some of ``Eli the Bearded'''s pseudoalphabets from \url{http://qaz.wtf/u/}.}
    \label{fig:pseudoalphabets}
\end{figure}

Finally, we can't speak of stupid Unicode tricks without mentioning
\textit{zalgo}: \texttt{i̸̬̦͙̒̌̓͝n̴͛̓͋̈́͂̇̑̅̕e̵̎̈́̎̀̐̓͆͘͝l̴̠͛͂̃́̅̄̚͝u̷̾́̄̇́̎͂̚͝c̴̡̟̜̉̈́̋͂̈́̎t̵́͊̂̑̑̇̒̃̄å̵̰͎̫̅̿̏͂̽b̷̛͋̎́͛́͋̈͌l̷̰͎̫͐̑̈͛͆͝e̴̱̋̾̃̒̏͘͘̕m̸̆͋͌̍̉͆͌͛̓ȍ̷̡̜̭͙̞̣̎̚d̴̐̃͆̈́̽̔̐̀͝ä̸́̑̐̒͊̔̾́̕l̶̨̡̙̳̭̮̠̼͝i̷̋̆̌̿̎̃̋̚̚ṯ̷̱̫̺͗͒̔͆̄ẏ̷̡̛̮͈͑̓́͊ȏ̵͋̅́̋͒̐̇͝f̸̛̃͐͒̈̾̆̕͠t̶͈̱̠͋̑̊̿̇͘h̷̰͚̳̳̤͗̓͗̄e̴͇͙̜̿͐́̎̌͘ṽ̶̌̈́́́̈́͌͂͘i̷̛͊̒̏̄̀̅͠͝s̸̯̘͍̝͓̈́ͅͅì̵̃͊̇͗̄̋̇͘b̷̃̎̓͛̊̾͘̚̚l̶̯̼͎͉̎͛̆͠e̷̙͊̇̿͗̃͊̓́}. Dump enough
diacritics into an EGC, and the result is a somewhat Cthulhian garble (it's not
quite \textit{squamous}, and I wouldn't go so far as to call it
\textit{eldritch}, but I suppose it's \textit{unheimlich}). Investigating zalgo
led down a Reddit hole of madness; I mention it only for completeness, and as
an excuse to include Figure~\ref{fig:zalgo}.

\begin{figure}[H]
    \centering
    \includegraphics[width=.5\linewidth]{media/zalgo.png}
    \caption{Past, present, future, all are one in Yog-Sothoth.}
    \label{fig:zalgo}
\end{figure}

\subsection{UTF-8}
Unicode Standard Annex \#17\cite{annex17} defines seven official Unicode
character encoding schemese: UTF-8, UTF-16, UTF-16BE, UTF-16LE, UTF-32, UTF-32BE,
and UTF-32LE. What a wealth of encodings! How is one to choose? The -16BE and
-16LE forms are simply UTF-16 with a known byte order; a UTF-16 stream can
(optionally!) be prefixed with a Byte-Order Mark, at which point the stream
reduces to -16LE or -16BE (in the absence of a BOM, the best advice is to follow
your heart). UTF-32 breaks down the same way. This question of endianness arises
from the fact that UTF-16 and UTF-32 are coded in terms of 16- and 32-bit units.
UTF-8, being coded in terms of individual bytes, has no need to define byte order.

``Well, that BOM sounds kinda annoying,'' I hear you asking. ``What other
advantages are offered by UTF-8?'' Remember how ANSI X3.4-1986 maps precisely
to the first 128 characters of UCS? UTF-8 (and \textit{only} in UTF-8, of
the official encodings) \textit{encodes} these 128 characters the same as
US-ASCII! Boom! Every ASCII document you have---including most source code,
configuration files, system files, etc.---is a perfectly valid UTF-8 document.
Furthermore, UTF-8 \textit{never encodes non-ASCII characters to the ASCII
bytes}. So an arbitrary UTF-8 document may have plenty of high-bit bytes that
your ASCII-aware, POSIX-locale program doesn't understand, but it never sees
a valid ASCII character where one wasn't intended. UTF-8 encodes ASCII's 0--0x7f
to 0--0x7f, and otherwise never produces a byte in that range. This includes
the all-important null character 0---Boom! Every nul-terminated C string is a
valid UTF-8 string. Every UTF-8 string can be passed through standard C APIs
cleanly, and they'll more or less work. It's furthermore self-synchronizing.
If you pick up a UTF-8 stream in the middle, you know after reading a single
byte whether you're in the middle of a multibyte character.

``Sweet! What's the catch? Does it waste space?'' RFC 3629\cite{rfc3629}
limits UTF-8's range to the $17*2^{16}$-ary code space of UCS, in which case
the maximum length of a single UTF-8-encoded UCS code point is four bytes\footnote{You
might hear six bytes, and indeed ISO/IEC 10646 specifies six bytes to handle
up through U+7FFFFFFF\textellipsis but only defines UCS to cover 17 planes.
Verify your \texttt{wctomb(3)} rejects inputs in excess of 0x10ffff before
exploiting RFC 3629's tighter bound.}. It's thus always as or more efficient
than UCS-32. When the ASCII characters are used, UTF-8 is more efficient than
either UTF-16 or UTF-32. Only for streams utterly dominated by BMP codepoints
requiring three or more bytes from UTF-8 can UTF-16 encode more efficiently.

``Sweet! What's the catch? Is it super slow?'' UTF-32, it is true, allows you
to index into a string by character in O(1) (UTF-16 \textit{does not}, unless
you're only dealing with BMP strings). UTF-32 also allows you to compute the
bytes necessary for encoding in O(1), given the number of Unicode codepoints,
but that's only because it's wasteful; if you're willing to be similarly
wasteful, you can do the same calculation with UTF-8 (and then trim any wastage
at the end, if you wish). Any advantage UTF-32 might hold in lexing simplicity
is likely a wash when UTF-8's usual space efficiency is taken into account,
owing to more effective use of cache and memory bandwidth. Nope, it's not slow.
\textbf{Always interoperate in UTF-8 by default.}

UTF-16 is some truly stupid shit, fit only for jabronies. It only ever passed
muster because people thought UCS was going to be a sixteen-bit character set.
The moment a second Plane was added, UTF-16 ought have been shown the door.
There's possibly an argument to be made for ripping it from the pages of books
in your local library, so it can't poison the youth. If you must work on a UTF-16
system, use UTF-16 at the boundary, and then keep it around as UTF-32 or UTF-8.
Always interoperate---including writing files---in UTF-8 by default.

There are a dozen-odd similarly-named encodings which are useful for nothing
but trivia. UCS-2 was UTF-16, but for only the BMP. UCS-4 is just UTF-32. UTF-7
is a seven-bit-clean UTF-8\footnote{The primary seven-bit-clean media of the
modern era is probably email sent without a MIME transfer encoding.}. UTF-1 is UTF-8's older, misshapen sister, locked away from
sight in the attic. UTF-5 and UTF-6 were proposed encodings for IDN, but
Punycode was selected instead. WTF-8 extends UTF-8 to handle invalid UTF-16
input. BOCU-1 and SCSU are compressing encodings that
don't compress as well as gzipped UTF-8. UTF-9 and UTF-18 were jokes. Is
UTF-EBCDIC a thing? Of course UTF-EBCDIC is a thing\footnote{Perhaps the
most cursed thing I'm aware of in computing\footnote{Also receiving votes:
Threaded INTERCAL, sendmail configuration files, old-skool XFree86
configuration files with the modeline bullshit, ActiveX controls, ``WebNFS'', and of course
Perl.} is ``UTFE'', a variable-length
UCS encoding that somehow requires six bytes sometimes, is used exclusively
on EBCDIC platforms, and furthermore exclusively only on (drum roll)\textellipsis.
EBCDIC. ORACLE. DATABASES. Ave Satanas!}

The one place where you won't interoperate with UTF-8 is for domain name lookup,
when converting IDNA into the LDH subset of ASCII. If you're interested,
consult RFC 3492, and Godspeed.

\begin{figure}[!htb]
    \centering
    \includegraphics[width=.75\linewidth]{media/control-char-standards.png}
    \caption{Flow of control characters through historic standards.}
\end{figure}

\cleardoublepage
%%%%%%%%%%%%%%%%%%%%%%%%%%%%%%%%%%%%%%%%%%%%%%%%%%%%%%%%%%%%%%%%%%%%%%%%

\pagebreak
\cleardoublepage
%%%%%%%%%%%%%%%%%%%%%%%%%%%%%%%%%%%%%%%%%%%%%%%%%%%%%%%%%%%%%%%%%%%%%%%%
\section{Using ncplanes}
\label{ncplane}
As mentioned in Chapter~\ref{sec:fullscreen}, \texttt{ncplane}s (henceforth
simply planes) are the fundamental drawing surface of Notcurses. A Notcurses
instance contains a z-axis on which planes are totally ordered\footnote{Future
releases of Notcurses might relax this to a partial ordering, allowing
multiple ncplanes to partition a logical level. See
\url{https://github.com/dankamongmen/notcurses/issues/184}.}. In addition, it
always contains at least one plane, the \textit{standard plane}. This plane's
origin is always defined to be the rendering area's origin. It is always
exactly as large as the rendering area, and it cannot be destroyed.

Chapter \ref{sec:notcursesfuncs} introduced one function that creates a new
plane: \texttt{ncplane\_new()}. In addition, there is \texttt{ncplane\_aligned()}.

\begin{listing}[!htbp]
\begin{minted}{C}
// Create a new ncplane at the specified 'yoff', of the specified 'rows' and 'cols'. Align this plane
// according to 'align' relative to 'n'.
struct ncplane* ncplane_aligned(struct ncplane* n, int rows, int cols, int yoff, ncalign_e align, void* opaque);
\end{minted}
\caption{Creating a new plane aligned relative to another.}
\end{listing}

A plane is defined by:
\begin{denseitemize}
\item{A packed ``framebuffer'' of \texttt{cell} structures. Cells are discussed
    in detail in~\ref{sec:cells}; it is enough now to know that each has an
    EGC, a set of attributes, and a fore- and background color.}
\item{A ``base cell'', rendered for any cell with a null EGC.}
\item{A cursor position relative to the plane's origin.}
\item{The plane's position relative to the visible area's origin.}
\item{A two-dimensional size.}
\item{An ``egcpool'' providing backing storage for the framebuffer.}
\item{An opaque pointer, controlled by the application.}
\item{A current set of attributes and colors.}
\item{A pointer to the plane below this one, or \texttt{NULL} for the bottommost plane.}
\end{denseitemize}

However a plane is created---including the standard plane---it is initialized
in the same way. All cells--both the base cell and those of the
framebuffer---are zeroed out. A zeroed cell has the null glyph (UTF-8
value ``00''), no attributes, and the default foreground and background color
(default colors are always opaque). Note that planes are thus by default
glyph-transparent but color-opaque. The cursor is placed at the origin. The
plane's current attributes and channels are likewise zeroed out. The plane is
pushed onto the top of the z-axis and assigned an initialized egcpool
(see~\ref{sec:egcpools}).

A plane can be duplicated with \texttt{ncplane\_dup()}. This will create a new
plane of the same geometry at the top of the z-axis. It will then have all
other properties duplicated, using its own egcpool.
\begin{listing}[!htbp]
\begin{minted}{C}
// Duplicate an existing ncplane. The new plane will have the same geometry, will duplicate all content, and
// will start with the same rendering state. The new plane will be immediately above the old one on the z axis.
struct ncplane* ncplane_dup(struct ncplane* n, void* opaque);
\end{minted}
\caption{Duplicating a plane.}
\end{listing}

Besides the default plane, planes may occupy any positive size (both the number
of rows and columns must be greater than zero), and have their origins any
integer offset from the visual origin. It is possible for a plane to be a
superset of the visual area, a subset, to exactly match the visual area, to
partially overlap it, or even to be entirely off-screen. A plane can be
moved to any coordinate, but the plane's cursor cannot be moved off the plane.

Any plane save the standard plane may be destroyed with \texttt{ncplane\_destroy()}.
All planes save the standard plane may be destroyed in one fell swoop with
\texttt{notcurses\_drop\_planes()}.
\begin{listing}[!htbp]
\begin{minted}{C}
// Destroy the specified ncplane. None of its contents will be visible after the next
// call to notcurses_render(). It is an error to attempt to destroy the standard plane.
int ncplane_destroy(struct ncplane* ncp);

// Destroy all ncplanes other than the standard plane.
void notcurses_drop_planes(struct notcurses* nc);
\end{minted}
\caption{Destroying planes.}
\end{listing}

\subsection{Moving and resizing planes}
Even the standard plane can be reordered along the z-axis. \texttt{ncplane\_move\_top()}
and \texttt{ncplane\_move\_bottom()} are absolute, moving the specified plane
to the top or bottom of the z-axis, respectively. \texttt{ncplane\_move\_above()}
and \texttt{ncplane\_move\_below()} are relative, moving the plane immediately
above or below another one. It is an error to try and move a plane below or above itself,
or above or below \texttt{NULL}. Likewise, an error will be returned if the relative
plane does not exist on the z-axis.
\begin{listing}[!htbp]
\begin{minted}{C}
// Splice ncplane 'n' out of the z-buffer, and reinsert it at the top or bottom.
int ncplane_move_top(struct ncplane* n);
int ncplane_move_bottom(struct ncplane* n);

// Splice ncplane 'n' out of the z-buffer, and reinsert it above 'above'.
int ncplane_move_above_unsafe(struct ncplane* restrict n, struct ncplane* restrict above);

static inline int
ncplane_move_above(struct ncplane* n, struct ncplane* above){
  if(n == above){
    return -1;
  }
  return ncplane_move_above_unsafe(n, above);
}

// Splice ncplane 'n' out of the z-buffer, and reinsert it below 'below'.
int ncplane_move_below_unsafe(struct ncplane* restrict n, struct ncplane* restrict below);

static inline int
ncplane_move_below(struct ncplane* n, struct ncplane* below){
  if(n == below){
    return -1;
  }
  return ncplane_move_below_unsafe(n, below);
}

// Return the plane above this one, or NULL if this is at the top.
struct ncplane* ncplane_below(struct ncplane* n);
\end{minted}
\caption{Moving planes on the z axis.}
\end{listing}

All planes other than the standard plane can be moved in the x- and y-dimensions.
\begin{listing}[!htbp]
\begin{minted}{C}
// Move this plane relative to the standard plane. It is an error to attempt to // move the standard plane.
int ncplane_move_yx(struct ncplane* n, int y, int x);
\end{minted}
\caption{Moving planes on the x and y axis.}
\end{listing}

\textbf{FIXME ncplane\_resize(), \_rotate()...}

Sometimes it is useful to translate coordinates between planes, or between the
visible area and planes (this latter is particularly useful when interpreting
mouse clicks; see Chapter~\ref{sec:input}).
\begin{listing}[!htbp]
\begin{minted}{C}
// provided a coordinate relative to the origin of 'src', map it to the same absolute coordinate
// relative to thte origin of 'dst'. either or both of 'y' and 'x' may be NULL.
void ncplane_translate(const struct ncplane* src, const struct ncplane* dst, int* restrict y, int* restrict x);

// Fed absolute 'y'/'x' coordinates, determine whether that coordinate is within the ncplane 'n'. If not, return false.
// If so, return true. Either way, translate the absolute coordinates relative to 'n'. If the point is not within 'n',
// these coordinates will not be within the dimensions of the plane.
bool ncplane_translate_abs(const struct ncplane* n, int* restrict y, int* restrict x);
\end{minted}
\caption{Translating coordinates between planes.}
\end{listing}

\subsection{Cells}
\label{sec:cells}
Each coordinate of an plane corresponds to a \texttt{cell}. The cell definition
is exposed to the application, though it should not generally be directly
manipulated. A multicolumn cell (a cell containing an EGC of $n$ columns where
$n>1$) overrides the $n-1$ following cells. Since there are always a fixed
number of cells, this means that the overridden cells are skipped during
rendering, as well as being zeroed out at the time the multicolumn EGC is
written to the cell.
\begin{listing}[!htbp]
\begin{minted}{C}
typedef struct cell {
  // These 32 bits are either a single-byte, single-character grapheme cluster (values 0–0x7f), or
  // an offset into a per-ncplane attached pool of varying-length UTF-8 grapheme clusters.
  uint32_t gcluster;          // 4B -> 4B
  // NCSTYLE_* attributes (16 bits) + 8 foreground palette index bits + 8 background palette index
  // bits. palette index bits are used only if the corresponding default color bit *is not* set,
  // and the corresponding palette index bit *is* set.
  uint32_t attrword;          // + 4B -> 8B
  // (channels & 0x8000000000000000ull): left half of wide character
  // (channels & 0x4000000000000000ull): foreground is *not* "default color"
  // (channels & 0x3000000000000000ull): foreground alpha (2 bits)
  // (channels & 0x0800000000000000ull): foreground uses palette index
  // (channels & 0x0700000000000000ull): reserved, must be 0
  // (channels & 0x00ffffff00000000ull): foreground in 3x8 RGB (rrggbb)
  // (channels & 0x0000000080000000ull): right half of wide character
  // (channels & 0x0000000040000000ull): background is *not* "default color"
  // (channels & 0x0000000030000000ull): background alpha (2 bits)
  // (channels & 0x0000000008000000ull): background uses palette index
  // (channels & 0x0000000007000000ull): reserved, must be 0
  // (channels & 0x0000000000ffffffull): background in 3x8 RGB (rrggbb)
  uint64_t channels;          // + 8B == 16B
} cell;
\end{minted}
\caption{The \texttt{cell} definition.}
\end{listing}
The \texttt{gcluster} field is a 32-bit number. If the value is less than 128,
it directly specifies its UTF-8 encoded character. Since Unicode's first 128
values are taken directly from ASCII, this means the entirety of ASCII can be
represented in-line. If the value is greater than or equal to 128, it is a
bias-128 index into the plane's associated egcpool. Since egcpools are per-plane,
this implies that it is unsafe to blindly copy a cell from one plane to another.

Applications generally need not work directly with cells, though sometimes it
is easiest to do so. The usual reason for working with a cell is either to set
all three properies of output at once (glyph, attributes, and colors), or to
receive all three properties at once when retrieving a coordinate's data.

As further discussed in Chapter~\ref{sec:channels}, the \texttt{channels}
variable is a 64-bit field packing together a number of properties. The high
32 bits apply to the foreground, and the low 32 bits to the background. They
can be set and queried as a channel (Listing~\ref{listing:cellchannels}).

\begin{listing}[!htbp]
\begin{minted}{C}
// Extract the 32-bit background channel from a cell.
static inline unsigned cell_bchannel(const cell* cl){
  return channels_bchannel(cl->channels);
}

// Extract the 32-bit foreground channel from a cell.
static inline unsigned cell_fchannel(const cell* cl){
  return channels_fchannel(cl->channels);
}

// Set the 32-bit background channel of a cell.
static inline uint64_t cell_set_bchannel(cell* cl, uint32_t channel){
  return channels_set_bchannel(&cl->channels, channel);
}

// Set the 32-bit foreground channel of a cell.
static inline uint64_t cell_set_fchannel(cell* cl, uint32_t channel){
  return channels_set_fchannel(&cl->channels, channel);
}
\end{minted}
\caption{The \texttt{cell} definition.}
\label{listing:cellchannels}
\end{listing}

RGB values consume 24 bits of each channel, 75\% of the 64 bits in \texttt{channels}.
RGB values can be blended, clipped, and otherwise dealt with arithmetically.

\begin{listing}[!htbp]
\begin{minted}{C}
// do not pass palette-indexed channels!
static inline uint64_t cell_blend_fchannel(cell* cl, unsigned channel, unsigned* blends){
  return cell_set_fchannel(cl, channels_blend(cell_fchannel(cl), channel, blends));
}

// Extract 24 bits of foreground RGB from 'cell', shifted to LSBs.
static inline unsigned cell_fg(const cell* cl){
  return channels_fg(cl->channels);
}

// Extract 2 bits of foreground alpha from 'cell', shifted to LSBs.
static inline unsigned cell_fg_alpha(const cell* cl){
  return channels_fg_alpha(cl->channels);
}

// Extract 24 bits of foreground RGB from 'cell', split into components.
static inline unsigned cell_fg_rgb(const cell* cl, unsigned* r, unsigned* g, unsigned* b){
  return channels_fg_rgb(cl->channels, r, g, b);
}

// Set the r, g, and b cell for the foreground component of this 64-bit
// 'cell' variable, and mark it as not using the default color.
static inline int cell_set_fg_rgb(cell* cl, int r, int g, int b){
  return channels_set_fg_rgb(&cl->channels, r, g, b);
}

// Same, but clipped to [0..255].
static inline void cell_set_fg_rgb_clipped(cell* cl, int r, int g, int b){
  channels_set_fg_rgb_clipped(&cl->channels, r, g, b);
}

// Same, but with an assembled 24-bit RGB value.
static inline int cell_set_fg(cell* c, uint32_t channel){
  return channels_set_fg(&c->channels, channel);
}
\end{minted}
\caption{\texttt{cell} foreground RGBA functionality.}
\label{listing:cellrgbfg}
\end{listing}

\begin{listing}[!htbp]
\begin{minted}{C}
static inline uint64_t cell_blend_bchannel(cell* cl, unsigned channel, unsigned* blends){
  return cell_set_bchannel(cl, channels_blend(cell_bchannel(cl), channel, blends));
}

// Extract 24 bits of background RGB from 'cell', shifted to LSBs.
static inline unsigned cell_bg(const cell* cl){
  return channels_bg(cl->channels);
}

// Extract 2 bits of background alpha from 'cell', shifted to LSBs.
static inline unsigned cell_bg_alpha(const cell* cl){
  return channels_bg_alpha(cl->channels);
}

// Extract 24 bits of background RGB from 'cell', split into components.
static inline unsigned cell_bg_rgb(const cell* cl, unsigned* r, unsigned* g, unsigned* b){
  return channels_bg_rgb(cl->channels, r, g, b);
}

// Set the r, g, and b cell for the background component of this 64-bit
// 'cell' variable, and mark it as not using the default color.
static inline int cell_set_bg_rgb(cell* cl, int r, int g, int b){
  return channels_set_bg_rgb(&cl->channels, r, g, b);
}

// Same, but clipped to [0..255].
static inline void cell_set_bg_rgb_clipped(cell* cl, int r, int g, int b){
  channels_set_bg_rgb_clipped(&cl->channels, r, g, b);
}

// Same, but with an assembled 24-bit RGB value.
static inline int cell_set_bg(cell* c, uint32_t channel){
  return channels_set_bg(&c->channels, channel);
}
\end{minted}
\caption{\texttt{cell} background RGBA functionality.}
\label{listing:cellrgbbg}
\end{listing}

It is also possible to make use of palette-indexed color (recall that the size
of the palette can be acquired with \texttt{notcurses\_palette\_size()}). Palette-indexed
color requires much less bandwidth than pure RGB, and allows for finer control
on terminals which don't faithfully implement RGB DirectColor. The terminal
palette can be manually reprogrammed with the palette256 API.

\begin{listing}[!htbp]
\begin{minted}{C}
// Set the cell's foreground palette index, set the foreground palette index
// bit, set it foreground-opaque, and clear the foreground default color bit.
static inline int cell_set_fg_palindex(cell* cl, int idx){
  if(idx < 0 || idx >= NCPALETTESIZE){
    return -1;
  }
  cl->channels |= CELL_FGDEFAULT_MASK;
  cl->channels |= CELL_FG_PALETTE;
  cl->channels &= ~(CELL_ALPHA_MASK << 32u);
  cl->attrword &= 0xffff00ff;
  cl->attrword |= (idx << 8u);
  return 0;
}

static inline unsigned cell_fg_palindex(const cell* cl){
  return (cl->attrword & 0x0000ff00) >> 8u;
}

// Set the cell's background palette index, set the background palette index
// bit, set it background-opaque, and clear the background default color bit.
static inline int cell_set_bg_palindex(cell* cl, int idx){
  if(idx < 0 || idx >= NCPALETTESIZE){
    return -1;
  }
  cl->channels |= CELL_BGDEFAULT_MASK;
  cl->channels |= CELL_BG_PALETTE;
  cl->channels &= ~CELL_ALPHA_MASK;
  cl->attrword &= 0xffffff00;
  cl->attrword |= idx;
  return 0;
}

static inline unsigned cell_bg_palindex(const cell* cl){
  return cl->attrword & 0x000000ff;
}

static inline bool cell_fg_palindex_p(const cell* cl){
  return channels_fg_palindex_p(cl->channels);
}

static inline bool cell_bg_palindex_p(const cell* cl){
  return channels_bg_palindex_p(cl->channels);
}
\end{minted}
\caption{\texttt{cell} palette-indexed color functionality.}
\label{listing:cellpalette}
\end{listing}

Finally, the default fore- and/or background color can be used, and is indeed
the default. Default colors can't be blended. Some terminals can be configured
to use a transparent background. Only in cells using the default background
color can this effect be seen.

\begin{listing}[!htbp]
\begin{minted}{C}
// Is the background using the "default background color"? The "default background color"
// must generally be used to take advantage of terminal-effected transparency.
static inline bool cell_bg_default_p(const cell* cl){
  return channels_bg_default_p(cl->channels);
}

// Is the foreground using the "default foreground color"?
static inline bool cell_fg_default_p(const cell* cl){
  return channels_fg_default_p(cl->channels);
}
\end{minted}
\caption{\texttt{cell} default color functionality.}
\end{listing}

\subsection{egcpools}
\label{sec:egcpools}

\subsection{Alpha blending and plane transparency}

\cleardoublepage

%%%%%%%%%%%%%%%%%%%%%%%%%%%%%%%%%%%%%%%%%%%%%%%%%%%%%%%%%%%%%%%%%%%%%%%%
% output and styling
\section{Output and styling}
\label{sec:output}
The most fundamental aspect of textual interfaces is, after all, text. All
valid UTF-8 can be written to a plane. If scrolling has been enabled for a
plane, any amount of text can be written. If scrolling has not been enabled,
output can only be generated through the end of the current line. All text
output functions return the number of screen columns written as their primary
return value (or a negative number on failure), and load the number of bytes
written into an auxiliary value-result parameter. If the supplied output would
exceed the line, a short number of columns will be returned. Scrolling is
disabled by default on the standard plane, and on all new planes.

\begin{listing}[!htbp]
\begin{minted}{C}
// Move the cursor to the specified position (the cursor needn't be visible).
// Returns -1 on error, including negative parameters, or ones exceeding the
// plane's dimensions.
int ncplane_cursor_move_yx(struct ncplane* n, int y, int x);

// Get the current position of the cursor within n. y and/or x may be NULL.
void ncplane_cursor_yx(struct ncplane* n, int* restrict y, int* restrict x);
\end{minted}
\caption{Cursor management. Each plane has its own cursor.}
\label{list:cursor}
\end{listing}

The cursor is advanced appropriately to the cell just beyond the output. If the
entire line was written, and scrolling is not enabled, the cursor will be
off-plane and must be repositioned before writing any further. If scrolling is
enabled, the cursor will either move to the first column of the next line, or
if the cursor is on the last line, the plane will be scrolled by one line (and
the cursor will move to the first column). If there is an error following some
output, the cursor will be positioned following the generated output. Comparing
the new cursor location to the old can thus reveal the amount of output generated
in the event of a failure.

To check that there was no failure, then, verify that the return value is not
negative. The entirety of the input was written if any of the following is
true:

\begin{denseitemize}
\item{The return value is equal to the number of columns required to represent the input.
    \texttt{mbswidth()}(Listing ~\ref{list:mbswidth}) can be used to find the number of columns required by
    the input.}
\item{The number of bytes written is equal to the number of bytes supplied. Since Notcurses
   supports only ASCII, \texttt{strlen()} can be used to find the number of
   bytes supplied.}
\item{The cursor is not beyond the end of the plane, and scrolling is disabled
 on the plane.}
\end{denseitemize}

\begin{listing}[!htbp]
\begin{minted}{C}
// Calculate the size in columns of the provided UTF8 multibyte string.
int mbswidth(const char* mbs);
\end{minted}
\caption{\texttt{mbswidth()} counts columns in a string.}
\label{list:mbswidth}
\end{listing}

\subsection{Writing text to planes}
\label{sec:outputtext}

Multiple families of functions are available for writing to planes. Only valid
encoded sequences from the active locale's encoding can be output. Attempting
to e.g. write UTF-8 characters while using the \texttt{ANSI\_X3.4-1968} encoding
will fail as soon as a non-ASCII character is submitted.

The base form of each family places its output at the plane's current cursor
location. Each family has a \texttt{\_xy()}-suffixed form which moves the
cursor as specified prior to beginning output. Supplying -1 to the \texttt{x}
and \texttt{y} parameters of these forms doesn't move the cursor on the
relevant axis. Supplying -1 to both decays to the base function of the family.
The \texttt{\_stainable()}-suffixed form updates the glyphs of a plane without
changing the attributes or channels.

\begin{listing}[!htbp]
\begin{minted}{C}
// Replace the cell at the specified coordinates with the provided cell 'c',
// and advance the cursor by the width of the cell (but not past the end of the
// plane). On success, returns the number of columns the cursor was advanced.
// On failure, -1 is returned.
int ncplane_putc_yx(struct ncplane* n, int y, int x, const cell* c);

// Call ncplane_putc_yx() for the current cursor location.
static inline int
ncplane_putc(struct ncplane* n, const cell* c){
  return ncplane_putc_yx(n, -1, -1, c);
}
\end{minted}
\caption{Output of \texttt{cell}s to planes.}
\label{list:putc}
\end{listing}

ASCII (and thus the lowest 128 UTF-8 encoded characters) can be written directly
with the \texttt{putsimple} family. Note that any control character will be replaced with
a space.

\begin{listing}[!htbp]
\begin{minted}{C}
// Replace the EGC underneath us, but retain the styling. The current styling
// of the plane will not be changed.
//
// Replace the cell at the specified coordinates with the provided 7-bit char
// 'c'. Advance the cursor by 1. On success, returns 1. On failure, returns -1.
// This works whether the underlying char is signed or unsigned.
int ncplane_putsimple_yx(struct ncplane* n, int y, int x, char c);

// Call ncplane_putsimple_yx() at the current cursor location.
static inline int
ncplane_putsimple(struct ncplane* n, char c){
  return ncplane_putsimple_yx(n, -1, -1, c);
}

// Replace the EGC underneath us, but retain the styling. The current styling
// of the plane will not be changed.
int ncplane_putsimple_stainable(struct ncplane* n, char c);
\end{minted}
\caption{Direct output of single-byte UTF-8 to planes.}
\label{list:putc}
\end{listing}

EGCs can be output one at a time. Supplying multiple EGCs in a single buffer
to the \texttt{putegc} family will only ever see the first one output.

\begin{listing}[!htbp]
\begin{minted}{C}
// Replace the cell at the specified coordinates with the provided EGC, and
// advance the cursor by the width of the cluster (but not past the end of the
// plane). On success, returns the number of columns the cursor was advanced.
// On failure, -1 is returned. The number of bytes converted from gclust is
// written to 'sbytes' if non-NULL.
int ncplane_putegc_yx(struct ncplane* n, int y, int x, const char* gclust, int* sbytes);

// Call ncplane_putegc() at the current cursor location.
static inline int
ncplane_putegc(struct ncplane* n, const char* gclust, int* sbytes){
  return ncplane_putegc_yx(n, -1, -1, gclust, sbytes);
}

// Replace the EGC underneath us, but retain the styling. The current styling
// of the plane will not be changed.
int ncplane_putegc_stainable(struct ncplane* n, const char* gclust, int* sbytes);
\end{minted}
\caption{Output of single EGCs to planes.}
\label{list:putc}
\end{listing}

\begin{listing}[!htbp]
\begin{minted}{C}
#define WCHAR_MAX_UTF8BYTES 6

// ncplane_putegc(), but following a conversion from wchar_t to UTF-8 multibyte.
static inline int
ncplane_putwegc(struct ncplane* n, const wchar_t* gclust, int* sbytes){
  // maximum of six UTF8-encoded bytes per wchar_t
  const size_t mbytes = (wcslen(gclust) * WCHAR_MAX_UTF8BYTES) + 1;
  char* mbstr = (char*)malloc(mbytes); // need cast for c++ callers
  if(mbstr == NULL){
    return -1;
  }
  size_t s = wcstombs(mbstr, gclust, mbytes);
  if(s == (size_t)-1){
    free(mbstr);
    return -1;
  }
  int ret = ncplane_putegc(n, mbstr, sbytes);
  free(mbstr);
  return ret;
}

// Call ncplane_putwegc() after successfully moving to y, x.
static inline int
ncplane_putwegc_yx(struct ncplane* n, int y, int x, const wchar_t* gclust,
                   int* sbytes){
  if(ncplane_cursor_move_yx(n, y, x)){
    return -1;
  }
  return ncplane_putwegc(n, gclust, sbytes);
}

// Replace the EGC underneath us, but retain the styling. The current styling
// of the plane will not be changed.
API int ncplane_putwegc_stainable(struct ncplane* n, const wchar_t* gclust, int* sbytes);
\end{minted}
\caption{Output of single \texttt{wchar\_t}s to planes.}
\label{list:putc}
\end{listing}

\begin{listing}[!htbp]
\begin{minted}{C}
// Write a series of EGCs to the current location, using the current style.
// They will be interpreted as a series of columns (according to the definition
// of ncplane_putc()). Advances the cursor by some positive number of cells
// (though not beyond the end of the plane); this number is returned on success.
// On error, a non-positive number is returned, indicating the number of cells
// which were written before the error.
API int ncplane_putstr_yx(struct ncplane* n, int y, int x, const char* gclustarr);

static inline int
ncplane_putstr(struct ncplane* n, const char* gclustarr){
  return ncplane_putstr_yx(n, -1, -1, gclustarr);
}

API int ncplane_putstr_aligned(struct ncplane* n, int y, ncalign_e align,
                               const char* s);
\end{minted}
\caption{Output of strings to planes.}
\label{list:putc}
\end{listing}

\begin{listing}[!htbp]
\begin{minted}{C}
// ncplane_putstr(), but following a conversion from wchar_t to UTF-8 multibyte.
static inline int
ncplane_putwstr_yx(struct ncplane* n, int y, int x, const wchar_t* gclustarr){
  // maximum of six UTF8-encoded bytes per wchar_t
  const size_t mbytes = (wcslen(gclustarr) * WCHAR_MAX_UTF8BYTES) + 1;
  char* mbstr = (char*)malloc(mbytes); // need cast for c++ callers
  if(mbstr == NULL){
    return -1;
  }
  size_t s = wcstombs(mbstr, gclustarr, mbytes);
  if(s == (size_t)-1){
    free(mbstr);
    return -1;
  }
  int ret = ncplane_putstr_yx(n, y, x, mbstr);
  free(mbstr);
  return ret;
}

static inline int
ncplane_putwstr_aligned(struct ncplane* n, int y, ncalign_e align,
                        const wchar_t* gclustarr){
  int width = wcswidth(gclustarr, INT_MAX);
  int xpos = ncplane_align(n, align, width);
  return ncplane_putwstr_yx(n, y, xpos, gclustarr);
}

static inline int
ncplane_putwstr(struct ncplane* n, const wchar_t* gclustarr){
  return ncplane_putwstr_yx(n, -1, -1, gclustarr);
}
\end{minted}
\caption{Output of wide strings to planes.}
\label{list:putc}
\end{listing}

Finally, analogues of \texttt{printf(3)} and \texttt{vprintf(3)} are provided.

\begin{listing}[!htbp]
\begin{minted}{C}
// The ncplane equivalents of printf(3) and vprintf(3).
API int ncplane_vprintf_aligned(struct ncplane* n, int y, ncalign_e align,
                                const char* format, va_list ap);

API int ncplane_vprintf_yx(struct ncplane* n, int y, int x,
                           const char* format, va_list ap);

static inline int
ncplane_vprintf(struct ncplane* n, const char* format, va_list ap){
  return ncplane_vprintf_yx(n, -1, -1, format, ap);
}

static inline int
ncplane_printf(struct ncplane* n, const char* format, ...)
  __attribute__ ((format (printf, 2, 3)));

static inline int
ncplane_printf(struct ncplane* n, const char* format, ...){
  va_list va;
  va_start(va, format);
  int ret = ncplane_vprintf(n, format, va);
  va_end(va);
  return ret;
}

static inline int
ncplane_printf_yx(struct ncplane* n, int y, int x, const char* format, ...)
  __attribute__ ((format (printf, 4, 5)));

static inline int
ncplane_printf_yx(struct ncplane* n, int y, int x, const char* format, ...){
  va_list va;
  va_start(va, format);
  int ret = ncplane_vprintf_yx(n, y, x, format, va);
  va_end(va);
  return ret;
}

static inline int
ncplane_printf_aligned(struct ncplane* n, int y, ncalign_e align,
                       const char* format, ...)
  __attribute__ ((format (printf, 4, 5)));

static inline int
ncplane_printf_aligned(struct ncplane* n, int y, ncalign_e align, const char* format, ...){
  va_list va;
  va_start(va, format);
  int ret = ncplane_vprintf_aligned(n, y, align, format, va);
  va_end(va);
  return ret;
}
\end{minted}
\caption{Formatted output to planes.}
\label{list:putc}
\end{listing}

\subsection{The 32-bit \texttt{attribute} value}
\label{sec:attribute}
\subsection{The 64-bit \texttt{channels} value}
\label{sec:channels}

\cleardoublepage

%%%%%%%%%%%%%%%%%%%%%%%%%%%%%%%%%%%%%%%%%%%%%%%%%%%%%%%%%%%%%%%%%%%%%%%%
\section{Lines, boxes, and fills}
\subsection{Linear interpolation (``lerping'') and lines}
\epigraph{On the plain behind him are the wanderers in search of bones and those who do not search and they move haltingly in the light like mechanisms whose movements are monitored with escapement and pallet so that they appear restrained by a prudence or reflectiveness which has no inner reality and they cross in their progress one by one that track of holes that runs to the rim of the visible ground and which seems less the pursuit of some continuance than the verification of a principle, a validation of sequence and causality as if each round and perfect hole owed its existence to the one before it there on that prairie upon which are the bones and the gatherers of bones and those who do not gather.}{Cormac McCarthy, \textit{Blood Meridian}}
\label{sec:lerps}
Following actual text, the most frequent need of a TUI is probably vertical
and horizontal lines. Notcurses allows such lines to be drawn using arbitrary
EGCs, though you'll usually want one of the Unicode Box Drawing characters
(see Figure~\ref{fig:linedrawing}) or the Block Elements
(Figure~\ref{fig:blockelements}).

\begin{figure}[!htb]
    \centering
    \includegraphics[width=.75\linewidth]{media/boxdrawing.png}
    \caption{Unicode box-drawing characters \textit{(source: Chininazu12, public domain)}.}
    \label{fig:linedrawing}
\end{figure}

Using \texttt{ncplane\_hline()} or \texttt{ncplane\_vline()}, a single cell is
provided, and this cell (including its attributes and channels) is reproduced
on each cell of the line\footnote{If you provide a multicolumn EGC\textellipsis
I have no idea what happens. Maybe it works? I should look into that, huh.}.
Two additional forms are provided: \texttt{ncplane\_hline\_interp()} and
\texttt{ncplane\_vline\_interp()} (see Listing~\ref{list:lines}). Both of these
accept two channel pairs, and perform a linear interpolation between the two
foreground and two background channels. Providing the equivalent values for
either channel will result in that channel remaining constant along the line's
length.

\begin{listing}[!htb]
\begin{minted}{C}
// Draw horizontal or vertical lines using the specified cell, starting at the current cursor position. The
// cursor will end at the cell following the last cell output (even, perhaps counter-intuitively, when
// drawing vertical lines), just as if ncplane_putc() was called at that spot. Return the number of cells
// drawn on success. On error, return the negative number of cells drawn.
int ncplane_hline_interp(struct ncplane* n, const cell* c, int len, uint64_t c1, uint64_t c2);

static inline int ncplane_hline(struct ncplane* n, const cell* c, int len){
  return ncplane_hline_interp(n, c, len, c->channels, c->channels);
}

int ncplane_vline_interp(struct ncplane* n, const cell* c, int len, uint64_t c1, uint64_t c2);

static inline int ncplane_vline(struct ncplane* n, const cell* c, int len){
  return ncplane_vline_interp(n, c, len, c->channels, c->channels);
}
\end{minted}
\caption{Functions for drawing lines.}
\label{list:lines}
\end{listing}

\subsection{Boxes}
\label{sec:boxes}
Rectangles are regularly required as borders and for grouping. Notcurses supports
flexible box drawing. Boxes have their upper-left corner at the current cursor
position, unless drawn with \texttt{ncplane\_perimeter()}, which draws along the
edges of the plane (and has its upper-left corner at the plane origin). Box-drawing
functions accept six \texttt{cell} objects, one for each corner, one for horizontal
lines, and one for vertical lines. It is possible to apply linear interpolation
between the corners, in which case the colors of the horizontal and vertical
line-drawing cells will be ignored. It is possible to draw none, all, or any set
of the four corners and none, all, or any set of the four sides. These configurable
behaviors are specified via the \texttt{ctlword} bitmask parameter.
\texttt{ctlword} is defined in the least significant byte, where bits 4--7 are
a gradient mask, and 0--3 are a border mask (see Table~\ref{table:boxes}).

\begin{table}[!htb]
  \centering
  \begin{tabular}{|l|l|l|}
    \hline
    Constant & Bit & Property \\
    \hline
    \hline
    \texttt{NCBOXMASK\_TOP} & 0x001 & Inhibit top \\
    \hline
    \texttt{NCBOXMASK\_RIGHT} & 0x002 & Inhibit right \\
    \hline
    \texttt{NCBOXMASK\_BOTTOM} & 0x004 & Inhibit bottom \\
    \hline
    \texttt{NCBOXMASK\_LEFT} & 0x008 & Inhibit left \\
    \hline
    \texttt{NCBOXGRAD\_TOP} & 0x010 & Left side linear interpolation \\
    \hline
    \texttt{NCBOXGRAD\_RIGHT} & 0x020 & Bottom side linear interpolation \\
    \hline
    \texttt{NCBOXGRAD\_BOTTOM} & 0x040 & Right side linear interpolation \\
    \hline
    \texttt{NCBOXGRAD\_LEFT} & 0x080 & Top side linear interpolation \\
    \hline
    x & 0x100 & Require 1 connecting edge to draw corner \\
    \hline
    x & 0x200 & Require 2 connecting edges to draw corner \\
    \hline
    x & 0x300 & Draw no corners \\
    \hline
  \end{tabular}
  \caption{\texttt{ctlword} parameter for box-drawing.}
  \label{table:boxes}
\end{table}

By default, vertexes are drawn whether their connecting edges are drawn or
not. The value of the bits corresponding to \texttt{NCBOXCORNER\_MASK} (0x300)
control this, and are interpreted as the number of connecting edges necessary to draw a
given corner. At 0 (the default), corners are always drawn. At 3, corners
are never drawn (as at most 2 edges can touch a box's corner).

\begin{listing}[!htb]
\begin{minted}{C}
int ncplane_box(struct ncplane* n, const cell* ul, const cell* ur, const cell* ll, const cell* lr,
                const cell* hline, const cell* vline, int ystop, int xstop, unsigned ctlword);

// Draw a box with its upper-left corner at the current cursor position, having dimensions 'ylen'x'xlen'.
// See ncplane_box() for more information. The minimum box size is 2x2, and it cannot be drawn off-screen.
static inline int
ncplane_box_sized(struct ncplane* n, const cell* ul, const cell* ur, const cell* ll,
                  const cell* lr, const cell* hline, const cell* vline, int ylen, int xlen, unsigned ctlword){
  int y, x;
  ncplane_cursor_yx(n, &y, &x);
  return ncplane_box(n, ul, ur, ll, lr, hline, vline, y + ylen - 1, x + xlen - 1, ctlword);
}

static inline int
ncplane_perimeter(struct ncplane* n, const cell* ul, const cell* ur, const cell* ll,
                  const cell* lr, const cell* hline, const cell* vline, unsigned ctlword){
  if(ncplane_cursor_move_yx(n, 0, 0)){
    return -1;
  }
  int dimy, dimx;
  ncplane_dim_yx(n, &dimy, &dimx);
  return ncplane_box_sized(n, ul, ur, ll, lr, hline, vline, dimy, dimx, ctlword);
}
\end{minted}
\caption{Functions for drawing rectilinear boxes.}
\label{list:boxes}
\end{listing}

It can be tedious to set up the six \texttt{cell} parameters to these
functions. Since boxes are typically drawn with one of a small number of sets
of EGCs, helper functions are provided for each set. I usually go with the
pleasantly rounded ``Light Arc'' Box Drawing codes
(Listing~\ref{list:roundboxes}). Or, should you prefer, there are
the strong, sure Double Box Drawing characters
(Listing~\ref{list:doubleboxes}).

\begin{listing}[!htb]
\begin{minted}{C}
static inline int cells_rounded_box(struct ncplane* n, uint32_t attr, uint64_t channels,
                                    cell* ul, cell* ur, cell* ll, cell* lr, cell* hl, cell* vl){
  return cells_load_box(n, attr, channels, ul, ur, ll, lr, hl, vl, "╭╮╰╯─│");
}

static inline int ncplane_rounded_box(struct ncplane* n, uint32_t attr, uint64_t channels,
                                      int ystop, int xstop, unsigned ctlword){
  int ret = 0;
  cell ul = CELL_TRIVIAL_INITIALIZER, ur = CELL_TRIVIAL_INITIALIZER;
  cell ll = CELL_TRIVIAL_INITIALIZER, lr = CELL_TRIVIAL_INITIALIZER;
  cell hl = CELL_TRIVIAL_INITIALIZER, vl = CELL_TRIVIAL_INITIALIZER;
  if((ret = cells_rounded_box(n, attr, channels, &ul, &ur, &ll, &lr, &hl, &vl)) == 0){
    ret = ncplane_box(n, &ul, &ur, &ll, &lr, &hl, &vl, ystop, xstop, ctlword);
  }
  cell_release(n, &ul); cell_release(n, &ur);
  cell_release(n, &ll); cell_release(n, &lr);
  cell_release(n, &hl); cell_release(n, &vl);
  return ret;
}

static inline int
ncplane_rounded_box_sized(struct ncplane* n, uint32_t attr, uint64_t channels, int ylen, int xlen, unsigned ctlword){
  int y, x;
  ncplane_cursor_yx(n, &y, &x);
  return ncplane_rounded_box(n, attr, channels, y + ylen - 1, x + xlen - 1, ctlword);
}
\end{minted}
\caption{Helpers for rounded-corner boxes.}
\label{list:roundboxes}
\end{listing}

\begin{listing}[!htb]
\begin{minted}{C}
static inline int cells_double_box(struct ncplane* n, uint32_t attr, uint64_t channels,
                                   cell* ul, cell* ur, cell* ll, cell* lr, cell* hl, cell* vl){
  return cells_load_box(n, attr, channels, ul, ur, ll, lr, hl, vl, "╔╗╚╝═║");
}

static inline int ncplane_double_box(struct ncplane* n, uint32_t attr, uint64_t channels,
                                                  int ystop, int xstop, unsigned ctlword){
  int ret = 0;
  cell ul = CELL_TRIVIAL_INITIALIZER, ur = CELL_TRIVIAL_INITIALIZER;
  cell ll = CELL_TRIVIAL_INITIALIZER, lr = CELL_TRIVIAL_INITIALIZER;
  cell hl = CELL_TRIVIAL_INITIALIZER, vl = CELL_TRIVIAL_INITIALIZER;
  if((ret = cells_double_box(n, attr, channels, &ul, &ur, &ll, &lr, &hl, &vl)) == 0){
    ret = ncplane_box(n, &ul, &ur, &ll, &lr, &hl, &vl, ystop, xstop, ctlword);
  }
  cell_release(n, &ul); cell_release(n, &ur);
  cell_release(n, &ll); cell_release(n, &lr);
  cell_release(n, &hl); cell_release(n, &vl);
  return ret;
}

static inline int ncplane_double_box_sized(struct ncplane* n, uint32_t attr, uint64_t channels,
                                           int ylen, int xlen, unsigned ctlword){
  int y, x;
  ncplane_cursor_yx(n, &y, &x);
  return ncplane_double_box(n, attr, channels, y + ylen - 1, x + xlen - 1, ctlword);
}
\end{minted}
\caption{Helpers for doubly-thicc boxes.}
\label{list:doubleboxes}
\end{listing}

\subsection{Gradients and polyfills}
\texttt{ncplane\_polyfill()} should be applied to a coordinate with no glyph
(Listing~\ref{list:polyfills}). That coordinate will be filled with the provided
\texttt{cell}. The function then effectively recurses on all cardinally
connected coordinates, thus filling a bounded region with the provided cell.
This operation is akin to the ``flood
fill'' of pixel graphics. Sometimes you'll want to destroy all content in the
plane, reinitializing its framebuffer and base cell without changing the
geometry. \texttt{ncplane\_erase()} allows you to do this in one fell swoop,
with the added bonus functionality of resetting the associated egcpool. A
freshly-reset egcpool can be much faster than one which has been heavily used,
requiring a search to find suitable free space. The state of the framebuffer is
exactly as it was when the plane was created---all cells hold the null EGC, all
attributes are 0, and all colors are defaults. All cells associated with this
plane are invalidated, so be sure you're not holding onto any. \texttt{ncplane\_polyfill\_yx()}
and \texttt{ncplane\_erase()} are detailed in Listing~\ref{list:polyfills}.

\begin{listing}[!htb]
\begin{minted}{C}
// Starting at the specified coordinate, if it has no glyph, 'c' is copied into it. We do the same to
// all cardinally-connected glyphless cells, filling in everything behind a boundary. Returns the
// number of cells polyfilled. An invalid initial y, x is an error.
int ncplane_polyfill_yx(struct ncplane* n, int y, int x, const cell* c);

void ncplane_erase(struct ncplane* n);
\end{minted}
\caption{Polyfills and plane erasure.}
\label{list:polyfills}
\end{listing}

A single EGC and attribute can be written to a rectangular region in any of
a single color, a vertical, horizontal, or diagonal gradient, or a 4-cornered
``inverted radial'' gradient (see Listing~\ref{list:gradients}). The gradient
operation is independently applied to both the fore- and background of 4 64-bit
channel parameters. Palette-indexed color is not yet supported for gradients.

\begin{listing}[!htb]
\begin{minted}{C}
// Draw a gradient with its upper-left corner at the current cursor position, stopping at 'ystop'x'xstop'.
// The glyph composed of 'egc' and 'attrword' is used for all cells. The channels specified by 'ul', 'ur',
// 'll', and 'lr' are composed into foreground and background gradients. To do a vertical gradient, 'ul'
// ought equal 'ur' and 'll' ought equal 'lr'. To do a horizontal gradient, 'ul' ought equal 'll' and 'ur'
// ought equal 'ul'. To color everything the same, all four channels should be equivalent. The resulting
// alpha values are equal to incoming alpha values.
//
// Preconditions for gradient operations (error otherwise):
//
//  all: only RGB colors, unless all four channels match as default
//  all: all alpha values must be the same
//  1x1: all four colors must be the same
//  1xN: both top and both bottom colors must be the same (vertical gradient)
//  Nx1: both left and both right colors must be the same (horizontal gradient)
int ncplane_gradient(struct ncplane* n, const char* egc, uint32_t attrword, uint64_t ul, uint64_t ur,
                     uint64_t ll, uint64_t lr, int ystop, int xstop);

// Do a high-resolution gradient using upper blocks and synced backgrounds. This doubles the number of
// vertical gradations, but restricts you to half blocks (appearing to be full blocks).
int ncplane_highgradient(struct ncplane* n, uint32_t ul, uint32_t ur,
                         uint32_t ll, uint32_t lr, int ystop, int xstop);

// Draw a gradient with its upper-left corner at the current cursor position, having dimensions
// 'ylen'x'xlen'. See ncplane_gradient for more information.
static inline int ncplane_gradient_sized(struct ncplane* n, const char* egc, uint32_t attrword, uint64_t ul,
                                         uint64_t ur, uint64_t ll, uint64_t lr, int ylen, int xlen){
  if(ylen < 1 || xlen < 1){
    return -1;
  }
  int y, x;
  ncplane_cursor_yx(n, &y, &x);
  return ncplane_gradient(n, egc, attrword, ul, ur, ll, lr, y + ylen - 1, x + xlen - 1);
}

static inline int ncplane_highgradient_sized(struct ncplane* n, uint64_t ul, uint64_t ur,
                                             uint64_t ll, uint64_t lr, int ylen, int xlen){
  if(ylen < 1 || xlen < 1){
    return -1;
  }
  int y, x;
  ncplane_cursor_yx(n, &y, &x);
  return ncplane_highgradient(n, ul, ur, ll, lr, y + ylen - 1, x + xlen - 1);
}
\end{minted}
\caption{Drawing gradients.}
\label{list:gradients}
\end{listing}

\subsection{Blitting}
\label{sec:blitting}
\begin{listing}[!htb]
\begin{minted}{C}
// Blit a flat array 'data' of BGRx 32-bit values to the ncplane 'nc', offset from the upper left by 'placey' and 'placex'.
// Each row ought occupy 'linesize' bytes (this might be greater than lenx * 4 due to padding). A subregion of the input
// can be specified with 'begy'x'begx' and 'leny'x'lenx'.
int ncblit_bgrx(struct ncplane* nc, int placey, int placex, int linesize, const unsigned char* data,
                int begy, int begx, int leny, int lenx);

// Blit a flat array 'data' of RGBA 32-bit values to the ncplane 'nc', offset from the upper left by 'placey' and 'placex'.
// Each row ought occupy 'linesize' bytes (this might be greater than lenx * 4 due to padding). A subregion of the input can
// be specified with 'begy'x'begx' and 'leny'x'lenx'.
int ncblit_rgba(struct ncplane* nc, int placey, int placex, int linesize, const unsigned char* data,
                int begy, int begx, int leny, int lenx);
\end{minted}
\caption{Blitting BGRx and RGBA.}
\label{list:blitting}
\end{listing}

Sometimes, you've got a chunk of RGBA or BGRx in memory, and just want to blast
it onto a plane as quickly as possible. Blitting functions (see Listing~\ref{list:blitting})
exist to transform such pixels into Unicode Block Elements. Every two input rows
become a single row, using half-blocks when necessary. Columns are mapped 1-to-1.
This functionality was used, for instance, to set up Notcurses as a rendering
backend for NEStopia and RetroArch, and these functions form the heart of the
multimedia functionality described in Chapter~\ref{sec:libav}. Note that these
functions do not offer any scaling capabilities.

\subsection{Staining}
\label{sec:staining}
Consult Chapter~\ref{sec:outputtext} for the ``stainable'' family of text output
functions, allowing EGCs to be replaced without affecting attributes or channels.
To modify attributes or channels in isolation but \textit{en masse}, Notcurses
provides \texttt{ncplane\_format()} and \texttt{ncplane\_stain()}.

\begin{listing}[!htb]
\begin{minted}{C}
// Set the given style throughout the specified region, keepying content and channels otherwise unchanged.
int ncplane_format(struct ncplane* n, int ystop, int xstop, uint32_t attrword);

// Set the given channels throughout the specified region, keepying content and attributes otherwise unchanged.
int ncplane_stain(struct ncplane* n, int ystop, int xstop, uint64_t ul, uint64_t ur, uint64_t ll, uint64_t lr);
\end{minted}
\caption{Changing attributes or channels in isolation.}
\label{list:stain}
\end{listing}

Staining could be accomplished without a special function call by reflecting
on each cell of interest using \texttt{ncplane\_at\_yx()}, changing the cell,
and writing it back, but \texttt{ncplane\_stain()} and \texttt{ncplane\_format()}
are faster and simpler. Creating a new plane of all null glyphs, setting the
channels as desired, and placing it atop the region would also accomplish a
staining, and can further be used as a ``moving'' stain. This wouldn't affect
the lower plane; it would purely be an effect of rendering. It is not possible
to simulate \texttt{ncplane\_format()} in this way, however---glyph and
attribute are a single dimension and cannot be independently transparent.

\cleardoublepage

%%%%%%%%%%%%%%%%%%%%%%%%%%%%%%%%%%%%%%%%%%%%%%%%%%%%%%%%%%%%%%%%%%%%%%%%
\section{Multimedia (images and videos)}
\label{sec:libav}

Media decoding and scaling is handled by libAV from FFmpeg, resulting in a
\texttt{ncvisual} object. This object generates frames, each one
corresponding to a renderable scene on the associated plane. If Notcurses
is built without FFmpeg support, these functions will all return error.
The flow of multimedia from the view of a Notcurses application is:
\begin{denseitemize}
\item{The media is opened with \texttt{ncplane\_visual\_open()} or
    \texttt{ncvisual\_open\_plane}. The latter creates a new plane suitable
    for rendering the media; the former will draw to a preexisting plane.
    Stream format and the codecs in use are determined during this step, but
    not necessarily the geometry of the visual data.}
\item{A frame is decoded. Several frames might have been decoded from the
    underlying stream, but the application sees them one at a time. This frame
    carries geometry information along with a flat matrix of pixels, along with
    some manner of timing information\footnote{This timing information can come
    in three different forms: ``frame number'' (which can be multiplied by
    some known frames per second to get a time offset), ``offset`` in some
    unit of time, and ``time to display'' for this frame in some unit of
    time. This last sadly cannot yield an O(1) time offset within the
    stream.}. \textbf{begin loop}}
\item{(Optional) Any subtitles associated with the frame are retrieved.}
\item{The frame may be rendered. A delay might be taken. \textbf{end loop}}
\end{denseitemize}

\begin{listing}[!htb]
\begin{minted}{C}
// Open a visual (image or video), associating it with the specified ncplane.
// Returns NULL on any error, writing the AVError to 'averr'.
struct ncvisual* ncplane_visual_open(struct ncplane* nc, const char* file, int* averr);

// Destroy an ncvisual. Rendered elements will not be disrupted, but the visual
// can be neither decoded nor rendered any further.
void ncvisual_destroy(struct ncvisual* ncv);

// Return the plane to which this ncvisual is bound.
struct ncplane* ncvisual_plane(struct ncvisual* ncv);
\end{minted}
\caption{Opening and destroying multimedia with \texttt{ncvisual}.}
\label{list:ncvisual}
\end{listing}

Any scaling is applied during the decoding step. It is thus sadly not possible
to redraw decoded media at a different size. This isn't usually a problem in
practice, since streaming media will provide a new frame (at the correct size)
shortly, and single-frame media can simply be decoded afresh.

\begin{listing}[!htb]
\begin{minted}{C}
// extract the next frame from an ncvisual. returns NULL on end of file,
// writing AVERROR_EOF to 'averr'. returns NULL on a decoding or allocation
// error, placing the AVError in 'averr'. this frame is invalidated by a
// subsequent call to ncvisual_decode(), and should not be freed by the caller.
struct AVFrame* ncvisual_decode(struct ncvisual* nc, int* averr);

// Render the decoded frame to the associated ncplane. The frame will be scaled
// to the size of the ncplane per the ncscale_e style. A subregion of the
// frame can be specified using 'begx', 'begy', 'lenx', and 'leny'. To render
// the rectangle formed by begy x begx and the lower-right corner, zero can be
// supplied to 'leny' and 'lenx'. Zero for all four values will thus render the
// entire visual. Negative values for any of the four parameters are an error.
// It is an error to specify any region beyond the boundaries of the frame.
int ncvisual_render(const struct ncvisual* ncv, int begy, int begx, int leny, int lenx);
\end{minted}
\caption{Decoding and rendering multimedia with \texttt{ncvisual}.}
\label{list:multimedia}
\end{listing}

Subtitles (considered ``metadata'' in FFmpeg) aren't advertised in any way to
the caller. If subtitles are to be displayed, you can simply call \texttt{ncvisual\_subtitle()}
at each frame. If a non-\texttt{NULL} string is returned, it is valid UTF-8
subtitle text. The subtitle will \textit{not} typically be repeated for all
frames where it ought be displayed, so it's best left persistent. Unfortunately,
there's no good way to know when it ought be struck from the display. Alas!

\begin{listing}[!htb]
\begin{minted}{C}
// If a subtitle ought be displayed at this time, returns a heap-allocated copy
// of the UTF8 text. Otherwise returns NULL.
char* ncvisual_subtitle(const struct ncvisual* ncv);
\end{minted}
\caption{Acquiring subtitles.}
\label{list:subtitles}
\end{listing}

\subsection{Streaming video/animated GIFs.}
\begin{listing}[!htb]
\begin{minted}{C}
// Called for each frame rendered from 'ncv'. If anything but 0 is returned, the streaming operation
// ceases immediately, and that value is propagated out.
typedef int (*streamcb)(struct notcurses* nc, struct ncvisual* ncv, void*);

// Shut up and display my frames! Provide as an argument to ncvisual_stream(). If you'd like subtitles to
// be decoded, provide an ncplane as the curry. If the curry is NULL, subtitles will not be displayed.
static inline int ncvisual_simple_streamer(struct notcurses* nc, struct ncvisual* ncv, void* curry){
  if(notcurses_render(nc)){
    return -1;
  }
  int ret = 0;
  if(curry){
    // need a cast for C++ callers
    struct ncplane* subncp = (struct ncplane*)curry;
    char* subtitle = ncvisual_subtitle(ncv);
    if(subtitle){
      if(ncplane_putstr_yx(subncp, 0, 0, subtitle) < 0){
        ret = -1;
      }
      free(subtitle);
    }
  }
  return ret;
}
\end{minted}
\caption{\texttt{streamcb} callback type and \texttt{ncvisual\_simple\_streamer()}.}
\label{list:streamingcb}
\end{listing}

Running the loop discussed above is good for the soul, but usually it's
sufficient to let Notcurses handle things, perhaps with a callback to your
program. The function \texttt{ncvisual\_stream()} (Listing~\ref{list:streaming})
does just that, honoring the timing hints embedded in the stream as best it can.
Use of this function is strongly recommended. For the simplest use, \texttt{ncvisual\_simple\_streamer()}
(Listing~\ref{list:streamingcb}) can be provided as the callback. It will print
subtitles in the upper left, and otherwise render frames as it receives them.
If you require more complex per-frame activity, provide your own callback of
type \texttt{streamcb}, which can carry a \texttt{void*} curry.

\begin{listing}[!htb]
\begin{minted}{C}
// Stream the entirety of the media, according to its own timing. Blocking, obviously. streamer may be NULL;
// it is otherwise called for each frame, and its return value handled as outlined for stream cb. If
// streamer() returns non-zero, the stream is aborted, and that value is returned. By convention, return a
// positive number to indicate intentional abort from within streamer(). 'timescale' allows the frame
// duration time to be scaled. For a visual naturally running at 30FPS, a 'timescale' of 0.1 will result in
// 300FPS, and a 'timescale' of 10 will result in 3FPS. It is an error to supply 'timescale' less than or
// equal to 0.
int ncvisual_stream(struct notcurses* nc, struct ncvisual* ncv, int* averr,
                    float timescale, streamcb streamer, void* curry);
\end{minted}
\caption{Media streaming and \texttt{ncvisual\_simple\_streamer()}.}
\label{list:streaming}
\end{listing}

\subsection{Scaling images and video}
\begin{listing}[!htb]
\begin{minted}{C}
// How to scale the visual in ncvisual_from_file(). NCSCALE_NONE will open a plane tailored to the visual's
// exact needs, which is probably larger than the visible screen (but might be smaller). NCSCALE_SCALE
// scales a visual larger than the visible screen down, maintaining aspect ratio. NCSCALE_STRETCH stretches
// and scales the image in an attempt to fill the visible screen.
typedef enum { NCSCALE_NONE, NCSCALE_SCALE, NCSCALE_STRETCH, } ncscale_e;

// Open a visual, extract a codec and parameters, and create a new plane suitable for its display at 'y','x'.
// If there is sufficient room to display the visual in its native size, or if NCSCALE_NONE is passed for
// 'style', the new plane will be exactly that large. Otherwise, the plane will be as large as possible (given
// the visible screen), either maintaining aspect ratio (NCSCALE_SCALE) or abandoning it (NCSCALE_STRETCH).
struct ncvisual* ncvisual_from_file(struct notcurses* nc, const char* file, int* averr, int y, int x, ncscale_e style);
\end{minted}
\caption{Scaling media onto a new plane.}
\label{list:scaling}
\end{listing}
Notcurses relies on FFmpeg for all scaling, and thus must tell it the size of
the rendering area. An \texttt{ncvisual} is always scaled according to the
geometry of its associated plane's entirety. The scaling target is twice the
rendering target's height in rows, and equal to the rendering target's width in
columns. If the media is smaller than this, it will be scaled up. If larger, it
will be scaled down. A lossless representation thus requires rendering to a
plane having exactly half the media's height, and its exact width. Since it's
not generally possible to know the media's size until it's been partially
decoded, Notcurses provides \texttt{ncvisual\_open\_plane()}, which creates
a new plane based on the media's dimensions (this plane can be retrieved from
the returned \texttt{ncvisual} using \texttt{ncvisual\_plane()}). For an
exactly-matched plane, supply \texttt{NCSCALE\_NONE} (this plane might of course
be bigger or smaller than the viewing area). Use \texttt{NCSCALE\_STRETCH} to
scale the image to the rendering area (just like \texttt{ncplane\_visual\_open()},
except with a new plane). \texttt{NCSCALE\_SCALE} is not yet implemented, but
will retain media aspect ratio whilst scaling to fit at least one dimension
of the rendering area.

If the plane on which your \texttt{ncvisual} is to be rendered is larger than
the rendering area, you can save time by rendering only a portion of the
decoded image. \texttt{ncvisual\_render()} accepts four arguments specifying a
rectangular subsection via origin and dimensions. The last two arguments can be
specified as -1 to render through the end of the decoded frame. This is used
in the \texttt{eagle} demo to progressively ``zoom'' in on a level map much
larger than a typical terminal; in general, this will be the most efficient means
of effecting parallax scrolling.

\subsection{Sprites}
Danny Hillis is probably best known to computer architects for Thinking Machines,
but to the great video game-playing masses, it was his coinage of the term
``sprite'' which will be remembered (or not)\cite{tms34010}. First used in
conjunction with the hallowed TMS34010 of Texas Instruments, sprites are simply
bitmaps composed into a larger scene; like the French \textit{esprit} or Celtic
\textit{spriggan}\footnote{Both derived from the Latin \textit{spiritus}.}, they
float ethereally above the ground (plane). Notcurses supports full-sized sprites
via the combination of transparency and independently moved planes, often
populated via an external image file (texture).

The basic recipe for a sprite is:
\begin{denseitemize}
\item{Create a new plane above whatever background is in use.}
\item{Set the base character of this plane transparent.}
\item{Draw your object, possibly via \texttt{ncvisual\_render()}.}
\end{denseitemize}
The new plane can be moved freely, and the base transparency will allow the
underlying background to show through wherever you haven't drawn. The plane can
be as large as is convenient, so long as it's transparent everywhere the
sprite isn't present. To test whether the sprite has been ``hit'', the plane
itself can be used as a bounding box, but much better accuracy can be had by
testing the plane for character presence with \texttt{ncplane\_at\_yx()}.

If your sprite has an animation cycle---common for e.g. walking figures---it
can often be easiest to render each frame of the cycle to a distinct plane,
and keep only one visible at a time. This is used in the \texttt{luigi} demo
for Luigi's running cycle.

\cleardoublepage

%%%%%%%%%%%%%%%%%%%%%%%%%%%%%%%%%%%%%%%%%%%%%%%%%%%%%%%%%%%%%%%%%%%%%%%%
\section{Collecting and dispatching input}
\label{sec:input}

To enter arbitrary Unicode using the keyboard, try pressing Ctrl+Shift+u. If
successful, you ought see an underlined or otherwise stylized lowercase 'u'.
The Unicode code point can be entered (each number will show up as you type it),
followed by Enter. The sequence will then reduce to an EGC.

The most fundamental call is \texttt{notcurses\_getc()}. This can operate as a
non-blocking, timed, or blocking call. Provide a \texttt{NULL}\texttt{struct timespec} '\texttt{ts}'.
\texttt{notcurses\_getc()} to block until input is received, or the call is
interrupted by a signal (prepare the \texttt{sigset\_t} parameter to mask signals
as necessary). Provide the desired timeout in '\texttt{ts}' for a timed call,
or zero out '\texttt{ts}' for a pure nonblocking call. On timeout, 0 is returned.
On an error, -1 is returned. Otherwise, a \texttt{char32\_t} is returned carrying
a single UTF-32 Unicode codepoint. If the \texttt{ncinput} parameter '\texttt{ni}'
is not \texttt{NULL}, it will be filled in with the codepoint, any applicable
keyboard modifiers, and the cell of the input\footnote{Coordinates are currently
reported only for pointing devices, not keyboards.}. Two helpers exist to simplify
standard use cases: \texttt{notcurses\_getc\_nblock()} and \texttt{notcurses\_getc\_blocking()}
do exactly what you'd expect.

\begin{listing}[!htb]
\begin{minted}{C}
// See ppoll(2) for more detail. Provide a NULL 'ts' to block at length, a 'ts' of 0 for non-blocking operation, and otherwise
// a timespec to bound blocking. Signals in sigmask (less several we handle internally) will be atomically masked and unmasked
// per ppoll(2). It should generally contain all signals. Returns a single Unicode code point, or (char32_t)-1 on error.
// 'sigmask' may be NULL. Returns 0 on a timeout. If an event is processed, the return value
// is the 'id' field from that event. 'ni' may be NULL.
char32_t notcurses_getc(struct notcurses* n, const struct timespec* ts, sigset_t* sigmask, ncinput* ni);

// 'ni' may be NULL if the caller is uninterested in event details. If no event is ready, returns 0.
static inline char32_t notcurses_getc_nblock(struct notcurses* n, ncinput* ni){
  sigset_t sigmask;
  sigfillset(&sigmask);
  struct timespec ts = { .tv_sec = 0, .tv_nsec = 0 };
  return notcurses_getc(n, &ts, &sigmask, ni);
}

// 'ni' may be NULL if the caller is uninterested in event details. Blocks until an event is processed or a signal is received.
static inline char32_t notcurses_getc_blocking(struct notcurses* n, ncinput* ni){
  sigset_t sigmask;
  sigemptyset(&sigmask);
  return notcurses_getc(n, NULL, &sigmask, ni);
}
\end{minted}
\caption{Input can be acquired in nonblocking, blocking, or timed fashion.}
\label{listing:input}
\end{listing}

Mouse events are reported only after a successful call to
\texttt{notcurses\_mouse\_enable()}, and will no longer be reported following
\texttt{notcurses\_mouse\_disable()}. Even when enabled, events are only
returned while a button is held. There will be one event for the initial button
press, one event for each cell into which the mouse moves while holding down
the button, and one when the button is released. See Listing~\ref{listing:mice}.

\begin{listing}[!htb]
\begin{minted}{C}
// Enable the mouse in "button-event tracking" mode with focus detection and UTF8-style extended coordinates. On failure,
// -1 is returned. On success, 0 is returned, and mouse events will be published to notcurses_getc().
int notcurses_mouse_enable(struct notcurses* n);

// Disable mouse events. Any events in the input queue can still be delivered.
int notcurses_mouse_disable(struct notcurses* n);
\end{minted}
\caption{Mouse events must be explicitly enabled, and can be disabled.}
\label{listing:mice}
\end{listing}

Input functions may be called concurrently with any output or read-only
functions, but only one thread at a time may call into the input layer via any
of its entry points.

\cleardoublepage

%%%%%%%%%%%%%%%%%%%%%%%%%%%%%%%%%%%%%%%%%%%%%%%%%%%%%%%%%%%%%%%%%%%%%%%%
\section{UI widgets}
Widgets provide ready-made tools for acquiring user input or displaying data.
As of Notcurses 1.2.4, there are four widgets: selector, multiselector, menu,
and reel. Multiple widgets can be in use at one time, and they are drawn onto
planes like any other output. Widgets can thus be moved up and down the z-axis.
It is not recommended to scribble on widgets, but nothing prevents you from
doing so. See Chapter~\ref{sec:input} for information about routing input to
widgets.

\label{sec:uiwidgets}
\subsection{Selectors and multiselectors}

The selector widget is an ncplane with a body section and optional title riser.
The body section is populated with options and descriptions, and supports
infinite scrolling up and down. The widget is automatically sized according to
the largest input provided. The keyboard and mouse wheel can scroll through
selections, and clicking on the arrows also scrolls. Selection and cancellation
are implemented by the caller. The currently-selected option can be retrieved
at any time. Option/description pairs can be added or removed while the
widget is active, even if the removed pair is currently selected. Removing the
last pair does not destroy the widget, and it is possible to create the widget
with no pairs.

\begin{figure}[!htb]
  \centering \includegraphics[width=.75\linewidth]{media/selector5.png}
  \caption{Naked selector.}
\end{figure}

\begin{figure}[!htb]
    \centering
    \includegraphics[width=.75\linewidth]{media/selector1.png}
    \caption{Selector with a long title.}
\end{figure}

\begin{listing}[!htb]
\begin{minted}{C}
struct selector_item {
  char* option;
  char* desc;
};

typedef struct selector_options {
  char* title; // title may be NULL, inhibiting riser, saving two rows.
  char* secondary; // secondary may be NULL
  char* footer; // footer may be NULL
  struct selector_item* items; // initial items and descriptions
  unsigned itemcount; // number of initial items and descriptions
  // default item (selected at start), must be < itemcount unless 'itemcount'
  // is 0, in which case 'defidx' must also be 0
  unsigned defidx;
  // maximum number of options to display at once, 0 to use all available space
  unsigned maxdisplay;
  // exhaustive styling options
  uint64_t opchannels;   // option channels
  uint64_t descchannels; // description channels
  uint64_t titlechannels;// title channels
  uint64_t footchannels; // secondary and footer channels
  uint64_t boxchannels;  // border channels
  uint64_t bgchannels;   // background channels, used only in body
} selector_options;

struct ncselector* ncselector_create(struct ncplane* n, int y, int x, const selector_options* opts);
\end{minted}
\caption{Selector creation.}
\end{listing}

\begin{figure}[!htb]
    \centering
    \includegraphics[width=.75\linewidth]{media/selector2.png}
    \caption{Short title intersecting with header.}
\end{figure}

\begin{figure}[!htb]
  \centering \includegraphics[width=.75\linewidth]{media/selector3.png}
    \caption{Selector with a long header.}
\end{figure}

\begin{figure}[!htb]
\centering \includegraphics[width=.75\linewidth]{media/selector4.png}
\caption{Selector with a long footer and no header.}
\end{figure}

\begin{listing}[!htb]
\begin{minted}{C}
int ncselector_additem(struct ncselector* n, const struct selector_item* item);
int ncselector_delitem(struct ncselector* n, const char* item);

// Return a reference to the selected option, or NULL if there are no items.
const char* ncselector_selected(const struct ncselector* n);

// Return a reference to the ncselector's underlying ncplane.
struct ncplane* ncselector_plane(struct ncselector* n);

// Move up or down in the list. A reference to the newly-selected item is
// returned, or NULL if there are no items in the list.
const char* ncselector_previtem(struct ncselector* n);
const char* ncselector_nextitem(struct ncselector* n);

// Offer the input to the ncselector. If it's relevant, this function returns
// true, and the input ought not be processed further. If it's irrelevant to
// the selector, false is returned. Relevant inputs include:
//  * a mouse click on an item
//  * a mouse scrollwheel event
//  * a mouse click on the scrolling arrows
//  * a mouse click outside of an unrolled menu (the menu is rolled up)
//  * up, down, pgup, or pgdown on an unrolled menu (navigates among items)
bool ncselector_offer_input(struct ncselector* n, const struct ncinput* nc);

// Destroy the ncselector. If 'item' is not NULL, the last selected option will
// be strdup()ed and assigned to '*item' (and must be free()d by the caller).
void ncselector_destroy(struct ncselector* n, char** item);
\end{minted}
\caption{Selector control.}
\end{listing}

\begin{figure}
    \centering
    \includegraphics[width=1\linewidth]{media/multiselector.png}
    \caption{Multiselector.}
\end{figure}

\subsection{Menus}
\label{sec:menus}
\begin{figure}
    \centering
    \includegraphics[width=.75\linewidth]{media/menutop.png}
    \caption{Menu along the top of the standard plane.}
\end{figure}
Horizontal menu bars are supported on the top and bottom rows of planes. If a
menu bar is longer than the bound plane, it will be only partially visible, but
any unrolled section will be visible. Menus may be either visible or invisible
by default. Set the `hiding` option to get an invisible menu. In the event of a
screen resize, menus will be automatically moved/resized.

Listing~\ref{list:menucreate} covers creation of menus, and Listing~\ref{list:menucontrol}
covers their control.

\begin{listing}[!htb]
\begin{minted}{C}
typedef struct menu_options {
  bool bottom;              // on the bottom row, as opposed to top row
  bool hiding;              // hide the menu when not being used
  struct {
    char* name;             // utf-8 c string
    struct {
      char* desc;           // utf-8 menu item, NULL for horizontal separator
      ncinput shortcut;     // shortcut, all should be distinct
    }* items;
    int itemcount;
  }* sections;              // array of menu sections
  int sectioncount;         // must be positive
  uint64_t headerchannels;  // styling for header
  uint64_t sectionchannels; // styling for sections
} menu_options;

struct ncmenu;

// Create a menu with the specified options. Menus are currently bound to an overall notcurses object
// (as opposed to a particular plane), and are implemented as ncplanes kept atop other ncplanes.
struct ncmenu* ncmenu_create(struct notcurses* nc, const menu_options* opts);
\end{minted}
\caption{Menu creation.}
\label{list:menucreate}
\end{listing}

\begin{figure}
    \centering
    \includegraphics[width=.75\linewidth]{media/menubottom.png}
    \caption{Menu along the bottom of the standard plane.}
\end{figure}

\begin{figure}
    \centering
    \includegraphics[width=.75\linewidth]{media/menuwarmech.png}
    \caption[WarMECH and a translucent menu.]{The \texttt{notcurses-demo} menu, unrolled \textit{in media res}. Luigi, pursued
      by WarMECH, is leaping through the ``Help'' menu. In the upper left is the HUD,
      and at the bottom the About text, both implemented as translucent planes.}
\end{figure}

\begin{listing}[!htb]
\begin{minted}{C}
// Unroll the specified menu section, making the menu visible if it was invisible, and rolling
// up any menu section that is already unrolled.
int ncmenu_unroll(struct ncmenu* n, int sectionidx);

// Roll up any unrolled menu section, and hide the menu if using hiding.
int ncmenu_rollup(struct ncmenu* n);

// Return the selected item description, or NULL if no section is unrolled. If 'ni' is not NULL,
// and the selected item has a shortcut, 'ni' will be filled in with that shortcut. This can allow faster matching.
const char* ncmenu_selected(const struct ncmenu* n, struct ncinput* ni);

// Return the ncplane backing this ncmenu.
struct ncplane* ncmenu_plane(struct ncmenu* n);

// Offer the input to the ncmenu. If it's relevant, this function returns true, and the input ought not be
// processed further. If it's irrelevant to the menu, false is returned. Relevant inputs include:
//  * mouse movement over a hidden menu
//  * a mouse click on a menu section (the section is unrolled)
//  * a mouse click outside of an unrolled menu (the menu is rolled up)
//  * left or right on an unrolled menu (navigates among sections)
//  * up or down on an unrolled menu (navigates among items)
//  * escape on an unrolled menu (the menu is rolled up)
bool ncmenu_offer_input(struct ncmenu* n, const struct ncinput* nc);

// Destroy a menu created with ncmenu_create().
int ncmenu_destroy(struct ncmenu* n);
\end{minted}
\caption{Menu control.}
\label{list:menucontrol}
\end{listing}

\subsection{Reels}
The ncreel\footnote{The term ``reel'' is borrowed from slot machines.} is a UI abstraction supported by Notcurses in which
dynamically-created and -destroyed toplevel entities (referred to as tablets)
are arranged on a ``cylinder'', allowing for infinite scrolling
(infinite scrolling can be disabled, resulting in a rectangle rather than a
cylinder). This works naturally with keyboard navigation, mouse scrolling wheels,
and touchpads (including the capacitive touchscreens of modern cell phones).
An ncreel initially has no tablets; at any given time thereafter, it has zero
or more tablets, and if there is at least one tablet, one tablet is focused
(and on-screen). If the last tablet is removed, no tablet is focused. A tablet
can support navigation within the tablet, in which case there is an in-tablet
focus for the focused tablet, which can also move among elements within the
tablet. Reels have a \texttt{struct ncreel\_options} object, passed to
\texttt{ncreel\_create()}; this struct is detailed in Listing~\ref{listing:reelopts}.

\begin{listing}[!htb]
\begin{minted}{C}
typedef struct ncreel_options {
  // require this many rows and columns (including borders). otherwise, a message will be displayed stating
  // that a larger terminal is necessary, and input will be queued. if 0, no minimum will be enforced. may
  // not be negative. note that ncreel_create() does not return error if given a plane smaller than these
  // minima; it instead patiently waits for the screen to get bigger.
  int min_supported_cols;
  int min_supported_rows;

  // use no more than this many rows and columns (including borders). may not be less than the
  // corresponding minimum. 0 means no maximum.
  int max_supported_cols;
  int max_supported_rows;

  // desired offsets within the surrounding WINDOW (top right bottom left) upon creation / resize. an
  // ncreel_move() operation updates these.
  int toff, roff, boff, loff;
  // is scrolling infinite (can one move down or up forever, or is an end reached?). if true, 'circular'
  // specifies how to handle the special case of an incompletely-filled reel.
  bool infinitescroll;
  // is navigation circular (does moving down from the last tablet move to the first, and vice versa)?
  // only meaningful when infinitescroll is true. if infinitescroll is false, this must be false.
  bool circular;
  // notcurses can draw a border around the ncreel, and also around the component tablets. inhibit
  // borders by setting all valid bits in the masks. partially inhibit borders by setting individual
  // bits in the masks. the appropriate attr and pair values will be used to style the borders. focused
  // and non-focused tablets can have different styles. you can instead draw your own borders, or
  // forgo borders entirely.
  unsigned bordermask; // bitfield; 1s will not be drawn (see bordermaskbits)
  uint64_t borderchan; // attributes used for ncreel border
  unsigned tabletmask; // bitfield; same as bordermask but for tablet borders
  uint64_t tabletchan; // tablet border styling channel
  uint64_t focusedchan;// focused tablet border styling channel
  uint64_t bgchannel;  // background colors
} ncreel_options;

struct ncreel;

// Create an ncreel according to the provided specifications. Returns NULL on failure. 'nc' must be a
// valid plane, to which offsets are relative. Note that there might not be enough room for the
// specified offsets, in which case the ncreel will be clipped on the bottom and right. A minimum number
// of rows and columns can be enforced via popts. efd, if non-negative, is an eventfd/pipe that ought be
// written to whenever ncreel_touch() updates a tablet (this is useful in the case of nonblocking input).
struct ncreel* ncreel_create(struct ncplane* nc, const ncreel_options* popts, int efd);
\end{minted}
\caption{Reel creation.}
\label{listing:reelopts}
\end{listing}

\begin{figure}
  \centering
  \includegraphics[width=.75\linewidth]{media/growlight1.png}
  \caption{\texttt{growlight}, a program built around reels.}
  \label{fig:growlight1}
\end{figure}

The ncreel object tracks the size, number, information depth, and order of
tablets, and the foci. It also draws the optional borders around tablets and
the optional border of the reel itself. It knows nothing about the actual
content of a tablet, save the number of lines it occupies at each information
depth. The typical control flow is that an application receives events (from
the UI or other event sources), and calls \texttt{ncreel\_touch()} on tablets
needing updates\footnote{This is one of the rare functions which can be called
concurrently with \texttt{notcurses\_render()}.}. Eventually, the application
calls \texttt{ncreel\_redraw()} to update the reel in its
entirety\footnote{A call to \texttt{notcurses\_render()} is still required to
update the display.}. Notcurses will call into the application for some number
of tablets, asking it to draw some line(s) from some tablet(s) at some
particular coordinate of that tablet's panel. Finally, control returns to the
application, and the cycle starts anew. The typedef for these callbacks is
defined in Listing~\ref{list:tabletcb}.

\begin{listing}[!htb]
\begin{minted}{C}
// Tablet draw callback, provided a tablet (from which the ncplane and userptr may be extracted),
// the first column that may be used, the first row that may be used, the first column that may not
// be used, the first row that may not be used, and a bool indicating whether output ought be
// clipped at the top (true) or bottom (false). Rows and columns are zero-indexed, and both are
// relative to the tablet's plane.
//
// Regarding clipping: it is possible that the tablet is only partially displayed on the screen. If
// so, it is either partially present on the top of the screen, or partially present at the bottom.
// In the former case, the top is clipped (cliptop will be true), and output ought start from the
// end. In the latter case, cliptop is false, and output ought start from the beginning.
//
// Returns the number of lines of output, which ought be less than or equal to
// maxy - begy, and non-negative (negative values might be used in the future).
typedef int (*tabletcb)(struct nctablet* t, int begx, int begy, int maxx, int maxy, bool cliptop);
\end{minted}
\caption{Tablet redraw callback function type.}
\label{list:tabletcb}
\end{listing}

Each tablet might be wholly, partially, or not on-screen. Notcurses always
places as much of the focused tablet as is possible on-screen (if the focused
tablet has more lines than the actual reel does, it cannot be wholly on-screen.
In this case, the focused subelements of the tablet are always on-screen). The
placement of the focused tablet depends on how it was reached (when moving to
the next tablet, offscreen tablets are brought onscreen at the bottom. When
moving to the previous tablet, offscreen tablets are brought onscreen at the
top. When moving to an arbitrary tablet which is neither the next nor previous
tablet, it will be placed in the center). Further ncreel functionality is detailed
in Listing~\ref{listing:reelcontrol}.

\begin{listing}[!htb]
\begin{minted}{C}
// Returns the ncplane on which this ncreel lives.
struct ncplane* ncreel_plane(struct ncreel* pr);

// Add a new tablet to the provided ncreel, having the callback object opaque. Neither, either, or both of
// after and before may be specified. If neither is specified, the new tablet can be added anywhere on the
// reel. If one or the other is specified, the tablet will be added before or after the specified tablet.
// If both are specified, the tablet will be added to the resulting location, assuming it is valid
// (after->next == before->prev); if it is not valid, or there is any other error, NULL will be returned.
struct nctablet* ncreel_add(struct ncreel* pr, struct nctablet* after, struct nctablet* before, tabletcb cb, void* opaque);

// Return the number of tablets.
int ncreel_tabletcount(const struct ncreel* pr);

// Indicate that the specified tablet has been updated in a way that would change its display.
// This will trigger some non-negative number of callbacks (though not in the caller's context).
int ncreel_touch(struct ncreel* pr, struct nctablet* t);

// Delete the tablet specified by t from the ncreel specified by pr. Returns -1 if the tablet cannot be found.
int ncreel_del(struct ncreel* pr, struct nctablet* t);

// Delete the active tablet. Returns -1 if there are no tablets.
int ncreel_del_focused(struct ncreel* pr);

// Move to the specified location within the containing plane.
int ncreel_move(struct ncreel* pr, int x, int y);

// Redraw the ncreel in its entirety, for instance after
// clearing the screen due to external corruption, or a SIGWINCH.
int ncreel_redraw(struct ncreel* pr);

// Return the focused tablet, if any tablets are present. This is not a copy;
// be careful to use it only for the duration of a critical section.
struct nctablet* ncreel_focused(struct ncreel* pr);

// Change focus to the next tablet, if one exists
struct nctablet* ncreel_next(struct ncreel* pr);

// Change focus to the previous tablet, if one exists
struct nctablet* ncreel_prev(struct ncreel* pr);

// Destroy an ncreel allocated with ncreel_create(). Does not destroy the
// underlying plane. Returns non-zero on failure.
int ncreel_destroy(struct ncreel* pr);

// Returns a pointer to a user pointer associated with this nctablet.
void* nctablet_userptr(struct nctablet* t);

// Access the ncplane associated with this tablet, if one exists.
struct ncplane* nctablet_ncplane(struct nctablet* t);
\end{minted}
\caption{Reel control.}
\label{listing:reelcontrol}
\end{listing}

Several widgets will probably be added, possibly before Notcurses 2.0:
\begin{denseitemize}
\item{A libreadline wrapper, or equivalent functionality.}
\item{Histogram and line plot capabilities using block elements and Braille.}
\item{A HUD for direct mode.}
\end{denseitemize}

\subsection{Example: let's rip off \texttt{whiptail}}
The \texttt{colloquy} program shipped with Notcurses implements a command-line
API similar to that of the Newt program \texttt{whiptail}, which itself ripped
off the NCURSES program \texttt{dialog}. All of these programs allow simple
user interfaces to be thrown up on the command line. \texttt{colloquy} is
written in Rust\footnote{\url{https://lib.rs/crates/colloquy}}, using the
Notcurses crate\footnote{\url{https://lib.rs/crates/notcurses}}, itself a wrapper
of the libnotcurses-sys\footnote{\url{https://lib.rs/crates/libnotcurses-sys}}
\texttt{bindgen}-generated Rust wrappers.

\section{Complex examples}
\subsection{Example: walking through \texttt{notcurses-demo}}
\label{sec:ncdemo}
The \texttt{notcurses-demo} program is built as part of Notcurses, and ought
have been installed alongside the library (on Debian, you'll need the
\texttt{notcurses-bin} package, and even then the demo has been somewhat
reduced in order to comply with the DFSG\cite{dfsg}). It demonstrates a wide
range of Notcurses capabilities, and its source code is most instructive.

It is best to run the demo in a terminal having geometry of at least 80x45,
though anything 80x24 or larger will more or less work (some content will be
clipped). It is also desirable to have 24-bit color enabled, assuming your
terminal supports it. Determine the number of colors advertised by your
terminal type using~\texttt{infocmp} (see Figure~\ref{fig:terminfocmp}).
Some relevant terminfo capabilities are described in Table~\ref{table:terminfo}.

\begin{table}[h]
  \begin{center}
    \begin{tabular}{ |c|c|c| }
      \hline
      \texttt{colors} & Integer & Number of colors. \\
      \hline
      \texttt{ccc} & Boolean & The palette can be programmed. \\
      \hline
      \texttt{RGB} & Boolean & Direct RGB values can be specified. \\
      \hline
    \end{tabular}
  \end{center}
  \caption{Relevant terminfo properties.}
  \label{table:terminfo}
\end{table}

Each demo makes use of a few different Notcurses capabilities. In addition,
a menu is present throughout. From this menu (or using keyboard shortcuts),
you can activate a HUD (H) and an informational help display (Ctrl+u). In
addition, you can restart the demo with Ctrl+R, or quit at any time (q). This
application serves admirably for benchmarking certain terminal behaviors, and
we'll do exactly that in Appendix~\ref{sec:termshade}. The performance
properties of various components are described at length therein.

\begin{figure}[h]
  \centering
  \includegraphics[width=.75\linewidth]{media/terminfocmp.png}
  \caption{Inspecting the terminfo database.}
  \label{fig:terminfocmp}
\end{figure}

Screenshots were taken using \texttt{scrot} 1.2 and a 80x45
\texttt{xfce4-terminal} 0.8.9.1 from Xfce 4.14+Compiz 0.8.16.1 atop Xorg
1.20.7 on NVIDIA 440.59. All of these are the unmodified Debian Unstable
x86\_64 binaries. My kernel is a custom 5.5.6 build. The terminal type is
\texttt{vte-256color}, and \texttt{COLORTERM} is defined to be
\texttt{24bit}. The terminal font was Hack 10, and the background is a 0.7
transparency.

\cleardoublepage

\begin{figure}
  \centering
  \begin{minipage}{0.45\textwidth}
    \includegraphics[width=1\linewidth]{media/demo-intro.png}
    \caption[``Intro''.]{``Intro''. Lerps on the perimeters. Inverse radial
            gradient plus vertical gradient. Full-screen fade.
            Cyclic glyphs. Italics.}
  \end{minipage}\hfill
  \begin{minipage}{0.45\textwidth}
    \includegraphics[width=1\linewidth]{media/demo-xray.png}
    \caption[``X-Ray''. Very large planes.]{``X-Ray''. Streaming video.
       Very large planes (the scrolling plane at the bottom is much larger than the visible screen).}
  \end{minipage}
\end{figure}

\begin{figure}
  \centering
  \begin{minipage}{0.45\textwidth}
    \includegraphics[width=1\linewidth]{media/demoeagle2.png}
    \caption{``Eagle'', first phase.\\
      Parallax scrolling on large image.}
  \end{minipage}\hfill
  \begin{minipage}{0.45\textwidth}
    \includegraphics[width=1\linewidth]{media/demoeagle1.png}
    \caption{``Eagle'', second phase.\\
      Sprites. Zoomed image.}
  \end{minipage}
\end{figure}

\begin{figure}
  \centering
  \begin{minipage}{0.30\textwidth}
    \includegraphics[width=1\linewidth]{media/demo-trans1.png}
    \caption[``Trans'', early phase.]{``Trans''. Transparent top plane. Window through to the desktop.}
  \end{minipage}\hfill
  \begin{minipage}{0.30\textwidth}
    \includegraphics[width=1\linewidth]{media/demo-trans2.png}
    \caption[``Trans'', middle phase.]{``Trans''. Opaque foreground, transparent background, no glyph.}
  \end{minipage}\hfill
  \begin{minipage}{0.30\textwidth}
    \includegraphics[width=1\linewidth]{media/demo-trans3.png}
    \caption[``Trans'', late phase.]{``Trans''. Transparent foreground and background with opaque glyph.}
  \end{minipage}\hfill
\end{figure}

\begin{figure}
  \centering
  \begin{minipage}{0.45\textwidth}
    \includegraphics[width=1\linewidth]{media/demo-highcon.png}
    \caption{``Highcon''. High-contrast text.}
  \end{minipage}\hfill
  \begin{minipage}{0.45\textwidth}
    \includegraphics[width=1\linewidth]{media/demo-grid.png}
    \caption{``Grid''. Max RGB density.}
  \end{minipage}\hfill
\end{figure}

\begin{figure}
  \centering
  \begin{minipage}{0.45\textwidth}
    \includegraphics[width=1\linewidth]{media/demo-box.png}
    \caption{``Box''. Lerped perimeters. Precise Unicode. Color sweeps.}
  \end{minipage}\hfill
  \begin{minipage}{0.45\textwidth}
    \includegraphics[width=1\linewidth]{media/demo-sliders.png}
    \caption{``Sliders''. Partial fades. Animation. Gradients.}
  \end{minipage}\hfill
\end{figure}

\begin{figure}
  \centering
  \begin{minipage}{0.45\textwidth}
    \includegraphics[width=1\linewidth]{media/demo-reels.png}
    \caption{``Reels''. The \texttt{ncreel} widget.}
  \end{minipage}\hfill
  \begin{minipage}{0.45\textwidth}
    \includegraphics[width=1\linewidth]{media/demo-whiteout.png}
    \caption{``Whiteout''. Translucency.}
  \end{minipage}\hfill
\end{figure}

\begin{figure}
  \centering \includegraphics[width=.65\linewidth]{media/demo-chunli1.png}
  \caption{``Chunli''. Sprite animation.}
\end{figure}

\begin{figure}
  \centering \includegraphics[width=.65\linewidth]{media/demo-chunli2.png}
  \caption{``Chunli''. Sprite animation.}
\end{figure}

\begin{figure}
  \centering
  \begin{minipage}{0.45\textwidth}
    \includegraphics[width=1\linewidth]{media/demo-uniblock1.png}
    \caption{``Uniblock''. Hangul syllables.}
  \end{minipage}\hfill
  \begin{minipage}{0.45\textwidth}
    \includegraphics[width=1\linewidth]{media/demo-uniblock2.png}
    \caption{``Uniblock''. Emoji.}
  \end{minipage}\hfill
\end{figure}

\begin{figure}
  \centering
  \begin{minipage}{0.45\textwidth}
    \includegraphics[width=1\linewidth]{media/demo-img1.png}
    \caption{``View''. Scaling an image.}
  \end{minipage}\hfill
  \begin{minipage}{0.45\textwidth}
    \includegraphics[width=1\linewidth]{media/demo-img2.png}
    \caption{``View''. Transparent images.}
  \end{minipage}\hfill
\end{figure}

\begin{figure}
  \centering
  \begin{minipage}{0.45\textwidth}
    \includegraphics[width=1\linewidth]{media/demo-view1.png}
    \caption{``View''. Streaming video with high-contrast text.}
  \end{minipage}\hfill
  \begin{minipage}{0.45\textwidth}
    \includegraphics[width=1\linewidth]{media/demo-view2.png}
    \caption{``View''. Notice the high-contrast kicking in.}
  \end{minipage}\hfill
\end{figure}

\begin{figure}
  \centering \includegraphics[width=1\linewidth]{media/demojungle.png}
  \caption[``Jungle''. Palette-indexed image.]{``Jungle''. Palette-indexed image. Very low-bandwidth animation via palette cycling.\\
    ``Ruins in Rain'' © Mark Ferrari/Living Worlds. Texelized with permission.}
\end{figure}


\begin{figure}
  \centering
  \begin{minipage}{0.45\textwidth}
    \includegraphics[width=1\linewidth]{media/demo-fallin1.png}
    \caption[``Fallin\''', early phase.]{``Fallin\'''. Color change, introspection, many planes.}
  \end{minipage}\hfill
  \begin{minipage}{0.45\textwidth}
    \includegraphics[width=1\linewidth]{media/demo-fallin2.png}
    \caption[``Fallin\''', late phase.]{``Fallin\'''. The underlying image is revealed.}
  \end{minipage}\hfill
\end{figure}

\begin{figure}
  \centering
  \begin{minipage}{0.45\textwidth}
    \includegraphics[width=1\linewidth]{media/demo-luigi.png}
    \caption{``Luigi''. Multiple sprites.}
  \end{minipage}\hfill
  \begin{minipage}{0.45\textwidth}
    \includegraphics[width=1\linewidth]{media/demo-outro.png}
    \caption{``Outro''. Fades atop video.}
  \end{minipage}\hfill
\end{figure}

\pagebreak
\cleardoublepage

\subsection{Example: let's rip off tetris}
\label{sec:casestudy}
Recall our exploration of tetriminos from Chapter~\ref{sec:simpleloop}. We
know enough now to turn this into an actual game of terminal Tetris\footnote{There
are a great many distinct Tetris implementations. We'll aim for loose
conformance to the NES version as published by Nintendo of America, because
I'm old. See \url{https://tetris.wiki/Tetris_(NES,_Nintendo)}. I'm not going
to obsess over details, though.}. Our implementation is fewer than 200 lines
total, yet it unites most of the techniques introduced in the past few
chapters. For fun, we'll do this in \CC\footnote{LOL how often does one hear
that said?}, using Marek Habersack's \CC wrappers (installed with Notcurses).

Our \texttt{main()} will parse command line options, set up a \texttt{Tetris}
object appropriately, and watch for input. The \texttt{Tetris} object provides
a thread function \texttt{Ticker} which moves the current piece down according
to timer events (calling \texttt{StuckPiece} to determine if the piece has been
placed, in which case a new piece enters the playing area), and the necessary
interface for the input loop:

\begin{denseitemize}
\item{\texttt{MoveLeft()} and \texttt{MoveRight()}
    (Listing~\ref{list:tetris-move}). These simply verify that the lateral
    move can be done, and then call \texttt{Plane::move()}.}
\item{\texttt{MoveDown()} (Listing~\ref{list:tetris-movedown}), which calls the same \texttt{StuckPiece()},}
\item{\texttt{Pause()}, and}
\item{\texttt{RotateCCw()} and \texttt{RotateCw()}, which verify that the rotation
      is possible, and then call \texttt{Plane::rotate\_cw()} or \texttt{Plane::rotate\_ccw()}.}
\end{denseitemize}

The general structure of our solution is thus:

\begin{denseitemize}
\item{A background is drawn onto the standard plane.}
\item{A new plane is created for the game area. Why make a new plane? Recall
    the beginning of Chapter~\ref{sec:planes}: \textit{we use a plane wherever we
    benefit from distinct state}. By keeping the game area on its own plane,
    we can trivially move it in response to terminal resizes, and likewise
    trivially test whether the current piece can move in a given direction.}
\item{Two tetrimino planes are created, one for the current piece, and one for
    the next piece. The current piece descends from the top of the game area.
    Upon reaching its final position, its plane is added to the game plane
    using \texttt{ncplane\_mergedown()}. The next piece is brought to the top
    of the game area, and the current piece is redrawn on its way to becoming
    the next piece.}
\item{We need no external state save the score and the order of pieces. The order
    of pieces is left up to the PRNG. We only need track the score and
    our four planes. The validity of a given movement can be checked entirely
    by reflection, using \texttt{ncplane\_at\_yx()}.}
\item{Our main thread loops on \texttt{Notcurses::getc()}, calling into the
    \texttt{Tetris} object. These calls will need to lock against the timer thread.}
\item{If \texttt{StuckPiece()} returns \texttt{true}, and the stuck piece
    is at the top of the playing area, the game is over.}
\end{denseitemize}

The playing area should not be visible while the game is paused, so whenever
the game is paused, we'll move the standard plane to the top of the z-axis. As
it is opaque and spans the visible area, this will hide the playing area. We'll
furthermore throw up a three-row plane centered on the screen; this plane will
contain a perimeter, and pulsing text reading ``Paused''. On an unpause event,
the attract plane is destroyed, and the standard plane moved back to the bottom
of the z-axis. Nothing else needs be touched, save inhibiting the operation
of \texttt{Ticker()}. This could be done any number of ways; we'll use a
condition variable plus a bit flag, checking it upon emerging from our timeout.

Our resize logic is pretty simple: on a resize event, after verifying that the
new geometry is large enough to play on (if not, we pause the game), we must
redraw the background, move the playing area to the new bottom center of the
visible area, and move the current piece to its same location relative to the
playing area. If the game is already paused when we resize, we ought
additionally recenter the attract plane\footnote{Imagine that we sized the playing area according to the
visible area. This would require resizing the game board upon a terminal
resize, a decidedly messier affair. It's \textit{doable}---\textgreek{ἢ τὰν ἢ ἐπὶ τᾶς}.}.

\pagebreak

\begin{listing}[!htb]
\inputminted[]{C}{code-tetris/gravity.h}
\inputminted[]{C}{code-tetris/stain.h}
\caption{Tetris helpers \texttt{Gravity()} and \texttt{StainBoard()}.}
\label{list:tetris-gravity}
\end{listing}

Let's do some boring groundwork first. We'll need the official constants for
``gravity'', the level-dependent rate at which pieces fall
(Listing~\ref{list:tetris-gravity}). We can implement this as a
\texttt{constexpr} function, not that it's likely to be of any real
advantage (especially since we only call \texttt{Gravity()} upon level changes).

\begin{listing}[!htb]
\inputminted[]{C}{code-tetris/clear.h}
\caption{\texttt{Tetris::LineClear()}.}
\label{list:tetris-lineclear}
\end{listing}

We'll need a function to tell us whether a line has been cleared
(Listing~\ref{list:tetris-lineclear}). We could track the state of the board
ourselves as a simple boolean matrix, but why bother when we can just ask the
source? Reflecting on a Notcurses plane is good practice of the
DRY\footnote{Don't Repeat Yourself.} principle. These aren't system calls,
just cheap indexed lookups into a framebuffer. It's unlikely that you can
do significantly better with your own implementation, so why bother? Let
Notcurses handle the state for you.

\begin{listing}[!htb]
\inputminted[]{C}{code-tetris/background.h}
\caption{Drawing the background and the gameplay plane.}
\label{list:tetris-background}
\end{listing}

Our background function (Listing~\ref{list:tetris-background}) just loads up
the provided image, stretched to fill the standard plane, blits it, and then
converts it to greyscale (so as not to be confused with actual playing pieces,
all of which are in color). The board is a double-lined box with the top
missing, and is its own plane. We also make a distinct plane for the score
and other textual info; this is useful as a last-ditch stopgap preventing
such text from spilling into the play area. Placing the board on its own
plane (as opposed to blitting it destructively onto the background) has two
major advantages: it simplifies responding to a terminal resize, and it allows
us to determine illegal moves via reflection on this plane.

\begin{listing}[!htb]
\inputminted[]{C}{code-tetris/stuck.h}
\caption{\texttt{Tetris::InvalidMove()}.}
\label{list:tetris-stuck}
\end{listing}

This brings us to \texttt{InvalidMove()} (Listing~\ref{list:tetris-stuck}).
All of our movement functions to come will need a means to test whether the
selected movement is legal, whether it's a lateral translation (\texttt{MoveLeft()}
and \texttt{MoveRight()}), a rotation (\texttt{RotateCw()} and \texttt{RotateCcw()}),
or falling towards the bottom (\texttt{MoveDown()}, called when the user presses
down and when the timer expires). Similarly to \texttt{LineClear()}, it's easiest
to leverage the Notcurses state. All five of these functions operate by speculatively
transforming the current piece, testing for overlap with the gameboard plane,
and pulling the piece back if there was an overlap.

\begin{listing}[!htb]
\inputminted[]{C}{code-tetris/movedown.h}
\caption{\texttt{Tetris::MoveDown()}.}
\label{list:tetris-movedown}
\end{listing}

As mentioned above, \texttt{MoveDown()} is called both by the timer thread,
and by the UI when the user initiates a drop. \texttt{MoveDown()} is special
among the movement functions: whereas the others undo an invalid move,
\texttt{MoveDown()} recognizes such as the current piece having bottomed out.
If the location is above the gameboard, the game is over. Otherwise, we
call \texttt{LockPiece()} to fuse the current piece with the stack, and bring
a new piece into play.

\begin{listing}[!htb]
\inputminted[]{C}{code-tetris/ticker.h}
\caption{\texttt{Tetris::Ticker()}.}
\label{list:tetris-ticker}
\end{listing}

The timer is managed in \texttt{Ticker()} (Listing~\ref{list:tetris-ticker}),
which is run inside its own thread.

\begin{listing}[!htb]
\inputminted[]{C}{code-tetris/lock.h}
\caption{\texttt{Tetris::LockPiece()}.}
\label{list:tetris-lock}
\end{listing}

When the current piece reaches its resting place, \texttt{LockPiece()} (Listing~\ref{list:tetris-lock})
is called. This merges the current piece down onto the playing area\footnote{The
``stack'', at least in Tetris parlance.}, scans for any cleared lines, removes
them, allows the material above these lines to fall, and finally repaints a
gradient onto the stack before rendering. The gradient slowly changes over
different levels---indeed, the gradient's computation limits us to 16 levels.

\begin{listing}[!htb]
\inputminted[]{C}{code-tetris/newpiece.h}
\caption{\texttt{Tetris::NewPiece()}.}
\label{list:tetris-newpiece}
\end{listing}

\texttt{NewPiece()} (Listing~\ref{list:tetris-newpiece}) is called at the
beginning of the game, and whenever a piece is locked in. We saw most of this
already in Chapter~\ref{sec:simpleloop}.

\begin{figure}
  \centering
  \begin{minipage}{0.45\textwidth}
    \includegraphics[width=1\linewidth]{media/tetris-prescore.png}
    \caption{Tetris---primed to score.}
  \end{minipage}\hfill
  \begin{minipage}{0.45\textwidth}
    \includegraphics[width=1\linewidth]{media/tetris-postscore.png}
    \caption{Tetris---boom!}
  \end{minipage}
\end{figure}

\begin{figure}
  \centering
  \begin{minipage}{0.45\textwidth}
    \inputminted[]{C}{code-tetris/moveleft.h}
  \end{minipage}\hfill
  \begin{minipage}{0.45\textwidth}
    \inputminted[]{C}{code-tetris/moveright.h}
  \end{minipage}
  \caption{\texttt{Tetris::MoveRight()} and \texttt{Tetris::MoveLeft()}.}
  \label{list:tetris-move}
\end{figure}

\begin{figure}
  \centering
  \begin{minipage}{0.45\textwidth}
    \inputminted[]{C}{code-tetris/rotate-cw.h}
  \end{minipage}\hfill
  \begin{minipage}{0.45\textwidth}
    \inputminted[]{C}{code-tetris/rotate-ccw.h}
  \end{minipage}
  \caption{\texttt{Tetris::RotateCcw()} and \texttt{Tetris::RotateCw()}.}
  \label{list:tetris-rotate}
\end{figure}

\begin{listing}[!htb]
\inputminted[]{C}{code-tetris/main.h}
\caption[]{Tetris \texttt{main()}.}
\label{list:tetris-main}
\end{listing}

Our \texttt{main()} is quite minimal. A \texttt{Tetris} instance is
constructed, initializting the game board and providing necessary state (score,
level, etc.). \texttt{Tetris::Ticker()} is launched as a {\CC}11 thread. We
then loop on input until either the game is over or the user presses 'q'.
Finally, we \texttt{join()} the \texttt{Ticker()} thread, and shut down
Notcurses. We accept function keys or vi keys for movement. 'z' and 'x' rotate
counterclockwise and clockwise respectively. Ctrl+L refreshes the screen to
round out our UI.

Note that, once again, a majority of our actual lines of code are devoted to
the detection and propagation of errors. Our \CC wrappers do not throw
exceptions themselves; I will likely add a further \CC wrapper which does in
the future. Should that come into play, all these error checks and manual
\texttt{throw} directives could be sweetly elided.

\pagebreak
\cleardoublepage
%%%%%%%%%%%%%%%%%%%%%%%%%%%%%%%%%%%%%%%%%%%%%%%%%%%%%%%%%%%%%%%%%%%%%%%%
\begin{appendices}

\section{A brief history of character graphics}
\label{sec:terminals}
The earliest terminals making use of glyphs\footnote{Konrad Zuse's Z3, generally
 considered the first programmable digital computer, communicated with its
operator through a matrix of blinkenlights and a not unsteampunkish keyboard that resembled the
Burroughs typewriters of its era\cite{zuse}.} printed them to paper, and are of
interest to us only so far as our modern term ``tty'' is rather dubiously
derived from ``TeleTYpewriter'', as these cantankerous contraptions were
known\footnote{Though we do hear of their Snoopy calendars in the songs of
legend\cite{quiche}.} (people had less experience abbreviating in those days).

These devices most typically printed 72 characters per line (CPL), a limit that
has persisted in strange places\cite{pandoc} through the modern era. Another constant
you'll see from time to time is 132 CPL, derived from line printers such as the
IBM 1403, the DEC LP11, and the Centronics 101\cite{ibm1403}. Most common,
however, is the 80 column line originating in 1928's 7¾x3¼x0.007in IBM
Computer Card (as designed by Clair D.\ Lake, deriving from the 1890 U.\ S.\
Census cards of Herman Hollerith\ldots themselves borrowing from Joseph
Jacquard's automation in 1804 of punched card loom control technology pioneered
by Basile Bouchon in 1725\cite{cards}). To this day, so long as your wacky
output device can do 80 columns, eh, that's good enough. In all these cases,
the limit arises from the number of characters that could be printed, using the
technology of the time, on their feeder paper (8.5in and 14in in the case of
printers).

On, then, to the ``Glass TTYs'' (ugh) and Visual Display Units of the 1970s.
Pictured in Figure~\ref{fig:terminals} are the Computer Terminal Corporation
Datapoint 3300, the Lear Sigler, Inc.\ ADM-3A, the Hazeltine 1500, and the
Soroc IQ-120. Lacking microcontrollers, and generally implementing no
independent control sequences, such devices are today often known as ``dumb
terminals'' (this term was originally a registered trademark of Lear Sigler,
see Figure~\ref{fig:adm3}). Already the 80x24 ``standard'' (it is not a standard) was emerging
(the DEC contemporaries listed were already pretty ``smart'', using proprietary
control codes):

\begin{table}[h]
\begin{center}
  \begin{tabular}{ |c|c|c| }
    \hline
    IBM 2260 Model 1 & 1965 & 40x6 \\
    \hline
    Datapoint 3300 & 1969 & 72x25 \\
    \hline
    DEC VT05 & 1970 & 72x24 \\
    \hline
    IBM 3277 Model 2 & 1971 & 80x24 \\
    \hline
    Textronix 4010 & 1972 & 74x35 \\
    \hline
    DECscope VT52 & 1974 & 80x24 \\
    \hline
    LSI ADM-3A & 1976 & 80x12, 80x24 \\
    \hline
    Hazeltine 1500 & 1977 & 80x24 \\
    \hline
    Sororc IQ-120 & 1977 & 80x24 \\
    \hline
  \end{tabular}
\caption{Some historical terminals and their resolutions.}
\end{center}
\end{table}

Why 80x24 (or 80x25, as you'll also see)\cite{infoworld80}? The 80 almost certainly arises from
the desire to display an entire punched card (this \textit{is} a standard---see
ANSI X3.21-1967/FIPS PUB 13, ``Rectangular Holes in Twelve-Row Punched
Cards'')\cite{sonicdelay}. The origin of 24 is less clear. 24 is highly
composite (it has more divisors than any smaller number), and it is the largest
integer divisible by all natural numbers not larger than its square root. There
are of course 24 hours in a day. 24 divides the scanline counts of both NTSC
and PAL at 480 and 576, respectively. 24 rows of 80 columns at a byte per
column utilize 93.75\% of a 2KiB memory, leaving exactly 128 bytes left over,
and everyone loves a good power of 2.

The Aaronites, Levite descendents of Moses's brother Aaron, the first
\texthebrew{כהן גדול}
(High Priest),
form the priestly \texthebrew{כֹּהֲנִים}; they were divided into 24 courses. The Buddha's Dharma Chakra (Wheel of Dhamma)
in its Ashoka form sends forth 24 spokes. But perhaps I grow esoteric, even
speculative\ldots in truth, 80x24 almost certainly owes its questionable
existence to IBM's punched cards, IBM 2260 and 3270 wanting compatibility with
IBM printers, the upstart DEC wanting compatibility with IBM software for their
VT52 and legendary VT100, and the VT100 subsequently becoming a \textit{de facto} standard for
four decades.

\begin{figure}
  \centering \includegraphics[width=.9\linewidth]{media/dumbterminals.jpg}
\caption[Dumb terminals of the 1970s.]{Clockwise starting from upper left: a dork and his LSI ADM (``American
  Dream Machine'', supposedly). That poor woman with the Sororc IQs looks stoned out
  of her gourd. Grizzly Adams rocks a Datapoint 3300, but really his mind is on
  seeing Skynyrd shred it this weekend. Finally, we have Frank the Cocaine
  Ranger and his Electric Hazeltine 1500 Band. \textit{Gott im Himmel}, the 70s were \textit{unseemly}.}
\label{fig:terminals}
\end{figure}

\begin{figure}
  \begin{minipage}{0.5\textwidth}
  \centering
    \includegraphics[width=.5\linewidth]{media/digital-terms.jpeg}
    \caption{Digital Equipment Corporation terminals of the 1970s and 1980s.}
  \end{minipage}\hfill
  \begin{minipage}{0.4\textwidth}
  \centering
    \includegraphics[width=.45\linewidth]{media/vt220-charset.png}
    \caption[VT220 glyph dump.]{The VT220's glyphs from a ROM dump. VT100 implemented most of the
      first seven columns. Note the existence of box-drawing characters\cite{crttypography}.}
  \end{minipage}\hfill
\end{figure}

\begin{figure}
  \centering
  \includegraphics[width=.8\linewidth]{media/adm3.jpeg}
  \caption{A strange time.}
  \label{fig:adm3}
\end{figure}

\subsection{The DEC VTxxx terminals and ANSI X3.64-1979}

Introduced in August 1978, the VT100 ushered in a new era of smart terminals
using commodity (Intel) microprocessors, implementing portions of the upcoming
ANSI X3.64 standard (itself based on 1976's first edition of ECMA-48) along
with DEC extensions\footnote{The VT100 \textit{did not} implement all of X3.64,
nor was X3.64 derived from the VT100. The VT100 didn't do color, nor did it
insert or delete lines. It furthermore implemented several features outside
the scope of ECMA-48's first edition.} This series would go on to sell over
six million units, and it was a rare vendor that didn't include some degree
of DEC VT compatibility. Each major iteration of the series was designed to
encompass all functionality of prior iterations, beginning with the VT100's
faithful emulation of the earlier era's VT52\cite{vt52}. The VT102 cut down on the cost
and size of the VT100, and included the 132-CPL mode by default; they were
otherwise essentially the same device\cite{vt100}.

Terminals had quite a heyday through the 1970s and 1980s, but as the price of
Wintel machines fell below \$1,000, their purpose rapidly eroded. They lived on,
especially in niche minicomputer and mainframe installations, but as of 2020 it's
difficult to find a new terminal. Digital sold their terminal business to SunRiver
Data Systems (now Boundless Technologies) in 1995; the \url{boundlessterminals.com}
site is not offered over HTTPS, and reads ``Now it is time for the last text
terminals to give way to newer technologies\cite{boundless}.'' Press F to pay
respects.

The original \texttt{xterm} (released in X10R3) was written as an emulator of
the VAXStation 100 (VS100), and slowly acquired scattered features from the
VT100, ANSI, and other sources\cite{xtermfaq}. Thomas E.\ Dickey (the current
maintainer of \Gls{ncurses} and xterm) began working on XTerm in the mid-90s,
and by 1996 had added the \texttt{decTerminalID} resource following the
addition of much VT220 compatibility. \texttt{xterm} can be built with support
for Sixel, ReGIS, Unicode, and an extraordinary number of archaic and/or
baroque mechanics with which I'm not personally familiar. Further information
is available in Mr.\ Dickey's
\href{https://invisible-island.net/xterm/xterm.faq.html}{XTerm FAQ}, which
makes for excellent reading\footnote{Mr. Dickey's XTerm and NCURSES FAQs are,
in my opinion, two of the finest pieces of technical documentation in
existence. It is my hope that this manuscript approaches their level.}.

I will not attempt to list the hundreds (probably thousands) of terminal
emulators that are available. Most claim some manner of ``ANSI'' or ``vt100''
(not the same thing!) compliance; take these claims with a large grain of salt.
It's worth knowing about:

\begin{denseitemize}
\item{\texttt{rxvt}, ``Rob's \texttt{xvt}''. An enhancement of \texttt{xvt} touted
    as a slimmed-down, simplified alternative to \texttt{xterm}. Implemented pixel-addressable
    graphics (in a scheme incompatible with Sixel). Like \texttt{xterm}, it uses
    X resources for configuration. Various forks add various essential technologies
    introduced since 2000.}
\item{Konsole, the official terminal of KDE. Having never used KDE substantially,
   I've never made much use of Konsole. I recall it having URL recognition
   when most terminal emulators didn't.}
\item{VTE terminals, a family of terminals built around GNOME's VTE library, built
  atop GTK3. GNOME and Xfce's terminals both wrap VTE, as do dozens of others.
  \texttt{xfce4-terminal} plays the role of our VTE terminal in benchmarks, because
  I run XFCE (atop Compiz).}
\item{\texttt{alacritty}, written in Rust by Joe Wilm. One of the new class of
    emulators written directly against OpenGL.}
\item{\texttt{kitty}, another OpenGL program, this one in Python by Kovid Goyal.}
\item{Terminology is the emulator of Enlightenment since E17, the long (twelve
    years!) awaited successor to E16. It seemed somewhat broken every time I
    tried it, and by 2014 I didn't want to spend time fixing terminal emulators.
    It does look really pretty on its website\cite{terminology}.}
\item{Terminator was written in Java, which\textellipsis life finds a way I guess.}
\end{denseitemize}

I haven't benchmarked \texttt{rxvt} because I didn't care to figure out which of
its ten thousand forks is the one you're supposed to use. Likewise, I didn't
benchmark Terminator because I would legitimately prefer setting myself on fire
to running a Java terminal emulator.

The best single source of terminal emulator (and terminal) information is
probably the ``Terminal Type Descriptions'' file distributed with NCURSES\cite{termdescript}.

\subsection{The Curses API}
Several APIs for TUIs and character (semi)graphics have emerged over the fifty
years of video terminals' existence. Of them, by far the most venerable,
portable, and proven is Curses, which has (for over twenty years) actually
been codified in the Single UNIX Specification. Notcurses is an obvious
intellectual descendant of Curses, and indeed ought be easy to pick up by any
experienced Curses programmer.

As mentioned in Chapter ~\ref{sec:terminfo}, Ken Arnold released the first BSD
Curses library shortly after extracting \texttt{vi}'s terminal abstraction routines
as libtermcap, and that first Curses made use of termcap\cite{cursesexplained}.
Similarly to termcap, Curses had its origins in Joy's \texttt{vi} code; unlike
termcap, Curses required significant reworking to be presented as a sensible
API. Meanwhile, over in AT\&T land, Mary Ann Horton (previously maintainer of
BSD's \texttt{vi} and \texttt{termcap}) arrived and implemented a new Curses,
this one making use of her terminfo. The two diverged over the years (with PDCurses
aka Public Domain Curses reimplementing AT\&T to work around licensing issues),
until X/Open codified an official vendor-neutral Curses, publishing it in
1996's Issue 4, version 2 (this specification derived primarily from SVR4 Curses).

Zeyd Ben-Halim adapted ncurses from the pcurses project of Pavel Curtis. Dickey
reports the 1.8.1 version released 1993-11-05 (and included with Slackware 2.0.1)
as the ``first widely-used version''. Numerous people (including the notorious
Eric S. Raymond and Ulrich Drepper) contributed significant code, but most NCURSES
development in recent years has been the work of Thomas Dickey.

The Curses API is best reviewed by reading the comprehensive man pages installed
with NCURSES. A few lines follow, summarizing major differences between Curses and Notcurses:

\begin{denseitemize}
\item{Curses has no built-in concept of a z-axis without use of its Panels
    extension, and its \texttt{stdscr} (analogous to the standard plane of
    Notcurses) is not considered part of the Panels stack. Curses does, however,
    allow windows to share memory, a capability Notcurses does not yet offer.}
\item{Panels can be ``hidden'', i.e.\ removed from the z-axis entirely. This cannot
    be done in Notcurses (but planes can be hidden underneath an opaque plane).}
\item{NCURSES defines its color API in terms of ``color pairs''. It is true that some
    terminals require both colors to be changed at the same time, but Notcurses
    hides this fact from the programmer, and instead allows foregrounds and backgrounds
    to be specified wholly independently.}
\item{Curses does not include the Panel, Menu, or Form extensions distributed
    with NCURSES. With that said, their presence in NCURSES effectively means
    they're in Curses. Notcurses is built around a Panel-like concept, and
    includes menus in its base, but lacks the rich Forms capabilities of NCURSES.}
\item{NCURSES supports only up to 256 colors at a time, as components of up to
    32,767 color pairs. There are no such limits in Notcurses (assuming terminal support).}
\item{NCURSES considers it an error to move a window or cursor off the screen. This
    is not an error in Notcurses.}
\end{denseitemize}

\pagebreak
\cleardoublepage

%%%%%%%%%%%%%%%%%%%%%%%%%%%%%%%%%%%%%%%%%%%%%%%%%%%%%%%%%%%%%%%%%%%
\section{Wherein shade is thrown at terminal emulators}
\label{sec:termshade}
\epigraph{And as he spoke, El-ahrairah's tail grew shining white and flashed like a star;
and his back legs grew long and powerful and he thumped the hillside until the
very beetles fell off the the grass stems. He came out of the hole and tore across
the hill faster than any creature in the world.}{Richard Adams, \textit{Watership Down}}
The \texttt{notcurses-demo} program detailed in Chapter~\ref{sec:ncdemo} serves
as an excellent testbed for benchmarking certain properties of various
terminals in various configurations. The \texttt{-c} argument ought always be
provided when benchmarking, so that the PRNG is seeded with the same
value\footnote{This only serves to eliminate variance among equal PRNG
implementations.}. The \texttt{-J} argument generates JSON-formatted
output suitable for machine processing. Finally, \texttt{-d} can be supplied
to reduce the amount of artificial delay: the default is \texttt{-d1} for 1.0x
the standard delay. \texttt{-d0} eliminates all artificial delay.

This binary is composed of over a dozen
different independent demos, which can be freely scheduled on the command line.
Different demos have different usefulness for benchmarking. Fixed-time, fixed-framecount
demos will generally only show differences in the amount of time spent
rendering. Fixed-framecount demos show difference in total time. Fixed-time demos
might show differences in framecount and time spent rendering. A greater
number of frames for the same demo is indicative of an advantage, as is completing
any particular demo in less time. The most generally useful stat to compare
across runs is average time per render and maximum time to render.

\begin{table}[!htbp]
  \centering
  \begin{tabular}{ |c|c|c| }
    \hline
    Demo & Type & Exercises \\
    \hline
    \hline
    Intro & Fixed-time & Fades, EGC draws \\
    \hline
    X-Ray & Fixed-time+frame & Plane movement (x), video \\
    \hline
    Eagle & Time-to-completion & Large renders, multiple sprites \\
    \hline
    Trans & Fixed-time & Plane movement (xy) \\
    \hline
    \rowcolor{blue!25}
    \textbf{Highcon} & Time-to-completion & Intense RGB, intense redraw \\
    \hline
    Box & Time-to-completion & Intense redraw \\
    \hline
    Chunli & Fixed-time+frame & Large sprite \\
    \hline
    \rowcolor{blue!25}
    \textbf{Grid} & Time-to-completion & Intense RGB, RGB variance \\
    \hline
    Reel & Fixed-time & Plane movement (y) \\
    \hline
    Whiteout & Fixed-time & Intense font rendering \\
    \hline
    Uniblock & Fixed-time & Intense font rendering \\
    \hline
    View & Fixed-time+frame & Video \\
    \hline
    Luigi & Time-to-completion & Sprites, background \\
    \hline
    \rowcolor{blue!25}
    \textbf{Fallin} & Time-to-completion & Many planes, plane movement (y) \\
    \hline
    Sliders & Time-to-completion & Many planes \\
    \hline
    Jungle & Fixed-time & Palette cycling \\
    \hline
    Outro & Fixed-time & Fades, video \\
    \hline
  \end{tabular}
\caption[Benchmarking properties of various demos.]{Benchmarking properties of
  various demos. I consider the colored demos particularly informative.}
\label{table:benchmarks}
\end{table}

Prior work benchmarking terminal emulators largely focused on ``scroll speed'',
a fairly useless and easily misleading statistic\cite{lookatterms}, and input
latency\cite{typingpleasure}.

All of the following graphs have a logarithmic y-axis. At an 80x52 geometry,
the number of bytes output by different demos spans three orders of magnitude,
and the time (when all artificial delays are removed) by four orders of
magnitude\footnote{With artificial delays, it's more like a single order of
magnitude.}. The font family used is always Hack\cite{hacktypeface} at 10 points,
except for \texttt{xterm}, which became effectively unusable with this TrueType
font. I instead allowed \texttt{xterm} to use its default font and size, leading
to a much smaller total window size at the 80x52 cell geometry. Nonetheless,
\texttt{xterm} still reliably delivered the poorest performance\footnote{\textbf{FIXME FIXME FIXME investigate this}}.
One interesting observation is that CPU usage of the \texttt{Xorg} server process
never exceeded 20\% with other terminal emulators, but regularly spiked above 60\%
with \texttt{xterm} using bitmapped fonts, and pegged the CPU(!) with TrueType
fonts. This ought be investigated.

All terminals were their Debian Unstable or Arch-packaged variants at the time
of testing, save \texttt{alacritty}, which was not yet present in Debian. Full
details are provided in Table~\ref{table:benchterms}.

\begin{table}
  \centering
  \begin{tabular}{|l|l|l|l|}
    \hline
    Emulator & Version & \texttt{TERM} & Comments \\
    \hline
    \hline
    \texttt{xfce4-terminal} & 0.8.9.1 & vte-256color & Based on GNOME's VTE\cite{gnomevte} \\
    \hline
    \texttt{xterm} & Patch \#353 & xterm-256color & \makecell[l]{See note above regarding regression\\
      to bitmapped fonts, yuck.} \\
    \hline
    \texttt{kitty} & 0.15.0 & xterm-kitty & OpenGL-based, Python \\
    \hline
    \texttt{alacritty} & \makecell[l]{0.5.0-dev\\(1ddd311)} & alacritty & OpenGL-based, Rust \\
    \hline
    \texttt{konsole} & 19.08.1 & konsole-direct & KDE's terminal \\
    \hline
  \end{tabular}
\caption{Terminal software used for benchmarking.}
\label{table:benchterms}
\end{table}

\begin{figure}[!htb]
\centering
\includegraphics[width=1\textwidth]{media/termsdemo.png}
\caption[Intel i7-8550U benchmarks, varying widths.]{\texttt{notcurses-demo} runtime against terminal width. Linux 5.5.2 + i915, Arch Xorg 1.20.7, standard delays. 3 runs each, at each width.
At this scale (about 100s), timings are very repeatable from run to run. Each run took about five hours.}
\label{fig:intel-full}
\end{figure}

\begin{figure}[!htb]
\centering
\includegraphics[width=1\textwidth]{media/nvidia-termsdemo.png}
\caption[NVIDIA GTX 1080 benchmarks, varying widths.]{\texttt{notcurses-demo} runtime against terminal width. Linux 5.5.9 + NVIDIA 440.64, Debian Xorg 1.20.7, no artificial delays. 3 runs each, at each width.}
\label{fig:nvidia-full}
\end{figure}

\begin{figure}[!htb]
\centering
\includegraphics[width=1\textwidth]{media/plotbytes.png}
\caption[Bytes output per demo per term.]{Bytes emitted on a 80x52 geometry (logarithmic y). As expected, the counts are generally equal across terms. If plotted against width, the counts increase linearly.}
\label{fig:nvidia-full}
\end{figure}

In order to prepare a machine for benchmarking, I disabled frequency scaling
by setting all cores' scaling governor to \texttt{performance}, and disabled
boosting by writing 1 to \texttt{/sys/devices/system/cpu/intel\_pstate/}\footnote{If using \texttt{acpi-cpufreq}, write 0 to \texttt{/sys/devices/system/cpu/cpufreq/boost}.}.
I disabled suspend mode on the Lenovo T580 laptop, and disabled screen blanking
on both machines (I'm not sure whether this latter has any impact, but wanted
to play it safe). Benchmarks always ran on the active virtual desktop. \texttt{cron} was
disabled, but I did \textit{not} disable systemd timers\footnote{All the timers I
had enabled are either daily or weekly, and disabling them might have
cumulative performance impact vs. typical behavior. I didn't go to any great
pains to avoid running benchmarks while these tasks were running, and doubt it
mattered much---they're minor tasks, and these are 8- and 20-core machines.}.
The runtimes were so repeatable (see e.g. Figure~\ref{fig:intel-full}, where
runs tracked so closely I thought I must have made an error) that I didn't
bother with things like CPU affinity or process priority.

\begin{figure}[!htb]
\centering
\includegraphics[width=1\textwidth]{media/i915-80x52.png}
\caption[80x52 Intel i7-8550U benchmarks.]{Runtimes on a 80x52 geometry (logarithmic y) atop Linux 5.5.2 + i915, Arch Xorg 1.20.7, no false delay. 5 runs each.}
\label{fig:intel-full}
\end{figure}

It's difficult to miss a distinct triangle pattern among the rising runtimes
as width increases. In all cases where it occurs, odd widths appear to run
faster than even widths. I suspect this to be a property of something within
\texttt{notcurses-demo} rather than a general truth about Notcurses, but have
not yet gotten to the root of this unexpected result. Beyond that, we also see
that there is a definite difference between both the performance at any width
between the different emulators, and furthermore a difference in the rates of
change. \texttt{xterm} appears likely to have quadratic behavior as a function
of width. \texttt{xfce4-terminal} looks more linear in its growth. \texttt{alacirtty}
barely notices a larger rendering area, with its time at 191 columns less than
10\% more than its time at 80. Over that same range, \texttt{xterm}'s runtime
has increased by closer to 67\%\footnote{Look into whether this is repeated when
the pixel count is held constant, but the cell count goes up---i.e. repeat these
tests with a smaller font at more rows X columns \textbf{FIXME}.}.


\begin{figure}[!htb]
\centering
\includegraphics[width=1\textwidth]{media/d0-large-nvidia.png}
\caption[382x74 NVIDIA GTX 1080 benchmarks.]{Runtimes on a 382x74 geometry (logarithmic y) atop Linux 5.5.9 + NVIDIA GTX1080 440.64, Debian Xorg 1.20.7, no false delay. 5 runs each. VTE appears to scale much more poorly than the GL-based Kitty and Alacritty.}
\label{fig:nvidia-full}
\end{figure}

\pagebreak
\cleardoublepage

%%%%%%%%%%%%%%%%%%%%%%%%%%%%%%%%%%%%%%%%%%%%%%%%%%%%%%%%%%%%%%%%%%%

\section{The Linux console}
The Linux console\footnote{The FreeBSD console is its own bag of wonders.} is
substantially different from the X and Wayland terminal emulators to which one
might be more accustomed\footnote{Muddying the issue is the fact that video
backends are sometimes described as consoles. The ``Linux console'' is a terminal
emulator running atop some video backend---on the x86, typically either VGA
Text Mode, or some trivial renderer atop a graphics-mode framebuffer
(e.\ g.\ EFIfb or vesafb.)}. Modern terminal emulators are generally more capable
than the Linux console in several ways:

\begin{denseitemize}
\item{While the Linux console accepts RGB specifiers, it downsamples them to
    far fewer colors.}
\item{Console font capabilities are extremely limited.}
\end{denseitemize}

Userspace alternatives to the Linux console include \texttt{fbterm} and
\texttt{kmscon}.

Like any interface to a termios\cite{termios} implementation, the \texttt{IUTF8}
flag should be set (consult \texttt{stty}). This can be accomplished with the
\texttt{IUTF8} termios flag (or \texttt{stty iutf8} on the command line). This
is necessary for the terminal to interpret your output as multibyte UTF-8. The
keyboard driver ought be placed into UTF-8 mode using the \texttt{KDSKBMODE}
ioctl; the \texttt{kbd-mode} tool does this when invoked with \texttt{-u}.
This is necessary for character erase to function properly in cooked mode. Some
keyboards generate scancodes beyond the essential 128 characters, and these
should be mapped to their UTF-8 equivalents. This can be accomplished with
\texttt{dumpkeys | loadkeys --unicode}\footnote{If you've ever seen the script
\texttt{unicode\_start}, this is exactly what it does.}. This functionality has
been supported since Linux 2.6.4, released 2004-03-11, and is almost certainly
already being done in your environment.

Ensure, as always, that \texttt{LANG} is properly set, that your program
initializes the locale with \texttt{setlocale(3)}, and that \texttt{TERM} is
properly set (in this case, to one of the ``linux*'' variants).

The console font supports only 256 characters, or 512 when colors are cut in
half. The built-in ``PSF''\footnote{The PC Screen Font of H. Peter Anvin and
  Andries Brouwer.} font supports the 256 characters of CP437 (remember
CP437?). The font can be displayed with the \texttt{showconsolefont} command,
and updated with \texttt{setfont}. The font is independent of whether the
console is supplied via \texttt{vgacon} or e.g. \texttt{fbcon}. It is worth
noting that this default font is missing several characters of which Notcurses
makes extensive default use, particularly the rounded and double box-drawing
characters. I intend to look into reprogramming the console font on the fly to
better support this environment\footnote{See \url{https://github.com/dankamongmen/notcurses/issues/201}.},
but for the moment, Notcurses on the console is definitely a weak spot.

Consult the \textit{The Linux Programmer's Manual} for more information,
particularly
\texttt{ioctl\_console(2)}\cite{ioctlconsole},
\texttt{ioctl\_tty(2)}\cite{ioctltty},
\texttt{termios(3)}\cite{termios},
\texttt{console\_codes(4)}\cite{consolecodes},
and
\texttt{charsets(7)}\cite{charsets7}.

\cleardoublepage
%%%%%%%%%%%%%%%%%%%%%%%%%%%%%%%%%%%%%%%%%%%%%%%%%%%%%%%%%%%%%%%%%%%
\section{Unicode 13}
The Unicode Consortium has scheduled Unicode 13.0 for a March 2020 release.
Chapters 3 and most of Chapter 4 of the Core Specification are normative. The
remainder is informative. The Unicode Standard consists of the Core
Specification\cite{unicode13}, the \href{https://www.unicode.org/charts/}{code charts},
the \href{https://unicode.org/versions/Unicode13.0.0/#Unicode_Standard_Annexes_nb}{Unicode Standard Annexes},
and the \href{http://www.unicode.org/Public/13.0.0/}{Unicode Character Database (UCD)}.

A Unicode Standard Annex (UAX) forms an integral part of the Unicode Standard,
but is published online as a separate document. The Unicode Standard may
require conformance to normative content in a Unicode Standard Annex, if so
specified in the Conformance chapter of that version of the Unicode Standard.
The version number of a UAX document corresponds to the version of the Unicode
Standard of which it forms a part.

\begin{table}[h]
\begin{center}
  \begin{tabular}{ |c|c| }
    \hline
    UAX \#9 & Unicode Bidirectional Algorithm \\
    \hline
    UTS \#10 & Unicode Collation Algorithm \\
    \hline
    UAX \#11 & East Asian Width \\
    \hline
    UAX \#14 & Unicode Line Breaking Algorithm \\
    \hline
    UAX \#15 & Unicode Normalization Forms \\
    \hline
    UAX \#24 & Unicode Script Property \\
    \hline
    UAX \#29 & Unicode Text Segmentation \\
    \hline
    UAX \#31 & Unicode Identifier and Pattern Syntax \\
    \hline
    UAX \#34 & Unicode Named Character Sequences \\
    \hline
    UAX \#38 & Unicode Han Database (Unihan) \\
    \hline
    UTS \#39 & Unicode Security Mechanisms \\
    \hline
    UAX \#41 & Common References for Unicode Standard Annexes \\
    \hline
    UAX \#42 & Unicode Character Database in XML \\
    \hline
    UAX \#44 & Unicode Character Database \\
    \hline
    UAX \#45 & U-Source Ideographs \\
    \hline
    UTS \#46 & Unicode IDNA Compatibility Processing \\
    \hline
    UAX \#50 & Unicode Vertical Text Layout \\
    \hline
    UTS \#51 & Unicode Emoji \\
    \hline
  \end{tabular}
\caption{Unicode 13.0.0 Standard Annexes and Synchronized Technical Standards.}
\end{center}
\end{table}

\cleardoublepage
%%%%%%%%%%%%%%%%%%%%%%%%%%%%%%%%%%%%%%%%%%%%%%%%%%%%%%%%%%%%%%%%%%%
\section{Relevant Standards}
All major standards regarding character sets, control sequences, and their
encoding have ISO/IEC versions, which should probably be considered their
canonical editions. Many of these ISO/IEC standards ratified or included ECMA
or ANSI standards. Table~\ref{table:standards} lists equivalencies.

\vspace{.5in}

\begin{table}[!h]
  \centering
  \begin{tabular}{|l|l|l|l|}
    \hline
    ISO/IEC & ANSI & ECMA & Topic \\
    \hline
    \hline
    646:1991 & X3.4-1986 & 6:1991 & \makecell[l]{7-bit character sets \\ Twinned with ITU T.50} \\
    \hline
    2022:1994 & x & 35:1994 & 8-bit character sets \\
    \hline
    2375:2003 & x & x & Registration of character sets \\
    \hline
    4873:1991 & x & 43:1991 & Multitiered 8-bit codes \\
    \hline
    6429:1992 & X3.64-1979 & 48:1991 & Control codes \\
    \hline
    8613-6:1994 & x & x & \makecell[l]{Graphic renditions in terms of SGR \\Twinned to ITU T.416} \\
    \hline
    8859:1998 & x & 94:1986 & \makecell[l]{Official 8-bit character sets\\ECMA 96 is only \textnumero 1 to \textnumero 4.} \\
    \hline
    10367:1991 & x & x & G0, G1, G2, and G3 \\
    \hline
    10646:2017 & x & x & The Universal Character Set \\
    \hline
    14755:1996 & x & x & Input methods for the UCS \\
    \hline
  \end{tabular}
  \caption{ECMA/ANSI/ISO standards referenced in this text.}
  \label{table:standards}
\end{table}

\vspace{.5in}

POSIX is a family of standards from IEEE 1003, first released as IEEE 1003.1-1988 as:
\begin{denseitemize}
\item{POSIX.1/IEEE 1003.1-1988: Core Services, incorporating ANSI C 1989}
\item{POSIX.2/IEEE 1003.2-1992: Shell and utilities}
\item{POSIX.1b/IEEE 1003.1b-1993: Real-time extensions}
\item{POSIX.1c/IEEE 1003.1c-1995: Threads}
\end{denseitemize}
Starting with the 1997's SUS2 (``UNIX 98''), it was agreed that the Austin
Group would take over development of POSIX. Releases of the standard would be
done under the SUS aegis, and then made POSIX standards upon ISO ratification.
Three releases have followed---POSIX.1-2001 and its amending 2004 Technical Corrigendae,
POSIX.1-2008 (aka ``Base Specifications Issue 7'') and \textit{its} two amendments,
and POSIX.1-2017.

The Single Unix Specification, then, is developed and issued by the Austin
Group, a combination of ISO JTC 1 SC22, the Open Group, and ISO 1003. The first
SUS emerged in 1995 as a ratification of the Open Group's Portability Guide (XPG)
4 version 2. SUS2 followed in 1997, as mentioned, and in 2001 came the glorious
unification: SUS3 aka POSIX.1-2001. SUS4 was built atop POSIX.1-2008, and has
seen three new editions in 2013, 2016, and 2018. If you comply with SUS4 including
both its Technical Corrigendae, you're officially ``UNIX V7'', and are out a few
tens of thousands of dollars in Open Group conformance testing.

\cleardoublepage
%%%%%%%%%%%%%%%%%%%%%%%%%%%%%%%%%%%%%%%%%%%%%%%%%%%%%%%%%%%%%%%%%%%

%\section{Notcurses header files}
%\subsection{The \texttt{notcurses.h} header}
%\bgroup
%\inputminted[linenos,breaklines=true]{C}{code/notcurses.h}
%\egroup
%
%\subsection{The \texttt{nckeys.h} header}
%\bgroup
%\inputminted[linenos,breaklines=true]{C}{code/nckeys.h}
%\egroup

\end{appendices}
\cleardoublepage
%%%%%%%%%%%%%%%%%%%%%%%%%%%%%%%%%%%%%%%%%%%%%%%%%%%%%%%%%%%%%%%%%%%%%%%%
\glsaddallunused
\printglossary[title={Glossary of terms}]
\cleardoublepage
%%%%%%%%%%%%%%%%%%%%%%%%%%%%%%%%%%%%%%%%%%%%%%%%%%%%%%%%%%%%%%%%%%%%%%%%
\addcontentsline{toc}{section}{References}
\printbibliography
\phantomsection
\cleardoublepage
%%%%%%%%%%%%%%%%%%%%%%%%%%%%%%%%%%%%%%%%%%%%%%%%%%%%%%%%%%%%%%%%%%%%%%%%
\vspace*{.5in}
\addcontentsline{toc}{section}{Acknowledgments}
\begin{center}\textbf{Acknowledgments}\end{center}
Hail Eris! All Hail Discordia!

This work would not have been possible without the early customers of
Dirty South Supercomputing; thank you Jeff Arnold at ShareCare,
Todd Wilson at Vakaros, Charles Brian Quinn at Greenzie, Carl Ledbetter
at SimpleRose, Keshav Attrey at PureStorage, Jaron Nix at Pathware,
Miguel Turner at Cyxtera, and the folks who can't be mentioned. Keep
churning through DGEMMs and blowing shit up, everybody. Yacin Nadji,
Scott Hughes, Paige Bailey, Brendan Dolan-Gavitt, Brett W.\ Thompson, Bill
Phillips, Lee Hall, Jon Paprocki, and Joe Lafiosca were particularly supportive of the Notcurses
endeavor. Mark Ferrari let me use his mindblowing palette-cycling pixel art,
and was awfully sweet to a weird dude emailing him from out of nowhere. GitHub
has remained shockingly useful and unoffensive for a Microsoft product. Robert
Edmonds helped me get Notcurses into Debian's NEW queue, and coached me in the
ways of the DFSG. Marek Habersack wrote the \CC wrapper, and has kept it up to
date despite my total lack of communication or warning before lunging in
entirely new API directions. Where would I be without jwz? Astrid Bin did the
awesome DSSCAW logo\footnote{Minus the boss purple gradient, which I applied
  against her passionate wishes. I stand by my call.}, and I couldn't pay her
in the end due to bullshit laws---like, I drunkenly called her up and was like
``FUCK DA POLICE i'll drop a brown paper sack of \$10 bills on your porch and
steal you a social security number, arrrrrrrr'' and yet she refused---thanks,
Astrid! Mary Ann Horton, the original author of terminfo, was gracious and
helpful in her responses. Thomas E. Dickey, author/maintainer of many venerable
trees including NCURSES, is a saint and an angel, a UNESCO treasure, and the
very model of conservative, thoughtful software stewardship. He graciously
answered my mails with lengthy and rigorous information. The free software
community owes him a great debt.

Shouts out to Midtown Yuppie Scum (ATL) and Stinkeye of the Tiger (NYC) Trivia
Clubs. Shouts out to Paul Johnson and Paul Judge. Shouts out to Outkast, Goodie
Mob, and RTJ. Shouts out to every freedom-loving person around the world
fighting the unceasing struggle against International Communism. God save the
Constitution and God save the Republic.
\vfill
\begin{center}
\includegraphics[width=.4\linewidth]{../common/dsscaw-purp-scaled.png}
\includegraphics[width=.5\linewidth]{../common/south.png}
\end{center}
\cleardoublepage
%%%%%%%%%%%%%%%%%%%%%%%%%%%%%%%%%%%%%%%%%%%%%%%%%%%%%%%%%%%%%%%%%%%%%%%%
\vspace*{.5in}
\addcontentsline{toc}{section}{About the author}
\begin{center}\textbf{About the Author}\end{center}
Aside from adventures in New York and Austin, Nick Black is a lifelong ATLien.
He began programming on an ATARI 400 sometime during childhood, and with no one
around to tell him better, developed an idiosyncratic, unorthodox style
involving more inline assembly and literary allusions than strictly
necessary, or perhaps even justifiable. He has since graduated several times
from the Georgia Institute of Technology, and dropped out almost as many times.
Approaching forty, he still manages to code about ten hours a day, every day,
ideally those free of the Daystar's malignant influence. He hopes to one day
destroy the sun. This is his first book as an adult. He lives in Midtown
Atlanta with wife Emily (both a finer engineer, and just plain finer),
along with their poofy penguin and several orca. Nick exclusively uses black
IBM Model-M Trackpoint II M-13 keyboards, and will happily buy any that you
might have laying around \textbf{CLACK CLACK CLACK}.

\begin{figure}
  \centering
\includegraphics[width=.75\linewidth]{media/theauthor.jpg}
%\caption{The author in a bathroom somewhere.}
\end{figure}
\cleardoublepage
%%%%%%%%%%%%%%%%%%%%%%%%%%%%%%%%%%%%%%%%%%%%%%%%%%%%%%%%%%%%%%%%%%%%%%%%
%\includepdf{media/backcover.pdf}
\end{document}
