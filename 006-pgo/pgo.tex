% oneside ought not be used for printed, bound material!
\documentclass[ebook,10pt,openany,oneside]{memoir}
\title{pgo}
\author{nick black}
%\special{pdf:minorversion 7}
%\usepackage{unicode-math}
\usepackage{amsmath}
\usepackage{fontspec}
\defaultfontfeatures{Scale=MatchUppercase,Ligatures=TeX,Renderer=HarfBuzz}
\usepackage{microtype}
\usepackage{bookmark}
\usepackage{ninecolors}
\usepackage[svgnames,HSB,table]{xcolor}
\selectcolormodel{rgb}
\usepackage[labelfont=bf]{caption} % for \captionsetup
\usepackage{polyglossia}
\usepackage{wrapfig}
\usepackage{pdfpages}
\usepackage{tabularray}
\usepackage[export]{adjustbox}
\usepackage{longtable}
\usepackage{tcolorbox}
\tcbuselibrary{most}
\usepackage{listings}
\usepackage{makecell}
\usepackage{varwidth}
\usepackage{ifthen}
\usepackage{multicol}
\usepackage{multirow}
\usepackage{biblatex}
\usepackage[siunitx]{circuitikz}
\usepackage[mode=text]{siunitx}
\setsansfont{Kanit}
\setmainfont{Gentium Book}
\setdefaultlanguage{american}
\newfontfamily\japanesefont[Script=CJK]{Noto Serif CJK JP}
\usepackage{newunicodechar}
\usepackage{realscripts} % typographical enhancement for sub/superscripts
\usepackage{luatex85} % otherwise pdfx blows up =\
\usepackage[a-3u]{pdfx}
\newfontfamily{\symbolfont}{Symbola}
\newunicodechar{🟉}{{\symbolfont🟉}}
\newcolumntype{g}{>{\columncolor{Gray!25}}c} % for alternate column coloring in tabels
\newcolumntype{C}[1]{>{\centering\arraybackslash}m{#1}} % centering within p-type column

% we're either 6x9" or (less desirably) 8.5x11"
\settrims{0pt}{0pt}
\setstocksize{8.75in}{5.625in}
\settrimmedsize{8.5in}{5.5in}{*}
\setlrmarginsandblock{.75in}{*}{1}
\setulmarginsandblock{1in}{1in}{1}       % 2" of total vertical margin
\setlength{\headwidth}{\textwidth}       % don't go to the end of paperback
\settypeblocksize{7in}{\textwidth}{*}    % want 7in of our 9
\checkandfixthelayout                     % print and check layout % W: Command terminated with space. (1)

\makeatletter
\newcommand*{\textoverline}[1]{$\overline{\hbox{#1}}\m@th$}
\makeatother

\chapterstyle{crosshead}

\renewcommand{\insertchapterspace}{}

\newcommand\Midrule[0]{\midrule}

\definecolor{Shadow}{HTML}{C247FB}
\definecolor{Bug}{HTML}{A6B91A}
\definecolor{Dark}{HTML}{575063}
\definecolor{Dragon}{HTML}{6F35FC}
\definecolor{Electric}{HTML}{F7D02C}
\definecolor{Fairy}{HTML}{D685AD}
\definecolor{Fighting}{HTML}{C22E28}
\definecolor{Fire}{HTML}{EE8130}
\definecolor{Flying}{HTML}{8D87DB}
\definecolor{Ghost}{HTML}{5066A3}
\definecolor{Grass}{HTML}{7AC74C}
\definecolor{Ground}{HTML}{D97A4B}
\definecolor{Ice}{HTML}{96D9D6}
\definecolor{Normal}{HTML}{9CA5AB}
\definecolor{Poison}{HTML}{A33EA1}
\definecolor{Psychic}{HTML}{F95587}
\definecolor{Rock}{HTML}{B6A136}
\definecolor{Steel}{HTML}{5E91A4}
\definecolor{Water}{HTML}{6390F0}


\def\LOGO {
  \textbf{
    \Huge Pokémon GO\\
    \Large A Quantitative Approach\\
  }}
\frontmatter
\begin{document}
  \pagestyle{empty}
  \begin{center}
  \LOGO
  \vspace{0.5in}
    \center{\Large{nick black}}
  \end{center}
  \vfill\hrule
  \begin{center}\textsf{Gold \& Appel Publishing}\end{center}
  \hrule
  \clearpage
  %\hfill\includegraphics{images/chao.png}

%\vspace*{\fill}
%{\normalsize
%\textbf{\textit{\textsc{dedicated}\\
%to all those of whom i am proud,\\
%and those who will one day get there.}}}

\bigskip

%\noindent \href{https://midnightssimulacra.com}{\TITLE}

{\small
\noindent copyright © 2025 nick black---all rights reserved

%\noindent{}isbn: 979-8-9895236-3-4
\noindent{}isbn: 979-8-9895236-4-1

\noindent{}version 1.0.3 released \today (original 2025-10-28)

\noindent{}published in atlanta, ga, usa

%\noindent{}Library of Congress Control Number: 2023922340

  %\clearpage
  \setcounter{page}{1}
  \pagestyle{plain} % want (roman) page numbers

\clearpage
\vfill
\noindent{}The electronic form of this book is distributed freely.\\
You may make copies, print it out, etc.\\
\\
Should you find it useful, feel encouraged to send an amount you think reasonable to
  \href{mailto:nickblack@linux.com}{nickblack@linux.com} using
  e.g.\ \href{https://paypal.me/dankamongmen}{paypal},
  or to buy the printed edition.
The printed edition does not contain this notice.\\
\\
Thanks! {\textbf{{\symbolfont♥} nick}}
\clearpage
\ifdefined\epub
\else
  \hypertarget{toc}{}%
  \bookmark[dest=toc]{Contents}
  \tableofcontents*
  %\clearpage
  %\hypertarget{lot}{}%
  %\bookmark[dest=lot]{list of tables}
  %\listoftables*
  %\hypertarget{lof}{}%
  %\bookmark[dest=lof]{list of figures} % if we use 3, it changes to i instead of iii [shrug]
  %\listoffigures*
  \fi
\chapter{Foreward}

\noindent{}This text is specific to Pokémon GO, though many details
apply to other games in the Pokémon family.
In particular, it is based on my experimentation with and analysis of
 the 0.361.0 Android release, using decompilation, debuggers, and
 network traffic analysis.
Where possible, I have relied on official Niantic communications
 and my own research.
I attempt to reproduce most claims, though some have been accepted at face value.\\
\\
\noindent{}Changes are frequently made to the game, some of them quite fundamental.
This text documents PGO as it is played in July 2025.
I have not generally bothered to explore variances with historical gameplay.\\
\\
\noindent{}Much of the information herein will be old news to experienced
 PGO players, but I hope that it will provide a valuable reference and central collection of wisdom.
The community established a large body of knowledge long before I
 came to the game.
Still, some of my theories regarding team selection and battle strategy might
  be new to you.
My novel contributions, such as they are, can be primarily found in
  \autoref{chap:unbounded}, \autoref{chap:bounded}, and \autoref{chap:simul}.

%--------------------------main text------------------------------------------%
%\pagestyle{companion}
%\chapter{The story thus far}
\label{chap:story}
%It's been at least a few decades, but authorities can't
%  agree exactly how long ago the great wars ended.
%Humanity and its output of centuries were largely undone by
% years of high explosives, biological war, the false stars
% of apocalyptic nuclear detonations, and the baneful
% waves of fission products.

%You are the newest Trainer hired by the
\textbf{FIXME finish this}

\chapter{Trainers}
A Pokémon GO account is synonymous with its associated Trainer.

New Trainers start at Level 1 (of 50) with 0 XP\@.
A new Level is conferred as soon as the Trainer's XP meets or exceeds
  the Level's XP threshold (see \autoref{table:xp40}).
\begin{table}[ht]
\begin{center}
\begin{tabular}{r r r|r r r}
  Level & kXP & ΔkXP & Level & kXP & ΔkXP \\
\Midrule
1 & 0 & 0 & 21 & 260 & 50 \\
2 & 1 & 1 & 22 & 335 & 75 \\
3 & 3 & 2 & 23 & 435 & 100 \\
4 & 6 & 3 & 24 & 560 & 125 \\
5 & 10 & 4 & 25 & 710 & 150 \\
6 & 15 & 5 & 26 & 900 & 190 \\
7 & 21 & 6 & 27 & 200 & 1,100 \\
8 & 28 & 7 & 28 & 250 & 1,350 \\
9 & 36 & 8 & 29 & 300 & 1,650 \\
10 & 45 & 9 & 30 & 350 & 2,000 \\
11 & 55 & 10 & 31 & 500 & 2,500 \\
12 & 65 & 10 & 32 & 500 & 3,000 \\
13 & 75 & 10 & 33 & 750 & 3,750 \\
14 & 85 & 10 & 34 & 1,000 & 4,750 \\
15 & 100 & 15 & 35 & 1,250 & 6,000 \\
16 & 120 & 20 & 36 & 7,500 & 7,500 \\
17 & 140 & 20 & 37 & 2,000 & 9,500 \\
18 & 160 & 20 & 38 & 2,500 & 12,000 \\
19 & 185 & 25 & 39 & 3,000 & 15,000 \\
20 & 210 & 25 & 40 & 5,000 & 20,000 \\
\end{tabular}
\caption{Requirements for Trainer Levels 1--40}
\label{table:xp40}
\end{center}
\end{table}
It is possible to advance multiple Levels via a single reward of XP\@.
A Trainer cannot control Pokémon of a Level exceeding the Trainer's by more than 10,
  i.e.\ a Level 22 Trainer cannot power a Pokémon past Level 32.
If such a Pokémon is captured or otherwise received, its Level will be
  reduced to the Trainer's plus 10.
The maximum Pokémon Level is 50, so all Pokémon Levels are available upon
  reaching Trainer Level 40.
Trainer Levels 41 and above have extra requirements beyond XP (see \autoref{table:xp41plus}).

Reaching a new Level is rewarded with items (\autoref{table:levelitems}),
  and sometimes opens up new elements of gameplay (\autoref{table:levelgates}).
Levels 1, 7, 8, 15, and 20 each come with Special Research.

\begin{table}[ht]
\begin{center}
\begin{tabular}{r r r|r r r}
\end{tabular}
\caption{Rewards for achieving Trainer Levels}
\label{table:levelitems}
\end{center}
\end{table}

\begin{table}[ht]
\begin{center}
  \begin{tabular}{r p{0.75\textwidth}}
  Level & Features unlocked \\
\Midrule
  2 & Nenab berries \\
  5 & Potions, Revives, Team selection, appraisal, Gyms, Raids \\
  8 & Razz berries, Team Rocket \\
  10 & Super Potions, Golden Razz berries, Evolution items, Trades, Trainer Battles \\
  12 & Great Balls \\
  15 & Hyper Potions, Fast TMs, Parties \\
  18 & Pinap berries \\
  20 & Silver Pinap berries, Ultra Balls, Mega Evolution \\
  25 & Max Potions, Charged TMs \\
  30 & Max Revives, Route creation \\
  31 & XL Candy \\
  37 & Pokéstop creation \\
\end{tabular}
\caption{Features unlocked by Trainer Levels}
\label{table:levelgates}
\end{center}
\end{table}

Upon reaching Level 5, a Trainer can join one of three Teams: Mystic, Valor, or Instinct.
A Trainer can change Team later by purchasing and making use of a Team Medallion.
The cost is 1,000 Pokécoins, and 365 days must pass between Medallion purchases.

\begin{table}[ht]
\begin{center}
  \begin{tabular}{r r r p{0.65\textwidth}}
Level & MXP & ΔMXP & Special requirements \\
\Midrule
41 & 26 & 6 & 20 Legendary/Mythical power ups,
                      30 Raid wins,
                      200 Pokémon caught in 24 hours,
                      5 Gold medals\\
42 & 33.5 & 7.5 & Evolve all 8 Eevee forms,
                      15 item-assisted evolutions,
                      3 Excellent throws,
                      200 Pokémon caught with berries \\
43 & 42.5 & 9 & 100 Stardust earned,
                      200 effective Charged Attacks,
                      5 Legendary/Mythical catches,
                      5 Platinum medals \\
44 & 53.5 & 11 & 30 Great League wins,
                       30 Ultra League wins,
                       30 Master League wins,
                       20 GO Battle League battles \\
45 & 66.5 & 13 & 100 Rocket Grunts wins,
                       100 Pokémon purified,
                       50 Rocket Leaders wins,
                       10 Platinum medals\\
46 & 82 & 15.5 & 100 Field Research tasks,
                       7 consecutive days w/snapshot,
                       50 Excellent throws,
                       30 Eggs hatched\\
47 & 100 & 18 & 30 Raid wins with heterogenous teams,
                        3{\symbolfont ✯}+ Raid win with CP1500-bounded team,
                        3 power ups to Max CP,
                        20 Platinum medals\\
48 & 121 & 21 & 10 Souvenirs from buddy,
                        300 hearts with buddy,
                        200km walked with buddy,
                        8 weeks with 25km walked\\
49 & 146 & 25 & 10 trades of 300km+ catch distance,
                        50 Lucky Pokémon received in trades,
                        500 Gifts sent,
                        35 Platinum medals\\
50 & 176 & 30 & 999 Excellent throws,
                        5 consecutive Legendary/Mythical catches,
                        3 Rocket Leader wins with CP2500-bounded teams,
                        Rank 10 in GO Battle League\\
\end{tabular}
\caption{Requirements for Trainer Levels 41--50}
\label{table:xp41plus}
\end{center}
\end{table}

All game mechanics, including the ability to fully power up Pokémon,
 are available upon reaching Level 40.
Levels 43, 45, 48, and 50 provide extra Challenge quests.
Levels 43, 45, 47, 49, and 50 open some cosmetic options for Trainer avatars.

\section{Capacities}
A Trainer enters this world with no Pokémon, no Pokécoins, no Stardust,
  no Max Particles, no Candy of any kind, no Mega Energy of any kind,
  no Primal Energy of any kind, and no Fusion Energy of any kind.
The Trainer has carrying capacity sufficient for 300 Pokémon; additional
  capacity can be added in increments of 50 up to a total of 10,500.
Each upgrade costs 200 Pokécoins.
A full upgrade will thus currently run you 40,800 Pokécoins.

The Trainer has a Bag capable of storing 350 items (strangely, not all items
  count against this limit, including Gifts, Stickers, and the Postcard Book).
The Bag comes with an Infinite Incubator, a Camera, 2 Incense, and 50 Poké Balls,
  and thus space for 296 new items.
The Bag can be upgraded to store up to 8,800 items.
Here again, each upgrade costs 200 Pokécoins, and increases capacity by 50 items.
The maximum Bag is thus 33,800 Pokécoins.

The Trainer has a Postcard Book with seven pages, each capable of retaining
 50 Postcards, for an inital limit of 350.
Pages can be added for 100 Pokécoins each, up to a maximum of 40 Pages
 supporting 2,000 total Postcards.
Such a monster will cost 3,300 Pokécoins.

The Trainer can carry up to 20 Gifts at a time. and up to 25 of each sticker.
To date, no one has found any use for the stickers.

The Trainer can accumulate no more than 9999 units of a given Mega Energy,
 Primal Energy, or Fusion Energy.

No limit is known for Candy nor Candy XL.

\section{Collecting Pokémon}
During the initial tour with Professor Willow, you are able to catch one of
 Bulbasaur, Charmander, Squirtle, or Pikachu.
You will be provided an infinite supply of Pokéballs for this encounter, and
 the catch rate (see \autoref{chap:spawn}) is set to 100\%.

\textbf{FIXME}

\section{Collecting XP}
\textbf{FIXME}

\section{Collecting Stardust}
\textbf{FIXME}

\section{Collecting Candy}
A Pokémon can be ``transferred to the Professor'' from your collection,
 whereupon its soul is converted into a tasty Candy (certain special
 Pokémon, including Kubfu, Marshadow and Zygarde, cannot be harvested).
Whenever a Pokémon is captured, Candy is awarded. The amount depends on
 the evolutionary state of the captured Pokémon, and the type of Berry
 applied at the time of capture.
\textbf{FIXME}

\section{Collecting Mega and Primal Energy}
The relevant Energies can be acquired by winning a Mega Raid of that type,
  walking with a Mega Evolved or Primal Reverted Buddy, and completing
  certain research tasks.

\section{Medals}
There is a large set of Medals, each of which can be achieved at the Bronze,
 Silver, Gold, or Platinum level.
\textbf{FIXME}

\chapter{Friends and buddies}
\label{chap:friends}
\textbf{FIXME}

\section{Gifts}
\label{sec:gifts}
\textbf{FIXME}

\section{Parties}
\label{sec:parties}
\textbf{FIXME}

\section{Trading}
\label{sec:trades}
\textbf{FIXME}

\section{Buddy Pokémon}
\label{sec:buddies}
\textbf{FIXME}


\chapter{Spawning and Catching}
\label{chap:spawn}

\section{Shiny Pokémon}
\label{section:shiny}

\chapter{Types}
\label{chap:types}
There are eighteen types, each with a representative icon.
In trainer battles, the types of active Pokémon are displayed
 using these icons, so it is important to memorize them.

\begin{table}[h!]
  \begin{center}
  \begin{tabular}{c c c c c c c c c}
  \includegraphics[scale=.25]{images/bug.png} &
  \includegraphics[scale=.25]{images/dark.png} &
  \includegraphics[scale=.25]{images/dragon.png} &
  \includegraphics[scale=.25]{images/electric.png} &
  \includegraphics[scale=.25]{images/fairy.png} &
  \includegraphics[scale=.25]{images/fighting.png} \\
  Bug & Dark & Dragon & Electric & Fairy & Fighting \\
  \includegraphics[scale=.25]{images/fire.png} &
  \includegraphics[scale=.25]{images/flying.png} &
  \includegraphics[scale=.25]{images/ghost.png} &
  \includegraphics[scale=.25]{images/grass.png} &
  \includegraphics[scale=.25]{images/ground.png} &
  \includegraphics[scale=.25]{images/ice.png} \\
  Fire & Flying & Ghost & Grass & Ground & Ice \\
  \includegraphics[scale=.25]{images/normal.png} &
  \includegraphics[scale=.25]{images/poison.png} &
  \includegraphics[scale=.25]{images/psychic.png} &
  \includegraphics[scale=.25]{images/rock.png} &
  \includegraphics[scale=.25]{images/steel.png} &
  \includegraphics[scale=.25]{images/water.png} \\
  Normal & Poison & Psychic & Rock & Steel & Water \\
\end{tabular}
\end{center}
\caption{The 18 base types}
\end{table}

Each species has either one or two distinct types, and each attack has a single type.
The attacker benefits if it has a type in common with the attack being used:
 the Same Type Attack Bonus (STAB), worth a 20\% boost.
The relationship between attack type and defender typing is more complex.
For each Type, the Attack Type can be \textit{very ineffective},
 \textit{ineffective}, \textit{standard}, or \textit{very effective}.
These are mapped to -2, -1, 0, and 1.
For each Type of the defender, determine the Attack Type's effectiveness
 on that Type, and add the results.
This is the total type effectiveness of the Attack Type on that Species.

There are 324 (18 × 18) total Type relationships, of which 204 (63\%) are standard.
That leaves 120 Type relations giving either attacker or defender an advantage,
 one that can easily decide a battle.
These relationships must also be memorized.


\textbf{FIXME: change to defenders along rows to match bigger table, and because we normally want to see what hits/misses a type}
\begin{table}[ht]
  \begin{center}
  \setlength{\tabcolsep}{1pt}
  \begin{tabular}{c g c g c g c g c g c g c g c g c g c}
  &
  \includegraphics[width=1em]{images/bug.png} &
  \includegraphics[width=1em]{images/dark.png} &
  \includegraphics[width=1em]{images/dragon.png} &
  \includegraphics[width=1em]{images/electric.png} &
  \includegraphics[width=1em]{images/fairy.png} &
  \includegraphics[width=1em]{images/fighting.png} &
  \includegraphics[width=1em]{images/fire.png} &
  \includegraphics[width=1em]{images/flying.png} &
  \includegraphics[width=1em]{images/ghost.png} &
  \includegraphics[width=1em]{images/grass.png} &
  \includegraphics[width=1em]{images/ground.png} &
  \includegraphics[width=1em]{images/ice.png} &
  \includegraphics[width=1em]{images/normal.png} &
  \includegraphics[width=1em]{images/poison.png} &
  \includegraphics[width=1em]{images/psychic.png} &
  \includegraphics[width=1em]{images/rock.png} &
  \includegraphics[width=1em]{images/steel.png} &
  \includegraphics[width=1em]{images/water.png}
    \\
    \rowcolor{Gray!25}
    \includegraphics[width=1em]{images/bug.png} & & 1 & & & −1 & −1 & −1 & −1 & −1 & 1 & & & & −1 & 1 & & −1 & \\ % 10
    \includegraphics[width=1em]{images/dark.png} & & −1 & & & −1 & −1 & & & 1 & & & & & & 1 & & & \\ % 5
    \rowcolor{Gray!25}
    \includegraphics[width=1em]{images/dragon.png} & & & 1 & & −2 & & & & & & & & & & & & −1 & \\ % 3
    \includegraphics[width=1em]{images/electric.png} & & & −1 & −1 & & & & 1 & & −1 & −2 & & & & & & & 1 \\ % 6
    \rowcolor{Gray!25}
    \includegraphics[width=1em]{images/fairy.png} & & 1 & 1 & & & 1 & −1 & & & & & & & −1 & & & −1 & \\ % 6
    \includegraphics[width=1em]{images/fighting.png} & −1 & 1 & & & −1 & & & −1 & −2 & & & 1 & 1 & −1 & −1 & 1 & 1 & \\ % 12
    \rowcolor{Gray!25}
    \includegraphics[width=1em]{images/fire.png} & 1 & & −1 & & & & −1 & & & 1 & & 1 & & & & −1 & 1 & −1 \\ % 8
    \includegraphics[width=1em]{images/flying.png} & 1 & & & −1 & & 1 & & & & 1 & & & & & & −1 & −1 & \\ % 6
    \rowcolor{Gray!25}
    \includegraphics[width=1em]{images/ghost.png} & & −1 & & & & & & & 1 & & & & −2 & & 1 & &  & \\ % 4
    \includegraphics[width=1em]{images/grass.png} & −1 & & −1 & & & & −1 & −1 & & −1 & 1 & & & −1 & & 1 & −1 & 1 \\ % 10
    \rowcolor{Gray!25}
    \includegraphics[width=1em]{images/ground.png} & −1 & & & 1 & & & 1 & −2 & & −1 & & & & 1 & & 1 & 1 & \\ % 8
    \includegraphics[width=1em]{images/ice.png} & & & 1 & & & & −1 & 1 & & 1 & 1 & −1 & & & & & −1 & −1 \\ % 8
    \rowcolor{Gray!25}
    \includegraphics[width=1em]{images/normal.png} & & & & & & & & & −2 & & & & & & & −1 & −1 & \\ % 3
    \includegraphics[width=1em]{images/poison.png} & & & &  & 1 &  & &  &  −1 & 1 & −1 & & & −1 & & −1 & -2 & \\ % 7
    \rowcolor{Gray!25}
    \includegraphics[width=1em]{images/psychic.png} & & −2 &  & & & 1 & & & & & & & & 1 & −1 & & −1 & \\ % 5
    \includegraphics[width=1em]{images/rock.png} & 1 &  &  & &  & −1 & 1 & 1 &  &  & −1 & 1 & & &  & & −1 & \\ % 7
    \rowcolor{Gray!25}
    \includegraphics[width=1em]{images/steel.png} & & & & −1 & 1 & & −1 & & & & & 1 & & & & 1 & −1 & −1 \\ % 7
    \includegraphics[width=1em]{images/water.png} & & & −1 & & & & 1 & & & −1 & 1 & & & & & 1 & & −1 \\ % 6
\end{tabular}
    \caption[Type relations]{Type relations. Rows attack, columns defend.}
\end{center}
\end{table}

Only eight relationships are very ineffective:
Normal → Ghost,
Ghost → Normal,
Fighting → Ghost,
Poison → Steel,
Ground → Flying,
Electric → Ground,
Psychic → Dark,
and Dragon → Fairy.
Note that the Ghost/Normal relationship is very ineffective in both directions.
On the other hand, Ground is effective against Electric, Fairy is effective
 against Dragon, and Dark is effective against Psychic, making these the most
 lopsided matchups between single Types.

\section{Memorizing the type relations}
Eleven of the eighteen types are self-active.
Of these, only Dragon and Ghost are
  self-effective\footnote{I use the mnemonic GD, as in ``Goddamnit!'', as in e.g.
  ``Goddamnit, I stupidly forgot Abomasnow can have Outrage, and am now less one Dragonite.''}.
Dark, Electric, Fire, Grass, Ice, Poison, Psychic,
 Steel, and Water are all self-ineffective\footnote{
Fire, Ice, and Water are easily enough remembered together.
For the remainder I employ the mnemonic PEGSDARKASS\@:
  Poison/Psychic, Electric, Steel, DARK, and GrASS\@.
You are encouraged to develop and use alternate mnemonics.}.

Whenever possible, remember two relationships as a single bidirectional one.
No types are mutually effective.
  only one pair of types is mutually ineffective (Bug↔Fighting),
  and only one pair is mutually very ineffective (Ghost↔Normal).
When one type $V$ is ineffective against type $K$, but $K$ is strong
 against $V$, I say $K$ ``kills'' $V$, or
 (if $V$ is very ineffective against $K$) ``slaughters'' $V$.
These twenty-nine relationships must be memorized and immediately recognized.
There are only three slaughter relations: Fairy → Dragon\footnote{Fairy was
  introduced in Generation VI in large part to balance Dragon.}, Dark → Psychic,
  and Ground → Electric.

\begin{figure}[h!]
  \begin{minipage}[t]{0.5\textwidth}
    \includegraphics[scale=.25]{out/circo/nature.dot.png}
    \caption{17 Natural relations}
    \label{fig:natural}
  \end{minipage}
  \begin{minipage}[t]{0.5\textwidth}
    \includegraphics[scale=.25]{out/circo/phases.dot.png}
    \caption{14 Elemental relations}
    \label{fig:elemental}
  \end{minipage}
\end{figure}
\noindent{}Poison kills Grass, which kills Ground, which kills Poison.
Ground kills Rock.
Bug kills Grass.
Rock is strong against Bug, but Ground is weak against it.
Bug is ineffective against Poison, which is ineffective against Rock (\autoref{fig:natural}).
Rock and Water kill Fire, which kills Ice, which kills nothing.
Electric is strong against Water, which is strong against Rock, which is strong against Ice,
 which once again is strong against nothing (\autoref{fig:elemental}).

\begin{figure}[ht]
\centering
\includegraphics[scale=.25]{out/circo/rational.dot.png}
\caption{16 Rational relations}
\label{fig:rational}
\end{figure}
\noindent{}Psychic kills Fighting.
Fighting kills Dark.
Dark kills Ghost, and slaughters Psychic.
Ghost is strong against Psychic.
Fighting, unique among the types, is strong against Normal (\autoref{fig:rational}).
\clearpage

\begin{figure}[t!]
\centering
\includegraphics[scale=.25]{out/circo/dragon.dot.png}
\caption{9 Dragon relations}
\label{fig:dragon}
\end{figure}
\noindent{}Fairy famously slaughters Dragon.
Ice is strong against Dragon.
Fighting, Grass, Electric, Water, and Fire are all weak against it (\autoref{fig:dragon}).
\vspace{.5in}

\begin{figure}[h!]
\centering
\includegraphics[scale=.25]{out/dot/death.dot.png}
\caption{24 relations of death}
\label{fig:death}
\end{figure}
\noindent{}The Graph of Death (\autoref{fig:death}).
Poison kills Fairy, which kills Dark.
Fairy also kills Fighting, which kills Rock.
Ground slaughters Electric, which kills Flying.
Flying strikes back, raining death from above and killing Fighting, Bug, and Grass.
Fire likewise kills Bug and Grass.
Grass kills Water (it is the only thing strong against Water besides Electric).
\clearpage

\begin{figure}[h!]
\centering
\includegraphics[scale=.25]{out/circo/steel.dot.png}
\caption{The 20 Steel relations}
\label{fig:steel}
\end{figure}
\noindent{}Ground and Fighting are strong against Steel, and Fire kills Steel.
Most everything else is weak against Steel, especially
 Poison, which is very ineffective against it.
Steel kills Ice, Rock, and Fairy---if Steel is strong against something, it kills it.
Steel is weak against Water and Electric.
Only two types have no relationships with Steel: Dark and Ghost (\autoref{fig:steel}).

\begin{figure}[ht]
\centering
\includegraphics[scale=.25]{out/dot/jumble.dot.png}
\caption{The 20 remaining interactions}
\end{figure}
The remaining interactions don't have any real structure, and must simply be
memorized as they are.

\begin{table}[ht]
  \begin{center}
    \begin{tabular}{llllll}
      \hline
      \multirow{3}{*}{Ice} & \multirow{3}{*}{effective} & Grass & & \\
      & & Flying & & \\
      & & Ground & effective & Fire \\
      \hline
      \multirow{4}{*}{Bug} & \multirow{2}{*}{effective} & Dark & & \\
      & & Psychic & effective & Poison \\
      & \multirow{2}{*}{ineffective} & Ghost \\
      & & Fairy & ineffective & Fire \\
      \hline
    \end{tabular}
  \end{center}
\end{table}

\section{Dual typing}
\label{section:dualtypes}
A species of Pokémon may be singly or doubly types.
It is not possible to dual type using a single type,
 and ordering of the types does not matter.
\[ C(18, 2) = \binom{18}{2} = \frac{18!}{2!16!} = 153 \]
There are thus 153 dual types in addition to the 18 base types,
  for a total of 171.
We could also arrive at this number by summing 1 through 18:
\[ \sum_{i=1}^{18} i = \frac{18 \times 19}{2} = 171 \]
Nine dual types are currently unpopulated (Normal/Steel, Normal/Ice, Normal/Rock,
 Normal/Bug, Poison/Ice, Ground/Fairy, Rock/Ghost, Bug/Dragon, Fire/Fairy),
 leaving 162 defender typings to consider.

Dual typing expands the type relation range, adding the possibilities
 of -3 and 2 (-4 is not possible, because no Type is very ineffective against
 two different Types).
It furthermore vastly expands the Type Effectiveness matrix,
 with 3,078 (18 × 171) relations of which only 1,490 (48.4\%) are standard.
Thankfully, these needn't be memorized, as they can all be calculated
 using the base relations matrix.
Completely unexpectedly, the best typings include Steel, with the top 16
 spots all making use of it.

\input{out/dualtypes.tex}

There are twenty triple resistances, five of which are against Fighting attacks.
Two of the typings exhibiting such resistances are unpopulated
  (Electricity+Ground\footnote{Sandy Shocks, Pokédex \#0989, is a Generation IX
  Electricity+Ground, but is not at this time present in Pokémon GO.} and Ghost+Rock).
\begin{table}[h]
  \begin{center}
    \begin{tabular}{cc}
Dragon → Fairy+Steel & Electricity → Grass+Ground \\
Electricity → Dragon+Ground & Electricity → Electricty+Ground \\
Fighting → Ghost+Poison & Fighting → Bug+Ghost \\
Fighting → Fairy+Ghost & Fighting → Flying+Ghost \\
Fighting → Ghost+Psychic & Ghost → Dark+Normal \\
Ground → Bug+Flying & Ground → Flying+Grass \\
Psychic → Dark+Psychic & Psychic → Dark+Steel \\
Normal → Ghost+Rock & Normal → Ghost+Steel \\
Poison → Ground+Normal & Poison → Poison+Steel \\
Poison → Rock+Steel & Poison → Ghost+Steel \\
    \end{tabular}
    \caption{The twenty triple resistances}
  \end{center}
\end{table}

\section{Weather boosting}
\label{section:weather}
Local weather ``boosts'' its associated types (\autoref{table:weathers}), affecting attack strength
 in PvE (\autoref{section:mbmult}),
 spawn rates (\autoref{section:spawns}), and catching (\autoref{section:catch}).
The game's assessment of your weather is communicated via an icon on the Map View.
\begin{table}[ht]
\begin{center}
  \begin{tabular}{ll}
    Weather & Types \\
    \Midrule
    Sun & Fire, Grass, Ground \\
    Rain & Bug, Electric, Water \\
    Partly Cloudy & Normal, Rock \\
    Cloudy & Fairy, Fighting, Poison \\
    Windy & Dragon, Flying, Psychic \\
    Snow & Ice, Steel \\
    Fog & Dark, Ghost \\
  \end{tabular}
  \label{table:weather}
  \caption{Weather-boosted types}
\end{center}
\end{table}

\section{Cover sets}

\chapter{Attacks}
A Pokémon deals damage in battle via its Attacks. Fast Attacks generate the
Energy required by Charged Attacks. Every Attack has a single Type, which
may or may not be a Type of the Attacker. Attacks having a Type in common
with the attacker enjoy a 1.2x multiplier, the Same Type Attack Bonus (STAB).
Energy is preserved across substitution, but not across battles.

Note that a single Attack usually has different stats for
 Raids/Max Battles and Trainer Battles.

\begin{table}[ht]
  \begin{center}
  \begin{tabular}{l r r r r r r r r r r r}
    & \multicolumn{4}{c}{Type → Type} & \multicolumn{7}{c}{Type → Typing} \\
    & \multicolumn{4}{c}{\downbracefill} & \multicolumn{7}{c}{\downbracefill} \\
    Type & -2 & -1 & 0 & 1 & -3 & -2 & -1 & 0 & 1 & 2 & ARA \\
    \Midrule
    Ground & 1 & 2 & 10 & 5 & 2 & 12 & 27 & 65 & 55 & 10 & 1 \\
    Rock & 0 & 3 & 11 & 4 & 0 & 3 & 36 & 78 & 48 & 6 & 1 \\
    Fairy & 0 & 3 & 12 & 3 & 0 & 3 & 39 & 87 & 39 & 3 & 0 \\
    Flying & 0 & 3 & 12 & 3 & 0 & 3 & 39 & 87 & 39 & 3 & 0 \\
    Fire & 0 & 4 & 10 & 4 & 0 & 6 & 44 & 71 & 44 & 6 & 0 \\
    Ice & 0 & 4 & 10 & 4 & 0 & 6 & 44 & 71 & 44 & 6 & 0 \\
    Water & 0 & 3 & 12 & 3 & 0 & 3 & 39 & 87 & 39 & 3 & 0 \\
    Ghost & 1 & 1 & 14 & 2 & 1 & 15 & 17 & 107 & 30 & 1 & -1 \\
    Steel & 0 & 4 & 11 & 3 & 0 & 6 & 48 & 78 & 36 & 3 & -1 \\
    Dark & 0 & 3 & 13 & 2 & 0 & 3 & 43 & 96 & 28 & 1 & -1 \\
    Fighting & 1 & 5 & 7 & 5 & 5 & 18 & 45 & 52 & 40 & 10 & -2 \\
    Psychic & 1 & 2 & 13 & 2 & 2 & 15 & 30 & 95 & 28 & 1 & -2 \\
    Dragon & 1 & 1 & 15 & 1 & 1 & 16 & 17 & 121 & 16 & 0 & -2 \\
    Electric & 1 & 3 & 12 & 2 & 3 & 11 & 41 & 89 & 26 & 1 & -3 \\
    Bug & 0 & 7 & 8 & 3 & 0 & 21 & 63 & 57 & 27 & 3 & -4 \\
    Grass & 0 & 7 & 8 & 3 & 0 & 21 & 63 & 57 & 27 & 3 & -4 \\
    Poison & 1 & 4 & 11 & 2 & 4 & 18 & 50 & 74 & 24 & 1 & -4 \\
    Normal & 1 & 2 & 15 & 0 & 2 & 17 & 34 & 118 & 0 & 0 & -4 \\
\end{tabular}
    \caption[Attack Type effectiveness summaries]{Attack Type effectiveness summaries (higher is better)}
  \end{center}
\end{table}

\section{Fast Attacks}

\section{Charged Attacks}
Every Pokémon knows at least one Charged Attack, and can be taught a second
Charged Attack at a cost of Stardust and Candy. Energy generated by the Fast
Attack goes into a pool from which either Charged Attack can draw.

\chapter{Species}

\begin{table}
\begin{tabular}{l l l l }
Level & CPM & Level & CPM \\
\Midrule

\end{tabular}
\caption{Combat Power Multiplier by half-level}
\label{table:cpm}
\end{table}

\section{Shadow Pokémon}

\chapter{Pokémon}
\label{chap:pokemon}
For purposes of battle, a Pokémon $P$ is defined as having:
\begin{itemize}
\item A species and form (and thus a typing and base stats)
\item An Individual Vector (IV) (\autoref{sec:ivs}) and level (\autoref{sec:plevel})
\item A set of two or three attacks, one of which is a fast attack
\item Shadow status, or the absence thereof
\item Some number of hit points not exceeding MHP
\end{itemize}
Visual presentation and size do not affect battle, nor does Purified status.
Hit Points are carried across PvE battles (including encounters with
  Team GO Rocket and Gym battles), but Trainer Battles always start with full HP\@.
A Pokémon with zero hit points has ``fainted''.
Fainted Pokémon cannot be used in battles of any kind, nor can they be left in
 Power Stops (see \autoref{sec:maxbattles}).

\section{Pokémon Level}
\label{sec:plevel}
Each Pokémon has its own level, taking one of the 99 integer and half-integer
 values between 0--50, inclusive.
A Trainer can ``power up'' their Pokémon in single level increments,
 not exceeding the Trainer's own level plus 10.
Each increase has a cost in Stardust and the Candy of that genus; beyond level 40, there
  is a further cost in Candy XL (\autoref{table:powerups}).
\begin{table}
  \footnotesize
  \begin{center}
    \begin{tabular}[ht]{rrrr|rrrr}
      Level &
        k\includegraphics[width=1em,height=1em]{images/stardust.png} &
        \includegraphics[width=1em,height=1em]{images/rarecandy.png} &
        \includegraphics[width=1em,height=1em]{images/rarecandyxl.png} &
      Level &
        k\includegraphics[width=1em,height=1em]{images/stardust.png} &
        \includegraphics[width=1em,height=1em]{images/rarecandy.png} &
        \includegraphics[width=1em,height=1em]{images/rarecandyxl.png} \\
      \Midrule\\\\
      1.5--3 & 0.2 & 1 & & 31.5--33 &   6 &  6 & \\
      3.5--5 & 0.4 & 1 & & 33.5--35 &   7 &  8 & \\
      5.5--7 & 0.6 & 1 & & 35.5--37 &   8 & 10 & \\
      7.5--9 & 0.8 & 1 & & 37.5--39 &   9 & 12 & \\
     9.5--11 &   1 & 1 & & 39.5--40 &  10 & 15 & \\
    11.5--13 & 1.3 & 2 & & 40.5--41 &  10 &    & 10 \\
    13.5--15 & 1.6 & 2 & & 41.5--42 &  11 &    & 10 \\
    15.5--17 & 1.9 & 2 & & 42.5--43 &  11 &    & 12 \\
    17.5--19 & 2.2 & 2 & & 43.5--44 &  12 &    & 12 \\
    19.5--21 & 2.5 & 2 & & 44.5--45 &  12 &    & 15 \\
    21.5--23 &   3 & 3 & & 45.5--46 &  13 &    & 15 \\
    23.5--25 & 3.5 & 3 & & 46.5--47 &  13 &    & 17 \\
    25.5--27 &   4 & 3 & & 47.5--48 &  14 &    & 17 \\ 
    27.5--29 & 4.5 & 4 & & 48.5--49 &  14 &    & 20 \\
    29.5--31 &   5 & 4 & & 49.5--50 &  15 &    & 20 \\
    \end{tabular}
  \end{center}
  \caption{Power-up costs for Pokémon levels}
  \label{table:powerups}
\end{table}
An increase in level improves the Pokémon's attack and defense, and
  will sometimes increase MHP.
See \autoref{chap:stats} for a quantitative treatment.

Powering up a fainted Pokémon will revive it, but with minimal HP\@:

\[ HP = \min{1, MHP_{New} - MHP_{Old} } \]

\section{Individual Vectors}
\label{sec:ivs}
Each Pokémon has three integers associated with it, each ranging from 0--15.
They are unchanging across the Pokémon's existence.

Once a Trainer has joined a team (\autoref{sec:levels}), the ``Appraise'' function can be used to
  get a visual representation of these three numbers (this representation,
  while annoying, is sufficient to determine the actual numbers).
Additionally, the Pokémon will be given a rating of 0--4, represented as
  one, two, or three stars (0 shows one star; 4 shows three, albeit in a different color).
A Pokémon rated 4 has an IV of 15/15/15, and is colloquially known as a ``hundo'' or ``100\%er''.
Such Pokémon have their own Pokédex (\autoref{sec:dexen}).
A Pokémon rated 0 has an IV of 0/0/0, and is colloquially known as a ``nundo'' or ``0\%er''.
They have no unique Pokédex, because they are worthless pieces of shit,
  and don't let anyone tell you otherwise\footnote{Consult \autoref{chap:optimal}
  to see that 0/0/0 is never an optimal IV.}.
While 15/15/15 is generally the optimal IV, this is not always true in CP-bounded
  competition; see \autoref{chap:bounded} for more information,
  and \autoref{chap:optimal} for tables of optimal IVs under different CP bounds.

\section{Generation of IV and level}
\label{sec:ivgeneration}
IVs and levels are generated when the Pokémon is ``created'', not when it is captured.
It is thus possible to glean information from the CP displayed for a Pokémon;
  one can sometimes even uniquely determine the IV\@.
The equation for CP is provided in \autoref{chap:unbounded}; reverse it and
  use \autoref{table:cpm} plus the base stats of the species to determine
  the set of IV+level pairs yielding that CP\@.
Various applications and websites can perform this analysis automatically.
Depending on the situation, there might be a floor for IVs.
Beyond that, they are uniformly randomly generated.
The chance of any given IV for a catch in the wild (without weather boosting)
  is exactly the same---1 out of 4096.
The probability of a random Pokémon being a hundo is actually
  greater than that of being a nundo, due to the existence of IV floors
  (e.g. the chance of a 15/15/15 for a lucky trade is 1 in 256, whereas
  a 0/0/0 is impossible).
\begin{table}
  \begin{center}
    \begin{tabular}{lcc}
      Context & IV floors & Level \\
      \Midrule \\
      \multirow{2}{*}{Wild catch} & 0 & 1--max(TL, 30) \\
      & 4 w/ weather boost & 1--max(TL, 30) + 5 \\
      Hatch & 10 & max(TL, 20) \\
      \multirow{2}{*}{Raid} & 10 & 20 \\
      & 6 (shadow) & 25 w/ weather boost \\
      Max Battle & 10 & 20 \\
      Research & 10 & 15 \\
      Team GO Rocket & 0 & 8 \\
      Grunts and Leaders & 4 w/ weather boost & 13 w/ weather boost \\
      \multirow{2}{*}{Giovanni} & \multirow{2}{*}{6} & 8\\
       & & 13 w/ weather boost\\
      GBL & 10 & 20\\
      \multirow{5}{*}{Trade} & 1 w/ Good Friend & \multirow{5}{*}{See \autoref{sec:trades}} \\
       & 2 w/ Great Friend & \\
       & 3 w/ Ultra Friend & \\
       & 5 w/ Best Friend & \\
       & 12 (lucky trade) & \\
    \end{tabular}
  \end{center}
\end{table}

\section{Lucky Pokémon}
\label{sec:lucky}
\textbf{FIXME}

\chapter{Meging and Maxing}
\label{chap:megmax}

\section{Mega forms}
\label{section:mega}
\input{out/mega}

\section{Primal forms}
\label{section:primal}

\section{Dynamax forms}
\label{section:dmax}
\input{out/dynas}

\section{Gigantamax forms}
\label{section:gmax}
Gigantamax forms are similar to Dynamax forms to the degree that they can
  employ Max Moves in Max Battles.
Unlike Dynamax forms, each Gigantamax knows a species-specific attack
  called a G-Max Move, which does not change to match the fast attack.
G-Max attacks of level 1, 2, and 3 have power of 350, 400, and 450
  respectively, 100 more than Dynamax attacks of the same level.
Gigantamax forms have their own visual representation.
\input{out/gigantas}

\chapter{Damage}
\label{chap:damage}
Each Attack reduces the opponent's Hit Points by an integer greater than zero.
This integer is called (inflicted) Damage, and calculated as a product of several factors.
The factors are different for Gym/Raid/Max Battles and Trainer Battles,
 but the core $D_C$ of the Damage equation is common to all contests.

\section{Typing multipliers}
We must first determine the multiplier due to Typing:

\[ M = M_{STAB} \times M_{T} \]

$M_{STAB}$ is the STAB bonus, equal to 1.2 when the Attack's Type matches any
  Type of the attacker, and 1 otherwise.
$T$ is the number calculated in \autoref{chap:types} using the Attack Type
 and the defender's Type.
$M_{T}$ is $1.6^{T}$.
For completeness I include a $T$ of -4 in \autoref{table:typemult},
 but remember that this is not currently realizable.

\begin{table}
\begin{center}
\begin{tabular}{c r r r r}
  Effectiveness & $M_{T}$ & Δ\% & $M$ w/ STAB & Δ\% \\
\Midrule
  -4 & 0.1526 & -84.74 & 0.1831 & -81.69 \\
  -3 & 0.2441 & -75.59 & 0.2930 & -70.7 \\
  -2 & 0.3906 & -60.94 & 0.4688 & -53.12 \\
  -1 & 0.625 & -37.5 & 0.75 & -25 \\
  0 & 1 & 0 & 1.2 & 20 \\
  1 & 1.6 & 60 & 1.92 & 92 \\
  2 & 2.56 & 156 & 3.072 & 207.2 \\
\end{tabular}
\caption{Typing multipliers with and without STAB}
\label{table:typemult}
\end{center}
\end{table}

Leaving out the hideous $TE = -4$ case, we see that $M$ ranges
 from 0.2441 (a 75.59\% reduction in Damage) to 3.072
 (a 207.2\% increase).
Considered another way, when $TE = -3$ and there is no STAB bonus,
 it takes about four times as many Attacks (and thus time) to subdue
 an opponent than would be expected without typing effects.
When $TE = 2$ and STAB is in play, it takes about one-third as many Attacks.

\section{The Damage equation}

\[ D_C = P \times \frac{Eff_A}{Eff_D} \times M \]

That is, the Power of the employed attack is scaled by the ratio of
 $Eff_A$ to $Eff_D$ and the product of Type Effectiveness multipliers.
Remembering their definitions, this can be rewritten as:

\[ D_C = P \times \frac{Mod_A \times CPM_A}{Mod_D \times CPM_D} \times M \]

which can of course be rewritten as:

\[ D_C = P \times \frac{Mod_A}{Mod_D} \times \frac{CPM_A}{CPM_D} \times M \]

From there, it's just some multipliers, a floor function (to ensure
 an integer), and a min function (to ensure at least 1 point of Damage
 is always inflicted):

\[ D = \min{1, \left\lfloor P \times \frac{Eff_A}{Eff_D} \times M \times M_M \right\rfloor } \]

Since this is a product, a proportional increase in any factor leads to
 the same change.
If we have a Power of 10, a $Mod_A$ of 140, and a CPM of 0.4628 (corresponding
 to Level 12), no STAB bonus, and no Type Effectiveness, $D_C$ is 647.92.
Increasing Power to 12 (a 20\% increase) yields 777.504.
Increasing $Mod_A$ to 168 (a 20\% increase) yields 777.504.
Switching to an Attack matching the attacker's type adds the
 STAB bonus, a 1.2x multiplier, yielding 777.504.
Increasing CPM 20\% to 0.55536 (somewhere between Levels 17 and 17.5; this
 exact change is not possible) yields 777.504.
A 20\% increase of 647.92 is, of course, 777.504.
Note that this is all independent of the opponent.

\section{Opponent-independent Damage}
We can break the Damage equation into those parts dependent
 upon opponents, and those which are independent.
Looking at the Damage equation, Power, $Eff_A$, and STAB do not depend
 upon an attacker's opponent.
Likewise $Eff_D$ and MHP for the defender.
The attacker knows neither $Eff_D$ nor MHP, while the defender doesn't know $Eff_A$.
Type effectiveness always requires knowing the opposing type.
We can thus define an independent core fitness evaluation:

\[ F = Eff_A \times P \times M_{STAB} \times Eff_D \times MHP \]

This is roughly equivalent ($MHP \approx Mod_S * CPM$) to

\[ Mod_A \times P \times M_{STAB} \times Mod_D \times Mod_S \times CPM^3 \]

Of course, a given Pokémon can generally select from multiple Fast and Charged Attacks,
  knowing up to two of the latter at a given time.
Furthermore, Fast and Charged Attacks are not used in equal ratio (there will
  always be multiple Fast Attacks per Charged Attack), and a Shield can
  nullify a Charged Attack.
How do we define $P$ and $M_{STAB}$?
As only one Fast Attack can be known at a time, it's easy enough to define one
  fitness for each possible Fast Attack, leaving out Charged Attacks for now.
All we need do is normalize the Power of the Fast Attack, using the number of
  Turns it requires.

\[ F = Eff_A \times \frac{P_{Fast}}{T_{Fast}} \times M_{STAB-Fast} \times Eff_D \times Eff_S \]

Integrating Charged Attacks is more complex.
First, we can simply normalize the Charged Attack as we did the Fast Attack,
 determine the number of Fast Attacks necessary to launch it:

\[ N_{Fast} = \lceil\frac{E_{Charged}}{E_{Fast}}\rceil \]

yielding the total Power of a cycle, known by the community as Total Damage Output (TDO):

\[ P_{cycle} = N_{Fast} \times P_{Fast} + P_{Charged} \]

We normalize this:

\[ F_{cycle} = \frac{P_{Cycle}}{N_{Fast} \times T_{Fast} + T_{Charged}} \]

There are two turns per second. Multiplying $F_{cycle}$ by two yields a
  stat the community calls Damage Per Second (DPS).

This has several problems. It doesn't account for the possibility of multiple
Charged Attacks. It doesn't account for leftover Energy, which will sometimes
be present whenever $E_{Charged}$ is not a multiple of $E_{Fast}$ (i.e. $N_{Fast}$
might fluctuate from one cycle to the next). It ignores the possibility that
the attacker might not use its Charged Attack immediately (and it is often
unwise to throw Charged Attacks as quickly as possible; see
\autoref{chap:strategy}). Finally, it presumes that the attacker can always get
off a full cycle of Fast Attacks followed by a Charged Attack. In reality, the
attacker might be knocked out or substituted long before its Charged Attack
becomes relevant, or the defender might use a Shield. It is tempting to ignore
Charged Attacks, but when they land, they tend to dominate our total Damage.

If we expand over multiple cycles of Fast and Charged Attack, we can
 generalize to situations with excess Energy. We know $E_{Charged} \times
 E_{Fast}$ will be a multiple of both the generated and consumed Energy, so
 simply consider $E_{Charged}$ cycles, each of which throws an average of
 $\frac{E_{Charged}}{E_{Fast}}$ Fast Attacks followed by one Charged Attack.

\[ P_{cycles} = E_{Charged} \times (\frac{E_{Charged}}{E_{Fast}} \times P_{Fast} + P_{Charged}) \]

 The number of Fast Attacks in a cycle is actually always either
 $\lfloor\frac{E_{Charged}}{E_{Fast}}\rfloor$
 or $\lceil\frac{E_{Charged}}{E_{Fast}}\rceil$ (iff $E_{Charged}$ is a multiple of
 $E_{Fast}$, these two expressions are equal, and the number of Fast Attacks
 per Charged Attack is constant). Normalize for the total time:

\[ F_{cycles} = \frac{P_{cycles}}{E_{Charged} \times (\frac{E_{Charged}}{E_{Fast}} \times T_{Fast} + T_{Charged})} \]

This has only exacerbated one of the problems we mentioned before: we may
  not get to throw all these Attacks!
Suppose we have some constant chance $0 < L_{KO} \leq 1$ of being knocked out or
 substituted following each Attack we launch, the probability of being
 in to throw the $N$th Attack is $(1 - L_{KO})^{N-1}$.
We could thus define an expected damage from Attack $N$:

\[ E_D(N) = \overline{P} \times (1 - L_{KO})^{N-1} \]

and a cumulative expected damage through $N$ Attacks:

\[ E_{TD}(N) = \sum^N_{i=1} \overline{P} \times (1 - L_{KO})^{i-1} \]

This is a geometric series where $a = \overline{P}$ and $r = 1 - L_{KO}$.
Since $0 \leq r < 1$, this series converges to

\[ E_{TD}(\infty) = \sum^\infty_{i=1} \overline{P} \times r^{i-1} = \frac{\overline{P}}{1 - L_{KO}} \]

Of course, our chance of being knocked out is usually not constant across
Attacks, but rather an immediate function of our remaining HP and any
Damage we are about to absorb.

\chapter{Battle Mechanics}
\label{chap:battle}
All battles work on quanta of 0.5s\footnote{This is a recent change for Raid-style battling.}.

\section{3x3 battles}
\label{section:3x3}
There is a four and a half minute (270 second) timer on Trainer Battles.
If this timer expires, the Trainer with the most Pokémon remaining wins.
If they have an equal number of Pokémon, the Trainer with less damage to their
  Pokémon (by percentage) wins.

One pays for charged attacks with energy, and fast attacks with time.

Switching out 

\textbf{FIXME}

\subsection{Team GO Rocket}
\label{section:rocket}
\textbf{FIXME}

\section{Nx1 battles}
\label{section:nx1}
\textbf{FIXME}

\subsection{Gyms}
\label{section:gyms}
\textbf{FIXME}

\subsection{Raids}
\label{section:raids}
\textbf{FIXME}

\subsection{Max battles}
\label{section:maxbattles}
\textbf{FIXME}

\chapter{Unbounded team selection}
\label{chap:unbounded}
Trainer Battles make very little use of random numbers (to the best of my
 knowledge, they only show up when checking for a Charged Attack side
 effect).
Their intrigue is instead rooted in ignorance: ignorance of
 the Pokémon you'll face, the order in which you'll face them,
 their statistics (beyond the static properties of the Species),
 their Attacks, and how their Trainer will employ them.

We've established that Pokémon battle fitness is highly dependent
 on opponent details.
We cannot name a globally superior Pokémon---primarily due to Type Effectiveness,
 there is no individual Pokémon that (assuming competent play by both Trainers)
 can defeat all others in even a 1-on-1 contest.
Yet the game provides a per-Pokémon measure of strength in battle: Combat Power (CP).
Obviously no single statistic can account for all possible opponents, but
 we will see that CP is a very questionable assessment for a PvP
 context (and not great for PvE, either).
A Pokémon's CP is defined as:
\[ CP = \max{10, \frac{Mod_A \times \sqrt{Mod_D} \times \sqrt{Mod_S} \times CPM^2}{10}} \]
CP growth is quadratic with respect to $CPM$, linear with $Mod_A$, and
  sublinear with $Mod_D$ and $Mod_S$.
Remember from our Damage equation that $Mod_A$ is divided by $Mod_D$
 to generate one of the factors.
This suggests that $Mod_D$ is undervalued by the CP equation, which
 would seem to suggest that $Mod_A$ is divided by $\sqrt{Mod_D}$.
We will exploit this discrepancy.
MHP isn't used in the Damage equation, but knocking out a Pokémon
 requires inflicting some average Damage $D$ $n$ times,
 where $n = \lceil\frac{MHP}{D}\rceil$.
It is thus similarly undervalued by CP\@.
The quadratic term for CPM makes sense from the perspective of the Damage
 equation, since CPM is used as a factor when calculating Damage in the
 case of both the attacker and defender.
But since CPM is also used to calculate MHP from $Mod_S$, an argument
 can be made that it ought be raised to the third power.
Since $CPM < 1$ for all Levels, $CPM^3 < CPM^2$, and thus it seems
 overvalued by CP\@.

Why would the game use such a flawed measure of power?
A single stat was clearly desired to cover both modes of Pokémon GO battling.
Since Raids allow substitution of defeated Pokémon, but are subject to a timer,
  the ability to defend and absorb Damage is less important than being able to
  quickly inflict damage.
Perhaps this explains the emphasis on attack, but some of the most important
  factors in the Damage equation are left out of CP entirely!
The Power and timing of the Pokémon's Attacks are not present, despite
  dominating the inflicted Damage and overall flow of the battle.
Type effectiveness and STAB have been likewise ignored, as have the
  properties of Shadow Pokémon.
It ought be clear that we must look beyond CP in our selection of Pokémon.

Change to Damage inflicted by an effective Attack is at least 60\%.
Change to Damage inflicted by an ineffective Attack may be as
 low as 37.5\%.
It is probably worth accepting an ineffective Fast Attack if it
 enables an effective Charged Attack, so long as you actually
 get that Charged Attack off.
Early in the battle, this might trick the opponent into leaving
 off a Shield, allowing a \textbf{FIXME}.

Remember that a type advantage is more impactful than a type
  disadvantage (60\% vs -37.5\%):
  all else being equal, it's more important to reduce your Pokémons'
  weaknesses than to increase their strengths.
Every typing is weak against at least one attack type---usually several.
Weakness against only a single type is uncommon: only seven typings manage it, and in two
 of those cases it's a double weakness (\autoref{table:singleweak}.
The only such typings with much population are Electric and Normal. 
\begin{table}[ht]
\begin{center}
\begin{tabular}{ll}
Typing & Weakness\\
\Midrule
Bug+Steel & Fire (doubly weak) \\
Ghost+Normal & Dark \\
Dark+Ghost & Fairy \\
Dark+Poison & Ground \\
Electric & Ground \\
Ground+Water & Grass (doubly weak) \\
Normal & Fighting \\
\end{tabular}
\end{center}
\caption{Single-weakness typings}
\label{table:singleweak}
\end{table}

\section{How important are IVs?}
As with most properties, we could answer ``the details are opponent-dependent''.
More usefully, we can say ``more important with low base stats, less important
  with higher base stats''.
Quantitatively, we can say ``An $IV_A$ of 15 is worth more than STAB to an
  attacker with ATK less than 75. It is worth exactly as much (20\%) as STAB
  to an attacker with ATK of exactly 75. It is worth less than STAB
  to an attacker with ATK greater than 75.''
Similarly, an $IV_A$ of 15 is equal to type effectiveness of 1 when ATK is 25.
For a more plausible ATK of 150, a maximum $IV_A$ is worth a 10\% bonus to attack.
In general, $IV_A$ adds a percentage boost of

\[ frac{IV_A}{ATK} \times 100 \]

\noindent{}to attack, and $IV_D$ lends the same (of course using DEF) to defense.

\textbf{FIXME: for each species, determine worst IV (IV that when taken to max level for a league has the lowest geommean, and difference between it and best}

\section{Team assembly}
Does this extend to teams?
At first, it seems that it might: after all, a team's overall fitness
 depends on the entire opposing team, involving still more unknowns.
We will see that the team mechanism actually allows us to mitigate certain
 unknowns, while adding new ones.
Comparing teams is highly nontrivial, and this greater complexity admits
 further opportunities for strategy and cunning.

It's generally very bad if all three team members are weak to the same type.
Ideally, no two members share a type weakness.

Lead with a species having few/uncommon weaknesses, to avoid having to make the first switch.

You'll use fast attacks more than anything else, so their quality is critical.
There is little use for a Pokémon with bad fast attacks.
Attacks of shorter duration might be thought preferable to those requiring
  several turns, due to greater responsiveness.
In truth, experienced players ought comprehend the universe of possibilities
  within the timescale of a single fast attack.
Attacks of longer duration allow the opponent to blunder with regard to the
  timing of their charged attacks, but experienced players will avoid
  such errors.
Short attacks credit damage more quickly than long ones.
There's no general rule: fast attack selection comes down to attack type
  (both for effectiveness and STAB), the Pokémon's charged attacks,
  and even the $\frac{P \times Eff_A}{Eff_D}$ of expected opponents.
Some Pokémon can learn two fast attacks of the same type and duration,
  where one is strictly inferior to the other in terms of power and
  energy.
Such garbage attacks ought be immediately changed with a Fast TM\@.

Each Pokémon ought know two charged attacks.
Ought they be of different types?
Some say that if they are of the same type, you can select for either time or power, depending on context.
This remains true, however, if they are of different types; the power simply depends on
  the different type relations.
It might indeed be that the more expensive charged move with higher DPE is no
  longer advantageous (if it is ineffective, and the other move is not),
  but it might also be the case that it is even more advantageous,
  or the cheaper move is now more powerful.
Either way, I consider wider type effectiveness to be more important than gains in DPE;
  after all, it is rare that two quality charged moves are separated by a DPE factor
  of 1.6 \textbf{verify this}!
It is best if they are not both ineffective against any typing, and one or the
  other is effective against many typings.
Remember, unlike defense, adding attack types can only be beneficial.
Pokémon that can't learn two good charged attacks having distinct types are best avoided.

  \textbf{FIXME: generate table of such}

\textbf{FIXME...}

\chapter{CP-bounded team selection}
\label{chap:bounded}
Several leagues (and some Research tasks) place a ceiling on the CP of
 participating Pokémon.
We have established that CP is a poor proxy for PvP performance, and thus
 expect that we might find strong candidates based on where CP fails.
If CP does not accurately represent a given Pokémon's chance to win, we ought
 be able to exploit this by assembling teams using undervalued Pokémon.
We first take the sorted list of Pokémon developed in \autoref{chap:unbounded},
 and remove all entries whose CP exceed the CP ceiling.
We then select a team using the strategies of that chapter. Since
 that list was sorted according to a general strength function, it
 ought apply in this reduced context.


\chapter{A worked example of team selection}
\label{chap:example}

Let's consider the Fossil Cup of summer 2025, which began while I was writing this book.
Like the Great League, the Fossil Cup has a CP1500 bound, and it
  restricts participants to those with Rock, Steel, or Water in their typings.
This limits teams to 51 of the 171 typings (29.8\%): the three monotypes, seventeen dualtypes
 involving Rock, sixteen dualtypes involving Steel (we already counted Rock+Steel), and
 fifteen dualtypes involving Water (we already counted Rock+Water and Steel+Water).
Consulting \autoref{table:defenders}, we see these 51 typings populated by 267 total Pokémon.
The biggest typing by far is pure Water, with 66.
Four typings are unpopulated: Fire+Water, Normal+Steel, Normal+Rock, and Ghost+Rock.

One hundred matches in, I was doing alright (despite far from optimal IVs)
  with the following team, formed largely off vibes:
\begin{center}
  \begin{tabular}{llrrrr}
    Pokémon & IV@Level (CP) & $Eff_A$ & $Eff_D$ & MHP & $\sqrt[3]{BP}$\\
    \Midrule
    Greninja & 15/14/12@20 (1500) & 142.18 & 99.17 & 112 & 115.27 \\
    Magcargo & 15/13/12@32.5 (1497) & 115.01 & 152.35 & 111 & 124.82 \\
    Gastrodon & 0/3/15@25.5 (1495) & 114.01 & 98.49 & 174 & 125.01 \\
  \end{tabular}
\end{center}
With attacks (Power+Energy@Turns and Power@Energy in parens):
\begin{center}
  \begin{tabular}{llp{.5\textwidth}}
    Greninja & Water Shuriken (7.2+14@3) & Hydro Cannon (100@40),\newline Night Slash (50@35)\\
    Magcargo & Incinerate (24+20@5) & Overheat (156@55),\newline Stone Edge (100@55)\\
    Gastrodon & Mud-Slap (14.4+10@3) & Earth Power (108@55)\\
  \end{tabular}
\end{center}
I was having particular trouble with teams making use of Jellicent, Quagsire,
  Lucario, and Poliwrath, all members of the ``meta'' (elite Cup participants
  as judged by the community).
How could I improve?

\section{Boosting the numbers}
Without changing my choice of species, the following configurations are desirable
  in the context of optimizing for $\sqrt[3]{BP}$:
\begin{center}
  \begin{tabular}{llrrrr}
    Pokémon & IV@Level (CP) & $Eff_A$ & $Eff_D$ & MHP & $\sqrt[3]{BP}$\\
    \Midrule
    Greninja & 3/15/12@20 (1500) & 138.35 & 102.23 & 115 & 117.60\\
    Magcargo & 0/15/14@38.5 (1498) & 108.67 & 161.05 & 118 & 127.34\\
    Gastrodon & 1/15/14@24.5 (1500) & 112.41 & 104.47 & 170 & 125.92\\
  \end{tabular}
\end{center}
Of course, it's impossible to modify IVs.
Even if I could, the benefits are meager: a 2.02\% increase for Greninja,
  2.02\% to Magcargo, and a mere 0.73\% for Gastrodon, all of it coming
  in defense and HP at the expense of a bit of attack.
The means themselves are unspectacular, but not bad: Gastrodon and Magcargo are up there
  in the thick of things, and Greninja is blasting away with that Hydro Cannon every nine turns.

\section{Covering the typings}
Remember what I said earlier: it is better to be effective against a type than for that
  type to be ineffective against you.
We want to maximize the number of relevant typings against which we are effective or
  very effective, and minimize the number of types which are effective or very effective against us.
Only as a secondary consideration need we minimize the number of relevant
  typings against which we are ineffective, and maximize the number of types
  which are ineffective against us.
Ideally we want to \textit{systematically force} an effective attack from each
  of our Pokémon against all possible typings.

As noted in \autoref{sec:typeleagues}, there exist two minimal coverings of the forty-seven
  remaining typings, each of size six:
\begin{center}
\begin{tabular}{c}
 Bug, Fairy, Fire, Grass, Ground, Rock\\
 Bug, Dragon, Fire, Grass, Ground, Rock\\
\end{tabular}
\end{center}
Bug isn't known for high DPT\@. Infestation manages two while achieving four EPT,
 but requires three turns. Bug Bite does three on both in a single turn.

Grass and Ground make up a cover of the core types, as do Fighting and Grass.
All three of these types (and only these three types) are strong against two of the three
  types making up the population.
Fighting and Ground are standard against the third type (Water), while Grass is ineffective
  against its third type (Steel).
Due to the huge number of potential participants with Water in their typing, we pretty
  much need Grass.

It is impossible to cover Grass's ineffectiveness against Steel with either Fighting or Ground,
  due dualtypes of Steel with any of Fighting or Ground's many weaknesses.
Similarly, Grass's strength against Rock and Water can be undone by dualtyping them with a Grass weakness.
We need something strong where Grass is weak.
Something like Rock.
Rock is ineffective against Steel, but standard against our other two cores.
Meanwhile, it's strong against three of Grass's weaknesses outside the core (Bug, Fire, and Flying),
  leaving only Water+Poison, Water+Grass, Water+Dragon, Rock+Poison, Rock+Grass, and Rock+Dragon
  as standard.
We're ineffective against anything with Steel except for Steel+Ground, Steel+Rock, and Steel+Water, which
  Grass can hit for standard damage.

  Poison+Water: 6 (Tentacool, Tenracruel, Qwilfish, Mareanie, Toxapex, Skrelp)
  Grass+Water: 3 (Lombre, Lotad, Ludicolo)
  Dragon+Water: 3 (Kingdra, Palkia, Origin Form Palkia)
  Poison+Rock: 1 (Nihilego)
  Grass+Rock: 2 (Lileep, Cradily)
  Dragon+Rock: 2 (Tyrunt, Tyrantrum)
  Steel+Fairy: 6
  Steel+Flying: 3
  Steel+Ghost: 1
  Steel+Dragon: 2
  Steel+Poison: 2
  Steel: 9
  Steel+Electric: 4
  Steel+Fire: 1
  Steel+Bug: 7
  Steel+Fighting: 3
  Steel+Normal: 0
  Steel+Dark: 3
  Steel+Grass: 3
  Steel+Psychic: 8
  Steel+Ice: 2

If our three Pokémon all have one Grass and one Rock charged attack,
  all three of them have a charged attack strong against all but
  71 of 267 potential opponents (73.4\%).
Typings with Grass or Rock in them will get STAB on one of the two attacks.
We do no better than standard against 17 (6.4\%), and both attacks are
  ineffective against 54 (20.2\%), among them most of anything involving
  Steel.
Rock gets us into the tournament, while Grass would have to pair with Steel,
  Rock, or Water.
Rock opens up weaknesses to Fighting, Grass, Ground, Steel, and Water.


\chapter{Battle strategy}
\label{chap:strategy}
One of the greatest advantages one can gain is knowledge.
Know Pokémon well, especially those on your team and those you regularly see.
Know what moves they can learn, and how hard they throw them (Power and $Eff_A$).
Knowing their bulk ($Eff_D$ times MHP) can help decide which charged attack to use.
Know type relations cold, and be able to calculate type effectiveness on the fly.
The flipside of this is exploitation of ignorance.
Opposing Trainers are less likely to know Pokémon outside the ``meta'',
 especially those with uncommon typings.
A Pokémon can only know two charged attacks at a time, but if
 chosen from a move pool of three or more charged, even an informed opponent
 can't know what's coming until it's thrown.

\begin{tcolorbox}[enhanced,title=A Tip regarding battle UI,halign title=flush center]
Without sufficient energy, pressing the charged attack control causes a fast
 attack to be thrown.
If you know you want to throw a charged attack as soon as possible, just press
 its control until thrown.
\end{tcolorbox}

Having two charged attacks is a tremendous advantage over only one.
If they are different types, you can gain type advantage over a wide range of opponents.
Since you choose which one to throw on the fly, resistances are much less of a concern.
\textbf{FIXME}

Change to Damage inflicted by an effective Attack is at least 60\%.
Change to Damage inflicted by an ineffective Attack may be as
 low as 37.5\%.
It is probably worth accepting an ineffective Fast Attack if it
 enables an effective Charged Attack, so long as you actually
 get that Charged Attack off.
Early in the battle, this might trick the opponent into leaving
 off a Shield, allowing a \textbf{FIXME}.

\textbf{Swap leadership and safe swaps}

\textbf{shield baiting}

\textbf{optimal charged move timing}

\chapter{Simulation}
\label{chap:simul}

It is not believed that simple closed form solutions exist describing
 the set of reasonable games between two arbitrary opponents (see \autoref{chap:unbounded}),
 let alone two teams.
For the highest level of predictive power, we must turn to machine-aided simulation.
Here we are lucky, for unlike most contests, Pokémon GO is concretely finite and discrete.
Its time evolves in half-second turns, each of which presents a small number of choices
  to both Trainers.
The use of random numbers is furthermore minimal.
Finally, it never diverges, instead advancing always towards a conclusion.
It is thus feasible to exhaustively simulate matches, examining every possible choice
  a Trainer might make, and the results.

\appendix
\chapter{Optimal League configurations}
\label{chap:optimal}
Forms\footnote{Forms disallowed in League play are not listed.} are sorted by $\sqrt[3]{BP}$, the harmonic mean of $Eff_A$, $Eff_D$, and MHP\@.
This ought not be interpreted to imply that a higher $\sqrt[3]{BP}$ ought
  defeat a lower one; other issues, such as type effectiveness and attack
  choice, are usually more important.
It is possible for multiple sets of IVs to score equally due to the discrete nature
  of MHP\@.
In this case, both optimal sets are listed (there are never more than two).

$A\%$ is the optimal configuration's advantage in terms of $\sqrt[3]{BP}$
  over the pessimal ``reasonable'' configuration, i.e. some IV
  and the highest level attainable given the CP bound.
For CP1500 (\autoref{table:cp1500}), when a form can max out, i.e. is not restricted by the CP bound,
  the maximum advantage tends to be 12\%--15\%, with weaker Pokémon usually
  seeing larger advantages.
It otherwise quickly drops to 4\% or less.
In CP2500 (\autoref{table:cp2500}), the advantage available from IVs is even smaller, as we would expect.
For maxed out Pokémon, A\% ranges from 6\%--11\%, increasing as we go down the table.
It is otherwise generally less than 3\%.
\textbf{FIXME: this is all invalidated by move to arithmetic mean!}

\autoref{table:cp1500}
\textbf{FIXME: add shadow forms}
\input{out/cp1500}
\input{out/cp2500}
\chapter{Species by typings}
\label{chap:speciesbytype}
Forms are sorted by typing, and within typing by Pokédex number.
I provide the base stats and their geometric mean, evolutionary line,
 Pokédex number, an image, cost to purify and to teach a second charged
 move, walk distance (\autoref{subsec:getcandy}), catch and flee rates,
 region, generation, optimal IV+level configurations for CP1500-
 and CP2500-bounded play, weather and Mega boost, and whether the
 form is available as Shiny and/or Shadow.
Forms have been combined when distinguished only visually.

\textbf{FIXME: need index by name}

\textbf{infobox for each species still needs: attacks, genus, evolves from, evolves to,
           requirements to evolve, catch rate, flee rate, region, generation, shadow?, dmax?,
           cost to purify/teach, walk distance, weather boost, candy/items to evolve,
           whether mega etc. forms exist}
\chapter{Species}

\begin{table}
\begin{tabular}{l l l l }
Level & CPM & Level & CPM \\
\Midrule

\end{tabular}
\caption{Combat Power Multiplier by half-level}
\label{table:cpm}
\end{table}

\section{Shadow Pokémon}

\chapter{Users of attacks}
\label{chap:attackemployers}
It is often useful to build a team around attacks, rather than Pokémon.
This appendix lists the Pokémon which can learn and use a given attack.
Pokémon in italics will \textit{not} have STAB with the attack.

\textbf{FIXME: need index attacks by name}

\section{Fast attacks}
\label{sec:usersfast}
Fast attacks are sorted by type, then duration, then by the product of DPT and EPT\@.
\input{out/fastusers}

\section{Charged attacks}
\label{sec:userscharged}
Charged attacks are sorted by type, then DPE\@.
\input{out/chargedusers}

\chapter{Permanent evolutions}
\label{chap:allevolutions}
\input{out/evolutions}

\backmatter
\clearpage
\openany
% page number gobbling ought be in effect, but just to be safe...
\pagestyle{empty}
%\hypertarget{author}{}%
%\bookmark[dest=author]{the author}
\chapter{The Author}
%\bookmark{the author}
\begin{figure}[!tbp]
  \centering
  \includegraphics[width=\textwidth]{images/arrayfire.jpg}
\end{figure}
\bigskip
\noindent\textbf{\theauthor} a/k/a \textbf{dank} holds numerous degrees from Georgia Tech, and has worked
at Nvidia, Google, Intel, and Microsoft, besides founding several companies.
He lives in Atlanta, and is a Debian Developer.\\
\\
\noindent{}Check out his 2024 novel, \textit{\href{https://www.amazon.com/midnights-simulacra-Nick-Black/dp/B0CSVJZB4R}{midnight's simulacra}}.
He has also written an esoteric and specialized textbook, which is not
 likely to be of interest to you: \textit{\href{https://www.amazon.com/Hacking-Planet-Notcurses-Character-Graphics/dp/B086PNVNC9}{Hacking the Planet with Notcurses: A Guide to TUIs and Character Graphics}.}

\bigskip\noindent{}He can be found at \url{https://nick-black.com}.

\end{document}
