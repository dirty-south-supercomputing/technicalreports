\chapter{Types\label{chap:types}}
\nopagecolor{}There are eighteen types, each with a representative icon.
In trainer battles, the types of active Pokémon are indicated
 using these icons.
It is absolutely necessary that you immediately recognize them.

\begin{table}[ht!]
\centering
\begin{tabular}{c c c c c c c c c}
\includegraphics[scale=.25]{images/bug.png} &
\includegraphics[scale=.25]{images/dark.png} &
\includegraphics[scale=.25]{images/dragon.png} &
\includegraphics[scale=.25]{images/electric.png} &
\includegraphics[scale=.25]{images/fairy.png} &
\includegraphics[scale=.25]{images/fighting.png} \\
Bug & Dark & Dragon & Electric & Fairy & Fighting \\
\includegraphics[scale=.25]{images/fire.png} &
\includegraphics[scale=.25]{images/flying.png} &
\includegraphics[scale=.25]{images/ghost.png} &
\includegraphics[scale=.25]{images/grass.png} &
\includegraphics[scale=.25]{images/ground.png} &
\includegraphics[scale=.25]{images/ice.png} \\
Fire & Flying & Ghost & Grass & Ground & Ice \\
\includegraphics[scale=.25]{images/normal.png} &
\includegraphics[scale=.25]{images/poison.png} &
\includegraphics[scale=.25]{images/psychic.png} &
\includegraphics[scale=.25]{images/rock.png} &
\includegraphics[scale=.25]{images/steel.png} &
\includegraphics[scale=.25]{images/water.png} \\
Normal & Poison & Psychic & Rock & Steel & Water \\
\end{tabular}
\caption{The 18 base types\label{table:basetypes}}
\end{table}

Each attack has a single type (\autoref{chap:attacks}).
Each species (more correctly each form, as typing can vary across a species; see \autoref{chap:species})
  has a \textit{typing} of either one or two distinct types.
The attacker benefits if it has a type in common with the attack being used:
  the Same Type Attack Bonus (STAB).
The relationship between attack type and defender typing is more complex.
For each type, the attack type can be \textit{very ineffective},
  \textit{ineffective}, \textit{standard}, or \textit{effective}.
These are mapped to −2, −1, 0, and 1.
Going the other way, a type is \textit{very weak}, \textit{weak},
  \textit{standard}, or \textit{strong} against each attack type;
  we will see that a typing can also be \textit{extremely weak} or
  \textit{very strong}.
For each member of the defender's typing, determine the attack type's effectiveness
  on that type, and add the results for net effectiveness.
See \autoref{sec:typemult} for a quantitative treatment of the effects.

There are 324 (18 × 18) total type relationships (\autoref{table:relations}),
  of which 204 (63\%) are standard.
That leaves 120 type relations giving either attacker or defender an advantage,
  one that can easily decide a battle\footnote{It has been said that Pokémon GO PvP is ``essentially Rock, Paper, Scissors.''}.
These relationships must also be memorized.

\input{out/typerels}

Eleven of the eighteen types are self-active.
Of these, only Dragon and Ghost are self-effective.
Dark, Electric, Fire, Grass, Ice, Poison, Psychic, Steel, and Water are all self-ineffective.%\footnote{Fire, Ice, and Water are easily enough remembered together.

Only eight relationships are very ineffective:
Normal → Ghost,
Ghost → Normal,
Fighting → Ghost,
Poison → Steel,
Ground → Flying,
Electric → Ground,
Psychic → Dark,
and Dragon → Fairy.

\section{Memorizing the type relations}
I memorized the type system fairly easily after developing the following seven graphs.
Other people use different methods.
Whenever possible, remember two relationships as a single bidirectional relation;
  this reduces sixty-two facts to thirty-one.
No types are mutually effective.
Only one pair of types is mutually ineffective (Bug↔Fighting),
  and only one pair is mutually very ineffective (Ghost↔Normal).
When one type $V$ is ineffective against type $K$, but $K$ is strong
 against $V$, I say $K$ ``kills'' $V$, or
 (if $V$ is very ineffective against $K$) ``slaughters'' $V$.
There are only three slaughter relations: Fairy → Dragon\footnote{Fairy was
  introduced in Generation VI in large part to balance Dragon.}, Dark → Psychic,
  and Ground → Electric.

\begin{figure}[h!]
\begin{minipage}[t]{0.5\textwidth}
\centering
\includegraphics[scale=.25]{out/circo/nature.dot.png}
\caption{17 natural relations\label{fig:natural}}
\end{minipage}
\begin{minipage}[t]{0.5\textwidth}
\centering
\includegraphics[scale=.25]{out/circo/rational.dot.png}
\caption{16 rational relations\label{fig:rational}}
\end{minipage}
\end{figure}
\noindent{}Poison kills Grass, which kills Ground, which kills Poison.
Rock is strong against Bug, which kills Grass, which is strong against Rock.
Ground kills Rock, but is weak against Bug.
Bug is ineffective against Poison, which is ineffective against Rock (\autoref{fig:natural}).
Psychic kills Fighting, which kills Dark, which slaughters Psychic.
Dark kills Ghost, which is strong against Psychic.
Fighting (unique among all types) is strong against Normal,
  but very weak against Ghost.
Normal and Ghost are very weak against one another (\autoref{fig:rational}).

\begin{figure}[ht]
\centering
\includegraphics[scale=.25]{out/circo/phases.dot.png}
\caption{The 17 elemental relations\label{fig:elemental}}
\end{figure}
\noindent{}Rock kills Flying, and Normal is weak against Rock.
Rock and Water kill Fire, which kills Ice, which kills nothing on this graph nor any other.
Electric is strong against Water, which is strong against Rock, which is strong against Ice,
 which on \autoref{fig:elemental} is strong against nothing.
\clearpage

\begin{figure}[t!]
\centering
\includegraphics[scale=.25]{out/circo/dragon.dot.png}
\caption{The 8 Dragon relations\label{fig:dragon}}
\end{figure}
\noindent{}Fairy famously slaughters Dragon.
Ice is strong against Dragon.
Grass, Electric, Water, and Fire are all weak against it (\autoref{fig:dragon}).

\begin{figure}[h!]
\centering
\includegraphics[scale=.25]{out/circo/steel.dot.png}
\caption{The 20 Steel relations\label{fig:steel}}
\end{figure}
\noindent{}Ground and Fighting are strong against Steel, and Fire kills Steel.
Most everything else is weak against Steel, especially
 Poison, which is very ineffective against it.
Steel kills Ice, Rock, and Fairy---if Steel is strong against something, it kills it.
Steel is weak against Water and Electric.
Only two types have no relationship with Steel: Dark and Ghost (\autoref{fig:steel}).
As Josef Stalin said, ``it's good to be Steel''.

\begin{figure}[h!]
\centering
\includegraphics[scale=.25]{out/dot/death.dot.png}
\caption{The 24 relations of death\label{fig:death}}
\end{figure}
\noindent{}The Graph of Death (\autoref{fig:death}).
Poison kills Fairy, which kills Dark.
Fairy also kills Fighting, which kills Rock.
Ground slaughters Electric, which kills Flying.
Flying strikes back, raining death from above and killing Fighting, Bug, and Grass.
Fire likewise kills Bug and Grass.
Grass kills Water (it is the only type strong against Water besides Electric).

\begin{figure}[ht]
\centering
\includegraphics[scale=.25]{out/dot/jumble.dot.png}
\caption{The 18 remaining interactions\label{fig:jumble}}
\end{figure}
\noindent{}The remaining interactions don't have any real structure, and must simply be
memorized as they are (\autoref{fig:jumble}).

Every type except Normal is killed by at least one other type.
Bug is killed by two types (Fire and Flying), as is Fire (Water and Rock).
Rock is killed by Steel, Fighting, and Ground.
Grass is killed by Fire, Flying, Bug, and Poison.
The killing is more concentrated: Ghost, Ice, and Normal don't kill
  any other type, while mighty Fire kills four.
It is important to remember that an opponent of some type might
  employ attacks of some other type, in which case the bidirectional
  relationship doesn't apply.
Still, recognizing kill situations can get you
  far in life and Pokémon GO\@.

\section{Typing\label{sec:dualtypes}}
Every Pokémon form is singly or doubly typed.
For dualtypes, ordering has no impact on function.
\[ C(18, 2) = \binom{18}{2} = \frac{18!}{2! \cdot 16!} = 153 \]
There are thus \textit{functionally} 153 dual types in addition to the 18 base types, for a total of 171.
We could also simply sum 1 through 18:
\[ \sum_{i=1}^{18} i = \frac{18 \cdot 19}{2} = 171 \]
Nine dual types are currently unpopulated (Normal/Steel, Normal/Ice, Normal/Rock,
 Normal/Bug, Poison/Ice, Ground/Fairy, Rock/Ghost, Bug/Dragon, Fire/Fairy),
 leaving 162 defender typings to consider.
Dual typing widens the type relation range, adding the possibilities
 of −3 and 2 (−4 is not possible, because no type is very ineffective against
 two different types).
It furthermore vastly expands the type effectiveness matrix,
 with 3,078 (18 × 171) relations of which only 1,490 (48.4\%) are standard.
Thankfully, these needn't be memorized, as they can all be calculated
 using the base relations matrix.
\input{out/dualtypes.tex}
Given $E_{n}, −3 \le n \le 2$ where $E_n$ is the number of types with
  a net relation of $n$ against a typing, we define \DRA\@:
\[  \DRA = \sum_{n=−3}^{2} \frac{E_{n}1.6^n}{18} \]
Completely unexpectedly, the defensive typings with highest DRA include Steel,
  with the top fiften spots all making use of it.
Don't be fooled by a metric like DRA, though; you will not be seeing
  randomly typed attacks.
Opposing Trainers will build their teams and make substitutions based on
  expected and demonstrated typing.
None of Steel's resistances matter when you're being hit by one of its
  weaknesses.
\input{out/dualsummaries.tex}
There are twenty triple resistances, five of which are against Fighting attacks (\autoref{table:triples}).
Ghost/Rock, triply resistant against Normal, is currently unpopulated.
There are sixty double weaknesses, including eight types with two such weaknesses (\autoref{table:doubledoubles}).
\begin{table}
\centering
\begin{tabular}{cc}
Dragon → Fairy/Steel & Ground → Bug/Flying \\
Electric → Dragon/Ground & Ground → Flying/Grass \\
Electric → Electric/Ground & Psychic → Dark/Psychic \\
Electric → Grass/Ground & Psychic → Dark/Steel \\
Fighting → Bug/Ghost & Normal → Ghost/Rock \\
Fighting → Fairy/Ghost & Normal → Ghost/Steel \\
Fighting → Flying/Ghost & Poison → Ghost/Steel \\
Fighting → Ghost/Poison & Poison → Ground/Steel \\
Fighting → Ghost/Psychic & Poison → Poison/Steel \\
Ghost → Dark/Normal & Poison → Rock/Steel \\
\end{tabular}
\caption{The twenty triple resistances\label{table:triples}}
\end{table}
\input{out/doubledoubles}
\section{Weather boosting\label{sec:weather}}
Local weather ``boosts'' its associated types (\autoref{table:weather}), affecting attack strength
 in PvE (\autoref{sec:mbmult}),
 spawn rates (\autoref{sec:spawns}), and catching (\autoref{sec:catch}).
The game's assessment of your weather is communicated via an icon on the Map View.
\begin{table}
\input{out/weather}
\end{table}
