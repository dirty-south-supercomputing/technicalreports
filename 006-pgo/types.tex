\chapter{Types}
There are eighteen Types, each with a representative icon.
In Trainer Battles, the Types of active Pokémon are displayed
 using these icons, so it is important to memorize them.
\begin{table}[h]
  \begin{tabular}{c c c c c c}
  \includegraphics[scale=.25]{images/bug.png} &
    \includegraphics[scale=.25]{images/dark.png} &
    \includegraphics[scale=.25]{images/dragon.png} &
    \includegraphics[scale=.25]{images/electric.png} &
    \includegraphics[scale=.25]{images/fairy.png} &
    \includegraphics[scale=.25]{images/fighting.png} \\
  Bug & Dark & Dragon & Electric & Fairy & Fighting \\
  \includegraphics[scale=.25]{images/fire.png} &
    \includegraphics[scale=.25]{images/flying.png} &
    \includegraphics[scale=.25]{images/ghost.png} &
    \includegraphics[scale=.25]{images/grass.png} &
    \includegraphics[scale=.25]{images/ground.png} &
    \includegraphics[scale=.25]{images/ice.png} \\
  Fire & Flying & Ghost & Grass & Ground & Ice \\
  \includegraphics[scale=.25]{images/normal.png} &
    \includegraphics[scale=.25]{images/poison.png} &
    \includegraphics[scale=.25]{images/psychic.png} &
    \includegraphics[scale=.25]{images/rock.png} &
    \includegraphics[scale=.25]{images/steel.png} &
    \includegraphics[scale=.25]{images/water.png} \\
  Normal & Poison & Psychic & Rock & Steel & Water \\
\end{tabular}
\end{table}
Each Species has either one or two distinct Types.
Each Attack has a single Type.
It is advantageous if either of an attacker's Types is the same Type as
 the Attack being used.
The relationship between Attack Type and defender typing is more complex.
For each Type, the Attack Type can be \textit{very ineffective},
 \textit{ineffective}, \textit{standard}, or \textit{very effective}.
These are mapped to -2, -1, 0, and 1.
For each Type of the defender, determine the Attack Type's effectiveness
 on that Type, and add the results.
This is the total type effectiveness of the Attack Type on that Species.

There are 324 (18 * 18) total Type relationships, of which 208 are standard.
That leaves 116 Type relations giving either attacker or defender an advantage,
 one that can easily define the winner of a battle.
These relationships must also be memorized.

\begin{table}[h]
  \renewcommand{\arraystretch}{0.5}
  \setlength{\tabcolsep}{1pt}
  \begin{tabular}{c c c c c c c c c c c c c c c c c c c c}
\captionlistentry[table]{Type relations}
  &
  \includegraphics[scale=.1]{images/bug.png} &
    \includegraphics[scale=.1]{images/dark.png} &
    \includegraphics[scale=.1]{images/dragon.png} &
    \includegraphics[scale=.1]{images/electric.png} &
    \includegraphics[scale=.1]{images/fairy.png} &
    \includegraphics[scale=.1]{images/fighting.png} &
  \includegraphics[scale=.1]{images/fire.png} &
    \includegraphics[scale=.1]{images/flying.png} &
    \includegraphics[scale=.1]{images/ghost.png} &
    \includegraphics[scale=.1]{images/grass.png} &
    \includegraphics[scale=.1]{images/ground.png} &
    \includegraphics[scale=.1]{images/ice.png} &
  \includegraphics[scale=.1]{images/normal.png} &
    \includegraphics[scale=.1]{images/poison.png} &
    \includegraphics[scale=.1]{images/psychic.png} &
    \includegraphics[scale=.1]{images/rock.png} &
    \includegraphics[scale=.1]{images/steel.png} &
    \includegraphics[scale=.1]{images/water.png} &
    \\
    \includegraphics[scale=.1]{images/bug.png} & & -1 & -1 & -1 & & & & -1 & -1 & -1 & & 1 & & 1 & & & 1 & -1 \\
    \includegraphics[scale=.1]{images/dark.png} & & -1 & & & & & & 1 & & & & & & 1 & & & -1 & -1 \\
    \includegraphics[scale=.1]{images/dragon.png} & & & & & & & & & -1 & & & & & & & 1 & & -2 \\
    \includegraphics[scale=.1]{images/electric.png} & & & 1 & & -2 & & & & & & 1 & -1 & -1 & & & -1 & & 0 \\
    \includegraphics[scale=.1]{images/fairy.png} & & 1 & & -1 & & & & & -1 & -1 & & & & & & 1 & 1 & 0 \\
    \includegraphics[scale=.1]{images/fighting.png} & 1 & & -1 & -1 & & 1 & -1 & -2 & 1 & & & & & -1 & 1 & -1 & 1 & -1 \\
    \includegraphics[scale=.1]{images/fire.png} & & & & & & -1 & 1 & & 1 & -1 & -1 & 1 & & & 1 & -1 & & 0 \\
    \includegraphics[scale=.1]{images/flying.png} & & 1 & & & & & 1 & & -1 & & & 1 & -1 & & & & & 0 \\
    \includegraphics[scale=.1]{images/ghost.png} & -2 & & & & & & & 1 & & & & & & 1 & & & -1 & 0 \\
    \includegraphics[scale=.1]{images/grass.png} & & & -1 & -1 & 1 & 1 & -1 & & -1 & -1 & 1 & -1 & & & & -1 & & 0 \\
    \includegraphics[scale=.1]{images/ground.png} & & & -2 & 1 & & 1 & -1 & & 1 & 1 & & -1 & 1 & & & & & 0 \\
    \includegraphics[scale=.1]{images/ice.png} & & & 1 & & 1 & & & & -1 & -1 & -1 & 1 & & & -1 & 1 & & 0 \\
    \includegraphics[scale=.1]{images/normal.png} & & & & & & -1 & & -2 & -1 & & & & & & & & & \\
    \includegraphics[scale=.1]{images/poison.png} & & & & -1 & -1 & -1 & & -1 & -2 & & & 1 & & & & & & 1 \\
    \includegraphics[scale=.1]{images/psychic.png} & & 1 & & 1 & & & & & -1 & & & & & -1 & & & -2 & &  \\
    \includegraphics[scale=.1]{images/rock.png} & & -1 & 1 & & -1 & & 1 & & -1 & 1 & & & & & 1 & & & 0 \\
    \includegraphics[scale=.1]{images/steel.png} & & & & & 1 & & & -1 & -1 & -1 & & -1 & & 1 & & & 1 \\
    \includegraphics[scale=.1]{images/water.png} & & & & 1 & 1 & & & & 1 & -1 & -1 & & & & -1 & & 0 \\
\end{tabular}
\end{table}

Only eight relationships are very ineffective:
Normal → Ghost,
Fighting → Ghost,
Poison → Steel,
Ground → Flying,
Ghost → Normal,
Electric → Ground,
Psychic → Dark,
Dragon → Fairy.

\section{Memorizing the type relations}
I've found it useful to break the 116 relationships of impact into a few directed graphs.
A thick green edge indicates effectiveness.
A thin red or black line indicates an ineffective or very ineffective relationship.

\begin{figure}[h]
\centering
\includegraphics[scale=.25]{out/circo/nature.dot.png}
\caption{The Nature relations}
\end{figure}
Poison and Grass are both self-ineffective.
Ground kills Rock and Poison.
Grass kills Ground.
Poison kills Grass.
Bug kills Grass.

\begin{figure}[h]
\centering
\includegraphics[scale=.25]{out/circo/phases.dot.png}
\caption{The Elemental relations}
\end{figure}
All are self-ineffective except Rock.
Water kills Fire.
Fire kills Ice.
Rock kills Fire.

\begin{figure}[h]
\centering
\includegraphics[scale=.25]{out/circo/rational.dot.png}
\caption{The Rational relations}
\end{figure}
Dark and Psychic are self-ineffective, while Ghost is self-effective.
Psychic kills Fighting.
Fighting kills Dark.
Dark kills Ghost, and slaughters Psychic.

\begin{figure}
\centering
\includegraphics[scale=.25]{out/circo/dragon.dot.png}
\caption{The Dragon relations}
\end{figure}
Dragon is self-effective.
Fairy slaughters Dragon.

\begin{figure}
\centering
\includegraphics[scale=.25]{out/circo/steel.dot.png}
\caption{The Steel relations}
\end{figure}
Steel is self-ineffective.
Fire kills Steel.
Steel kills Ice, Rock, and Fairy.

\begin{figure}[h]
\centering
\includegraphics[scale=.25]{out/dot/death.dot.png}
\caption{The Death Graph}
\end{figure}
The Graph of Death.
Fairy kills Dark and Fighting.
Poison kills Fairy.
Fighting kills Rock.
Flying kills Fighting, Bug, and Grass.
Fire kills Bug and Grass.
Grass kills Water.
Electric kills Flying.
Ground slaughters Electric.

\begin{figure}[h]
\centering
\includegraphics[scale=.25]{out/dot/jumble.dot.png}
\caption{Remaining interactions}
\end{figure}
The remaining interactions don't have any real structure, and simply must be
memorized as they are.
