\chapter{Types}
\label{chap:types}
There are eighteen Types, each with a representative icon.
In Trainer Battles, the Types of active Pokémon are displayed
 using these icons, so it is important to memorize them.
\begin{table}[h]
  \begin{tabular}{c c c c c c}
  \includegraphics[scale=.25]{images/bug.png} &
    \includegraphics[scale=.25]{images/dark.png} &
    \includegraphics[scale=.25]{images/dragon.png} &
    \includegraphics[scale=.25]{images/electric.png} &
    \includegraphics[scale=.25]{images/fairy.png} &
    \includegraphics[scale=.25]{images/fighting.png} \\
  Bug & Dark & Dragon & Electric & Fairy & Fighting \\
  \includegraphics[scale=.25]{images/fire.png} &
    \includegraphics[scale=.25]{images/flying.png} &
    \includegraphics[scale=.25]{images/ghost.png} &
    \includegraphics[scale=.25]{images/grass.png} &
    \includegraphics[scale=.25]{images/ground.png} &
    \includegraphics[scale=.25]{images/ice.png} \\
  Fire & Flying & Ghost & Grass & Ground & Ice \\
  \includegraphics[scale=.25]{images/normal.png} &
    \includegraphics[scale=.25]{images/poison.png} &
    \includegraphics[scale=.25]{images/psychic.png} &
    \includegraphics[scale=.25]{images/rock.png} &
    \includegraphics[scale=.25]{images/steel.png} &
    \includegraphics[scale=.25]{images/water.png} \\
  Normal & Poison & Psychic & Rock & Steel & Water \\
\end{tabular}
\end{table}
Each Species has either one or two distinct Types.
Each Attack has a single Type.
The attacker benefits if it has a Type in common with the Attack being used.
The relationship between Attack Type and defender typing is more complex.
For each Type, the Attack Type can be \textit{very ineffective},
 \textit{ineffective}, \textit{standard}, or \textit{very effective}.
These are mapped to -2, -1, 0, and 1.
For each Type of the defender, determine the Attack Type's effectiveness
 on that Type, and add the results.
This is the total type effectiveness of the Attack Type on that Species.

There are 324 (18 * 18) total Type relationships, of which 204 are standard.
That leaves 120 Type relations giving either attacker or defender an advantage,
 one that can easily define the winner of a battle.
These relationships must also be memorized.

\begin{table}[h]
  \begin{center}
  \setlength{\tabcolsep}{1pt}
  \begin{tabular}{c c g c g c g c g c g c g c g c g c g c}
  &
  \includegraphics[width=1em]{images/bug.png} &
    \includegraphics[width=1em]{images/dark.png} &
    \includegraphics[width=1em]{images/dragon.png} &
    \includegraphics[width=1em]{images/electric.png} &
    \includegraphics[width=1em]{images/fairy.png} &
    \includegraphics[width=1em]{images/fighting.png} &
  \includegraphics[width=1em]{images/fire.png} &
    \includegraphics[width=1em]{images/flying.png} &
    \includegraphics[width=1em]{images/ghost.png} &
    \includegraphics[width=1em]{images/grass.png} &
    \includegraphics[width=1em]{images/ground.png} &
    \includegraphics[width=1em]{images/ice.png} &
  \includegraphics[width=1em]{images/normal.png} &
    \includegraphics[width=1em]{images/poison.png} &
    \includegraphics[width=1em]{images/psychic.png} &
    \includegraphics[width=1em]{images/rock.png} &
    \includegraphics[width=1em]{images/steel.png} &
    \includegraphics[width=1em]{images/water.png} &
    \\
    \includegraphics[width=1em]{images/bug.png} & & -1 & -1 & -1 & & & & -1 & -1 & -1 & & 1 & & 1 & & & 1 & -1 \\ % 10
    \rowcolor{Gray!25}
    \includegraphics[width=1em]{images/dark.png} & & -1 & & & & & & 1 & & & & & & 1 & & & -1 & -1 \\ % 5
    \includegraphics[width=1em]{images/dragon.png} & & & & & & & & & -1 & & & & & & & 1 & & -2 \\ % 3
    \rowcolor{Gray!25}
    \includegraphics[width=1em]{images/electric.png} & & & 1 & & -2 & & & & & & 1 & -1 & -1 & & & -1 & & \\ % 6
    \includegraphics[width=1em]{images/fairy.png} & & 1 & & -1 & & & & & -1 & -1 & & & & & & 1 & 1 & \\ % 6
    \rowcolor{Gray!25}
    \includegraphics[width=1em]{images/fighting.png} & 1 & & -1 & -1 & & 1 & -1 & -2 & 1 & & & & & -1 & 1 & -1 & 1 & -1 \\ % 12
    \includegraphics[width=1em]{images/fire.png} & & & & & & -1 & 1 & & 1 & -1 & -1 & 1 & & & 1 & -1 & & \\ % 8
    \rowcolor{Gray!25}
    \includegraphics[width=1em]{images/flying.png} & & 1 & & & & & 1 & & -1 & & & 1 & -1 & & & & & \\ % 5
    \includegraphics[width=1em]{images/ghost.png} & -2 & & & & & & & 1 & & & & & & 1 & & & -1 & \\ % 4
    \rowcolor{Gray!25}
    \includegraphics[width=1em]{images/grass.png} & & & -1 & -1 & 1 & 1 & -1 & & -1 & -1 & 1 & -1 & & & & -1 & & \\ % 10
    \includegraphics[width=1em]{images/ground.png} & & & -2 & 1 & & 1 & -1 & & 1 & 1 & & -1 & 1 & & & & & \\ % 8
    \rowcolor{Gray!25}
%\textbf{FIXME: ROWS ABOVE ARE WRONG}
    \includegraphics[width=1em]{images/ice.png} & & & 1 & & & & -1 & 1 & & 1 & 1 & -1 & & & & & -1 & -1 \\ % 8
    \includegraphics[width=1em]{images/normal.png} & & & & & & & & & -2 & & & & & & & -1 & -1 & \\ % 3
    \rowcolor{Gray!25}
    \includegraphics[width=1em]{images/poison.png} & & & &  & 1 &  & &  &  -1 & 1 & -1 & & & -1 & & -1 & & \\ % 7
    \includegraphics[width=1em]{images/psychic.png} & & -2 & &  & & 1 & & &  & & & & & 1 & -1 & & -1 & &  \\ % 5
    \rowcolor{Gray!25}
    \includegraphics[width=1em]{images/rock.png} & 1 &  &  & &  & -1 & 1 & 1 &  &  & -1 & 1 & & &  & & -1 & \\ % 7
    \includegraphics[width=1em]{images/steel.png} & & & & -1 & 1 & & -1 & & & & & 1 & & & & 1 & -1 & -1 \\ % 7
    \rowcolor{Gray!25}
    \includegraphics[width=1em]{images/water.png} & & & -1 & & & & 1 & & & -1 & 1 & & & & & 1 & & -1 \\ % 6
\end{tabular}
    \caption[Type relations]{Type relations. Rows attack, columns defend.}
\end{center}
\end{table}

Only eight relationships are very ineffective:
Normal → Ghost,
Fighting → Ghost,
Poison → Steel,
Ground → Flying,
Ghost → Normal,
Electric → Ground,
Psychic → Dark,
and Dragon → Fairy.
Note that the Ghost/Normal relationship is very ineffective in both directions.
On the other hand, Ground is effective against Electric, and Fairy is effective
 against Dragon, making these the most lopsided matchups between single Types.

\begin{table}[h]
  \begin{center}
  \begin{tabular}{l r r r r r r r r r r r}
    Type & -2 & -1 & 0 & 1 & DRA \\
    \Midrule
    Steel & 1 & 10 & 4 & 3 & -5 \\
    Ghost & 2 & 2 & 12 & 2 & -0.\textoverline{2} \\
    Fairy & 1 & 3 & 12 & 2 & -0.1\textoverline{6} \\
    Poison & 0 & 5 & 11 & 2 & -0.1\textoverline{6} \\
    Dragon & 0 & 3 & 10 & 5 & -0.\textoverline{1} \\
    Electric & 0 & 3 & 14 & 1 & -0.\textoverline{1} \\
    Fire & 0 & 6 & 9 & 3 & -0.\textoverline{1} \\
    Flying & 1 & 3 & 11 & 3 & -0.\textoverline{1} \\
    Water & 0 & 4 & 12 & 2 & -0.\textoverline{1} \\
    Dark & 1 & 2 & 12 & 3 & -0.0\textoverline{5} \\
    Ground & 1 & 2 & 12 & 3 & -0.0\textoverline{5} \\
    Normal & 1 & 0 & 16 & 1 & -0.0\textoverline{5} \\
    Bug & 0 & 3 & 12 & 3 & 0 \\
    Fighting & 0 & 3 & 12 & 3 & 0 \\
    Grass & 0 & 4 & 9 & 5 & 0.0\textoverline{5} \\
    Psychic & 0 & 2 & 13 & 3 & 0.0\textoverline{5} \\
    Rock & 0 & 3 & 10 & 5 & 0.\textoverline{1} \\
    Ice & 0 & 1 & 13 & 4 & 0.1\textoverline{6} \\
\end{tabular}
    \caption[Defender Type effectiveness summaries]{Defender Type effectiveness summaries (lower is better)}
  \end{center}
\end{table}

In addition to the 18 base types, there are 153 dual types (it is not possible
to dual typed using a single Type, and ordering does not matter) for a total
of 171. Nine are currently unpopulated (Normal/Steel, Normal/Ice, Normal/Rock,
Normal/Bug, Poison/Ice, Ground/Fairy, Rock/Ghost, Bug/Dragon, Fire/Fairy). When
dual typing is considered, the type relation ranges from -3 to 2 (-4 is not
possible, because no Type is very ineffective against two different Types).

% i'd like toppers explaining that 4 are against single types and 6 against all types
\begin{table}[h]
  \begin{center}
  \begin{tabular}{l r r r r r r r r r r r}
    Type & -2 & -1 & 0 & 1 & -3 & -2 & -1 & 0 & 1 & 2 & ARA \\
    \Midrule
    Ground & 1 & 2 & 10 & 5 & 2 & 12 & 27 & 65 & 55 & 10 & 1 \\
    Rock & 0 & 3 & 11 & 4 & 0 & 3 & 36 & 78 & 48 & 6 & 1 \\
    Fairy & 0 & 3 & 12 & 3 & 0 & 3 & 39 & 87 & 39 & 3 & 0 \\
    Fire & 0 & 4 & 10 & 4 & 0 & 6 & 44 & 71 & 44 & 6 & 0 \\
    Ice & 0 & 4 & 10 & 4 & 0 & 6 & 44 & 71 & 44 & 6 & 0 \\
    Water & 0 & 3 & 12 & 3 & 0 & 3 & 39 & 87 & 39 & 3 & 0 \\
    Flying & 0 & 2 & 13 & 3 & 0 & 1 & 28 & 97 & 42 & 3 & -1 \\
    Ghost & 1 & 1 & 14 & 2 & 1 & 15 & 17 & 107 & 30 & 1 & -1 \\
    Steel & 0 & 4 & 11 & 3 & 0 & 6 & 48 & 78 & 36 & 3 & -1 \\
    Dark & 0 & 3 & 13 & 2 & 0 & 3 & 43 & 96 & 28 & 1 & -1.0\textoverline{5} \\
    Psychic & 1 & 2 & 13 & 2 & 2 & 15 & 30 & 95 & 28 & 1 & -2 \\
    Fighting & 1 & 6 & 6 & 5 & 6 & 22 & 41 & 51 & 42 & 10 & -2.2\textoverline{7} \\
    Dragon & 1 & 1 & 15 & 1 & 1 & 16 & 17 & 121 & 16 & 0 & -2.\textoverline{3} \\
    Electric & 1 & 3 & 12 & 2 & 3 & 11 & 41 & 89 & 26 & 1 & -2.\textoverline{4} \\
    Bug & 0 & 7 & 8 & 3 & 0 & 21 & 63 & 57 & 27 & 3 & -4 \\
    Grass & 0 & 7 & 8 & 3 & 0 & 21 & 63 & 57 & 27 & 3 & -4 \\
    Poison & 1 & 4 & 11 & 2 & 4 & 18 & 50 & 74 & 24 & 1 & -4 \\
    Normal & 1 & 2 & 15 & 0 & 2 & 17 & 34 & 118 & 0 & 0 & -4.\textoverline{1} \\
\end{tabular}
    \caption[Attack Type effectiveness summaries]{Attack Type effectiveness summaries (higher is better)}
  \end{center}
\end{table}


\section{Memorizing the type relations}
I've found it useful to break the 120 relationships of impact into a few directed graphs.
A thick green edge indicates effectiveness.
A thin red or black edge indicates an ineffective or very ineffective relationship, respectively.

\begin{figure}[h]
\centering
\includegraphics[scale=.25]{out/circo/nature.dot.png}
\caption{The 17 Nature relations}
\end{figure}
Poison and Grass are both self-ineffective.
Ground kills Rock and Poison.
Grass kills Ground.
Poison kills Grass.
Bug kills Grass.

\begin{figure}[h]
\centering
\includegraphics[scale=.25]{out/circo/phases.dot.png}
\caption{The 14 Elemental relations}
\end{figure}
All are self-ineffective except Rock.
Water kills Fire.
Fire kills Ice.
Rock kills Fire.

\begin{figure}[h]
\centering
\includegraphics[scale=.25]{out/circo/rational.dot.png}
\caption{The 16 Rational relations}
\end{figure}
Dark and Psychic are self-ineffective, while Ghost is self-effective.
Psychic kills Fighting.
Fighting kills Dark.
Dark kills Ghost, and slaughters Psychic.

\begin{figure}
\centering
\includegraphics[scale=.25]{out/circo/dragon.dot.png}
\caption{The 9 Dragon relations}
\end{figure}
Dragon is self-effective.
Fairy slaughters Dragon.

\begin{figure}
\centering
\includegraphics[scale=.25]{out/circo/steel.dot.png}
\caption{The 20 Steel relations}
\end{figure}
Steel is self-ineffective.
Fire kills Steel.
Steel kills Ice, Rock, and Fairy.

\begin{figure}[h]
\centering
\includegraphics[scale=.25]{out/dot/death.dot.png}
\caption{The Death Graph's 24 relations}
\end{figure}
The Graph of Death.
Fairy kills Dark and Fighting.
Poison kills Fairy.
Fighting kills Rock.
Flying kills Fighting, Bug, and Grass.
Fire kills Bug and Grass.
Grass kills Water.
Electric kills Flying.
Ground slaughters Electric.

\begin{figure}[h]
\centering
\includegraphics[scale=.25]{out/dot/jumble.dot.png}
\caption{The 20 remaining interactions}
\end{figure}
The remaining interactions don't have any real structure, and simply must be
memorized as they are.

\section{Cover sets}
