\chapter{Types}
\label{chap:types}
There are eighteen types, each with a representative icon.
In trainer battles, the types of active Pokémon are indicated
 using these icons, and it is necessary to immediately recognize
 them on sight.

\begin{table}[h!]
  \begin{center}
  \begin{tabular}{c c c c c c c c c}
  \includegraphics[scale=.25]{images/bug.png} &
  \includegraphics[scale=.25]{images/dark.png} &
  \includegraphics[scale=.25]{images/dragon.png} &
  \includegraphics[scale=.25]{images/electric.png} &
  \includegraphics[scale=.25]{images/fairy.png} &
  \includegraphics[scale=.25]{images/fighting.png} \\
  Bug & Dark & Dragon & Electric & Fairy & Fighting \\
  \includegraphics[scale=.25]{images/fire.png} &
  \includegraphics[scale=.25]{images/flying.png} &
  \includegraphics[scale=.25]{images/ghost.png} &
  \includegraphics[scale=.25]{images/grass.png} &
  \includegraphics[scale=.25]{images/ground.png} &
  \includegraphics[scale=.25]{images/ice.png} \\
  Fire & Flying & Ghost & Grass & Ground & Ice \\
  \includegraphics[scale=.25]{images/normal.png} &
  \includegraphics[scale=.25]{images/poison.png} &
  \includegraphics[scale=.25]{images/psychic.png} &
  \includegraphics[scale=.25]{images/rock.png} &
  \includegraphics[scale=.25]{images/steel.png} &
  \includegraphics[scale=.25]{images/water.png} \\
  Normal & Poison & Psychic & Rock & Steel & Water \\
\end{tabular}
\end{center}
\caption{The 18 base types}
\end{table}

Each attack has a single type (\autoref{chap:attacks}).
Each species (more correctly each form, as typing can vary across a species; see \autoref{chap:species})
  has a \textit{typing} of either one or two distinct types.
The attacker benefits if it has a type in common with the attack being used:
  the Same Type Attack Bonus (STAB).
The relationship between attack type and defender typing is more complex.
For each type, the attack type can be \textit{very ineffective},
  \textit{ineffective}, \textit{standard}, or \textit{effective}.
These are mapped to -2, -1, 0, and 1.
Going the other way, a type is \textit{very weak}, \textit{weak},
  \textit{standard}, or \textit{strong} against each attack type;
  we will see that a typing can also be \textit{extremely weak} or
  \textit{very strong}.
For each member of the defender's typing, determine the attack type's effectiveness
  on that type, and add the results for net effectiveness.
See \autoref{sec:typemult} for a quantitative treatment of the effects.

There are 324 (18 × 18) total type relationships (\autoref{table:relations}),
  of which 204 (63\%) are standard.
That leaves 120 type relations giving either attacker or defender an advantage,
  one that can easily decide a battle\footnote{It has been said that Pokémon GO PvP is ``essentially Rock, Paper, Scissors.''}.
These relationships must also be memorized.

\input{out/typerels}

Eleven of the eighteen types are self-active.
Of these, only Dragon and Ghost are
  self-effective
 %\footnote{I use the mnemonic GD, as in ``Goddamnit!'', as in e.g. ``Goddamnit, I stupidly forgot Abomasnow can have Outrage, and am now less one Dragonite.''}
 .
Dark, Electric, Fire, Grass, Ice, Poison, Psychic,
 Steel, and Water are all self-ineffective%\footnote{Fire, Ice, and Water are easily enough remembered together.
%For the remainder I employ the mnemonic PEGSDARKASS\@: Poison/Psychic, Electric, Steel, DARK, and GrASS\@.
%You are encouraged to develop and use alternate mnemonics.}
.

Only eight relationships are very ineffective:
Normal → Ghost,
Ghost → Normal,
Fighting → Ghost,
Poison → Steel,
Ground → Flying,
Electric → Ground,
Psychic → Dark,
and Dragon → Fairy.
Note that the Ghost/Normal relationship is very ineffective in both directions.
On the other hand, Ground is effective against Electric, Fairy is effective
 against Dragon, and Dark is effective against Psychic, making these the most
 lopsided matchups between single types.

\section{Memorizing the type relations}
I memorized the type system fairly easily after developing the following seven graphs.
Other people use different methods.
Whenever possible, remember two relationships as a single bidirectional one.
No types are mutually effective.
  Only one pair of types is mutually ineffective (Bug↔Fighting),
  and only one pair is mutually very ineffective (Ghost↔Normal).
When one type $V$ is ineffective against type $K$, but $K$ is strong
 against $V$, I say $K$ ``kills'' $V$, or
 (if $V$ is very ineffective against $K$) ``slaughters'' $V$.
These twenty-nine relationships must be memorized and immediately recognized.
There are only three slaughter relations: Fairy → Dragon\footnote{Fairy was
  introduced in Generation VI in large part to balance Dragon.}, Dark → Psychic,
  and Ground → Electric.

\begin{figure}[h!]
  \begin{minipage}[t]{0.5\textwidth}
    \includegraphics[scale=.25]{out/circo/nature.dot.png}
    \caption{17 Natural relations}
    \label{fig:natural}
  \end{minipage}
  \begin{minipage}[t]{0.5\textwidth}
    \includegraphics[scale=.25]{out/circo/phases.dot.png}
    \caption{14 Elemental relations}
    \label{fig:elemental}
  \end{minipage}
\end{figure}
\noindent{}Poison kills Grass, which kills Ground, which kills Poison.
Ground kills Rock.
Bug kills Grass.
Rock is strong against Bug, but Ground is weak against it.
Bug is ineffective against Poison, which is ineffective against Rock (\autoref{fig:natural}).
Rock and Water kill Fire, which kills Ice, which kills nothing.
Electric is strong against Water, which is strong against Rock, which is strong against Ice,
 which once again is strong against nothing (\autoref{fig:elemental}).

\begin{figure}[ht]
\centering
\includegraphics[scale=.25]{out/circo/rational.dot.png}
\caption{16 Rational relations}
\label{fig:rational}
\end{figure}
\noindent{}Psychic kills Fighting.
Fighting kills Dark.
Dark kills Ghost, and slaughters Psychic.
Ghost is strong against Psychic.
Fighting, unique among the types, is strong against Normal (\autoref{fig:rational}).
\clearpage

\begin{figure}[t!]
\centering
\includegraphics[scale=.25]{out/circo/dragon.dot.png}
\caption{9 Dragon relations}
\label{fig:dragon}
\end{figure}
\noindent{}Fairy famously slaughters Dragon.
Ice is strong against Dragon.
Fighting, Grass, Electric, Water, and Fire are all weak against it (\autoref{fig:dragon}).
\vspace{.5in}

\begin{figure}[h!]
\centering
\includegraphics[scale=.25]{out/dot/death.dot.png}
\caption{24 relations of death}
\label{fig:death}
\end{figure}
\noindent{}The Graph of Death (\autoref{fig:death}).
Poison kills Fairy, which kills Dark.
Fairy also kills Fighting, which kills Rock.
Ground slaughters Electric, which kills Flying.
Flying strikes back, raining death from above and killing Fighting, Bug, and Grass.
Fire likewise kills Bug and Grass.
Grass kills Water (it is the only type strong against Water besides Electric).
\clearpage

\begin{figure}[h!]
\centering
\includegraphics[scale=.25]{out/circo/steel.dot.png}
\caption{The 20 Steel relations}
\label{fig:steel}
\end{figure}
\noindent{}Ground and Fighting are strong against Steel, and Fire kills Steel.
Most everything else is weak against Steel, especially
 Poison, which is very ineffective against it.
Steel kills Ice, Rock, and Fairy---if Steel is strong against something, it kills it.
Steel is weak against Water and Electric.
Only two types have no relationships with Steel: Dark and Ghost (\autoref{fig:steel}).
As Josef Stalin said, ``it's good to be Steel''.

\begin{figure}[ht]
\centering
\includegraphics[scale=.25]{out/dot/jumble.dot.png}
\caption{The 20 remaining interactions}
\end{figure}
The remaining interactions don't have any real structure, and must simply be
memorized as they are.

\begin{table}[ht]
  \begin{center}
    \begin{tabular}{llllll}
      \hline
      \multirow{3}{*}{Ice} & \multirow{3}{*}{effective} & Grass & & \\
      & & Flying & & \\
      & & Ground & effective & Fire \\
      \hline
      \multirow{4}{*}{Bug} & \multirow{2}{*}{effective} & Dark & & \\
      & & Psychic & effective & Poison \\
      & \multirow{2}{*}{ineffective} & Ghost \\
      & & Fairy & ineffective & Fire \\
      \hline
    \end{tabular}
  \end{center}
\end{table}

\section{Dual typing}
\label{section:dualtypes}
A species of Pokémon may be singly or doubly types.
It is not possible to dual type using a single type,
 and ordering of the types does not matter.
\[ C(18, 2) = \binom{18}{2} = \frac{18!}{2!16!} = 153, 153 + 18 = 171 \]
There are thus 153 dual types in addition to the 18 base types,
  for a total of 171.
We could also arrive at this number by summing 1 through 18:
\[ \sum_{i=1}^{18} i = \frac{18 \times 19}{2} = 171 \]
Nine dual types are currently unpopulated (Normal/Steel, Normal/Ice, Normal/Rock,
 Normal/Bug, Poison/Ice, Ground/Fairy, Rock/Ghost, Bug/Dragon, Fire/Fairy),
 leaving 162 defender typings to consider.

Dual typing expands the type relation range, adding the possibilities
 of -3 and 2 (-4 is not possible, because no type is very ineffective against
 two different types).
It furthermore vastly expands the type effectiveness matrix,
 with 3,078 (18 × 171) relations of which only 1,490 (48.4\%) are standard.
Thankfully, these needn't be memorized, as they can all be calculated
 using the base relations matrix.
Completely unexpectedly, the best typings include Steel, with the top 16
 spots all making use of it.

\input{out/dualtypes.tex}

There are twenty triple resistances, five of which are against Fighting attacks (\autoref{table:triples}).
Two of the typings exhibiting such resistances are unpopulated
  (Electricity+Ground\footnote{Sandy Shocks, Pokédex \#0989, is a Generation IX
  Electricity+Ground, but is not at this time present in Pokémon GO.} and Ghost+Rock).
\begin{table}[h]
  \begin{center}
    \begin{tabular}{cc}
Dragon → Fairy+Steel & Electricity → Grass+Ground \\
Electricity → Dragon+Ground & Electricity → Electricty+Ground \\
Fighting → Ghost+Poison & Fighting → Bug+Ghost \\
Fighting → Fairy+Ghost & Fighting → Flying+Ghost \\
Fighting → Ghost+Psychic & Ghost → Dark+Normal \\
Ground → Bug+Flying & Ground → Flying+Grass \\
Psychic → Dark+Psychic & Psychic → Dark+Steel \\
Normal → Ghost+Rock & Normal → Ghost+Steel \\
Poison → Ground+Normal & Poison → Poison+Steel \\
Poison → Rock+Steel & Poison → Ghost+Steel \\
    \end{tabular}
    \caption{The twenty triple resistances}
    \label{table:triples}
  \end{center}
\end{table}

\section{Weather boosting}
\label{section:weather}
Local weather ``boosts'' its associated types (\autoref{table:weather}), affecting attack strength
 in PvE (\autoref{section:mbmult}),
 spawn rates (\autoref{section:spawns}), and catching (\autoref{section:catch}).
The game's assessment of your weather is communicated via an icon on the Map View.
\begin{table}[ht]
\begin{center}
  \begin{tabular}{lcl}
    Weather & Icon & Types \\
    \Midrule
    Clear &
     \includegraphics[width=1em,height=1em]{images/weatherclearday.png}
     \includegraphics[width=1em,height=1em]{images/weatherclearnight.png}
     & Fire, Grass, Ground \\
    Rain &
     \includegraphics[width=1em,height=1em]{images/weatherrain.png}
     & Bug, Electric, Water \\
    Partly Cloudy &
     \includegraphics[width=1em,height=1em]{images/weatherpartlycloudyday.png}
     \includegraphics[width=1em,height=1em]{images/weatherpartlycloudynight.png}
     & Normal, Rock \\
    Cloudy &
     \includegraphics[width=1em,height=1em]{images/weathercloudy.png}
     & Fairy, Fighting, Poison \\
    Windy &
     \includegraphics[width=1em,height=1em]{images/weatherwindy.png}
     & Dragon, Flying, Psychic \\
    Snow &
     \includegraphics[width=1em,height=1em]{images/weathersnow.png}
     & Ice, Steel \\
    Fog &
     \includegraphics[width=1em,height=1em]{images/weatherfoggy.png}
     & Dark, Ghost \\
  \end{tabular}
  \caption{Weather-boosted types}
  \label{table:weather}
\end{center}
\end{table}
