\chapter{Pokémon}
\label{chap:pokemon}
For purposes of battle, a Pokémon $P$ is defined as having:
\begin{itemize}
\item Some form of some species, and thus a typing and base stats
\item An Individual Vector (IV) (\autoref{sec:ivs})
\item A level
\item Shadow status, or the absence thereof
\item Hit Points (HP). A Pokémon with zero HP has ``fainted'', and is generally unusable.
\end{itemize}
Visual presentation does not affect battle, nor does Purified status.
Hit Points are carried across PvE battles, but Trainer Battles always start
 with full HP\@.
Fainted Pokémon cannot be used in battles of any kind.

\section{Pokémon Level}
\label{sec:plevel}
Each Pokémon has its own level, taking one of the 99 integer and half-integer
 values from 0--50, inclusive.
A Trainer can ``power up'' their Pokémon in single level increments,
 not exceeding the Trainer's own level plus 10.
Each increase has a cost in Stardust and the Candy of that genus; beyond level 40, there
  is a further cost in Candy XL (\autoref{table:powerups}).
An increase in level improves the Pokémon's attack and defense, and
  will sometimes increase MHP.
See \autoref{chap:stats} for a quantitative treatment.

Powering up a fainted Pokémon will revive it, but with minimal HP\@:

\[ HP = \min{1, MHP_{New} - MHP_{Old} } \]

\section{Individual Vectors}
\label{sec:ivs}
Each Pokémon has three integers associated with it, each ranging from 0--15.
These serve to distinguish members of a species from one another.
See \autoref{chap:stats} for a quantitative treatment.

Once a Trainer has joined a team (\autoref{sec:levels}), the ``Appraise'' function can be used to
  get a visual representation of these three numbers (this representation,
  while annoying, is sufficient to determine the actual numbers).
Additionally, the Pokémon will be given a rating of 0--4, represented as
  one, two, or three stars (0 shows one star; 4 shows three, albeit in a different color).
A Pokémon rated 4 has an IV of 15/15/15, and is colloquially known as a ``hundo'' or ``100\%er''.
Such Pokémon have their own Pokédex (\autoref{sec:dexen}).
A Pokémon rated 0 has an IV of 0/0/0, and is colloquially known as a ``nundo'' or ``0\%er''.
They have no unique Pokédex, because they are worthless pieces of shit,
  and don't let anyone tell you otherwise\footnote{Consult \autoref{chap:optimal}
  to see that 0/0/0 is never an optimal IV.}.
While 15/15/15 is generally the optimal IV, this is not always true in CP-bounded
  competition; see \autoref{chap:bounded} for more information,
  and \autoref{chap:optimal} for tables of optimal IVs under different CP bounds.

\section{Generation of IV and level}
\label{sec:ivgeneration}
IVs and levels are generated when the Pokémon is ``created'', not when it is captured.
It is thus possible to glean information from the CP displayed for a Pokémon;
  one can sometimes even uniquely determine the IV\@.
The equation for CP is provided in \autoref{chap:unbounded}; reverse it and
  use \autoref{table:cpm} plus the base stats of the species to determine
  the set of IV+level pairs yielding that CP\@.
Various applications and websites can perform this analysis automatically.
Depending on the situation, there might be a floor for IVs.
Beyond that, they are uniformly randomly generated.
The chance of any given IV for a catch in the wild (without weather boosting)
  is exactly the same---1 out of 4096.
The probability of a random Pokémon being a hundo is actually
  greater than that of being a nundo, due to the existence of IV floors
  (e.g. the chance of a 15/15/15 for a lucky trade is 1 in 256, whereas
  a 0/0/0 is impossible).
\begin{table}
  \begin{center}
    \begin{tabular}{lcc}
      Context & IV floors & Level \\
      \Midrule \\
      \multirow{2}{*}{Wild catch} & 0 & 1--max(TL, 30) \\
      & 4 w/ weather boost & 1--max(TL, 30) + 5 \\
      Hatch & 10 & max(TL, 20) \\
      \multirow{2}{*}{Raid} & 10 & 20 \\
      & 6 (shadow) & 25 w/ weather boost \\
      Max Battle & 10 & 20 \\
      Research & 10 & 15 \\
      Team GO Rocket & 0 & 8 \\
      Grunts and Leaders & 4 w/ weather boost & 13 w/ weather boost \\
      \multirow{2}{*}{Giovanni} & \multirow{2}{*}{6} & 8\\
       & & 13 w/ weather boost\\
      GBL & 10 & 20\\
      \multirow{5}{*}{Trade} & 1 w/ Good Friend & \multirow{5}{*}{See \autoref{sec:trades}} \\
       & 2 w/ Great Friend & \\
       & 3 w/ Ultra Friend & \\
       & 5 w/ Best Friend & \\
       & 12 (lucky trade) & \\
    \end{tabular}
  \end{center}
\end{table}

\section{Lucky Pokémon}
\label{sec:lucky}
\textbf{FIXME}
