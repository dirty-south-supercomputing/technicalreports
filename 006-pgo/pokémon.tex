\chapter{Pokémon}
\label{chap:pokemon}
For purposes of battle, a Pokémon $P$ is defined as:
\begin{itemize}
\item A Species, possibly taking some Form
\item An ``Individual Vector'' (\autoref{sec:ivs})
\item Shadow status
\item Current Hit Points. A Pokémon with zero HP has ``fainted'', and is generally unusable.
\end{itemize}
Visual presentation does not affect battle.
Hit Points are carried across PvE battles, but Trainer Battles always start
 with full HP\@.
Fainted Pokémon cannot be used in battles of any kind.

\section{Individual Vectors}
\label{sec:ivs}
Each Pokémon has three integers associated with it, each ranging from 0--15.
These serve to distinguish members of a species from one another.
See \autoref{chap:stats} for a quantitative treatment.

Once a Trainer has joined a team, the ``Appraise'' function can be used to
  get a visual representation of these three numbers (this representation,
  while annoying, is sufficient to determine the actual numbers).
Additionally, the Pokémon will be given a rating of 0--4, represented as
  one, two, or three stars (0 shows one star; 4 shows three, albeit in a different color).
A Pokémon rated 4 has an IV of 15/15/15, and is colloquially known as a ``hundo'' or ``100\%er''.
Such Pokémon have their own Pokédex (\autoref{sec:dexen}).
A Pokémon rated 0 has an IV of 0/0/0, and is colloquially known as a ``nundo'' or ``0\%er''.
They have no unique Pokédex, because they are worthless pieces of shit,
  and don't let anyone tell you otherwise\footnote{Consult \autoref{chap:optimal}
  to see that 0/0/0 is never an optimal IV.}.
While 15/15/15 is generally the optimal IV, this is not always true in CP-bounded
  competition; see \autoref{chap:bounded} for more information,
  and \autoref{chap:optimal} for tables of optimal IVs under different CP bounds.

\section{Pokémon Level}
\label{sec:plevel}
\textbf{FIXME}
Powering up a fainted Pokémon will revive it, but with minimal HP\@:

\[ HP = \min{1, MHP_{New} - MHP_{Old} } \]

\section{Lucky Pokémon}
\label{sec:lucky}
\textbf{FIXME}
