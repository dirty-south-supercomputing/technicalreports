\chapter{Pokémon\label{chap:pokemon}}
For purposes of battle\footnote{Attributes not directly relevant to battle include height, weight, region,
generation, Shiny status, Purified status, Lucky status, Legendary status, Mythological status, spawn
reason, capture method, capture location, and capture time.}, a Pokémon $P$\, is defined as having:
\begin{itemize}
\item A species and form (and thus a typing and base stats)
\item An Individual Vector (IV) (\autoref{sec:ivs}) and level (\autoref{sec:plevel})
\item A set of two or three attacks, one of which is a fast attack
\item Modifiers to attack and defense due charged attacks' stat effects (\autoref{sec:buffs})
\item Shadow status, or the absence thereof
\item Some number of hit points not exceeding \MHP
\end{itemize}
Hit Points are carried across PvE battles (including encounters with
  Team GO Rocket and Gym battles), but Trainer Battles always start with full \HP\@.
A Pokémon with zero hit points has ``fainted''.
Fainted Pokémon cannot be used in battles of any kind, nor can they be left in
 Power Stops (see \autoref{sec:maxbattles}).

\section{Pokémon Level\label{sec:plevel}}
Each Pokémon has its own level, taking one of the 99 integer and half-integer
 values between 0--50, inclusive.
A Trainer can ``power up'' their Pokémon in single level increments,
 not exceeding the Trainer's own level plus 10.
Each increase has a cost in Stardust and the Candy of that family; beyond level 40, there
  is a further cost in Candy XL (\autoref{table:powerups}).
\begin{table}
\centering
\begin{tabular}[ht]{rrr|rrrr}
Level &
\includegraphics[width=1em,height=1em]{images/stardust.png} &
\includegraphics[width=1em,height=1em]{images/rarecandy.png} &
Level &
\includegraphics[width=1em,height=1em]{images/stardust.png} &
\includegraphics[width=1em,height=1em]{images/rarecandy.png} &
\includegraphics[width=1em,height=1em]{images/rarecandyxl.png} \\
\Midrule
1.5--3 &   200 & 1 & 31.5--33 &  6,000 &  6 & \\
3.5--5 &   400 & 1 & 33.5--35 &  7,000 &  8 & \\
5.5--7 &   600 & 1 & 35.5--37 &  8,000 & 10 & \\
7.5--9 &   800 & 1 & 37.5--39 &  9,000 & 12 & \\
9.5--11 & 1,000 & 1 & 39.5--40 & 10,000 & 15 & \\
11.5--13 & 1,300 & 2 & 40.5--41 & 10,000 &    & 10 \\
13.5--15 & 1,600 & 2 & 41.5--42 & 11,000 &    & 10 \\
15.5--17 & 1,900 & 2 & 42.5--43 & 11,000 &    & 12 \\
17.5--19 & 2,200 & 2 & 43.5--44 & 12,000 &    & 12 \\
19.5--21 & 2,500 & 2 & 44.5--45 & 12,000 &    & 15 \\
21.5--23 & 3,000 & 3 & 45.5--46 & 13,000 &    & 15 \\
23.5--25 & 3,500 & 3 & 46.5--47 & 13,000 &    & 17 \\
25.5--27 & 4,000 & 3 & 47.5--48 & 14,000 &    & 17 \\ 
27.5--29 & 4,500 & 4 & 48.5--49 & 14,000 &    & 20 \\
29.5--31 & 5,000 & 4 & 49.5--50 & 15,000 &    & 20 \\
\end{tabular}
\caption{Power-up costs for Pokémon levels\label{table:powerups}}
\end{table}
An increase in level improves the Pokémon's attack and defense, and
  will sometimes increase \MHP\@.
See \autoref{sec:staticstats} for a quantitative treatment.

Powering up a fainted Pokémon will revive it, but with minimal \HP\@:
\[ \HP = \min{1, \MHP_\mathrm{New} - \MHP_\mathrm{Old} } \]
Advancing a normal Pokémon from level 1 to level 50 is not a cheap endeavor (\autoref{table:sumcosts}),
  though Shadow Pokémon of course have a minimum level of eight, purified Pokémon are always
  at least level 25, and most captures will be well above level 1.
Note that pushing level 40 to 50 requires more Stardust than going from 1 to 40.
\begin{table}[ht]
\centering
\begin{tabular}{lrrr}
  Type & Candy & Candy XL & Stardust\\
\Midrule
        Normal & 304 & 296 & 520,000\\
        Shadow & 406 & 360 & 624,000\\
      Purified & 294 & 272 & 468,000\\
         Lucky & 304 & 296 & 260,000\\
Lucky purified & 294 & 272 & 234,000\\
\end{tabular}
\caption{Total costs for powering up from level 1 to 50\label{table:sumcosts}}
\end{table}

\section{Individual Vectors\label{sec:ivs}}
Each Pokémon has three associated integers, each ranging from 0--15.
Once a Trainer has joined a team (\autoref{sec:levels}), the ``Appraise'' function can be used to
  get a visual representation of these three numbers (this representation,
  while annoying, is sufficient to determine the actual numbers).
Additionally, the Pokémon will be given a rating of 0--4 (\autoref{table:appraisal}), represented as
  zero, one, two, or three stars (4 shows three, albeit in a different color).
A Pokémon rated 4 has an IV of 15/15/15, and is colloquially known as a ``hundo'' or ``100\%er''.
Such Pokémon have their own Pokédex (\autoref{sec:dexen})\footnote{A Pokémon with an IV of 0/0/0
 is colloquially known as a ``nundo'' or ``0\%er''. They have no unique Pokédex, because they are worthless pieces of shit,
  and don't let anyone tell you otherwise. Consult \autoref{chap:optimal} to see that 0/0/0 is never an optimal IV.}.
While 15/15/15 is generally the optimal IV, this is not always true in \CP-bounded
  competition; see \autoref{chap:bounded} for more information,
  and \autoref{chap:optimal} for tables of optimal IVs under different \CP{} bounds.
\begin{table}
\centering
\begin{tabular}{ll}
Appraisal & Minimum sum of IVs\\
\Midrule
4🟉 & 45\\
3🟉 & 37\\
2🟉 & 30\\
1🟉 & 23\\
0🟉 & 0\\
\end{tabular}
\caption{Appraisals for various IVs\label{table:appraisal}}
\end{table}

Three things can change a Pokémon's IV, which is otherwise constant across its existence.
Purification (\autoref{sec:shadow}) raises each component by two, not to exceed fifteen.
Trades (\autoref{sec:trades}) reroll the IV entirely.
Finally, Gold Bottle Caps allow a Pokémon of at least Best Buddy status (\autoref{sec:buddies}) 
  to enter ``hypertraining''.
The Trainer specifies a new, higher IV, and is assigned a set of Research-like tasks.
If they are completed within the specified time, the Pokémon acquires the new IV.
At the time of this writing, Gold Bottle Caps have been available only through paid Research.

\section{Generation of IV and level\label{sec:ivgeneration}}
IVs and levels are generated when the Pokémon is ``created'', not when it is captured.
It is thus possible to glean information from the \CP{} displayed for a Pokémon;
  one can sometimes even uniquely determine the IV\@.
The equation for \CP{} is provided in \autoref{chap:unbounded}; reverse it and
  use \autoref{table:cpm} plus the base stats of the species to determine
  the set of IV+level pairs yielding that \CP\@.
Various applications and websites can perform this analysis automatically.
Depending on the situation, there might be a floor for IVs.
Beyond that, they are uniformly randomly generated.
The chance of any given IV for a catch in the wild (without weather boosting)
  is exactly the same---1 out of 4096.
The probability of a random Pokémon being a hundo is actually
  greater than that of being a nundo, due to the existence of IV floors
  (e.g. the chance of a 15/15/15 for a lucky trade is 1 in 256, whereas
  a 0/0/0 is impossible).
\begin{table}
\centering
\begin{tabular}{lcc}
Context & IV floors & Level \\
\Midrule
\multirow{2}{*}{Wild catch} & 0 & 1--max(TL, 30) \\
& 4 w/ weather boost & 1--max(TL, 30) + 5 \\
Hatch & 10 & max(TL, 20) \\
\multirow{2}{*}{Raid} & 10 & 20 \\
& 6 (shadow) & 25 w/ weather boost \\
Max Battle & 10 & 20 \\
Research & 10 & 15 \\
Team GO Rocket & 0 & 8 \\
Grunts and Leaders & 4 w/ weather boost & 13 w/ weather boost \\
\multirow{2}{*}{Giovanni} & \multirow{2}{*}{6} & 8\\
& & 13 w/ weather boost\\
GBL & 10 & 20\\
\multirow{5}{*}{Trade} & 1 w/ Good Friend & \multirow{5}{*}{See \autoref{sec:trades}} \\
& 2 w/ Great Friend & \\
& 3 w/ Ultra Friend & \\
& 5 w/ Best Friend & \\
& 12 (lucky trade) & \\
\end{tabular}
\caption{IV and level floors for acquired Pokémon\label{table:ivfloors}}
\end{table}

\section{Lucky Pokémon\label{sec:lucky}}
Trades sometimes result in both Pokémon becoming ``Lucky''.
Lucky Pokémon have an IV floor of 12/12/12, and enjoy a 50\% discount on Stardust for power-ups
  (this does not extend to learning second charged attacks).
This discount stacks with the 10\% purification discount, so costs for
  powering up Lucky Purified Pokémon run 45\% of normal.

\section{Static stats\label{sec:staticstats}}
We'll define some statistics for each Pokémon.
These values are used as inputs to battle mechanics.
First, sum the Species's ATK, DEF, and STA with the Pokémon's
IVs to generate \Mod{A}, \Mod{D}, and \Mod{S}:
\begin{align*}
 \Mod{A} &= \mathit{ATK} + \mathit{\IV{A}} \\
 \Mod{D} &= \mathit{DEF} + \mathit{\IV{D}} \\
 \Mod{S} &= \mathit{STA} + \mathit{\IV{S}}
\end{align*}
These are typically scaled by the Pokémon's Combat Power Multiplier.
\section{Combat Power Multiplier\label{sec:cpm}}
Recall that each Pokémon has its own Level, ranging from 1.0
 to 50.0, defined on half integers\footnote{I would rather they
 simply be integer Levels 1--99, but this is settled terminology.}.
This Level is used as an index into a table of positive rational constants.
The value accessed is that Pokémon's Combat Power Multiplier (\CPM).
\CPM{} maxes out at Level 50 with a value of 0.8403, and takes its
  0.094 minimum at Level 1.
At Level 10, \CPM{} is 0.4225, almost half of its maximum value.
At Level 34, \CPM{} is 0.7557, about 90\% of its maximum value.
The extremal ratios are 0.1119 (L1 / L50) and 8.9394 (L50 / L1).

\begin{table}
\centering
\begin{tabular}{l l|l l|l l}
Level & \CPM & Level & \CPM & Level & \CPM \\
\Midrule
1.0 & 0.0940 & 17.5 & 0.5588 & 34.0 & 0.7557 \\ 
1.5 & 0.1351 & 18.0 & 0.5668 & 34.5 & 0.7586 \\
2.0 & 0.1664 & 18.5 & 0.5746 & 35.0 & 0.7616 \\
2.5 & 0.1927 & 19.0 & 0.5823 & 35.5 & 0.7645 \\
3.0 & 0.2157 & 19.5 & 0.5899 & 36.0 & 0.7674 \\
3.5 & 0.2366 & 20.0 & 0.5974 & 36.5 & 0.7703 \\
4.0 & 0.2557 & 20.5 & 0.6048 & 37.0 & 0.7732 \\
4.5 & 0.2735 & 21.0 & 0.6122 & 37.5 & 0.7761 \\
5.0 & 0.2902 & 21.5 & 0.6194 & 38.0 & 0.7789 \\
5.5 & 0.3061 & 22.0 & 0.6266 & 38.5 & 0.7818 \\
6.0 & 0.3211 & 22.5 & 0.6336 & 39.0 & 0.7846 \\
6.5 & 0.3354 & 23.0 & 0.6407 & 39.5 & 0.7875 \\
7.0 & 0.3492 & 23.5 & 0.6476 & 40.0 & 0.7903 \\
7.5 & 0.3625 & 24.0 & 0.6544 & 40.5 & 0.7928 \\
8.0 & 0.3752 & 24.5 & 0.6612 & 41.0 & 0.7953 \\
8.5 & 0.3876 & 25.0 & 0.6679 & 41.5 & 0.7978 \\
9.0 & 0.3996 & 25.5 & 0.6746 & 42.0 & 0.8003 \\
9.5 & 0.4112 & 26.0 & 0.6812 & 42.5 & 0.8028 \\
10.0 & 0.4225 & 26.5 & 0.6877 & 43.0 & 0.8053 \\
10.5 & 0.4329 & 27.0 & 0.6941 & 43.5 & 0.8078 \\
11.0 & 0.4431 & 27.5 & 0.7005 & 44.0 & 0.8103 \\
11.5 & 0.4531 & 28.0 & 0.7069 & 44.5 & 0.8128 \\
12.0 & 0.4628 & 28.5 & 0.7132 & 45.0 & 0.8153 \\
12.5 & 0.4723 & 29.0 & 0.7194 & 45.5 & 0.8178 \\
13.0 & 0.4817 & 29.5 & 0.7256 & 46.0 & 0.8203 \\
13.5 & 0.4909 & 30.0 & 0.7317 & 46.5 & 0.8228 \\
14.0 & 0.4999 & 30.5 & 0.7347 & 47.0 & 0.8253 \\
14.5 & 0.5087 & 31.0 & 0.7378 & 47.5 & 0.8278 \\
15.0 & 0.5174 & 31.5 & 0.7408 & 48.0 & 0.8303 \\
15.5 & 0.5259 & 32.0 & 0.7438 & 48.5 & 0.8328 \\
16.0 & 0.5344 & 32.5 & 0.7468 & 49.0 & 0.8353 \\
16.5 & 0.5426 & 33.0 & 0.7498 & 49.5 & 0.8378 \\
17.0 & 0.5508 & 33.5 & 0.7527 & 50.0 & 0.8403 \\
\end{tabular}
\caption{Combat Power Multiplier for Pokémon Levels\label{table:cpm}}
\end{table}

\section{Effective stats\label{sec:effectivestats}}
Multiplying $\mathit{Mod_S}$ by CPM yields $\mathit{Eff_S}$, which determines a Pokémon's Maximum Hit Points (MHP).
When a Pokémon's Hit Points are reduced to 0, it is knocked out, and cannot be
 used (nor made a Buddy) until revived.
Hit Points are persistent across Raids, Max Battles, and Team Rocket encounters,
 and can be restored (up to MHP) with Potions\footnote{Trainer Battles (other
 than those with Team Rocket) always start with full HP.}.
MHP is defined such that it is always an integer not less than ten:
\begin{align*}
 \mathit{Eff_S} &= \mathit{Mod_S} \cdot CPM \\
 MHP &= \min{10, \lfloor \mathit{Eff_S} \rfloor}
\end{align*}
Multiplying \Mod{S} by CPM yields \Eff{ARaw}, used to determine charged move priority (\autoref{sec:3x3}).
$M_{Shadow}$ is $\frac{6}{5}$ if the Pokémon is a Shadow Pokémon, and 1 otherwise.
Multiplying \Eff{ARaw} by $M_{Shadow}$ yields \Eff{A}.
Multiplying \Mod{D} by CPM and dividing by $M_{Shadow}$ gives us \Eff{D}.
This manifests a 20\% bonus to \Eff{A} and a 16.7\% penalty to \Eff{D}.
\begin{align*}
  Eff_{ARaw} &= Mod_A \cdot CPM\\
  Eff_A &= Eff_{ARaw} \cdot M_{Shadow} \\
 Eff_D &= \frac{Mod_D \cdot CPM}{M_{Shadow}}
\end{align*}
Is a Shadow Pokémon superior to its unshadowed equivalent, all else being equal?
Evaluating using a product function such as the geometric mean will show no difference;
 the two multipliers cancel one another.
\begin{align*}
  P &= Eff_A \cdot Eff_D\\
  &= \frac{6}{5}Mod_A \cdot \frac{5}{6}Mod_D \cdot CPM^2
\end{align*}
Using a sum function such as the arithmetic mean will usually show a difference.
When $Mod_A < \frac{5}{6}Mod_D$, the average goes down.
When $Mod_A = \frac{5}{6}Mod_D$, the average is unchanged,
 and the average increases if $Mod_A > \frac{5}{6}Mod_D$.

\section{Searching the Pokédex\label{sec:searching}}
The search interface for Pokédexen and storage is surprisingly capable.
Most strings are matched as substrings in both nicknames and species names,
 but there are a number of keywords (\autoref{table:searchterms}).
Most are self-evident.
Prefix a species with \texttt{+} to search for its family.
Searching for a type will show any Pokémon with that type in their typing,
 \texttt{<type} matches Pokémon weak against \texttt{type},
 and \texttt{>type} those strong against it.
\texttt{x*} for \texttt{x} between 0 and 4 matches Pokémon with that appraisal.
\texttt{buddy0} matches Pokémon which have never been your Buddy,
 \texttt{buddy1} those made your Buddy but having earned zero hearts,
 and \texttt{buddyx} for \texttt{x} between 2 and 5 to match buddy
 levels \texttt{x} - 1.
\texttt{favorite} matches those marked ``favorite'' using a star system that
 seems like it could have handled in the general tagging system just fine
 but whatever.
Speaking of, \texttt{\#TAG} searches the tag space, and \texttt{\#} by itself
 matches anything tagged.
\texttt{@weather} shows Pokémon with at least one attack boosted by current
 weather, \texttt{@ATTACK} shows Pokémon who know \texttt{ATTACK}, and \texttt{@TYPE}
 any who know some attack of type \texttt{TYPE}.
Be more specific with \texttt{@xTYPE} for \texttt{x} between 1 and 3 to check
  for a fast, first charged, or second charged attack of the specified type.
Range suffixes take the form \texttt{NUM} for an exact match of \texttt{NUM},
  \texttt{NUMX-NUMY} for an inclusive match between \texttt{NUMX} and \texttt{NUMY},
  and \texttt{-NUM} for an inclusive ceiling.
\texttt{cpRANGE} and \texttt{hpRANGE} match Combat Power and hit points.
Use \texttt{distanceRANGE} and \texttt{ageRANGE} to match catch location
  distance in kilometers and catch age in days.
\texttt{yearRANGE} matches the catch year.
Union over searches is accomplished with the infix ';' operator (',' and ':' are synonyms),
  and intersection with '\&'\footnote{Oddly, '|' also seems to function as AND. I say ``oddly'' because it means OR everywhere else.
  Except the UNIX shell, where it means ``pipe''.
  And set theory (intensional comprehensions).
  And number theory (``divides'').
  And the bra-ket notation of quantum mechanics.
  OK, not everywhere, but that doesn't mean you use it for intersection.}.
It does not seem possible to search for regional forms using substrings,
 e.g.\ a search for ``alolan'' does not bring up my Alolan Exeggutor,
 but each region is itself a search keyword.

I nickname everything with a suffix indicating their IV's distance from the ideal.
If it's the ideal IV, the distance is zero.
\begin{table}
\centering\small
\begin{tabular}{ll}
Keyword & Matches \\
\Midrule
background & Special background \\
candyxl & Level 40 and above \\
costume & Costumed event form \\
defender & In a gym \\
dynamax & Can Dynamax\\
eggsonly & Baby Pokémon \\
evolve & Can evolve now\\
evolvenew & Evolution will fill a Pokédex entry\\
female & Females \\
gbl & Acquired via GO Battle League rewards\\
genderunknown & Ungendered \\
gigantamax & Can Gigantamax\\
hatched & Obtained from egg \\
item & Evolution requires an item\\
legendary & Legendary \\
lucky & Lucky Pokémon \\
male & Males\\
megaevolve & Can Mega or Primal evolve\\
megaraid & Caught in Mega raid\\
mythical & Mythical\\
primalraid & Caught in Primal raid\\
purified & Purified Pokémon\\
raid & Caught in any kind of raid\\
remoteraid & Caught in remote raid\\
research & Obtained in research\\
rocket & Rescued from Team GO Rocket\\
shadow & Unpurified Shadow Pokémon\\
shiny & Shiny Pokémon\\
special & Elite move(s)\\
tradeevolve & Evolution discount if traded\\
traded & Acquired via trade\\
ultrabeasts & Ultra Beasts (\autoref{sec:regions})\\
xl/xs & Extra large / extra small\\
xxl/xxs & Extra extra large / extra extra small\\
\end{tabular}
  \caption{Search terms for Pokémon storage/Pokédex\label{table:searchterms}}
\end{table}
