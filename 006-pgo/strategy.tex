\chapter{Tactics}
\label{chap:strategy}
One of the greatest advantages one can gain is knowledge.
Know Pokémon well, especially those on your team and those you regularly see.
Know what moves they can learn, and how hard they throw them (Power and $Eff_A$).
Knowing their bulk ($Eff_D$ times MHP) can help decide which charged attack to use.
Know type relations cold, and be able to calculate type effectiveness on the fly.
The flipside of this is exploitation of ignorance.
Opposing Trainers are less likely to know Pokémon outside the ``meta'',
 especially those with uncommon typings.
A Pokémon can only know two charged attacks at a time, but if
 chosen from a move pool of three or more charged, even an informed opponent
 can't know what's coming until it's thrown.

\begin{tcolorbox}[enhanced,title=A Tip regarding battle UI,halign title=flush center]
Without sufficient energy, pressing the charged attack control causes a fast
 attack to be thrown.
If you know you want to throw a charged attack as soon as possible, just press
 its control until thrown.
\end{tcolorbox}

Having two charged attacks is a tremendous advantage over only one.
If they are different types, you can gain type advantage over a wide range of opponents.
Since you choose which one to throw on the fly, resistances are much less of a concern.
\textbf{FIXME}

Change to Damage inflicted by an effective Attack is at least 60\%.
Change to Damage inflicted by an ineffective Attack may be as
 low as 37.5\%.
It is probably worth accepting an ineffective Fast Attack if it
 enables an effective Charged Attack, so long as you actually
 get that Charged Attack off.
Early in the battle, this might trick the opponent into leaving
 off a Shield, allowing a \textbf{FIXME}.

\subsection{Catching charged attacks}
When a Pokémon drops to very low health (one or two more fast attacks will knock
 them out), they can be subbed out.
Later, when you know a charged attack to be imminent, bring them back in.
If the charged attack is spent knocking out the original Pokémon, its total
 damage is greatly lessened, and you can immediately return to the Pokémon
 you had up.
If you can get an effective fast attack out of the sacrifice, all the better.
This requires sufficient time between the sub out and the catch, and that the
 opponent doesn't knock out your sacrifice Pokémon with fast attacks.

\subsection{Saving charged attacks}
If your Pokémon has almost enough energy to throw a charged attack, but is close to fainting, it can be useful to sub them out.
When they come back, the situation might have changed so that they can get the attack off,
 rather than wasting the builtup energy.
A downside is that an incoming charged attack will probably inflict more total
 damage on a healthy Pokémon than one with little HP to lose.

\textbf{Swap leadership and safe swaps}

\textbf{shield baiting}

\textbf{optimal charged move timing}
