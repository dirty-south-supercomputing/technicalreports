\chapter{Species and forms\label{chap:species}}
Pokémon are partitioned into species.
Each species is identified by its own Pokédex number.
We will see that this number is the only thing guaranteed to be shared among all the population of a species.\\
\\
\autoref{chap:speciesbytype} has instructions for interpreting form cards.

\section{Forms\label{sec:forms}}
Some species boast multiple forms and genders, all sharing one common Pokédex entry.
Form can be a mere visual distinction, but it is sometimes reflected in different stats,
 attacks, and/or typing.
All Pokémon of a form share Attack (ATK), Defense (DEF), and Stamina (STA) base statistics.
These statistics are always positive integers.

Mysterious energies power temporary and persistent changes of form.
Dynamax and Gigantamax forms operate on the scale of attacks (\autoref{sec:dmaxgmax}).
Mega and Primal forms are assumed for hours (\autoref{sec:mega}).
Fused and Crowned forms persist until explicitly undone.
Regional variants are also seen: Alolan, Galarian, Hisuian, and Paldean.
Gender differentiation is typically small changes to appearance or call,
 but it affects a few evolutionary paths (\autoref{table:sexevolutions}).
Meowstic and Indeedee have different possible attacks based on their gender,
 Oinkologne genders have different stats\footnote{They are the only forms of a species with different Stamina stats.}, and Nidoran two entirely different
 species: ``Nidoran♂'' and ``Nidoran♀'' (I have Questions about Nidoran biology).
\begin{table}
\footnotesize
\centering
\begin{tabular}{lll}
Base & Requirements & Result \\
\Midrule
Female Combee	& 50 Combee Candy & Vespiquen\\
Female Salandit & 50 Salandit Candy & Salazzle\\
Female Snorunt & 100 Snorunt Candy + \includegraphics[width=1em,height=1em]{images/sinnohstone.png} & Froslass\\
Male Snorunt & 100 Snorunt Candy & Glalie\\
Female Kirlia & 100 Ralts Candy & Gardevoir\\
Male Kirlia & 100 Ralts Candy + \includegraphics[width=1em,height=1em]{images/sinnohstone.png} & Gallade\\
Female Burmy & 50 Burmy Candy & Wormadam\\
Male Burmy & 50 Burmy Candy & Mothim\\
\end{tabular}
\caption[Gender-dependent evolutions]{Certain evolution paths depend on gender. Note that male Combees and male Salandits cannot evolve.\label{table:sexevolutions}}
\end{table}

\subsection{Mega and Primal forms\label{sec:mega}\label{sec:primal}}
Mega Evolution and Primal Reversion are almost equivalent; where I write Mega,
 read ``Mega (evolution) or Primal (reversion)'' unless otherwise noted.
Species with corresponding Mega or Primal Energy (no species has both) can take on the relevant form for eight hours.
Only one of a Trainer's Pokémon can be in its Mega form at any given time---Mega
  evolving Pokémon will cause any existing Mega forms to revert.
Mega evolution revives and fully heals the evolved Pokémon, and persists across
  fainting (i.e. the evolved Pokémon can be revived into the Mega form).
Fainting does not stop the eight hour counter.
When the evolution expires, the Pokémon has a rest period before it can Mega evolve again.
Mega Pokémon cannot be used in Max Battles, cannot defend gyms, and cannot be traded while in Mega form.
The first evolution of a Pokémon per calendar day counts towards its Mega level,
  which affects the cost to evolve, the downtime, and bonuses (\autoref{table:megalevels}).
Mega levels are reset upon a trade.
When actively battling in a raid, Mega Pokémon provide a 30\% boost to attacks of aligned
  type by \textit{other Trainers' Pokémon}, and 10\% to unaligned attacks,
  \textit{except for} Mega Rayquaza and Primal forms, which provide the boost just
  by being in the raiding party (i.e.\ they needn't be active)\footnote{Just
  for fun, Mega Rayquaza provides this bonus to Psychic attacks in addition to
  Flying and Dragon; Primal Groudon adds Grass to its Ground and Fire;
  Primal Kyogre gives a bonus to Water, Bug, and Electric. Don't blame me, man; I didn't do it.}.
This boost is never applied to the Trainer's own Pokémon, and they do not stack.
Mega forms typically have higher ATK and DEF than their base forms.
All three base stats are higher for Primal forms.

\begin{table}
\centering
\begin{tabular}{lrrrp{.4\textwidth}}
Level & Evols & Rest days & Cost & Bonuses\\
\Midrule
0 &  0 & 14 & 100\% & 1 Candy\\
1 &  1 &  7 & 20\% & 1 Candy\\
2 &  7 &  5 & 10\% & 1 Candy\newline{}10\% Candy XL chance\newline{}50 XP\\
3 & 30 &  3 &  5\% & 2 Candy\newline{}35\% Candy XL chance\newline{}100 XP\\
\end{tabular}
\caption{Mega evolution and Primal reversion levels\label{table:megalevels}}
\end{table}
\clearpage
\input{out/mega}
\input{out/gigantas}
\clearpage
\nopagecolor
\subsection{Dynamax and Gigantamax forms\label{sec:dmaxgmax}}
Dynamax and Gigantamax Pokémon can participate in Max Battles (\autoref{sec:maxbattles}),
 as can Eternatus and the Crowned forms of Zacian and Zamazenta.
There are no stat or type changes.
Normally acquired in Research, Max Battles, or via trade,
 Dynamax Pokémon also spawn around Power Stops with installed Gigantamax Pokémon.
They have access to three Max Moves: Max Attack, Max Guard, and Max Spirit.
Max Moves have four levels: Locked, 1, 2, and 3.
A freshly acquired Maxable Pokémon has Max Attack at level 1, while Max Guard and Max Spirit are both locked.
Unlocking and upgrading moves costs Max Particles and Candy of the Pokémon's family (\autoref{table:maxupgrades});
 the amount of Candy depends on the Pokémon's cost group (\autoref{sec:costgroups}).
Eternatus has its own cost schedule; the Max Particles required is the same,
 but unlocking costs 400 Candy, Level 2 runs 1200 Candy, and
 Level 3 is an absolutely punishing 1200 Candy plus 320 Candy XL.
There are eighteen Max Attacks, one for each type, and the Max Attack known by a Dynamax Pokémon matches its fast attack
 (Hidden Power always matches up with Max Strike).
Max Attacks of level 1, 2, and 3 have power 250, 300, and 350 respectively.
Gigantamax Pokémon instead know a species-specific G-Max attack which cannot change.
Max Move levels persist across evolution and use of Fast TMs, but are reset in a trade.
G-Max attacks have power of 350, 400, and 450, 100 more than Dynamax attacks of the same level.
Max Guard bestows a protective shield (the Guard) on all active Pokémon
  with fewer than three Guards.
This Guard will absorb 20, 40, or 60 points of damage, depending on Max Guard's level.
Max Spirit restores \HP{} to active Pokémon.
The amount restored is a percentage of each affected Pokémon's \MHP\@;
  the percentage (8\%, 12\%, or 16\%) is based on Max Spirit's level.

\subsection{Crowned forms\label{sec:crowned}}
Crowning is a reversible, persistent evolution available to Zacian and Zamazenta
 so long as they know the charged attack Iron Head.
Upon coronation, stats are modified, and Iron Head is replaced with Behemoth Blade (Zacian)
  or Behemoth Bash (Zamazenta).
This new attack cannot be changed with even an Elite Charged TM\@.
A Crowned form can fight in Max Battles (though it \textit{cannot} be left at Power Stops), with Behemoth Blade/Bash as its Max Attack.
This Max Attack follows the same rules for leveling, power, and other Dynamax attacks,
 except leveling consumes Stardust (25k, 50k, and 100k) rather than Max Particles.
Crowning consumes 1,000 Crown Energy, but once a Pokémon has been crowned,
  that Pokémon can freely change between the base and crowned
  forms\footnote{I'm not sure why one would uncrown. Perhaps to change charged attacks?
    Eliminate Steel typing? Drop CP under a limit? Most likely to trade the Pokémon.}, even if traded.
Crowned Shield Zamazenta can stack four Guards rather than the typical three.
If it has Max Guard unlocked, it starts Max Battles with one Guard already in place.

\subsection{Fused forms\label{sec:fusion}}
Certain Pokémon can be reversibly fused, yielding a new form.
The fusion process requires 1,000 units of Fusion Energy particular to the output\footnote{They all kinda sound like Mountain Dew flavors.},
 and 30 Candy of each input species (\autoref{table:fusion}).
There is no cost to undo the fusion, but subsequent fusions cost the same amount as the initial fusion.
Any powering up performed on the fused form will apply only to the primary
  Pokémon if the fusion is undone, the fused form's IV is taken from the primary Pokémon,
  and fusion replaces the first charged attack of the primary Pokémon.

\begin{table}[ht]
\centering
\footnotesize
\begin{tabular}{lllll}
Primary & Secondary & Energy & Output & Attack\\
\Midrule
Kyurem & Zekrom & Volt & Black Kyurem & Freeze Shock\\
Kyurem & Reshiram & Blaze & White Kyurem & Ice Burn\\
Necrozma & Solgaleo & Solar & Dusk Mane Necrozma & Sunsteel Strike\\
Necrozma & Lunala & Lunar & Dawn Wings Necrozma & Moongeist Beam\\
\end{tabular}
\caption{Fusion paths and special charged attacks\label{table:fusion}}
\end{table}

\subsection{Zygarde\label{sec:zygarde}}
Zygarde is received from research in its ``10\%'' form.
``Zygarde Cells'' are found on Routes, and can be stored in the Zygarde Cube
 (received at the same time as Zygarde 10\%).
The Cube can hold up to 300 cells, and cannot be removed from the bag.
Neither the Cube nor Cells count against bag limits.
Up to three cells can be collected per calendar day (more than three can spawn).
A route will not necessarily present a cell.
Cells are only generated the first time a route is taken each day, and more than one per route is very rare.
Cells usually show up near the end of a route, and disappear if the route is marked complete.
It can change to a ``50\%'' form for 50 Cells, and a subsequent ``Complete''
 form for 200 more Cells, all with different stats (but the same attacks).
Both forms are reversible, at the cost of 10 Zygarde Candy and 2,000 Stardust each.

\subsection{Hoopa\label{subsec:hoopa}}
The Mythical Pokémon Hoopa has two forms: Confined and Unbound.
They have different typing (Ghost+Psychic and Dark+Psychic) and stats,
  and Hoopa Unbound can learn Dark Pulse in place of Hoopa Confined's Psybeam.
Unbinding costs 50 Hoopa Candy and 10,000 Stardust.
Confining is a relative bargain at 10 Hoopa Candy and 2,000 Stardust.

\subsection{Shaymin}
Shaymin's Sky form has the same attacks as its Land form, but adds Flying to its typing.
Conversion between forms is 25 Shaymin Candy and 10,000 Stardust either way.

\section{Evolution\label{sec:evolution}}
Under the correct conditions, some species can evolve into others.
We call the set of Pokémon related by evolution operations a family.
Change of form is not an evolution, since the species and Pokédex number remain the same.
We call a Pokémon that has undergone $N$ evolutions a Stage $N+1$ Pokémon.
No Pokémon of Stage 4 or higher exist.
Evolution (unlike some form changes) is irreversible.
Evolution does not necessarily preserve typing, i.e.\ typing is not always constant within a family (\autoref{table:heteroevolve})\footnote{It is interesting
  that Sharpedo changes from Water+Dark to Mega Sharpedo's Dark+Water,
  a functionally equivalent typing. When this kind of thing happens,
  I assume it due to conformance with other Pokémon games, but never
  rule out simple fuckups. ALSO, no species lose Dark in an evolution,
  though several gain it. Is Nintendo hinting at the inevitable triumph of
  evil over good, or commenting on the blasphemies central to evolutionary theory?
  Impossible to know, but I'm pretty sure it's one of those things.}.
Evolution infrequently changes the cost group---always in the more expensive direction\footnote{Most
  cost-changing evolutions involve Baby Pokémon.} (\autoref{table:heterocost}).
\input{out/hetero}
Almost all evolutions require Candy of that family (Gimmighoul is an exception),
  and some evolutions depend upon some condition (\autoref{table:condevolutions}),
  or walking the Pokémon while your Buddy (\autoref{table:walkevolutions}),
  or a catalyzing item, which is consumed (\autoref{table:itemevolutions}).
Some evolutions can be performed at no Candy cost if the Pokémon was received by trade
 (\autoref{table:tradeevolution}).
Evolution usually improves base stats (though there are many exceptions to this rule),
  and typically makes available new, more powerful attacks.
Evolution revives a Pokémon if it is fainted, and always fully restores \HP\@.
\begin{table}
\footnotesize
\centering
\begin{tabular}{lll}
  Base & Requirements & Result \\
  \Midrule
  Magneton & 100 Magnemite candy + active \includegraphics[width=1em,height=1em]{images/magneticlure.png} & Magnezone\\
  Nosepass & 50 Nosepass candy + active \includegraphics[width=1em,height=1em]{images/magneticlure.png} & Probopass\\
  Charjabug & 100 Grubbin candy + active \includegraphics[width=1em,height=1em]{images/magneticlure.png} & Probopass\\
  Sliggoo	& 100 Goomy candy + rain or active \includegraphics[width=1em,height=1em]{images/rainylure.png}& Goodra\\
  Bisharp	& 100 Pawniard candy + win 15 \includegraphics[width=1em,height=1em]{images/dark.png} or \includegraphics[width=1em,height=1em]{images/steel.png} raids & Kingambit\\
  Primeape & 100 Mankey candy + defeat 30 \includegraphics[width=1em,height=1em]{images/psychic.png} or \includegraphics[width=1em,height=1em]{images/ghost.png} & Annihilape\\
  Floette	& 100 Flabébé candy + earn 20 ♥ & Florges\\
  Snom & 400 Snom candy + earn 10 ♥, evolve at night & Frosmoth \\
  Charcadet	& 50 Charcadet candy + defeat 30 \includegraphics[width=1em,height=1em]{images/psychic.png}& Armarouge\\
  Charcadet	& 50 Charcadet candy + defeat 30 \includegraphics[width=1em,height=1em]{images/ghost.png}& Ceruledge\\
  Poipole & 200 Poipole candy + capture 20 Dragon-type & Naganadel\\
  Hisuian Qwilfish & 50 Qwilfish candy + win 10 raids & Overqwil\\
  Kubfu	& 200 Kubfu candy + win 30 \includegraphics[width=1em,height=1em]{images/water.png} raids/Max battles & Rapid Strike Urshifu\\
  Kubfu	& 200 Kubfu candy + win 30 \includegraphics[width=1em,height=1em]{images/dark.png} raids/Max battles & Single Strike Urshifu\\
  Inkay	& 50 Inkay candy + rotating the mobile device & Malamar\\
  Amaura & 50 Amaura candy, evolve at night & Aurorus\\
  Tyrunt & 50 Tyrunt candy, evolve during the day & Tyrantrum\\
  Pancham	& 50 Pancham candy + capture 32 \includegraphics[width=1em,height=1em]{images/dark.png} & Pangoro\\
  Ursaring & 100 Teddiursa candy, evolve during full moon & Ursaluna\\
  Galarian Farfetch'd & 50 Farfetch'd candy + 10 Excellent throws & Sirfetch'd \\
  Spritzee & 50 Spritzee candy + use incense & Aromatisse\\
  Swirlix & 50 Swirlix candy + feed 25 treats & Slurpuff\\
  Galarian Yamask & 50 Yamask candy + win 10 raids & Runerigus\\
  Galarian Slowpoke & 50 Slowpoke candy + capture 30 \includegraphics[width=1em,height=1em]{images/poison.png} & Galarian Slowbro\\
  Galarian Slowpoke & 50 Slowpoke candy + capture 30 \includegraphics[width=1em,height=1em]{images/psychic.png} & Galarian Slowking\\
\end{tabular}
  \caption{Evolutions dependent upon a condition\label{table:condevolutions}}
\end{table}
\begin{table}
\footnotesize
\centering
\begin{tabular}{lll}
  Base & Usual requirement & Result \\
\Midrule
Kadabra & 100 Abra candy & Alakazam\\
Machoke & 100 Machop candy & Machamp\\
  Graveler & 100 Geodude candy & Golem\\
  Alolan Graveler & 100 Geodude candy & Alolan Golem\\
  Haunter & 100 Gastly candy & Gengar\\
  Boldore & 200 Roggenrola candy & Gigalith\\
  Gurdurr & 200 Timburr candy & Conkeldurr\\
  Karrablast & 200 Karrablast candy & Escavalier\\
  Shelmet & 200 Shelmet candy & Accelgor\\
  Phantump & 200 Phantump candy & Trevenant\\
  Pumpkaboo & 200 Pumpkaboo candy & Gourgeist\\
\end{tabular}
  \caption{Trade-assisted evolutions\label{table:tradeevolution}}
\end{table}
\begin{table}
\footnotesize
\centering
\begin{tabular}{lll}
  Base & Requirements & Result \\
\Midrule
  Woobat & 50 Woobat candy + 1 km & Swoobat\\
  Hisuian Sneasel & 100 Sneasel candy + 7 km, evolve during the day & Sneasler\\
  Mime Jr. & 50 Mime Jr. candy + 15 km & Mr. Mime\\
  Bonsly & 50 Bonsly candy + 15 km & Sudowoodo\\
  Happiny & 25 Chansey candy + 15 km & Chansey\\
  Feebas & 100 Feebas candy + 20 km & Milotic\\
  Pawmo & 100 Pawmi candy + 25 km & Pawmot\\
\end{tabular}
  \caption{Evolutions requiring walking of Buddy Pokémon\label{table:walkevolutions}}
\end{table}
\begin{table}
\footnotesize
\centering
  \begin{tabular}{lll}
    Base & Requirements & Result \\
    \Midrule
    Gloom & 100 Oddish candy + \includegraphics[width=1em,height=1em]{images/sunstone.png} & Bellossom \\
    Sunkern & 50 Sunkern candy + \includegraphics[width=1em,height=1em]{images/sunstone.png} & Sunflora \\
    Cottonee & 50 Cottonee candy + \includegraphics[width=1em,height=1em]{images/sunstone.png} & Whimsicott \\
    Petilil & 50 Petilil candy + \includegraphics[width=1em,height=1em]{images/sunstone.png} & Lilligant \\
    Helioptile & 50 Helioptile candy + \includegraphics[width=1em,height=1em]{images/sunstone.png} & Heliolisk \\
    Poliwhirl & 100 Poliwag candy + \includegraphics[width=1em,height=1em]{images/kingsrock.png} & Politoed \\
    Slowpoke & 50 Slowpoke candy + \includegraphics[width=1em,height=1em]{images/kingsrock.png} & Slowking \\
    Onix & 50 Onix candy + \includegraphics[width=1em,height=1em]{images/metalcoat.png} & Steelix \\
    Scyther & 50 Scyther candy + \includegraphics[width=1em,height=1em]{images/metalcoat.png} & Scizor \\
    Seadra & 100 Horsea candy + \includegraphics[width=1em,height=1em]{images/dragonscale.png} & Kingdra \\
    Porygon & 25 Porygon candy + \includegraphics[width=1em,height=1em]{images/upgrade.png} & Porygon2 \\
    Poygon2 & 100 Porygon candy + \includegraphics[width=1em,height=1em]{images/sinnohstone.png} & Porygon-Z \\
    Lickitung & 100 Lickitung candy + \includegraphics[width=1em,height=1em]{images/sinnohstone.png} & Lickilicky \\
    Tangela	& 100 Tangela candy + \includegraphics[width=1em,height=1em]{images/sinnohstone.png} & Tangrowth \\
    Electabuzz & 100 Elekid candy + \includegraphics[width=1em,height=1em]{images/sinnohstone.png} & Electivire	\\
    Magmar & 100 Magby candy + \includegraphics[width=1em,height=1em]{images/sinnohstone.png} & Magmortar	\\
    Sneasel & 100 Sneasel candy + \includegraphics[width=1em,height=1em]{images/sinnohstone.png} & Weavile	\\
    Togetic & 100 Togepi candy + \includegraphics[width=1em,height=1em]{images/sinnohstone.png} & Togekiss	\\
    Yanma & 100 Yanma candy + \includegraphics[width=1em,height=1em]{images/sinnohstone.png} & Yanmega	\\
    Gligar & 100 Gligar candy + \includegraphics[width=1em,height=1em]{images/sinnohstone.png} & Gliscor	\\
    Murkrow & 100 Murkrow candy + \includegraphics[width=1em,height=1em]{images/sinnohstone.png} & Honchkrow	\\
    Misdreavus & 100 Misdreavus candy + \includegraphics[width=1em,height=1em]{images/sinnohstone.png} & Mismagius	\\
    Piloswine & 100 Swinub candy + \includegraphics[width=1em,height=1em]{images/sinnohstone.png} & Mamoswine	\\
    Dusclops & 100 Duskull candy + \includegraphics[width=1em,height=1em]{images/sinnohstone.png} & Dusknoir	\\
    Aipom & 100 Aipom candy + \includegraphics[width=1em,height=1em]{images/sinnohstone.png} & Ambipom	\\
    Roselia & 100 Budew candy + \includegraphics[width=1em,height=1em]{images/sinnohstone.png} & Roserade	\\
    Applin & 200 Applin candy + 20 \includegraphics[width=1em,height=1em]{images/tartapple.png} & Flapple \\
    Applin & 200 Applin candy + 20 \includegraphics[width=1em,height=1em]{images/sweetapple.png} & Appletun \\
    Zygarde 10\% & 50 \includegraphics[width=1em,height=1em]{images/zygardecell.png} & Zygarde 50\% \\
    Zygarde 50\% & 200 \includegraphics[width=1em,height=1em]{images/zygardecell.png} & Zygarde Complete \\
    Gimmighoul & 999 \includegraphics[width=1em,height=1em]{images/gcoin.png} & Gholdengo \\
  \end{tabular}
  \caption{Evolutions and form changes dependent upon items\label{table:itemevolutions}}
\end{table}

\subsection{Eevolution}
\begin{figure}
\end{figure}
Evolution within the Eevee family is a complicated and unique affair.
Eevee (the species) can evolve eight different ways.
Evolution can be driven by nicknames, deterministic across all eight targets,
  but this can be done only once per target per Trainer.
Otherwise, five targets require some condition, and are deterministic.
If none of the conditions are met, the evolution is nondeterministic,
  with three possible results.
For the nickname-based mechanic, give the Eevee you wish to evolve the nickname
  specified for the desired target from \autoref{table:eevee}.
When multiple deterministic evolutions are possible, all will be shown,
  and the Trainer can choose the target.
It is impossible to execute nondeterministic evolution when conditions for deterministic evolution are met.
When in doubt, check---if the evolution is deterministic, the ``Evolve'' button will show the target (or a silhouette thereof).
It will otherwise show a question mark.
Eevee's evolutions are strong Pokémon in the early game.
\begin{table}
\centering
\begin{tabular}{lll}
  Target & Nickname & Condition\\
  \Midrule
  Vaporeon & Rainer & Random\\
  Jolteon & Sparky & Random\\
  Flareon & Pyro & Random\\
  Sylveon & Kira & Earn 70 ♥ \\
  Espeon & Sakura & 10 km, evolve during the day\\
  Umbreon & Tamao & 10 km, evolve at night\\
  Leafeon & Linnea & Evolve near an active \includegraphics[width=1em,height=1em]{images/rainylure.png} \\
  Glacion & Rea & Evolve near an active \includegraphics[width=1em,height=1em]{images/glaciallure.png} \\\\
\end{tabular}
\caption{Eevolution (all require 25 Eevee candy)\label{table:eevee}}
\end{table}

\section{Shadow and Purified Pokémon\label{sec:shadow}}
Team Rocket uses ``Shadow'' Pokémon modified for more attack
 and less defense capability than their base forms (for more details,
 see \autoref{sec:damage}).
Shadow Pokémon cost 20\% more Stardust than normal to teach second charged attacks or power up.
When captured from Team Rocket, they enter their own Shadow Pokédex.
Captured Shadow Pokémon always know the charged attack Frustration.
Only during certain special events can this charged attack be replaced,
 at which time a Charged TM is required as normal.
It's a pretty terrible Charged Attack, and this really degrades the
 Shadow Pokémon until it can be replaced.
A Shadow Pokémon can be taught a second Charged Attack at any time.
Frustration is preserved across evolution, as is Shadow status itself.
Shadow Pokémon cannot be traded.

Purification of a Shadow Pokémon has a cost in Stardust and Candy.
Purified Pokémon cost 10\% less than normal to evolve, teach second charged attacks, or power up.
Purification enters the Pokémon into its own Purified Pokédex,
 advances it to level 25 if not yet there,
 eliminates the attack bonus and defense penalty,
 replaces the primary charged attack with Return (exclusive to purified Pokémon),
 and increases each IV component by 2  (up to the usual max of 15).
Return is preserved across evolution, as is Purified status itself.
Purified Pokémon can be traded, but constitute a Special Trade (\autoref{sec:trades}).

It ought be obvious, but if you intend to purify a Shadow Pokémon, it
  is best to do so prior to any other development.
Shadow Pokémon are more expensive than normal to develop, while Purified Pokémon are cheaper.

\section{Regions, generations, myths, legends, and beasts\label{sec:regions}}
Each Pokémon is associated with one of eleven regions and one of nine generations.
Both can be determined by their Pokédex index (\autoref{table:regions}).
Whether Meltan and Melmetal (of the ``Unknown'' region) are Generation VII
  or Generation VIII is a raging and utterly inconsequential debate.
I place them in Generation VII because doing so saves a keystroke when using Roman numerals.
If you disagree, I encourage you to FAX your senator.
\input{out/regions}
Legendary (\textjapanese{伝説のポケモン}) and Mythical (\textjapanese{幻のポケモン}) Pokémon
 are not generally seen in the wild.
They are members of the Pokémon mythos, heralds from times past and partly forgotten,
  some so rare that their existence is doubted.
Ultra Beasts (\textjapanese{ウルトラビースト}) descend into our dimension from
  another via ``Ultra Wormholes''.
Such Pokémon never hatch from eggs (\autoref{sec:eggs}), and are not usually included in wild spawn groups (\autoref{sec:spawns}).
They are instead typically Research rewards, or captured in raids.
They cannot be left in gyms (\autoref{sec:gyms}).
It is not usually possible to trade or accumulate arbitrary numbers of Mythical
  Pokémon\footnote{Marshadow, Hoopa, Celebi, Mew, Shaymin, Victini, Meloetta, Keldeo,
  Jirachi, Deoxys, Phione, Manaphy, Darkrai, Diancie, Zarude, and Volcanion.}, nor Legendaries Zygarde and
  Eternatus, making them prime candidates for Bottle Caps.
Many of these Pokémon have been collected into (often unofficial) named sets (\autoref{table:namedmyths}).

\begin{table}
\begin{tabular}{lp{.65\textwidth}}
Group & Members\\
\Midrule
Legendary Birds & Articuno, Zapdos, Moltres\\
Galarian Birds & Galarian Articuno, Galarian Zapdos, Galarian Moltres\\
Legendary Beasts & Raikou, Entei, Suicune\\
Tower Duo & Lugia, Ho-Oh\\
Giants of Legend & Regigigas, Registeel, Regice, Regirock, Regieleki, Regidrago\\
Eon Duo & Latias, Latios\\
Super-Ancients & Kyogre, Groudon, Rayquaza (also known as the ``Weather Trio'')\\
Lake Guardians & Uxie, Mesprit, Azelf\\
Pokémon of Myth & Dialga, Palkia, Giratina (also known as the ``Creation Trio'' and ``Three Sinnoh Dragons'')\\
Lunar Duo & Cresselia, Darkrai\\
Swords of Justice & Cobalion, Terrakion, Virizion, Keldeo\\
Forces of Nature & Tornadus, Thundurus, Landorus, Enamorus\\
Tao Trio & Reshiram, Zekrom, Kyurem\\
Aura Trio & Xerneas, Yveltal, Zygarde\\
Guardian Deities & Tapu Koko, Tapu Fini, Tapu Lele, Tapu Bulu\\
Light Trio & Solgaleo, Lunala, Necrozma\\
Hero Duo & Zacian, Zamazanta\\
Sea Guardians & Phione, Manaphy\\
Ultra Beasts & Nihilego, Buzzwole, Pheromosa, Xurkitree, Celesteela, Kartana, Guzzlord,
               Poipole, Naganadel, Stakataka, Blacephalon \\
\end{tabular}
\caption{Named collections of Legendary Pokémon\label{table:namedmyths}}
\end{table}

\begin{tipbox}[title=Kanto Starters]
The ``Kanto Starters'' refer to Bulbasaur, Charmander, and Squirtle, and sometimes
  also Pikachu and Eevee.
They are the initial partner Pokémon from various games.
\end{tipbox}

\section{Trends among species}
One thing to take away from \autoref{table:populations} is that the largest
  typings by population are the monotypes, representing 485 forms.
A majority of Pokédex have dual typing, but their diversity of typing means few dual types have much of a population.
Only Ground+Water, Flying+Normal, Bug+Poison, Grass+Ghost, Grass+Poison, Rock+Water, and Bug+Flying
  have more than ten members.
All monotypes have at least ten members except for Steel (9) and Flying (4).
\input{out/populations}
Scattering defense (max 400) against attack (max 420) for each type
  (including all species with that type in their typing) shows general
  balance.

\noindent{}\includegraphics[width=.5\textwidth,keepaspectratio]{graph/BugDvA.png}
\includegraphics[width=.5\textwidth,keepaspectratio]{graph/DarkDvA.png}
\includegraphics[width=.5\textwidth,keepaspectratio]{graph/DragonDvA.png}
\includegraphics[width=.5\textwidth,keepaspectratio]{graph/ElectricDvA.png}
\includegraphics[width=.5\textwidth,keepaspectratio]{graph/FairyDvA.png}
\includegraphics[width=.5\textwidth,keepaspectratio]{graph/FightingDvA.png}
\includegraphics[width=.5\textwidth,keepaspectratio]{graph/FireDvA.png}
\includegraphics[width=.5\textwidth,keepaspectratio]{graph/FlyingDvA.png}
\includegraphics[width=.5\textwidth,keepaspectratio]{graph/GhostDvA.png}
\includegraphics[width=.5\textwidth,keepaspectratio]{graph/GrassDvA.png}
\includegraphics[width=.5\textwidth,keepaspectratio]{graph/GroundDvA.png}
\includegraphics[width=.5\textwidth,keepaspectratio]{graph/IceDvA.png}
\includegraphics[width=.5\textwidth,keepaspectratio]{graph/NormalDvA.png}
\includegraphics[width=.5\textwidth,keepaspectratio]{graph/PoisonDvA.png}
\includegraphics[width=.5\textwidth,keepaspectratio]{graph/PsychicDvA.png}
\includegraphics[width=.5\textwidth,keepaspectratio]{graph/RockDvA.png}
\includegraphics[width=.5\textwidth,keepaspectratio]{graph/SteelDvA.png}
\includegraphics[width=.5\textwidth,keepaspectratio]{graph/WaterDvA.png}
\begin{figure}[h]
\centering
\includegraphics[width=\textwidth,keepaspectratio]{graph/AllDvA.png}
  \caption{Defense vs Attack, all species\label{figure:alldva}}
\end{figure}

Now we scatter stamina (max 500) against attack (max 420):

\noindent{}\includegraphics[width=.5\textwidth,keepaspectratio]{graph/BugSvA.png}
\includegraphics[width=.5\textwidth,keepaspectratio]{graph/DarkSvA.png}
\includegraphics[width=.5\textwidth,keepaspectratio]{graph/DragonSvA.png}
\includegraphics[width=.5\textwidth,keepaspectratio]{graph/ElectricSvA.png}
\includegraphics[width=.5\textwidth,keepaspectratio]{graph/FairySvA.png}
\includegraphics[width=.5\textwidth,keepaspectratio]{graph/FightingSvA.png}
\includegraphics[width=.5\textwidth,keepaspectratio]{graph/FireSvA.png}
\includegraphics[width=.5\textwidth,keepaspectratio]{graph/FlyingSvA.png}
\includegraphics[width=.5\textwidth,keepaspectratio]{graph/GhostSvA.png}
\includegraphics[width=.5\textwidth,keepaspectratio]{graph/GrassSvA.png}
\includegraphics[width=.5\textwidth,keepaspectratio]{graph/GroundSvA.png}
\includegraphics[width=.5\textwidth,keepaspectratio]{graph/IceSvA.png}
\includegraphics[width=.5\textwidth,keepaspectratio]{graph/NormalSvA.png}
\includegraphics[width=.5\textwidth,keepaspectratio]{graph/PoisonSvA.png}
\includegraphics[width=.5\textwidth,keepaspectratio]{graph/PsychicSvA.png}
\includegraphics[width=.5\textwidth,keepaspectratio]{graph/RockSvA.png}
\includegraphics[width=.5\textwidth,keepaspectratio]{graph/SteelSvA.png}
\includegraphics[width=.5\textwidth,keepaspectratio]{graph/WaterSvA.png}

\begin{figure}[hb]
\centering
\includegraphics[width=\textwidth,keepaspectratio]{graph/AllSvA.png}
  \caption{Stamina vs Attack, all species\label{figure:allsva}}
\end{figure}

\section{Cost groups\label{sec:costgroups}}
Each species is in one of four cost groups.
Group membership determines the cost of teaching Pokémon a second charged move,
  upgrading Max Moves (\autoref{sec:dmaxgmax}),
  walk distance for Candy,
  and purification.
All Legendary and Mythical Pokémon are in group 4 (the most expensive group).
\begin{table}
\centering
\begin{tabular}{lrrrrrr}
  Group & 2nd move   & Max 1 & Max 2 & Max 3 & Km & Purify\\
\Midrule
      1 & 10k + 25   & 50    & 100   & 40 XL & 1  &       \\
  Shadow& 12k + 30   &       &       &       & 1  & 1k + 1\\
Purified& 8k + 8   &       &       &       & 1  &         \\
      2 & 50k + 50   & 60    & 110   & 45 XL & 3  &       \\
  Shadow& 60k + 60   &       &       &       & 3  & 3k + 3\\
  Purified& 48k + 48   &       &       &       & 3  &     \\
      3 & 75k + 75   & 70    & 120   & 50 XL & 5  &       \\
  Shadow& 90k + 90   &       &       &       & 5  & 5k + 5\\
  Purified & 60k + 60 &   &    &   & 5 &                  \\
      4 & 100k + 100 & 80    & 130   & 55 XL & 20 &       \\
  Shadow& 120k + 120 &       &       &       & 20 & 20k + 20\\
Purified& 80k + 80 &   &    &   & 20 & \\
\end{tabular}
  \caption{Costs for the four groups\label{table:costs}}
\end{table}

\section{Freak Pokémon a/k/a Freakémonics\label{sec:freaks}}
A few Pokémon have unique behaviors or mechanics.

\subsection{Morpeko\label{subsec:morpeko}}
Morpeko changes its form between ``Full Belly Mode'' and ``Hangry Mode''
  each time it uses a charged attack.
Upon becoming active, Morpeko is in Full Belly Mode, and Aura Wheel is an Electric attack.
In Hangry Mode, Aura Wheel is a Dark attack.
Morpeko cannot defend gyms, assault gyms, or participate in raids.

\subsection{Smeargle\label{subsec:smeargle}}
Smeargle is not generally available in the wild\footnote{When captured in the wild, it knows Splash and Struggle.}.
Instead, it sometimes photobombs pictures taken of your Buddy Pokémon.
Smeargle will then spawn nearby, exclusive to the affected Trainer.
If caught, this Smeargle knows the same moves as whatever Pokémon was being photographed.
Smeargle will not appear more than once per day.
Smeargle cannot be taught a second charged attack.
Smeargle cannot copy certain fast attacks, which will be randomly replaced with
 a fast attack it can learn.
Via this means, Smeargle can know combinations of fast and charged attacks available
 to no other Pokémon.

\subsection{Ditto and Zorua\label{subsec:ditto}}
At any time, there is a set of Pokémon which might actually be Ditto,
  a Pokémon impersonator (impokénator?).
When captured, the Pokémon will be revealed as Ditto, and Ditto Candy will take
  the place of expected Candy.
Ditto status is shared across Trainers, i.e. it is a property of the spawn, not the encounter.
In battle, Ditto adopts the moves, typing, ATK, and DEF of its opponent, but not its \MHP\@.
The result is almost universally garbage, and Ditto can't be used in 3x3 anyway\footnote{It is one of two species banned from PvP, the other being Shedinja.}.
The spiteful fox Zorua copies your Buddy Pokémon when it spawns; if you have no Buddy
  Pokémon, it spawns as itself.

\subsection{Cherrim}
Cherrim's Sunshine form normally spawns only in sunlight.
Its Overcast form spawns otherwise.
Only the Sunshine form knows the Weather Ball attack.
Both forms, incidentally, are boosted by clear weather.

\subsection{Rotom}
Rotom is Electric+Ghost, but its Heat, Wash, Fan, Frost, and Mow forms
 combine Electric with Fire, Water, Flying, Ice, and Grass respectively.
The five variants all have 204 ATK and 219 DEF, compared to Rotom's
 185 and 159 (all share 137 STA).
Rotom and Mow Rotom have the same attacks, but Heat, Wash, and Frost
  trade Overheat, Hydro Pump, and Blizzard for Ominous Wind,
  while Fan Rotom can learn Air Slash instead of Thunder Shock.

\subsection{Burmy}
Burmy has three ``cloaks'', varying only in appearance: Plant, Sandy, and Trash.
The corresponding Wormadam evolutions have different typing, different attacks,
 and---in the case of Trash---different stats.

\subsection{Meltan\label{subsec:meltan}}
Meltan only appears when the Mystery Box is in use\footnote{Though a Meltan encounter ends the ``Let's GO, Meltan'' Special Research.}.
Charging the Mystery Box requires linking a Pokémon HOME account and transporting
  a Pokémon to it.
Meltan evolves into the interesting Melmetal, and the Mystery
  Box spawns pretty continuously for an hour should you need harvest some Stardust
  the hard way.

\subsection{Vivillon\label{subsec:vivillon}}
Remember postcards (\autoref{sec:capacities})?
Each Gift received from another Trainer bears a postcard, indicating the Gift's origin.
These postcards can be saved.
The globe is partitioned into eighteen regions (\autoref{figure:vivillonregions}),
  and a different Vivillon form is associated with each.
Upon pinning three, nine, or multiples of fifteen postcards from a region,
  that region's Vivillon Collector medal (\autoref{sec:medals}) increases
  by a tier, and the Trainer encounters Scatterbug.
This does not happen more than thrice per calendar day.
The Vivillon into which this Scatterbug evolves (via Spewpa) takes that region's form.
\begin{figure}
\centering
\includegraphics[keepaspectratio,width=\textwidth]{images/vivillonregions.png}
\caption{Munda est omnis divisia in partes duodeviginti\label{figure:vivillonregions}}
\end{figure}

\subsection{Palkia, Dialga, and Giratina}
Each Pokémon of Myth has an ``Origin'' form with different stats and attacks, including
 an exclusive attack.
Origin Forme Giratina and Altered Forme Giratina both boast Shadow Force, Origin Forme Palkia Spacial Rend,
 and Origin Forme Dialga Roar of Time.
All three require an Elite Charged TM, but at least they all have STAB\@.

\subsection{Castform\label{subsec:castform}}
Castform has four weather-specific forms.
Under sunshine, it spawns its Sunny form.
Rain favors its Rain form, and I'll give you three guesses as to its Snowy form.
These have Fire, Water, and Ice monotyping respectively.
Otherwise, Castform spawns as its Normal type.
All have the fast attack Tackle, and the appropriate variant of the charged attack Weather Ball.
Each also has its own attacks.

\subsection{Deoxys\label{subsec:deoxys}}
Deoxys has three alternate forms: Defense, Attack, and Speed.
The Defense and Attack forms are heavily weighted towards DEF and ATK, respectively.
The normal form is also ATK-heavy, but not so much as Attack.
The Speed form is the most balanced of the four.
All can learn Zen Headbutt and Psycho Boost, but their attacks are otherwise somewhat distinct.

\subsection{Gimmighoul\label{subsec:gimmighoul}}
Another one requiring interaction with the wider Pokémon empire, this time
 the Nintendo Switch games \textit{Pokémon Scarlet} or \textit{Pokémon Violet}.
Sending postcards to one of these games (whatever that means) earns you the Coin Bag,
 which functions as a Gimmighoul-specific incense for half an hour.
The Gimmighoul coins necessary to evolve into Gholdengo can be acquired by capturing
 Gimmighoul, or spinning Golden Pokéstops.
Pokéstops become golden with active Golden Lures.
Golden Lures are acquired by sending multiple postcards to these same games.

\subsection{Aegislash\label{subsec:aegislash}}
Aegislash has Blade and Shield forms which switch the other's ATK and DEF stats.
Outside of battle, it is always in the Shield form (97 ATK, 291 DEF).
Immediately before throwing a charged move, it takes on its Blade form.
Upon using a shield, or being substituted, it reverts to Shield form.
Since its Shield form (considered for League CP bounds) has much lower CP,
 this allows a high ATK to be snuck under a CP bound (though it will only
 be effective for charged attacks).
Aegislash cannot defend gyms, assault gyms, or participate in raids.

\begin{tipbox}[title=Galar First Partners]
The ``Galar First Partners'' refer to Grookey, Scorbunny, and Sobble, the
initial partner Pokémon from \textit{Sword and Shield}.
\end{tipbox}

\subsection{Tyrogue and the Hitmons\label{subsec:tyrogue}}
Tyrogue's evolution is based on its IV.
If $I_A$ has the highest value, it evolves into Hitmonlee.
If $I_D$ has the highest value, it evolves into Hitmonchan.
Hitmontop completes the trio for $I_S$.
If two components tie for the highest value, the evolution takes one or the other
 paths with equal probability.
In the case of a three-way tie, all three evolutions are possible, but it is believed that Hitmontop
 dominates\footnote{This is a Silph Road claim from 2018; I cannot corroborate it.}.

\subsection{Furfrou, Spinda, and Unown\label{subsec:furfrou}}
Easily my most despised Pokémon, Furfrou has many forms.
Some are event-specific.
Some are region-specific.
All are worthless.
Should one for some reason wish to do so, a Trainer can change Furfrou's form
  at any time for 25 Furfrou Candy and 10,000 Stardust,
Furfrou at least has the decency to be easily caught, unlike the contemptible Unown.
Unown has 28 forms: one for each letter of the Latin alphabet, and two
  resembling deformed sperm.
In a grim future Trainers hunt new Unown, trading Shiny Octothorpes and
  Crowned Ampersands and Mega U+03CC GREEK SMALL LETTER OMICRON WITH TONOS.
Spinda has twenty visually distinct forms, but only nine have been released.
We wait with bated breath for the other eleven.

%\vfill{\centering\includegraphics[width=\linewidth,keepaspectratio]{images/carnivine.png}}
