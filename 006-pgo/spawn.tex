\chapter{Spawning and catching\label{chap:spawn}}
Pokémon GO is intended to be played on the move.
Trainers primarily interface with the Map screen, onto which is projected
  some approximation of the local geography.
Pokémon spawning on this map will be a Trainer's major source of raw materials.
In addition to the Map, encounters follow victories against Team GO Rocket,
  wins in raids and Max Battles, and completion of certain research.
Encounters feature a summary display including the Pokémon's \CP{} (\autoref{sec:cp}),
  and icons indicating any weather leading to boosting (\autoref{sec:weather}),
  any type of lure that led to the spawn,
  Shiny status,
  and that the species has been previously captured\footnote{These traits are stored, and can be seen when browsing inventory.}.
The \CP{} will be replaced with ``???'' if the species has never been seen, or if
  the \CP{} exceeds that of any owned Pokémon.

\section{Wild spawns\label{sec:spawns}}
Decorations around a Pokémon indicate how it spawned (\autoref{table:spawnreasons}).
\begin{table}
\centering
\begin{tabular}{lll}
  Decoration & Reason for spawn & Visible to\\
\Midrule
  Single white circle & Daily spawn & Trainer\\
  Three white circles & Spawn point & All\\
  Rotating blue spiral & Weather-boosted & All\\
  Pink circles & Incense & Trainer\\
  Blue circles & Adventure incense & Trainer\\
  Swirling petals & Lure & All\\
  Orange circles & Sunsteel Strike & Trainer\\
  Dark blue circles & Moongeist Beam & Trainer\\
\end{tabular}
  \caption{Spawn decorations\label{table:spawnreasons}}
\end{table}
Weather-boosted (\autoref{sec:weather}) types enjoy higher spawn rates,
  in addition to higher levels and IVs (\autoref{table:ivfloors}).

\subsection{Lures and incense\label{subsec:lures}}
Spawn rates can be increased by means of incense and lures.
Incense lasts for an hour, affects only the Trainer using it (i.e.\ other Trainers
  will neither see nor be able to interact with resulting spawns),
  and moves with the Trainer.
A Trainer can make use of only one incense at a time\footnote{The
  Mystery Box, Coin Bag, Sunsteel Strike Adventure Effect, and
  Moongeist Beam Adventure Effect similarly cannot be used
  alongside incense, or each other.}.
Incense will result in a spawn every five minutes for immobile Trainers.
Otherwise, incense will result in a spawn every 200 meters moved.
Incense is awarded at certain Trainer levels and for some Research,
  and can be bought for 40 Pokécoins (or eight for 250).

Once unlocked, the Trainer can use Adventure incense.
This special incense results in a greater diversity of spawns, including
  a small chance of the ``Legendary Galarian Birds'' (\autoref{sec:regions}).
Adventure incense functions only if the Trainer is moving.
So long as it is completely consumed by midnight, it is automatically replenished
  upon the change of day.
Adventure incense lasts fifteen minutes, so it must be applied before 2345h
  to receive the next day's portion.
Application of Adventure incense grants thirty Poké Balls if the Trainer's
  bag holds fewer than thirty total Poké, Great, and Ultra Balls (\autoref{table:balls}),
  and can accommodate the balls.

Six kinds of lures (\autoref{table:lures}) can be installed into Pokéstops.
A Pokéstop supports only one lure at a time.
Flower petals will swirl around a Pokéstop with an active lure,
  and also any Pokémon spawned as a result of the lure.
Unlike incense, spawns due to lures are available to all Trainers.
A lure is active for thirty minutes, and then consumed.
Once installed into a Pokéstop, a lure cannot be removed nor deactivated.
Some evolutions (\autoref{sec:evolution}) can only take place near an active lure
 (\autoref{table:condevolutions}).
\begin{table}
\centering
\begin{tabular}{lp{.75\textwidth}}
  Cause & Effects\\
  \Midrule
  Lure & General increase in spawns\\
  Glacial Lure & Increased Ice and Water spawns\\
  Magnetic Lure & Increased Electric, Rock, and Steel spawns\\
  Mossy Lure & Increased Bug, Grass, and Poison spawns\newline{}Apple drops\\
  Rainy Lure & Increased Bug, Electric, and Water spawns\\
  Golden Lure & Attracts Roaming Form Gimmighoul\newline{}Gimmighoul Coins from spins \newline{}Extra items from spins\\
  Mystery Box & Meltan spawns\newline{}Movement not required\\
  Coin Bag & Gimmighoul spawns\newline{}Enhanced Gimmighoul catch rate\\
  Moongeist Beam & Nighttime spawns\newline{}Perform nighttime evolutions any time\\
  Sunsteel Strike & Daytime spawns\newline{}Perform daytime evolutions any time\\
\end{tabular}
  \caption{Lures and their effects\label{table:lures}}
\end{table}
Pokémon spawned due to incense or lures do not show up in the matrix
  of nearby Pokémon.

\section{Catching\label{sec:catch}}
Catching a Pokémon requires throwing a ball close enough to capture it,
 and that the Pokémon not escape from the ball.
Each form has a \texttt{CollisionRadiusM} and \texttt{CollisionHeightM}; this (possibly scaled by
 the individual Pokémon's height) defines the size of the hitbox.
Each species has dodge, movement, and attack characteristics, a distance, a catch rate $C$, and a flee rate
 $F$\footnote{See the \texttt{PokemonEncounterAttributes} protobuf.}.
Attacks during encounters cannot hurt the Trainer, but the Pokémon cannot be caught
  while attacking (the ball will bounce away, though it might be saved by a Great Buddy).
Catching can still occur during a dodge or move.
The catch rate determines likelihood of the Pokémon escaping a ball,
  while the flee rate determines likelihood of the Pokémon leaving the encounter altogether.
Pokémon flee only immediately following an escape; they will not flee after a throw which
  does not result in capture.
Pokémon that flee do not reappear on the map\footnote{The Trainer does receive 25 consolation XP points.}.
Research and GBL award encounters cannot flee.
Encounters following raids, Max Battles, and Team GO Rocket have a fixed number
 of balls; Pokémon cannot flee until they are exhausted, at which point they always flee.
While holding a ball, a target ring will encircle the Pokémon, looping from a maximum to a minimum size.
Its color (red, orange, or green) is an indicator of the capture probability $P_C$:
\begin{align*}
  P_\mathrm{C} &= 1-\left(1 - \frac{C}{2\cdot{}\CPM}\right)^{M_\mathrm{C}}\\
  M_\mathrm{C} &= M_\mathrm{Ball} \cdot M_\mathrm{Berry} \cdot M_\mathrm{Throw} \cdot M_\mathrm{Medal} \cdot M_\mathrm{Adv} \cdot M_\mathrm{Enc}
\end{align*}
\begin{itemize}
  \item \CPM{} is the Combat Power Multiplier (\autoref{sec:cpm}).
  \item $M_\mathrm{Ball}$ is the multiplier due the ball:
\begin{table}[h!]
\centering
\begin{tabular}{lr}
Ball & Multiplier\\
\Midrule
Poké & 1\\
Great & 1.5\\
Ultra & 2\\
Master & ∞\\
\end{tabular}
  \caption{Balls and their capture multipliers\label{table:balls}}
\end{table}
\item $M_\mathrm{Berry}$ is the multiplier due a berry (\autoref{table:berries}).
\item $M_\mathrm{Throw}$ is the multiplier due throw accuracy. Throw accuracy is summarized
   with ``Excellent!'', ``Great!'', ``Nice!'', or silence.
    This summary is a low-resolution signal of $M_\mathrm{Throw}$, with a range over [1.0, 2.0].
\begin{table}[h!]
\centering
\begin{tabular}{lr}
Summary & $M_\mathrm{Throw}$ range\\
\Midrule
``Excellent!'' & [1.7, 2.0]\\
``Great!'' & [1.3, 1.7)\\
``Nice!'' & [1.0, 1.3)\\
No summary & 1\\
\end{tabular}
  \caption{The throw summary is an indication of $M_\mathrm{Throw}$\label{table:throw}}
\end{table}
$M_\mathrm{Throw}$ is multiplied by 1.7 for a curveball.
\item $M_\mathrm{Medal}$ is the multiplier due medals (\autoref{sec:medals}).
\item $M_\mathrm{Adv}$ is 1.5 for Ice Burn, 1.25 for Freeze Shock, and 1 otherwise.
\item $M_\mathrm{Enc}$ is 2 for research encounters, and 1 otherwise.
\end{itemize}
If you throw your ball too directly at Tyrunt, it'll eat it.
If a response is not received from the server in a timely fashion, the Pokémon will escape.
If GPS indicates determines sustained travel over a certain speed, the Pokémon will escape, and probably flee.
In neither case is there an indication of the escape's irregular cause.

Raids and Team GO Rocket encounters use special ``Premier Balls'' rather than your inventory.
Max Battles use ``Power Balls'' with the same principles.
Premier Balls cannot be saved after the encounter, and regular balls from
  a Trainer's bag cannot be used in these encounters (though berries can).
The number of Premier Balls available for an encounter depends on a number of factors (\autoref{table:premierrocket}
  and \autoref{table:premierraid}), and their catch rate increases up to two times normal as more are thrown in an encounter.
Shiny (\autoref{sec:shiny}) Legendary raid bosses have a 100\% catch rate.
\begin{table}
\centering
\begin{tabular}{lr}
Criterion & Balls\\
\Midrule
Defeat Grunt & 2\\
Defeat Leader & 5\\
Remaining Pokémon & 1--3\\
Hero medal & 0--4\\
Purifier medal & 0--4\\
\end{tabular}
  \caption{Premier balls for Team GO Rocket encounters\label{table:premierrocket}}
\end{table}
\begin{table}
\centering
\begin{tabular}{lr}
Criterion & Balls\\
\Midrule
  Defeat boss & 6\\
  Damage contribution & 0--6\\
  Gym control & 0--2\\
  Gym level & 0--3\\
  Friendship bonus & 0--4\\
  Speed bonus & 2--8\\
\end{tabular}
  \caption{Premier balls for raid bonus challenges\label{table:premierraid}}
\end{table}
Any capture, whether a result of spawns, battles, or research, awards XP (\autoref{table:catchxp}),
 Stardust (\autoref{table:stardust}), and Candy (\autoref{subsec:getcandy}).
\begin{table}
\centering
\begin{tabular}{lr}
Event & XP \\
\Midrule
Basic capture, unless\ldots & 100\\
``Nice!'' capture, unless\ldots & 120\\
``Great!'' capture, unless\ldots & 200\\
``Excellent!'' capture & 1000\\
Curveball capture bonus & 20\\
First throw bonus & 50\\
100th collection of species bonus & 100\\
AR Plus bonus & 300\\
First capture of the day bonus, unless\ldots & 1500\\
First capture of the 7th day of streak & 6000\\
Master Ball bonus & 1000\\
\end{tabular}
  \caption{XP awards for captures\label{table:catchxp}}
\end{table}

\subsection{Berries\label{sec:berries}}
Berries can enhance the catching process (\autoref{table:berries}).
Only one berry can be applied to a Pokémon at a time.
If a berry is active, its icon will be shown on the encounter summary.
The berry is consumed upon use, but persists on the Pokémon across Pokéballs
  and even encounters (i.e., you can leave the encounter, and if you return,
  the berry will still be applied).
If the Pokémon escapes a Pokéball, however, any applied berry is gone forever.
A new berry can be used in this case.
\begin{table}
  \centering
  \begin{tabular}{lp{.75\textwidth}}
Berry & Effect \\
\Midrule
Razz  & 150\% catch percentage\\
Nanab & Immobilizes Pokémon\\
Pinap & Boosts Candy rewards: 3→7, 5→11, 10→23\\
Silver pinap & 180\% catch percentage\newline Boosts Candy rewards: 3→7, 5→11, 10→23\\
Golden razz & 250\% catch percentage\newline Restore full \HP{} to Gym defender\\
\end{tabular}
  \caption{Berries and their uses\label{table:berries}}
\end{table}
No aiming is necessary for berries\footnote{Aiming \textit{is} necessary when feeding a Buddy Pokémon.
If you miss, the Pokémon will gaze forlornly at the berry as it disappears into nothingness.}.
The berry control always displays a Razz berry, even if there
  are no Razz berries in inventory---don't get tricked!

\section{Eggs\label{sec:eggs}}
Eggs can be received from gifts, spinning Pokéstops, Adventure Sync rewards,
  and defeating Team GO Rocket leaders.
Eggs placed in an incubator will hatch after sufficient travel.
An incubator can hold only one egg at a time, and eggs cannot be removed from incubators.
Incubators cannot be loaded without an open slot in Pokémon storage, which the egg will fill
 until it hatches.
Trainers start with an incubator with unlimited uses.
Limited incubators and Super incubators can be used only three times.
Super incubators cut the necessary travel distance by 33\%.
Pokémon hatched from eggs use their egg's origin for their ``catch location'',
 which can be useful in trades (\autoref{sec:trades}).
The hatchling's IV, size, and shininess are determined when the egg is created.
Level is set at the same time, to the lesser of the Trainer's level and 20.
Hatching an egg rewards XP, Stardust, and Candy (\autoref{table:eggrewards}).
\begin{tipbox}[title=Baby Pokémon]
``Baby Pokémon'' are available only from eggs (or trades).
This includes Pichu, Cleffa, Igglybuff, Togepi, Tyrogue, Smoochum, Elekid, Magby,
 Azurill, Wynaut, Chingling, Budew, Riolu, Bonsly, Mime Jr., Happiny, Munchlax, and Mantyke.
\end{tipbox}
\begin{table}
\centering
\begin{tabular}{llrrr}
Km & Source & XP & Stardust & Candy\\
\Midrule
  2 & Pokéstops and gyms & 500 & 400--800 & 5--15\\
  5 & Pokéstops, gyms, Adventure Sync & 1000 & 800--1600 & 10--21\\
  7 & Gifts (friends and Mateo) & 1000 & 800--1600 & 10--21\\
  10 & Pokéstops, gyms, Adventure Sync & 2000 & 1600--3200 & 16--32\\
  12 & Team GO Rocket leaders & 4000 & 3200--6400 & 16--32\\
\end{tabular}
\caption{Eggs\label{table:eggrewards}}
\end{table}
There is sufficient storage for nine eggs of any type, plus three ``bonus''
 spots for eggs won from Team GO Rocket Leaders or as Adventure Sync rewards.
Whatever their origin, eggs will first fill up the main storage.
Different classes of egg source from different pools of potential Pokémon.
Pools are divided into multiple tiers according to rarity.
A tier is selected using a weighted random number, and the particular hatchling
 is randomly chosen from that tier.
Egg pools change on a seasonal basis, and are often changed for events (\autoref{sec:timeline}).
\begin{tipbox}[title=Seeing the egg pool]
Press on an egg and scroll down to see the various tiers for that egg, ranging
  from one egg (most common) to five eggs (least common).
\end{tipbox}

\section{Shiny Pokémon and other visual forms\label{sec:shiny}}
Many Pokémon have an alternate presentation known as their ``Shiny'' form (\autoref{chap:species}).
These forms are collected in their own Pokédex (\autoref{sec:dexen}).
Shadow and Purified Pokémon can be Shiny.
Shininess is preserved across evolution, purification, and trades.
Shininess is determined for individual Trainers' encounters: two Trainers might disagree on whether the same spawn is Shiny.
Shiny forms are fairly rare (\autoref{table:shiny}), but various events
  enhance the probability of their generation\footnote{I have been unable to validate these rates myself; they are taken from the Fandom wiki and Reddit.
  Strong evidence exists that certain forms enjoy persistently boosted Shiny rates, but---as far as I can tell---Niantic
  has never confirmed this. Check out \href{https://shinyrates.com}{shinyrates.com}.}.
\begin{table}
\centering
\begin{tabular}{ll}
Context & Probability of shine \\
\Midrule
  Wild spawns & 1/512 (0.195\%) \\
  Team GO Rocket Grunt shadows & 1/256 (0.39\%) \\
  Team GO Rocket Leader shadows & 1/64 (1.56\%) \\
  Wild spawn capable of Mega Evolution & 1/64 (1.56\%)\\
  Ditto & 1/64 (1.56\%)\\
  Community Day spotlight & 1/25 (4\%)\\
  5🟉 raid & 1/20 (5\%) \\
\end{tabular}
  \caption{Likelihood of Shiny Pokémon\label{table:shiny}}
\end{table}
Premier Balls have 100\% catch rates against Shiny Legendary raid bosses, (but \textit{not} shiny raid bosses in general).
Shiny Legendaries encountered in the wild (in particular, the Galarian Birds) do not typically flee\footnote{They \textit{will} flee if the speed limit is exceeded.}.
Zorua and Ditto (\autoref{subsec:ditto}) do not indicate Shiny status until revealed (an encounter
  with a bare Zorua \textit{will} immediately indicate Shiny status).
Despite offering no advantage in battle, and being in no way associated with
  higher levels or IVs, Shiny forms are of great importance to many players.
Personally, unless I expect to trade them, I transfer them to the Professor;
 his usual culinary magic leads to a particularly shiny feast of harvested Pokésoul.
For those determined to collect Shiny forms, the key is always to maximize encounters.
Enter as many encounters as possible, check the Pokémon summary
  for the Shiny icon (\calign{\includegraphics[width=1em,keepaspectratio]{images/shiny.png}}),
  and exit the encounter immediately if it's absent.
Pokémon thus encountered can be distinguished by their turning to face the Trainer on the map.
When raiding for Shiny Pokémon, try to avoid weather-boosted raid bosses, as they will require more time to defeat.

Events sometimes make possible special backgrounds or clothed forms.
Pikachu gets dressed up in something every few months.
``Costumed'' Pokémon often cannot be evolved (\autoref{sec:evolution}).

\vfill
\begin{tipbox}[title=An aside regarding independent events,title style={color=Green!50!black}]
\addcontentsline{toc}{section}{An aside regarding independent events}
\textbf{Q:} How many raids need I do to guarantee at least one Shiny?

\textbf{A:} No finite number of events guarantees the desired result.

\textbf{Q:} OK, but if I do, like, twenty raids, and it's a 5\% chance, it basically has to happen, right?

\textbf{A:} Alas, twenty is finite. 36\% chance of zero gets.

\textbf{Q:} Bullshit, what if I do fifty raids?

\textbf{A:} Alas, fifty is finite. There is about a 7.7\% chance of no love.

\textbf{Q:} My friend got four shiny Pokémon in three raids.

\textbf{A:} That's not a question. It is furthermore untrue.

\textbf{Q:} I'm pretty sure---

\textbf{A:} It's a bullshit patty between two slices of lies is what it is.

\textbf{Q:} Maybe it was four raids.

\textbf{A:} There's a 0.000625\% chance of four wins in four tries. They're independent of your tests,
             so the chance of your friend pulling juice four times in a row while you ate shit given
             the same 5\% chance is 0.000509\%. That's about five times out of a million.
             If you find a lottery offering those odds on the jackpot, well, I'm not saying buy---

\textbf{Q:} Oh shit, does this affect my chance of winning the lottery?\\

Put another way, if you have a 5\% chance of an outcome, and do tests in sets of fourteen,
 over the long term you'll see the outcome in a little over half those sets.
Put yet another way, over the long run, you can expect a little over two Shiny
 Pokémon for every thousand wild spawns.\\

The probability of an outcome with probability $P$\, never being seen in $N$ independent
  events is ${(1 - P)}^N$. For a 5\% likelihood, $1 - P = 0.95$. For thirteen events,

  \[ 0.95^{13} ≈ 0.513 \]

There is about a 51.3\% chance that thirteen raids generate no Shiny Pokémon,
  so please don't go on Reddit claiming conspiracies on the part of
  random number generators or Niantic, looking like an innumerate dumbass.
\end{tipbox}
