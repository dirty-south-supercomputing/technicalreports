\chapter{Optimizing for CP limits}

Several leagues (and some Research tasks) place a ceiling on the CP of
 participating Pokémon.
If CP does not accurately represent a given Pokémon's chance to win, we ought
 be able to exploit this by assembling teams using undervalued Pokémon.
We will see that CP is indeed a very incomplete assessment for PvP.

A Pokémon's CP is defined as:
\[ CP = Mod_A \times \sqrt{Mod_D} \times \sqrt{Mod_S} \times CPM^2 \]

First off, notice that CP grows quadratically with $CPM$,
 linearly with $Mod_A$, and only logarithmically with
 $Mod_D$ and $Mod_S$.
Remember from our Damage equation that $Mod_A$ is divided by $Mod_D$
 to generate one of the factors.
This suggests that $Mod_D$ is undervalued by the CP equation
The CP equation would suggest that $Mod_A$ is divided by $\sqrt{Mod_D}$,
 and we will exploit this discrepancy\footnote{Why would the game use such
 a flawed measure of power? A single stat was clearly desired to cover both
 modes of Pokémon GO battling. Since Raids allow substitution of defeated
 Pokémon, but are subject to a timer, the ability to defend and absorb
 Damage is less important than the capability of inflicting damage.}
MHP isn't used in the Damage equation, but knocking out a Pokémon
 requires inflicting some average Damage $D$ $n$ times,
 where $n = \lceil\frac{MHP}{D}\rceil$.
It is thus similarly undervalued by CP\@.
The quadratic term for CPM makes sense from the perspective of the Damage
 equation, since CPM is used as a factor when calculating Damage in the
 case of both the attacker and defender.
Indeed, since CPM is also used to calculate MHP from $Mod_S$, an argument
 can be made that it ought be raised to the third power.
