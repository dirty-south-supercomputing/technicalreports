% arara: xelatex: {shell: true}
% arara: biber
% arara: xelatex: {shell: true}
% arara: xelatex: {shell: true}
% arara: xelatex: {shell: true}
\documentclass[letterpaper,10pt]{article}
\usepackage[margin=1in]{geometry}
\usepackage{newfloat}
\usepackage{hyperref}
\usepackage{graphicx}
\usepackage[font=small,labelfont=bf]{caption}
%\usepackage{draftwatermark}
\usepackage{fancyhdr}
\usepackage{makecell}
\usepackage{wrapfig}
\usepackage{parskip}
\usepackage[section]{placeins}
\usepackage{epigraph}
%\usepackage{sourcecodepro}
\usepackage{fontspec}
\usepackage[toc,nonumberlist,xindy]{glossaries}
\usepackage{relsize}
%\setmonofont[Scale=0.7]{Source Code Pro}
\setmonofont[Scale=0.8]{Unifont}
\defaultfontfeatures{Ligatures=TeX}
\usepackage[table]{xcolor}
%\usepackage{pdfpages}
\usepackage[titletoc,title]{appendix}
\usepackage{tipa}
\newfontfamily\greekfont{Noto Sans}
\definecolor{dsscawpurp}{HTML}{b079b0}
\definecolor{dsscawpurpcap}{HTML}{6c286c}
\usepackage[font={color=dsscawpurpcap},labelfont={sc}]{caption}
\usepackage[backend=biber,
date=iso,
seconds=true,
style=numeric,
bibencoding=utf8,
]{biblatex}

\tracinglostchars=2

%\SetWatermarkText{DRAFT}
%\SetWatermarkScale{1}

\hypersetup{
  colorlinks=true,
  linkcolor=blue,
  pdftitle={Proposal for a Prefix ZWJ},
}

\addbibresource{\jobname.bib}
\newenvironment{denseitemize}{
  \begin{itemize}
      \setlength{\itemsep}{0pt}
}{
  \end{itemize}
}

\pagestyle{fancy}
\rhead{
  \includegraphics[height=\fontcharht\font`\D,keepaspectratio=true]{../dsscaw-hdr.pdf}
  \textcolor{dsscawpurp}{DSSCAW Technical Report \#005}
}

\title{Proposal for a Prefix ZWJ}
\author{Nick Black, Consulting Scientist\\
\texttt{nickblack@linux.com}
}

%%%%%%%%%%%%%%%%%%%%%%%%%%%%%%%%%%%%%%%%%%%%%%%%%%%%%%%%%%%%%%%%%%%%%%%%
\begin{document}
%%%%%%%%%%%%%%%%%%%%%%%%%%%%%%%%%%%%%%%%%%%%%%%%%%%%%%%%%%%%%%%%%%%%%%%%
%\includepdf{media/frontcover.pdf}
%\date{March 24, 2020}
\maketitle
\date{}
\vspace{1in}
\begin{abstract}
When processing a stream of Unicode, there are several cases where it
is useful or even necessary to know that a codepoint is only part of a
ZWJ sequence. This is not possible with existing ZWJ semantics, since
the ZWJ arrives only after each non-terminating combined character. I
propose a stacking, prefixing ZWJ, indicating that the subsequent two
encoded codepoints and/or ZWJ sequences are to be combined into a single
ZWJ sequence.
\end{abstract}
\thispagestyle{empty}
%%%%%%%%%%%%%%%%%%%%%%%%%%%%%%%%%%%%%%%%%%%%%%%%%%%%%%%%%%%%%%%%%%%%%%%%

\section{Introduction}
As the author of a modern library for TUIs and character graphics\cite{notcurses}
library, I've spent significant time over the past two years working with
complex Unicode in interactive environments. Interactivity, and more
fundamentally streaming, adds complexity to handling Unicode, especially
with regard to combining characters, ZWJ sequences\cite{zwjseqs}, and
segmentation\cite{segmentation}.

\pagenumbering{roman}

\end{document}
