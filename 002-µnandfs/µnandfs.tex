% arara: xelatex: {shell: true}
% arara: biber
% arara: xelatex: {shell: true}
% arara: xelatex: {shell: true}
\documentclass[letterpaper,10pt]{article}
\usepackage[margin=1in]{geometry}
\usepackage{hyperref}
\usepackage{graphicx}
\usepackage[justification=centering,font=small,labelfont=bf]{caption}
\usepackage{fancyhdr}
\usepackage{fontspec}
\setmonofont{Inconsolata}[Scale=MatchLowercase]
\defaultfontfeatures{Ligatures=TeX}
\usepackage{xeCJK}
\usepackage{minted}
\usepackage[backend=biber,
date=iso,
seconds=true,
style=numeric,
bibencoding=utf8,
]{biblatex}

\addbibresource{\jobname.bib}
\usemintedstyle{colorful}
\newenvironment{denseitemize}{
  \begin{itemize}
      \setlength{\itemsep}{0pt}
}{
  \end{itemize}
}

\pagestyle{fancy}
\rhead{DSSCAW Technical Report \#002}

\title{µnandfs:\\
A NAND Blobstore for Memory-Starved Platforms\thanks{
 \href{https://www.dsscaw.com/}{Dirty South Supercomputing} on behalf
 of \href{https://www.vakaros.com/}{Vakaros} of Atlanta, GA.
}\\
}
\author{Nick Black, Consulting Scientist\\
\texttt{nickblack@linux.com}
}

%%%%%%%%%%%%%%%%%%%%%%%%%%%%%%%%%%%%%%%%%%%%%%%%%%%%%%%%%%%%%%%%%%%%%%%%
\begin{document}
%%%%%%%%%%%%%%%%%%%%%%%%%%%%%%%%%%%%%%%%%%%%%%%%%%%%%%%%%%%%%%%%%%%%%%%%
\maketitle
\thispagestyle{fancy}
\date{}
\begin{abstract}
I was tasked with designing and implementing a persistent associative array
mapping names to arbitrary data---i.e. a single-directory filesystem, often
called a \textit{blobstore}---using the Nordic Semiconductor nRF52840 and two
Winbond W25N01GV gigabit SLC NAND chips. The contract also required necessary
QSPI drivers. The requirements permitted 4KB of RAM,
allowed no use of other persistent storage, and mandated a fully asynchronous
API running on ``bare metal'' (no OS, realtime or otherwise). I detail my
resulting deliverable, µnandfs, and demonstrate its generally performant
and robust fulfillment of these specs. I also describe its pathological worst
case behaviors.
\end{abstract}
%%%%%%%%%%%%%%%%%%%%%%%%%%%%%%%%%%%%%%%%%%%%%%%%%%%%%%%%%%%%%%%%%%%%%%%%
\section{Introduction}
The client's initial request was simply ``a filesystem on the nRF52840 using
two W25N01GV NANDs, plus any necessary drivers, plus an entirely asynchronous
C++ API, in as little RAM as possible''. Refinement of these requirements
determined that:
\begin{denseitemize}
\item Most files would be on the order of 1KB, with a few files
       on the order of tens of megabytes. These larger files would grow
       over time, via substantial (page-sized) appends.
\item It was not thought necessary to have directories, nor links (either hard
       or symbolic), but it must be possible to remove files, reclaiming both
       their space and name.
\item It must be possible to have multiple files open at once, but it is not
       necessary that a single file support multiple open handles. Writing
       to a file need not be visible to an existing reader.
\item No more than 4KB of RAM was to be consumed, and ideally not more than 2KB
       would be persistently consumed. Callers could be required to supply
       an additional 2KB for the duration of their operation.
\item Wear-leveling must be as close to uniform as possible. Ideally, no block
       would be erased two times more than any other block. Robustness in
       general is at a premium.
\end{denseitemize}

The nRF52840\parencite{nrf52840} SoC pairs an ARM Cortex-M4F with 1MB of
NOR flash and 256KB of RAM, along with a wealth of interconnection
capabilities. This storage is shared with the ``S140
SoftDevice''\parencite{s140}, a closed-source BlueTooth stack, which consumes
slightly more than 100KB of RAM and significant flash.
Two Winbond W25N01GV\parencite{winbond}
128MB NANDs, each capable of QSPI at up to 104MHz, were added to the PCB.
One QSPI and three SPI masters are available, each clocked at 32MHz (ignoring
overhead, 32MHz QSPI moves an ideal 16MB/s, SPI a respectable 4MB/s). Nordic's nRF5 SDK\parencite{nrf52sdk}
version 15.3.0 was linked into our binary, and the DUT was probed via 10-pin
J-Link\parencite{segger} connection from an nRF52-DK\parencite{nrf52dk}.

\section{Details of NAND flash}

NAND flash---typically packaged as a collection of chips, and often managed via
an on-board controller---makes up the majority of modern solid state drives,
flash drives, and memory cards. It is cheaper and denser than NOR flash,
faster than spinning disk, \textit{much} faster than EEPROMs, but typically
less reliable than all three. There are severe constraints on how it can be
used: a chip of NAND flash is divided into some number of \textit{blocks}, which are
themselves divided into \textit{pages}. Erasing a block changes all bits within to
1s\footnote{Nothing about the NAND memory cell itself---floating-gate MOSFETs
connected in series---requires large-scale operations. Larger blocks mean
faster operations (per bit), cheaper chips, and less power draw\ldots plus
amplified errors, and reduced flexibility.}. Data can be written a page at a
time within a block, but usually only in ascending page order---out-of-order
page programming can upset data in adjacent wordlines (a ``program
interference``\parencite{interference}). Data can be read a page at a time from
anywhere within the block, but reading a page too many times can upset data in
an adjacent page (a ``read disturb''\parencite{readdisturb}). SLC NAND flash stores one bit per cell,
and blocks can be erased on the order of $10^5$ times. MLC and TLC variants
encode more bits per cell, and can be reliably erased far fewer times (they
also tend to be slower, and to require more ECC bits per data bit). As a block
wears out, programming time will decrease, while erase latency
increases\parencite{needtoknow}. Finally, it is common for MLC and TLC NAND to
exhibit ``fast pages'' and ``slow pages'', but this arises from the
distribution of bits within a multilevel cell to different pages, and thus does
not affect SLC NAND\parencite{anomalies} such as our Winbond.

\subsection{Details of the Winbond W25N01GV}
The W25N01GVxxIG/IT\footnote{The ``xx'' is a package code, one of ZE (WSON), SF (SOIC),
or TB/TC (TFBGA). IG/IT differentiates between devices which reset into ``continuous mode''
or ``buffered mode''. We always use the (default) buffered mode on our IGs.} is
organized as $2^{10}$ blocks of $2^6$ 2KB pages each ($2^{17}$ bytes per block), for
a total of $2^{16}$ pages and $2^{27}$ data bytes (128MB). Blocks must be
erased before their pages can be programmed with 0s, and pages must be
programmed in ascending order. A page may be read from any block at any time.
In addition to the 2KB of data, each page has a 64 byte ``Spare Area'',
primarily used for hardware-managed ECC\footnote{Those who long to live
dangerously can disable hardware ECC. We don't.}, but offering 16 ECC-protected bytes
(in 4 discontiguous 4-byte chunks) for our use. A bad block mapper (BBM) of 20
LUTs is at our disposal, providing transparent remapping of block addresses.
Finally, there are 10 ``One-Time Program'' pages, which can be written to only
once.

Block erasure operations can fail, setting the FAIL-E bit; likewise page
programming and FAIL-P. Both of these bits are reset when the next operation
is started. Page read operations that fail their ECC check result in one or
more ECC error bits being set. These error bits remain high until the chip is
reset. Most operations set the BUSY bit, and only when this bit goes low is it
safe to consider an operation completed.

\subsection{Interactions with the nRF52840}
The Winbond's command set bears no resemblance to the ONFI
standard\parencite{onfi}, so there's no need to learn the latter. Unfortunately,
it also differs wildly from that supported by the nRF52840's QSPI
interface. Extensive use of the {\texttt{CINSTR}} ``custom instruction''
facility was needed, but eventually full-speed communication was had over
QSPI with the two NANDs. Alas, the Nordic unit can only deal with pages of
either 256 or 512 bytes. Using the maximum 512 byte setting, the last 512 bytes
written anywhere in the page would be returned for subsequent reads from the
page of any length. Eventually, it was confirmed with Nordic support that their
QSPI controller simply can't drive a 2KB page for more than 25\% of its
capacity. A bit-banging approach, were it even possible to do robustly,
would consume far too many CPU cycles (remember, all interfaces were to be
asynchronous). Faced with either a 75\% reduction in speed or a 75\% reduction
in capacity, the clients chose to retain capacity, and we moved to the SPI interfaces.
Note that hanging both chips off the QSPI interface meant that only one would
be used at a time (aside from a mirrored configuration employing
``superblocks''\parencite{superblocks}), but placing the two on distinct SPI
interfaces allows for parallelism. Full, concurrent utilization of both SPIs would represent
50\% of the original QSPI capacity.

\section{Blobstore API}
I had some ideas regarding the internal design, taking inspiration from
classic work on log-structured filesystems\parencite{sprite}, particularly
as implemented in NAND-focused projects such as F2FS\parencite{f2fs},
JFFS2\parencite{jffs}, and YAFFS\parencite{yaffs}. At the same time, I wanted
a simpler scheme, one implementable in 2KB of RAM, by one engineer, in four
weeks of part-time work. These ideas informed the proposed API, which happily
ended up pretty well-suited to the final implementation.

\begin{listing}[ht]
\caption{{\texttt{VK\_FS}} class public API ({\texttt{NANDDev}} defines CS pin and SPI device)}
\begin{minted}{C++}
constexpr size_t BLOB_NAME_MAXLEN = 58; // Including mandatory NUL byte!
constexpr auto NAND_PAGE_DATABYTES = 2048u;
constexpr auto NAND_PAGE_SPAREBYTES = 64u;
constexpr auto NAND_PAGE_SIZE = NAND_PAGE_DATABYTES + NAND_PAGE_SPAREBYTES;

using blob_t = uint64_t;
using NANDPage = std::array<uint8_t, NAND_PAGE_SIZE>;
using Handler = void(*)(void *, int);
using FSCallback = void(*)(void*, blob_t, size_t);

class VK_FS {
 static int Format(Handler fxn, void* vctx, NANDDev* nand);
 template<class It> int Init(Handler fxn, void* vctx, It b, It e, bool mirror);
 int Fsck(Handler fxn, void* vctx, bool checkdata);
 // flags include BLOB_CREAT, BLOB_EXCL, and BLOB_KILL
 blob_t OpenBlob(const std::string& name, unsigned flags,
                 FSCallback cb, void* vctx);
 blob_t ExtendBlob(blob_t blob, const void* buf, size_t len,
                   FSCallback cb, void* vctx, NANDPage& scratch);
 blob_t ReadBlob(blob_t blob, void* buf, size_t len, unsigned offset,
                 FSCallback cb, void* vctx, NANDPage& scratch);
 blob_t Bloblen(blob_t blob, FSCallback cb, void* vctx, NANDPage& scratch);
 blob_t CloseBlob(blob_t blob, FSCallback cb, void* vctx);
 blob_t ReplaceNamedBlob(const std::string& name, const void* buf, size_t len,
                         FSCallback cb, void* vctx);
 blob_t RemoveNamedBlob(const std::string& name, const void* buf, size_t len,
                        FSCallback cb, void* vctx);
 blob_t ListBlobs(blob_t blob, FSCallback cb, void* vctx, NANDPage& scratch);
 blob_t Sync(FSCallback cb, void* vctx);
 int Reset();
}
\end{minted}
\end{listing}

A {\texttt{NANDDev}} is a small structure, often {\texttt{constexpr}}, tying
together a SPI master identifier and a Chip Select line (numerous slaves can be
connected to a single SPI master, and selected via this line). The same {\texttt{NANDDev}}
must not be provided to two {\texttt{VK\_FS}} objects. To {\texttt{Format}} a
NANDDev is simply to erase all of its blocks, restoring all bits to their default
1. {\texttt{Init}} is handed one or more {\texttt{NANDDev}}s, and the boolean
{\texttt{mirror}} (if false, the NANDs are put in linear combination). All
specified devices must be freshly formatted, or they must make up an existing
filesystem (by virtue of having previously been fed together to
{\texttt{Init}})\footnote{In most mirrors, we'd want to be able to add a
  replacement device, but that won't be happening with our PCB in the field.}.
If {\texttt{Init}} is able to recognize the devices, it calls back with
success, and the object can be used until a call to {\texttt{Reset}}.

The primary handle provided to callers is the {\texttt{blob\_t}}\footnote{We encode
all handle state in the \texttt{blob\_t}, held by the client. ``File descriptors''
imply indices into space proportional to the number of open handles, a luxury we're not afforded.}.
This 64-bit value is opaque to callers except through the following two
functions:

\begin{listing}[ht]
\caption{{\texttt{VK\_FS}} functionality for inspecting {\texttt{blob\_t}}}
\begin{minted}{C++}
// Returns true iff the blob_t is an error code
static inline bool BlobErrorP(blob_t code) {
  return code != (code & ~0x3ul);
}

// Errors can be any arbitrary distinct blob_t with a 1 in the LSB. Each of
// these is all 1s, except for one low bit (for differentiation).
constexpr blob_t BLOBD_EINPROGRESS = ~(0x1u << 22); // request not yet complete
constexpr blob_t BLOBD_EIO = ~(0x1u << 23); // error talking to device
constexpr blob_t BLOBD_ENOSPC = ~(0x1u << 24); // no room to write
constexpr blob_t BLOBD_ENOENT = ~(0x1u << 25); // no BLOB_CREAT, but blob did not exist
constexpr blob_t BLOBD_ENAMETOOLONG = ~(0x1u << 26); // no NUL in 58 bytes of name
constexpr blob_t BLOBD_EINVAL = ~(0x1u << 27); // bad flags or invalid blobd or invalid read length
constexpr blob_t BLOBD_EEXIST = ~(0x1u << 28); // blob existed with BLOB_EXCL
constexpr blob_t BLOBD_INVALID = ~0ul; // generic error

static inline const char* BlobStrerror(blob_t res) {
  switch(res){
    case BLOBD_EINPROGRESS: return strerror(EINPROGRESS);
    case BLOBD_EIO: return strerror(EIO);
    case BLOBD_ENOSPC: return strerror(ENOSPC);
    case BLOBD_ENOENT: return strerror(ENOENT);
    case BLOBD_ENAMETOOLONG: return strerror(ENAMETOOLONG);
    case BLOBD_EINVAL: return strerror(EINVAL);
    case BLOBD_EEXIST: return strerror(EEXIST);
    case BLOBD_INVALID: [[fallthrough]];
    default:
      if(BlobErrorP(res)){
        return "Unknown unandfs error";
      }else{
        return strerror(0);
      }
  }
}
\end{minted}
\end{listing}

Functions will generally return {\texttt{BLOBD\_EINPROGRESS}}, and call back
asynchronously with the true handle. Callers must be prepared for an immediate
return, however, in which case no callback will be invoked. The proper idiom
is thus typically to invoke the function, check if the result is
{\texttt{BLOBD\_EINPROGRESS}}, and if not, directly invoke the callback
from the original callsite.

{\texttt{OpenBlob}} searches for the specified name, and then functions according
to the provided flags. If {\texttt{BLOB\_CREAT}} is provided, the blob will be
created if it does not yet exist (without \texttt{BLOB\_CREAT}, {\texttt{BLOBD\_ENOENT}} would
be returned). If {\texttt{BLOB\_EXCL}} is provided (providing {\texttt{BLOB\_EXCL}}
without {\texttt{BLOB\_CREAT}} results in {\texttt{BLOBD\_EINVAL}}), the
blob \textit{must not} exist, or \texttt{BLOBD\_EEXIST} will be returned. If
\texttt{BLOB\_KILL} is provided (providing \texttt{BLOB\_KILL} with \texttt{BLOB\_CREAT}
results in \texttt{BLOBD\_EINVAL}), the blob will be removed if it exists (otherwise,
\texttt{BLOBD\_ENOENT} is returned). All other errors can also occur with this
function. The handle can now be used in calls to \texttt{ReadBlob}, \texttt{ExtendBlob},
\texttt{Bloblen}, and \texttt{CloseBlob}. Behavior is undefined if the blob
is modified through another handle or function. Failure to call \texttt{CloseBlob}
does \textit{not} cause a resource leak, but timely use of \texttt{CloseBlob}
can result in fewer wasted NAND pages.

\texttt{ExtendBlob} adds up to a page worth of data to the blob. The callback
specifies a new handle which should be used for further extensions (extending
again on the same blob will replace the extended data). No matter the extension's
length, a full page will be consumed for the data, plus an inode (1/32 of a page)
to describe the data page; it is thus desirable for callers to buffer a full page
worth of output. A blob can grow as large as 16MB-1 (this is the sum of actual
data, not consumed pages). Note that other handles will not be updated; it is
not possible for a reader to notice the new data save by calling \texttt{OpenBlob}
anew. The amount written is returned as the third argument to the callback.

\texttt{ReadBlob} allows up to a page worth of data to be read from the blob,
at the specified offset. If the requested amount of data is not available, a
short read will be returned as the third argument to the callback. \texttt{Bloblen}
gets the length of the blob through this handle (again, concurrent extensions
will not be visible). \texttt{CloseBlob} hints that no more extensions will
be performed through this handle.

\texttt{RemoveNamedBlob} and \texttt{ReplaceNamedBlob} skip the name lookup
procedure, forcefully seizing the name and either marking it dead, or
initializing it with up to one page of data. Extending a handle which has been
\texttt{Remove}d or \texttt{Replace}d leads to frightful shenanigans (typically,
the old blob will ``return from the grave'').

\texttt{ListBlobs} ought first be called with \texttt{BLOBD\_INVALID}. It will
populate \texttt{scratch} with between 1 and 31 blob names and lengths, and
return a nonce \texttt{blob\_t}. This can be used in a subsequent call, which
will return a new nonce, and a new set of blob names and lengths. Eventually,
\texttt{BLOBD\_ENOENT} will be returned, and all blobs will have been
enumerated. I have not bothered to include the functions for extracting these
data.

\texttt{Sync} ought be called prior to \texttt{Reset} or process exit. Any
metadata not yet written to NAND will be flushed. Calling \texttt{Sync} can
waste up to 62 pages, so do it only when necessary. Failure to call \texttt{Sync}
can currently result in up to 61 pages being lost. \texttt{Fsck} can be used to
perform an intensive validation of the filesystem, requiring a read of every
metadata page (1/32 of the pages). If \texttt{checkdata} is true, \texttt{Fsck}
will read every page of the NAND, looking for ECC errors. In the absence of
any other load, this requires roughly 42s per NAND.

All of these functions can involve some asynchronous I/O. In the best case,
necessary information is readily available in the active metadata, and the
return is immediate. Extending a blob can result in up to two pages being
written out in the active block, and a full block copy if space needed to be
found in a used block (but will usually result in one page being written).
In the worst case, where the named blob does not exist on a full NAND, a name
lookup requires reading every metadata page. Creating (or removing) a file
usually leads to no I/O, but in the worst case leads to a page being written
out followed by a block copy. Reading a blob requires traversing some number of
metadata pages; in the worst case, where a blob has grown fat off partial
writes, this could again require reading every metadata page (see
Section \ref{futurework} for plans on improving this linear behavior on reads).

\texttt{BLOB\_NAME\_MAXLEN} is derived from filesystem internals, as we shall
now see.

\section{Filesystem design}
I knew I wanted to provide at least a few dozen bytes' worth of name per blob.
I knew 16 bits of blob length were too little, and 32 bits too much.
I considered metadata compression, and said ``no'' to madness.

I knew I wanted to march purely forward writing within a chip---i.e., in
addition to the mandatory in-order programming of pages, I hoped to erase
blocks in order, with no special meaning for any given block. Any such scheme
is guaranteed perfectly uniform wear leveling, and should require minimal state
(I knew that any frequently-changing state couldn't be persisted without either
redundancy or special-purpose blocks). I considered bundling metadata with
data. Such a scheme would be very flexible; I could write the metadata length
and true length into the Spare Area, grow the data up from the bottom and the
metadata down from the top, and have type-tagged metadata tuned to the particular
blob. I could even pack multiple small blobs into a single page. Unless I
wanted to buffer data across calls there was no way for a caller to know how
much data to optimally pass with variable-length, in-page metadata. Buffering
would require RAM linear with the number of incomplete writes, though, so it
was out. Metadata and data would live on different pages, and I'd burn my
allotted 2KB building up the next metadata page in memory.

A problem presented itself, however: how would I determine state on
initialization? As noted above, if I persisted my state
to a known location, those locations would see significant wear. Under the
mixed pages plan, every page had identifiable metadata, and that metadata could
carry a decreasing nonce. Find the discontinuity in nonces, and you've found
your active page (more on this momentarily). By placing metadata pages at the
32nd and 64th page of each block, nonces can be recorded in these pages only, and
the same O(lgN) algorithm performed (on 1/32 as many pages). I call these
32-page regions \textit{zones}. This yielded up a natural 64-byte fixed inode
size for each of the 31 previous pages, plus 64 bytes for zone metadata. Three
bytes were dedicated to ``blob length through this inode'', two bytes to
``previous data page'' (the data page governed by an inode can be derived in
O(1), as can the inode corresponding to a data
page), and one byte to a status bitfield. The remaining 58 bytes became
\texttt{BLOB\_NAME\_MAXLEN}\footnote{Metadata placed flexibly, only when
  appropriate for recent I/O, could be more efficient than periodic metadata
  with a fixed inode size (fewer pages used for metadata, fewer pages needing
  reading to describe a given file). Metadata and arbitrary data pages could be
  differentiated via a single bit in the Spare Area. Finding the metadata page
  describing a given data page becomes O(lgN) as opposed to O(1).}.

We define a few useful functions before proceeding:

\begin{listing}[ht]
\caption{Functions for navigating data and metadata pages, all O(1)}
\begin{minted}{C++}
constexpr size_t INODE_SIZE = 64;
static_assert(NAND_PAGE_DATABYTES % INODE_SIZE == 0);
constexpr auto PAGES_PER_ZONE = NAND_PAGE_DATABYTES / INODE_SIZE;
static_assert(PAGES_PER_ZONE > 1 && ((PAGES_PER_ZONE & (PAGES_PER_ZONE - 1)) == 0);
constexpr auto DATA_PAGES_PER_ZONE = PAGES_PER_ZONE - 1;
constexpr auto LAST_METADATA_PAGE = NAND_BLOCK_COUNT - 1;

static inline unsigned // Given a raw page, return its associated metadata page and zidx
MetadataForPage(unsigned pidx, unsigned* zidx) {
  if(zidx){
    *zidx = pidx % PAGES_PER_ZONE;
  }
  return pidx | DATA_PAGES_PER_ZONE;
}

static inline unsigned // Given a metadata page and inode number, return described page
PageMetadataOwns(unsigned pidx, unsigned zidx) {
  return (pidx & ~DATA_PAGES_PER_ZONE) + zidx;
}

static inline unsigned // Return metadata page of next zone
NextMetadataPage(unsigned pidx) {
  pidx = MetadataForPage(pidx, nullptr);
  return (pidx == LAST_METADATA_PAGE) ? DATA_PAGES_PER_ZONE : pidx + PAGES_PER_ZONE;
}

static inline unsigned // Return metatdata page of previous zone
PrevMetadataPage(unsigned pidx) {
  pidx = MetadataForPage(pidx, nullptr);
  return (pidx == DATA_PAGES_PER_ZONE) ? LAST_METADATA_PAGE : pidx - PAGES_PER_ZONE;
}

static inline unsigned // Return epoch corresponding to nonce
EpochForNonce(uin32_t nonce) {
  return ((uint64_t)0xffffffffull - nonce + NAND_BLOCK_COUNT - 1) / NAND_BLOCK_COUNT;
}
\end{minted}
\end{listing}

\subsection{Phased algorithm for block movement}
That's all well and good, but what becomes of our ineluctable march when we
wrap around to the front of the NAND, and can't blow away live data? We
sacrifice one block worth of capacity, designating it the \textit{victim
  block}. On first initialization, this is the last block. We keep a counter,
initialized to 0 and increased by 1 each time we erase the first bloc,
designating it the \textit{epoch}. Note that we can derive the epoch from a
nonce in O(1), and that epochs partition the nonce space.

Whenever our current block becomes the victim block (this won't happen until the
last block, but it will happen every other zone afterwards), designate the subsequent
block to be the \textit{Dunkirk block}. Load the Dunkirk block's first zone's
metadata page as the working page. Beginning with the first inode, check to
see if it's dead. If so, it doesn't need rescuing; use its slot in the victim
to write the next data page. If it is live, copy it into the victim block to
fight another day. Copy pages in bursts when appropriate, and when the zone is
complete, write metadata to the victim block. Repeat this with the second zone.
Designate the survivor block the victim block, and start over.

This maintains perfect wear leveling, and ought even perform reasonably well
when there are plenty of deletions. But what about outstanding handles to the
Dunkirk block? Their data just moved out from underneath them! Twenty bits in
each \texttt{blob\_t} are thus used to encode their origin epoch, i.e. the epoch
at the time of \texttt{OpenBlob}. If a \texttt{blob\_t} is handed to us with
our epoch, there's no way it's been moved. Otherwise, it has been moved zero or
more times; this can be computed in O(1), and applied to the incoming physical
location as an O(1) displacement modulo \texttt{NAND\_BLOCK\_COUNT}.

\subsection{Initialization algorithm}
Knowing that freshly-formatted or otherwise erased blocks are all 1s, and
knowing that we write a decreasing nonce to each zone we write out, and knowing
that we write zones in order, a valid µnandfs can take one of three forms:
\begin{denseitemize}
\item \textit{More than one zone has a nonce of all 1s, these zones are
    contiguous, and they include the last zone. All other zones form a
    decreasing procession.} This is a µnandfs which was interrupted before
    filling its NAND the first time. The current zone is the first zone with an
    all-1s nonce.
\item \textit{Exactly one or two zones, within the same block, have a nonce of
    all 1s. All other zones form a decreasing procession.} The µnandfs was interrupted
  either before writing to the last zone its first time through the NAND, or after
  erasing the current block.  The current zone is the first zone with an all-1s
  nonce.
\item \textit{All zones form a decreasing procession, save for a single discontinuity,
    when the nonce increases by the number of zones.} The µnandfs, on at least
    its second trip through, was interrupted before erasing the current block.
    The current zone is the zone with the greatest nonce.
\end{denseitemize}
If we're not concerned with verifying these properties in full (that's left to
the optional \texttt{Fsck}), we can recover the current zone with a single
O(lgN) algorithm, one very much like binary search.

\section{Future work} \label{futurework}
It is desirable to encode large files more efficiently. Currently, a file of
the maximum $2^{24}-1$ bytes requires 8192 inodes, and thus (in the best case)
530 pages of metadata, and attendant reads. When all of a zone is a single
blob, some different scheme ought be employed to encode this fact.
Unfortunately, it's unlikely that such a write could ever be performed as a
single ExtendBlob() operation, due to the RAM requirements of such a buffer.

It is not currently possible to use the two chips in parallel as a single,
unified blobstore. If placed on distinct SPI masters, they can be used in
parallel as two blobstores, or as a mirrored set. In any configuration---even
a single SPI master---they can be combined as a linear device. A unified
2Gbit namespace accessible at 2x32MHz, however, is not yet possible. Mirrored
sets could be implemented via superblocks for
maximum performance, requiring only a single SPI master. This would be especially
advantageous if the QSPI interface had been capable of driving the Winbonds.

An ECC failure reading a metadata page currently results in the blobstore being
brought offline. Given that ECC failures can often be recognized immediately
after programming, metapages should probably be read back following write,
remapped using the BBM LUTs, and written anew. In any case, a more graceful
recovery seems desirable.

We ought be able to handle shutdown without a call to \texttt{Sync}, at least
for pages which actually wrote out data \textbf{DETAILS...}.

%%%%%%%%%%%%%%%%%%%%%%%%%%%%%%%%%%%%%%%%%%%%%%%%%%%%%%%%%%%%%%%%%%%%%%%%
\printbibliography
\end{document}
