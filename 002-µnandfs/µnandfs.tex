\documentclass[letterpaper,10pt]{article}
\usepackage[margin=1in]{geometry}
\usepackage{hyperref}
\usepackage{graphicx}
\usepackage[justification=centering,font=small,labelfont=bf]{caption}
\usepackage{fancyhdr}
\usepackage{fontspec}
\defaultfontfeatures{Ligatures=TeX}
\usepackage{xeCJK}
\usepackage[backend=biber,
date=iso,
seconds=true,
style=numeric,
bibencoding=utf8,
]{biblatex}
\addbibresource{\jobname.bib}
\newenvironment{denseitemize}{
  \begin{itemize}
      \setlength{\itemsep}{0pt}
}{
  \end{itemize}
}

\pagestyle{fancy}
\rhead{DSSCAW Technical Report \#002}

\title{µnandfs:\\
A NAND Blobstore for Memory-Starved Platforms\thanks{
 \href{https://www.dsscaw.com/}{Dirty South Supercomputing} on behalf
 of \href{https://www.vakaros.com/}{Vakaros} of Atlanta, GA.
}\\
}
\author{Nick Black, Consulting Scientist\\
\texttt{nickblack@linux.com}
}

%%%%%%%%%%%%%%%%%%%%%%%%%%%%%%%%%%%%%%%%%%%%%%%%%%%%%%%%%%%%%%%%%%%%%%%%
\begin{document}
%%%%%%%%%%%%%%%%%%%%%%%%%%%%%%%%%%%%%%%%%%%%%%%%%%%%%%%%%%%%%%%%%%%%%%%%
\maketitle
\thispagestyle{fancy}
\date{}
\begin{abstract}
I was tasked with designing and implementing a persistent associative array
mapping names to arbitrary data---i.e. a single-directory filesystem, often
called a \textit{blobstore}---using the Nordic Semiconductor nRF52840 and two
Winbond W25N01GV gigabit SLC NAND chips. The contract also required necessary
QSPI drivers. The requirements permitted 4KB of RAM, permitted 4KB of RAM,
allowed no use of other persistent storage, and mandated a fully asynchronous
API running on ``bare metal'' (no OS, realtime or otherwise). I detail my
resulting deliverable, µnandfs, and demonstrate its generally performant
and robust fulfillment of these specs. I also describe its pathological worst
case behaviors.
\end{abstract}
%%%%%%%%%%%%%%%%%%%%%%%%%%%%%%%%%%%%%%%%%%%%%%%%%%%%%%%%%%%%%%%%%%%%%%%%
\section{Introduction}
The client's initial request was simply ``a filesystem on the nRF52840 using
two W25N01GV NANDs, plus any necessary drivers, plus an entirely asynchronous
C++ API, in as little RAM as possible''. Refinement of these requirements
determined that:
\begin{denseitemize}
\item Most files would be quite small (on the order of 1KB), and a few files
       would be large (on the order of multiple megabytes).
\item It was not thought necessary to have directories, nor symlinks.
\item It must be possible to remove files, reclaiming their space.
\item It must be possible to have multiple files open at once, but it is not
       necessary that a single file support multiple open handles.
\item No more than 4KB of RAM was to be consumed, and ideally persistent use
       would be not more than 2KB. Callers, however, could be required to supply
       an additional 2KB for the duration of their call.
\end{denseitemize}

The nRF52840\parencite{nrf52840} SoC pairs an ARM Cortex-M4F with 1MB of
NOR flash and 256KB of RAM, along with a wealth of interconnection
capabilities. This storage is shared with the ``S140
SoftDevice''\parencite{s140}, a closed-source BlueTooth stack, which consumes
slightly more than 100KB of RAM and significant flash. One QSPI and three SPI
masters are available, and can be clocked up to 32MHz. Two Winbond W25N01GV\parencite{winbond}
128MB NAND flashes were added to the PCB, each capable of QSPI at up to 104MHz.
Nordic's nRF5 SDK\parencite{nrf52sdk} version 15.3.0 was linked into our binary,
and the DUT was probed via 10-pin J-Link\parencite{segger} connection from
an nRF52-DK\parencite{nrf52dk}.

\section{Future work}
It is desirable to encode large files more efficiently. Currently, a file of
the maximum $2^{24}-1$ bytes requires 8192 inodes, requiring (in the best case)
530 pages of metadata, and attendant reads. When all of a zone is a single
blob, some different scheme ought be employed to encode this fact.
Unfortunately, it's unlikely that such a write could ever be performed as a
single ExtendBlob() operation, due to the RAM requirements of such a buffer.

It is not currently possible to use the two chips in parallel as a single,
unified blobstore. If placed on distinct SPI masters, they can be used in
parallel as two blobstores, or as a mirrored set. In any configuration---even
a single SPI master---they can be combined as a linear device. A unified
2Gbit namespace accessible at 2x34MHz, however, is not yet possible.

%%%%%%%%%%%%%%%%%%%%%%%%%%%%%%%%%%%%%%%%%%%%%%%%%%%%%%%%%%%%%%%%%%%%%%%%
\printbibliography
\end{document}
